\documentclass[11pt]{amsart}

\usepackage{amsthm, amssymb,amsmath}
\usepackage{graphicx}

\theoremstyle{definition}  % Heading is bold, text is roman
\newtheorem{theorem}{Theorem}
\newtheorem{definition}{Definition}
\newtheorem{example}{Example}

\newcommand{\ojo}[1]{{\sffamily\bfseries\boldmath[#1]}}

\newcommand{\Z}{\mathbb{Z}}
\newcommand{\N}{\mathbb{N}}
\newcommand{\Q}{\mathbb{Q}}
\newcommand{\R}{\mathbb{R}}
\newcommand{\C}{\mathbb{C}}

\newcommand{\nullspace}{\mathrm{null}}
\newcommand{\rank}{\mathrm{rank}}


\oddsidemargin 0pt
\evensidemargin 0pt
\marginparwidth 0pt
\marginparsep 10pt
\topmargin -10pt
\headsep 10pt
\textheight 8.4in
\textwidth 7in

%\input{../header}
\newcommand{\range}{\mathrm{range}}
\newcommand{\NULL}{\mathrm{null}}
\newcommand{\IP}[1]{\left \langle\, #1 \,\right \rangle}
\newcommand{\LL}{\mathcal{L}}
\newcommand{\MM}{\mathcal{M}}
\newcommand{\PP}{\mathcal{P}}
\newcommand{\MATRIX}{\mathcal{M}_{n\times n}}
\newcommand{\trace}{\mathrm{tr}}


\begin{document}

%\homework{}{Homework X}
\begin{center}
\Large{Math 725 -- Advanced Linear Algebra}\\
\large{Paul Carmody}\\
Assignment \#11 -- Due 12/8/23
\end{center}
\vskip 1.0cm

%\homework{}{Homework XI}

\noindent
{\bf 1.} Let $A$ be an invertible square matrix.  Show that $|\det A| = \sigma_1 \sigma_2 \cdots \sigma_n$. 
\begin{align*}
	A &= U \Sigma V^T \\
	|\det A| &= |\det(U \Sigma V^T)|\\
	&= |\det(U)\det( \Sigma)\det (V^T)| \\
	&= |\det(\Sigma)| \\
	&= |\sigma_1\cdot\sigma_2\cdots\sigma_n| \\
	&= \sigma_1\cdot\sigma_2\cdots\sigma_n 
\end{align*}because $U$ and $V$ are orthonormal, their determinant is one and all $\sigma_i>0$.
\\


\vskip 0.1cm
\noindent
{\bf 2.}  Let $A$ be a nonzero $m \times n$ matrix. Prove that $\sigma_1 \, = \, \max \{|| Au || \, : \, || u || = 1\}$. \\ 
\\
Suppose that this is not true.  Then let $\mu = \max \{|| Au || \, : \, || u || = 1\}$ and let $u'$ be such that $\mu = ||Au'||$ and $||u'|| = 1$.  Then, $\sigma_1-\mu \ge 0$ because $\sigma_1$ is the greatest eigenvalue which implies that $||Au'|| = \mu < ||\sigma_1 u'|| = \sigma_1$ hence a contradiction.\\

\vskip 0.1cm
\noindent
{\bf 3.}  Let $A$ and $A'$ be two nonzero $m \times n$ matrices with respective largest singular values $\sigma_1$ and 
$\sigma_1'$. Prove that the largest singular value of $A + A'$ is bounded above by $\sigma_1 + \sigma_1'$. 
\begin{align*}
	||(A+A')x|| &\le ||Ax||+||A'x||\\
%	&\le ||\sigma_1 x||+||\sigma'_1 x||\\
	&\le (||A||+||A'||)||x|| \\
	&\le (\sigma_1 + \sigma'_1)||x||
\end{align*}
 

\vskip 0.1cm
\noindent
{\bf 4.} Suppose $A$ is an $m \times n$ matrix and $B$ is $n \times m$ matrix obtained by rotating $A$ ninety degrees clockwise on paper
(not a standard matrix operation). Do $A$ and $B$ have the same singular values? Prove or give a counterexample. \\
\begin{align*}
	\text{Let } A &= \left( \begin{array}{ccccc}
		\lambda_1 & 0 & \cdots & 0 & 0\\
		0 & \lambda_2 & \cdots & 0 & 0\\
		\vdots & \vdots & \ddots & \vdots & \vdots \\
		0 & 0 & \cdots & \lambda_n & 0
	\end{array}	 \right) \\
	\text{Then } B &= \left(\begin{array}{cccc}
		0 & \cdots & 0 & \lambda_1 \\
		0 & \cdots & \lambda_2 & 0 \\
		\vdots & \vdots & \vdots & \vdots \\
		\lambda_n & 0 & \cdots & 0 \\
		0 & 0 & \cdots & 0
	\end{array}	 \right)
\end{align*}Clearly, $A$ has distinct eigenvalues of $\lambda_i$ which are not at all the same for $B$.


\vskip 0.1cm
\noindent
{\bf 5.} Let $A$ be an $m \times n$ matrix of rank $r > 0$ with singular values $\sigma_1, \ldots, \sigma_r$. 
Show that $|| A ||_F = \sqrt{\sigma_1^2 + \ldots + \sigma_r^2}$. \\
\\
Let the SVD of $A = U\Sigma V^T$ then
\begin{align*}
	||A||_F &= \sqrt{\sum_{i=1}^n\sum_{j=1}^m A_{ij}^2 }\\
	& = ||A^T|| \\
	||A^T A|| &= || (U\Sigma V^T)^TU\Sigma V^T|| \\
	&= ||V\Sigma^T U^TU\Sigma V^T|| \\
	&= ||V\Sigma^T \Sigma V^T|| \\
	&= ||\Sigma^T \Sigma || \\
	&= \sum_{i=1}^r \Sigma_{ii}^2 \\
	&= \sum_{i=1}^r \sigma_{i}^2 \\
\end{align*}







\end{document}