\documentclass[10pt,a4paper]{report}
\usepackage[utf8]{inputenc}
\usepackage{amsmath}
\usepackage{amsfonts}
\usepackage{amssymb}
\usepackage{amsthm}
\usepackage{hyperref}

\usepackage{multicol}
\usepackage{fancyhdr}
\usepackage[inline]{enumitem}
\usepackage{tikz}
\usepackage{tikz-cd}
\usetikzlibrary{calc}
\usetikzlibrary{shapes.geometric}
\usepackage[margin=0.5in]{geometry}
\usepackage{xcolor}

\hypersetup{
    colorlinks=true,
    linkcolor=blue,
    filecolor=magenta,      
    urlcolor=cyan,
    pdftitle={Tensors},
    pdfpagemode=FullScreen,
    }

%\urlstyle{same}

\newcommand{\CLASSNAME}{Abstract Algebra}
\newcommand{\STUDENTNAME}{Paul Carmody}
\newcommand{\ASSIGNMENT}{Dummitt and Foote}
\newcommand{\DUEDATE}{May 6, 20222}
\newcommand{\SEMESTER}{Spring 2021}
\newcommand{\SCHEDULE}{T/Th 2:00 -- 3:20}
\newcommand{\ROOM}{Remote}

\pagestyle{fancy}
\fancyhf{}
\chead{ \fancyplain{}{\CLASSNAME} }
%\chead{ \fancyplain{}{\STUDENTNAME} }
\rhead{\thepage}
\newcommand{\LET}{\text{Let }}
%\newcommand{\IF}{\text{if }}
\newcommand{\AND}{\text{ and }}
\newcommand{\OR}{\text{ or }}
\newcommand{\FORSOME}{\text{ for some }}
\newcommand{\FORALL}{\text{ for all }}
\newcommand{\WHERE}{\text{ where }}
\newcommand{\WTS}{\text{ WTS }}
\newcommand{\WLOG}{\text{ WLOG }}
\newcommand{\BS}{\backslash}
\newcommand{\DEFINE}[1]{\textbf{\emph{#1}}}
\newcommand{\IF}{$(\Rightarrow)$}
\newcommand{\ONLYIF}{$(\Leftarrow)$}
\newcommand{\ITH}{\textsuperscript{th} }
\newcommand{\FST}{\textsuperscript{st} }
\newcommand{\SND}{\textsuperscript{nd} }
\newcommand{\TRD}{\textsuperscript{rd} }
\newcommand{\INV}{\textsuperscript{-1} }

\newcommand{\XXX}{\mathfrak{X}}
\newcommand{\MMM}{\mathfrak{M}}
%\newcommand{\????}{\textfrak{A}}
%\newcommand{\????}{\textgoth{A}}
%\newcommand{\????}{\textswab{A}}

\DeclareMathOperator{\DER}{Der}
\DeclareMathOperator{\SGN}{sgn}

%%%%%%%
% derivatives
%%%%%%%

\newcommand{\PART}[2]{\frac{\partial #1}{\partial #2}}
\newcommand{\SPART}[2]{\frac{\partial^2 #1}{\partial #2^2}}
\newcommand{\DERIV}[2]{\frac{d #1}{d #2}}
\newcommand{\LAPLACIAN}[1]{\frac{\partial^2 #1}{\partial x^2} + \frac{\partial^2 #1}{\partial y^2}}

%%%%%%%
% sum, product, union, intersections
%%%%%%%

\newcommand{\SUM}[2]{\underset{#1}{\overset{#2}{\sum}}}
\newcommand{\PROD}[2]{\underset{#1}{\overset{#2}{\prod}}}
\newcommand{\UNION}[2]{\underset{#1}{\overset{#2}{\bigcup}}}
\newcommand{\INTERSECT}[2]{\underset{#1}{\overset{#2}{\bigcap}}}
\newcommand{\FSUM}{\SUM{n=-\infty}{\infty}}
       

%%%%%%%
% supremum and infimum
%%%%%%%

\newcommand{\SUP}[1]{\underset{#1}\sup \,}
\newcommand{\INF}[1]{\underset{#1}\inf \,}
\newcommand{\MAX}[1]{\underset{#1}\max \,}
\newcommand{\MIN}[1]{\underset{#1}\min \,}

%%%%%%%
% infinite sums, limits
%%%%%%%

\newcommand{\SUMK}{\SUM{k=1}{\infty}}
\newcommand{\SUMN}{\SUM{n=1}{\infty}}
\newcommand{\SUMKZ}{\SUM{k=0}{\infty}}
\newcommand{\LIM}[1]{\underset{#1}\lim\,}
\newcommand{\IWOB}[1]{\LIM{#1 \to \infty}}
\newcommand{\LIMK}{\IWOB{k}}
\newcommand{\LIMN}{\IWOB{n}}
\newcommand{\LIMX}{\IWOB{x}}
\newcommand{\NIWOB}{\LIM{n \to \infty}}
\newcommand{\LIMSUPK}{\underset{k\to\infty}\limsup \,}
\newcommand{\LIMSUPN}{\underset{n\to\infty}\limsup \,}
\newcommand{\LIMINFK}{\underset{k\to\infty}\liminf \,}
\newcommand{\LIMINFN}{\underset{n\to\infty}\liminf \,}
\newcommand{\ROOTRULE}[1]{\LIMSUPK \BARS{#1}^{1/k}}

\newcommand{\CUPK}{\bigcup_{k=1}^{\infty}}
\newcommand{\CAPK}{\bigcap_{k=1}^{\infty}}
\newcommand{\CUPN}{\bigcup_{n=1}^{\infty}}
\newcommand{\CAPN}{\bigcap_{n=1}^{\infty}}

%%%%%%%
% number systems (real, rational, etc.)
%%%%%%%

\newcommand{\REALS}{\mathbb{R}}
\newcommand{\RATIONALS}{\mathbb{Q}}
\newcommand{\IRRATIONALS}{\REALS \backslash \RATIONALS}
\newcommand{\INTEGERS}{\mathbb{Z}}
\newcommand{\NUMBERS}{\mathbb{N}}
\newcommand{\COMPLEX}{\mathbb{C}}
\newcommand{\DISC}{\mathbb{D}}
\newcommand{\HPLANE}{\mathbb{H}}

\newcommand{\R}{\mathbb{R}}
\newcommand{\Q}{\mathbb{Q}}
\newcommand{\Z}{\mathbb{Z}}
\newcommand{\N}{\mathbb{N}}
\newcommand{\C}{\mathbb{C}}
\newcommand{\T}{\mathbb{T}}
\newcommand{\COUNTABLE}{\aleph_0}
\newcommand{\UNCOUNTABLE}{\aleph_1}


%%%%%%%
% Arithmetic/Algebraic operators
%%%%%%%


\DeclareMathOperator{\MOD}{mod}
%\newcommand{\MOD}[1]{\mod #1}
\newcommand{\BAR}[1]{\overline{#1}}
\newcommand{\LCM}{\text{ lcm}}
\newcommand{\ZMOD}[1]{\Z/#1\Z}
\DeclareMathOperator{\VAR}{Var}
%%%%%%%
% complex operators
%%%%%%%

\DeclareMathOperator{\RR}{Re}
%\newcommand{\RE}{\text{Re}}
\DeclareMathOperator{\IM}{Im}
%\newcommand{\IM}{\text{Im}}
\newcommand{\CONJ}[1]{\overline{#1}}
\DeclareMathOperator{\LOG}{Log}
%\newcommand{\LOG}{\text{ Log }}
\newcommand{\RES}[2]{\underset{#1}{\text{res}} #2}

%%%%%%%
% Group operators
%%%%%%%

\newcommand{\AUT}{\text{Aut}\,}
\newcommand{\KER}{\text{ker}\,}
\newcommand{\END}{\text{End}}
\newcommand{\HOM}{\text{Hom}}
\newcommand{\CYCLE}[1]{(\begin{array}{cccccccccc}
		#1
	\end{array})}
\newcommand{\SUBGROUP}{\underset{\text{group}}\subseteq}	
%\newcommand{\SUBGROUP}{\subseteq_g}
\newcommand{\SUBRING}{\underset{\text{ring}}\subseteq}
\newcommand{\SUBMOD}{\underset{\text{mod}}\subseteq}
\newcommand{\SUBFIELD}{\underset{\text{field}}\subseteq}
\newcommand{\ISO}{\underset{\text{iso}}\longrightarrow}
\newcommand{\HOMO}{\underset{\text{homo}}\longrightarrow}

%%%%%%%
% grouping (parenthesis, absolute value, square, multi-level brackets).
%%%%%%%

\newcommand{\PAREN}[1]{\left (\, #1 \,\right )}
\newcommand{\BRACKET}[1]{\left \{\, #1 \,\right \}}
\newcommand{\SQBRACKET}[1]{\left [\, #1 \,\right ]}
\newcommand{\ABRACKET}[1]{\left \langle\, #1 \,\right \rangle}
\newcommand{\BARS}[1]{\left |\, #1 \,\right |}
\newcommand{\DBARS}[1]{\left \| \, #1 \,\right \|}
\newcommand{\LBRACKET}[1]{\left \{ #1 \right .} 
\newcommand{\RBRACKET}[1]{\left . #1 \right \]}
\newcommand{\RBAR}[1]{\left . #1 \, \right |}
\newcommand{\LBAR}[1]{\left | \, #1 \right .}
\newcommand{\BLBRACKET}[2]{\BRACKET{\RBAR{#1}#2}}
\newcommand{\GEN}[1]{\ABRACKET{#1}}
\newcommand{\BINDEF}[2]{\LBRACKET{\begin{array}{ll}
     #1\\
     #2
\end{array}}}

%%%%%%%
% Fourier Analysis
%%%%%%%

\newcommand{\ONEOTWOPI}{\frac{1}{2\pi}}
\newcommand{\FHAT}{\hat{f}(n)}
\newcommand{\FINT}{\int_{-\pi}^\pi}
\newcommand{\FINTWO}{\int_{0}^{2\pi}}
\newcommand{\FSUMN}[1]{\SUM{n=-#1}{#1}}
%\newcommand{\FSUM}{\SUMN{\infty}}
\newcommand{\EIN}[1]{e^{in#1}}
\newcommand{\NEIN}[1]{e^{-in#1}}
\newcommand{\INTALL}{\int_{-\infty}^{\infty}}
\newcommand{\FTINT}[1]{\INTALL #1 e^{2\pi inx\xi} dx}
\newcommand{\GAUSS}{e^{-\pi x^2}}

%%%%%%%
% formatting 
%%%%%%%

\newcommand{\LEFTBOLD}[1]{\noindent\textbf{#1}}
\newcommand{\SEQ}[1]{\{#1\,\}}
\newcommand{\WIP}{\footnote{work in progress}}
\newcommand{\QED}{\hfill\square}
\newcommand{\ts}{\textsuperscript}
\newcommand{\HLINE}{\noindent\rule{7in}{1pt}\\}

%%%%%%%
% Mathematical note taking (definitions, theorems, etc.)
%%%%%%%

\newcommand{\REM}{\noindent\textbf{\\Remark: }}
\newcommand{\DEF}{\noindent\textbf{\\Definition: }}
\newcommand{\THE}{\noindent\textbf{\\Theorem: }}
\newcommand{\COR}{\noindent\textbf{\\Corollary: }}
\newcommand{\LEM}{\noindent\textbf{\\Lemma: }}
\newcommand{\PROP}{\noindent\textbf{\\Proposition: }}
\newcommand{\PROOF}{\noindent\textbf{\\Proof: }}
\newcommand{\EXP}{\noindent\textbf{\\Example: }}
\newcommand{\TRICKS}{\noindent\textbf{\\Tricks: }}


%%%%%%%
% text highlighting
%%%%%%%

\newcommand{\B}[1]{\textbf{#1}}
\newcommand{\CAL}[1]{\mathcal{#1}}
\newcommand{\UL}[1]{\underline{#1}}

%%%%%%
% Linear Algebra
%%%%%%

\newcommand{\COLVECTOR}[1]{\PAREN{\begin{array}{c}
#1
\end{array} }}
\newcommand{\TWOXTWO}[4]{\PAREN{ \begin{array}{c c} #1&#2 \\ #3 & #4 \end{array} }}
\newcommand{\DTWOXTWO}[4]{\BARS{ \begin{array}{c c} #1&#2 \\ #3 & #4 \end{array} }}
\newcommand{\THREEXTHREE}[9]{\PAREN{ \begin{array}{c c c} #1&#2&#3 \\ #4 & #5 & #6 \\ #7 & #8 & #9 \end{array} }}
\newcommand{\DTHREEXTHREE}[9]{\BARS{ \begin{array}{c c c} #1&#2&#3 \\ #4 & #5 & #6 \\ #7 & #8 & #9 \end{array} }}
\newcommand{\NXN}{\PAREN{ \begin{array}{c c c c} 
			a_{11} & a_{12} & \cdots & a_{1n} \\
			a_{21} & a_{22} & \cdots & a_{2n} \\
			\vdots & \vdots & \ddots & a_{1n} \\
			a_{n1} & a_{n2} & \cdots & a_{nn} \\
		\end{array} }}
\newcommand{\SLR}{SL_2(\R)}
\newcommand{\GLR}{GL_2(\R)}
\DeclareMathOperator{\TR}{tr}
\DeclareMathOperator{\BIL}{Bil}
\DeclareMathOperator{\SPAN}{span}

%%%%%%%
%  White space
%%%%%%%

\newcommand{\BOXIT}[1]{\noindent\fbox{\parbox{\textwidth}{#1}}}


\newtheorem{theorem}{Theorem}[section]
\newtheorem{corollary}{Corollary}[theorem]
\newtheorem{lemma}[theorem]{Lemma}

\theoremstyle{definition}
\newtheorem{definition}[theorem]{Definition}
\newtheorem{prop}[theorem]{Proposition}

\theoremstyle{remark}
\newtheorem{remark}[theorem]{Remark}
\newtheorem{example}[theorem]{Example}
%\newtheorem*{proof}[theorem]{Proof}



\newcommand{\RED}[1]{\textcolor{red}{#1}}
\newcommand{\BLUE}[1]{\textcolor{blue}{#1}}
\newcommand{\COV}[1]{(#1_1 \quad \cdots \quad #1_n)}
\newcommand{\CONTRA}[1]{\PAREN{\begin{array}{c}
              #1^1\\
              \vdots \\ 
              #1^n
         \end{array}}}
\newcommand{\SPECIAL}[1]{\begin{center}
	{\Large \textbf{\textit{\\#1}} }
\end{center}
}

\title{Topology without Tears}
\author{Sidney A. Morris}
\date{June 2020}

\begin{document}

\maketitle

\tableofcontents

\chapter{Topology Spaces}
\section{Topology -- Exercises }

\begin{enumerate}
\item Let $x = \{a,b,c,d,e,f\}$.  Determine whether or not each of the following collections of subsets of $X$ is a topology on $X$:
\begin{enumerate}

	\item $\tau_1= \{X, \emptyset, \{a\}, \{a,f\},\{b,f\},\{a,b,f\}\};$
	
	No, $\{a,f\} \cap \{b,f\} = \{f\} \not \in \tau$.
	
	\item $\tau_2 = \{X, \emptyset, \{a,b,f\},\{a,b,d\},\{a,b,d,f\}\};$
	
	No, $\{a,b,f\}\cap\{a,b,d\} \not \in \tau$.
	
	\item $\tau_3 = \{X, \emptyset, \{f\}, \{e,f\}, \{a,f\}\}	;$
	
	No, $\{e,f\} \cup \{a,f\} = \{a,e,f\} \not \in \tau$.
\end{enumerate}

\item Let $X=\{a,b,c,d,e,f\}$.  Which of the following collections of subsets of $X$ is a topology on $X$? (Justify your answer.)

\begin{enumerate}
	\item $\tau_1 = \{X,\emptyset, \{c\},\{b,d,e\}, \{b,c,d,e\}, \{b\}\};$
	\item $\tau_2 = \{X, \emptyset, \{a\}, \{b,d,e\}, \{a,b,d\},\{a,b,d,e\}\};$
	\item $\tau_3 = \{X, \emptyset, \{b\}, \{a,b,c\}, \{d,e,f\}, \{b,d,e,f\}\};\}$
\end{enumerate}

\item If $X=\{a,b,c,d,e,f\}$, and $\tau$ is the discrete topology on $X$, which of hte following statements are true?

\noindent
\begin{enumerate*}
	\item $X \in \tau;$ YES
	\item $\{X\} \in \tau;$ ???
	\item $\{\emptyset\} \in \tau;$ ???
	\item $\emptyset \in \tau;$ YES\\
	\item $\emptyset \in X;  NO$
	\item $\{\emptyset\} \in X; NO$
	\item $\{a\}\in \tau; YES$
	\item $a \in \tau;$ NO\\
	\item $\emptyset \in X;$ NO
	\item $\{a\}\in X;$ NO
	\item $\{\emptyset\} \subseteq X;$ YES
	\item $a \in X;$ YES\\ 
	\item $X\subseteq \tau;$ YES
	\item $\{a\}\subseteq \tau;$ YES
	\item $\{X\} \subseteq \tau;$ YES
	\item $a \subseteq \tau;$ NO
\end{enumerate*}

\item Let $(X, \tau)$ be any topological space.  Verify that \textbf{\textit{the intersection of any finite number of members of $\tau$ is a member of $\tau$}}.

\item Let $\R$ be the set of all real numbers.  Prove that each of the following collections of subsets of $\R$ is a topology
\begin{enumerate}[label=(\roman*)]
	\item $\tau_1$ consists of $\R, \emptyset$, and every interval $(-n,n)$, for $n$ any positive integer, where $(-n,n)$ denotes the set $\{x \in\R:-n<x<n\};$
	\item $\tau_2$ consists of $\R, \emptyset$, and every interval $[-n,n]$, for $n$ any positive integer, where $[-n,n]$ denotes the set $\{x \in\R:-n\le x\le n\};$
	\item $\tau_3$ consists of $\R, \emptyset$, and every interval $[n,\infty)$, for $n$ any positive integer, where $[n,\infty)$ denotes the set $\{x \in\R:n\le x\};$
\end{enumerate}

\item 
\begin{enumerate}[label=(\roman*)]
	\item $\tau_1$ consists of $\N, \emptyset$, and every set $\{1,2,\dots,n\}$, for $n$ any positive integer. (This is called \textbf{\textit{initial segment topology}}).
	\item $\tau_2$ consists of $\N,\emptyset$, and every $\{n, n+1, \dots\}$, for $n$ any positive integer. (This is called the \textbf{\textit{final segment topology}}.)
\end{enumerate}

\item List all possible topologies on the following sets:
\begin{enumerate}
	\item $X = \{a,b\};$
	\item $Y = \{a,b,c\};$
\end{enumerate}

\item Let $X$ be an infinite set and $\tau$ a topology on $X$.  If every infinte subset of $X$ is in $\tau$, prove that $\tau$ is the discrete topology.

\item Let $\R$ be the set of all real numbers Precisely three of the followin ten collections are subsets of $\R$ are topologies.  Identify these and justifey your answer.
\begin{enumerate}[label=(\roman*)]
	\item $\tau_1$ consists of $\R, \emptyset$, and every interval $(a,b)$, for $a$ and $b$ any real numbers where $a<b$.
	\item $\tau_2$ consists of $\R, \emptyset$ and every interval $(-r,r)$, for $r$ any positive real number.
	\item $\tau_3$ consists of $\R,\emptyset$, and every interval $(-r,r)$, for $r$ any positive rational number;
	\item $\tau_4$ consists of $\R,\emptyset$, and every interval $[-r,r]$, for $r$ any positive rational number;
	\item $\tau_5$ consists of $\R,\emptyset$, and every interval $(-r,r)$, for $r$ any positive irrational number;
	\item $\tau_6$ consists of $\R,\emptyset$, and every interval $[-r,r]$, for $r$ any positive irrational number;
	\item $\tau_7$ consists of $\R,\emptyset$, and every interval $[-r,r)$, for $r$ any positive real number;
	\item $\tau_8$ consists of $\R,\emptyset$, and every interval $(-r,r]$, for $r$ any positive real number;
	\item $\tau_9$ consists of $\R,\emptyset$, and every interval $[-r,r]$,  and every interval $(-1,r)$, for $r$ any positive real number;
	\item $\tau_{10}$ consists of $\R, \emptyset$, every internval $[-n,n]$, and every interval $(-r,r)$, for $n$ any positive intger and $r$ any positive real number.
\end{enumerate}

\end{enumerate}
%end of 1.1 Exercises

\newpage
\section{Open Sets - Exercises}

\begin{enumerate}
\item List all 64 subsets of the set $X$ in Example 1.1.2.  Write down, next to each set, whether it is (i) clopen, (ii) neither open nor closed; (iii) open but not closed; (iv) closed but not open.

\newcommand{\OPEN}{\text{,open}}
\newcommand{\CLOSED}{\text{,closed}}
\newcommand{\CLOPEN}{\text{,clopen}}
\newcommand{\NOTT}{\text{,neither}}
\textbf{\textit{Example 1.1.2}}: Let $X=\{a,b,c,d,e,f\}$ and \[ \tau_1=\{X,\emptyset, \{a\}, \{c,d\},\{a,c,d\},\{b,c,d,e,f\}\}.\]
\begin{itemize}
	\item size one
	\begin{align*}
		\begin{array}{cccccc}
			\{a\}\CLOPEN&\{b\}\NOTT&\{c\}\NOTT&\{d\}\NOTT&\{e\}\NOTT&\{f\}\NOTT
		\end{array}
	\end{align*}
	
	\item size two
	\begin{align*}
		\begin{array}{ccccc}
			\{a,b\}\NOTT&\{a,c\}&\{a,d\}&\{a,e\}&\{a,f\}\\
			\{b,c\}&\{b,d\}&\{b,e\}&\{b,f\}\\
			\{c,d\}\OPEN&\{c,e\}&\{d,f\}\\
			\{d,e\}&\{d,f\}\\
			\{e,f\}\\
		\end{array}
	\end{align*}
	
	\item size three
	\begin{align*}
		\begin{array}{cccc}
			\{a,b,c\} & \{a,b,d\} & \{a,b,e\} & \{a,b,f\}\\
			\{a,c,d\}\OPEN & \{a,c,e\} & \{a,c,f\}\\
			\{a,d,e\} & \{a,d,f\}\\
			\{a,e,f\}\\
			\{b,c,d\} & \{b,c,e\} & \{b,c,f\}\\
			\{b,d,e\} & \{b,d,f\}\\
			\{b,e,f\}\\
			\{c,d,e\} & \{c,d,f\}\\
			\{c,e,f\}\\
			\{d,e,f\}
		\end{array}
	\end{align*}
	
	\item size four
	\begin{align*}
		\begin{array}{ccc}
			\{a,b,c,d\} & \{a,b,c,e\} & \{a,b,c,f\}\\
			\{a,b,d,e\} & \{a,b,d,f\}\\
			\{a,b,e,f\}\\
			\{b,c,d,e\} & \{b,c,d,f\}\\
			\{c,d,e,f\}
		\end{array}
	\end{align*}
	
	\item size five
	\begin{align*}
		\begin{array}{cc}
			\{a,b,c,d,e\} & \{a,b,c,d,f\} \\
			\{a,b,c,e,f\}\\
			\{a,b,d,e,f\}\\
			\{a,c,d,e,f\}\\
			\{b,c,d,e,f\}\CLOPEN
		\end{array}
	\end{align*}
	
	\item size six
	
		$\{a,b,c,d,e,f\}\OPEN$
\end{itemize}

\newpage
\item Let $(X,\tau)$ be a topological space with the property that every subset is closed.  Prove that it is a discrete space.

\begin{align*}
	S \subseteq X \implies X\backslash X \text{ is open } \implies X\backslash S \in \tau \\
	T \in \tau \implies X\backslash T \text{ is closed} \implies T \subseteq X
\end{align*}

\item Observe that if $(X,\tau)$ is a discrete space or an indiscrete space, then every opne set is a clopen set.  Find a topology $\tau$ on the set $X=\{a,b,c,d\}$ which is not discrete and is not indiscrete but has the property that every open set is clopen.

Let $\tau = \{X, \emptyset, \{a\}, \{b,c,d\}\}$

\item Let $X$ be an infinite set.  If $\tau$ is a topology on $X$ such that every infinite subset of $X$ is closed, prove that $\tau$ is the discrete topology.

\begin{align*}
	S \subseteq X \text{ and } |S| = \infty\\
	| X\backslash S | < \infty \implies X\backslash S \text{ is open}\\
\end{align*}there are an infinite number of finite subsets whose compliment is infinite and closed.  These are precisely what make up a discrete topology.

\item Let $X$ be an infinite set and $\tau$ a topology on $X$ with the property that the only infinite subset of $X$ which is open is $X$ itself.  Is $(X,\tau)$ necessarily an indiscrete space?

\item \begin{enumerate}[label=(\roman*)]
	\item Let $\tau$ be a topology on a set $X$ such that $\tau$ consists of precisely for sets; that is, $\tau = \{X,\emptyset, A,B\}$, where $A$ and $B$ are non-empty distinct proper subsets of $X$.  [$A$ is a \textbf{\textit{proper subset}} of $X$ means that $A \subseteq X$ and $A \ne X$.  This si denoted by $A\subset X$.]  Prove that $A$ and $B$ must satisfy exactly one of the following conditions.
	\begin{center}
	(a) $B=X\backslash A;$    (b) $A \subset B$;    (c) $B\subset A$;	
	\end{center}
[Hint. Firstly show that $A$ and $B$ must satisfy at least one of the conditions and then show that they cannot satisfy more than one of the conditions.]

\item Using (i) list all topologies on $X =\{1,2,3,4\}$ which consist of exactly four sets.
	
\end{enumerate}

\item \begin{enumerate}[label=(\roman*)]
	\item A recorded in \url{http://en.wikipedia.org/wiki/Finite_topological_space}, the number of distinct topologies on a set with $n \in \N$ points can be very large even for small $n$; namely when $n =2$, there are 4 topologies;  when $n=3$, there are 29 topologies:  when $n=4$, there are 355 topologies;  when $n=5$, there are 6942 topologies etc.  Using mathematical induction, prove that as $n$ increases the number of topologies increases.\\
%	[Hint. It suffices to show that if a set with $n$ points has $M$ distinct topolgies, then a set with $n+1$ points has at least $M+1$ topologies.]
	
	\item Using mathematical induction prove that if the finite set $X$ has $n \in \N$ then it has at least $(n-1)!$ distinct topolgies.\\
%	[Hint. Let $X = \{x_1,\dots,x_n\}$ and $Y=\{x_1,\dots, x_n,x_{n+1}\}$.  If $\tau$ is any topologyon $X$, fix an $i \in \{1,2,\dots,n\}.$  Definte a topology $\tau_i$ on $Y$ as follows: For each open set $U \in \tau$, definte $U_i$ by replacign any occurrence of $x_i$ in $U$ by $x_{n+1}$; then $\tau_i$ consists of all $U_i$ plus the set $Y$.  Verify that $\tau_i$ is indeed a topology on $y$.  Deduce that for each topology on $X$, there are at least $n$ distinct topologies on $Y$.]
	
	\item If $X$ is any infintie set of cardinality $\mathfrak{N}$, prove that there are at least $2^\mathfrak{N}$ distinct topologies on $X$.  Deduce that every infinite set has an uncountable number of distinct topolgies on it.
	%[Hint. Prove that there are at least $2^\mathfrak{N}$ distinct topologies with precisely 3 open sets.  For an introduction to cardinal numbers, see Appendix 1.]
\end{enumerate}
\end{enumerate}

\newpage
\section{Finite Closed Topology -- Exercises}

\begin{enumerate}
	\item Let $f$ be a function from a set $X$ into a set $Y$.  Then we stated in Example 1.3.9 that
\begin{align}
	f^{-1}\left(\bigcup_{j\in J}B_j \right) = \bigcup_{j\in J}f^{-1}(B_j)
\end{align} and 
\begin{align}
	f^{-1}\left( B_1 \bigcap B_2\right) = f^{-1}(B_1)\cap f^{-1}(B_2)
\end{align} 
for any subsets $B_j$ of $Y$ and any index set $J$.
\begin{enumerate}
	\item Prove that (1.1) is true
	
	\begin{align*}
		\text{Let } y &\in \bigcup_{j \in J} B_j\\
		\exists k \in J &\to y \in B_k \\
		f^{-1}(y) &\in f^{-1}\left(\bigcup_{j\in J} B_j\right) \text{ and } f^{-1}(y) \in f^{-1}(B_k) \\
		f^{-1}(B_k) & \subseteq f^{-1}\left(\bigcup_{j\in J} B_j\right)
	\end{align*}since there MUST be a $k$ for each $y$ then it must be that all $\cup_{j\in J}f^{-1}(B_j) \subseteq f^{-1}\left(\bigcup_{j\in J} B_j\right)$
	
	\item Prove that (1.2) is true.
	\item Find (concrete) sets $A_1,A_2, X$, and $Y$ and a function $f: X \to Y$ such that $f(A_1\cap A_2)\ne f(A_1)\cap f(A_2)$, where $A_1 \subseteq X$ and $A_2 \subseteq X$.
\end{enumerate}

\item Is the topology $\tau$ described in Exercises 1.1 \#6 (ii) the finite-closed topology?

$\tau_2$ consists of $\N,\emptyset$, and every $\{n, n+1, \dots\}$, for $n$ any positive integer. (This is called the \textbf{\textit{final segment topology}}.)

\SPECIAL{$T_1$-spaces}

\item A topological space $(X, \tau)$ is said to be a \textbf{$T_1$-space} if every singleton set $\{x\}$ is closed in $(X,\tau)$.  Show that precisely two of the following nine topological spaces are $T_1$-spaces. (Justify your answer).
\begin{enumerate}[label=(\roman*)]
	\item a discrete space.
	\item an indiscrete space with at least two points.
	\item an infinite set with the finite-closed topology.
	\item Exampe 1.1.2;
	\item Exercise 1.1 \#5 (i)
	
	$\tau_1$ consists of $\R, \emptyset$, and every interval $(-n,n)$, for $n$ any positive integer, where $(-n,n)$ denotes the set $\{x \in\R:-n<x<n\};$
	
	\item Exercise 1.1 \#5 (ii)
	
	$\tau_2$ consists of $\R, \emptyset$, and every interval $[-n,n]$, for $n$ any positive integer, where $[-n,n]$ denotes the set $\{x \in\R:-n\le x\le n\};$
	
	\item Exercise 1.1 \#5 (iii)
	
	$\tau_3$ consists of $\R, \emptyset$, and every interval $[n,\infty)$, for $n$ any positive integer, where $[n,\infty)$ denotes the set $\{x \in\R:n\le x\};$
	
	\item Exercise 1.1 \#6 (i)
	
	$\tau_1$ consists of $\N, \emptyset$, and every set $\{1,2,\dots,n\}$, for $n$ any positive integer. (This is called \textbf{\textit{initial segment topology}}).
	
	\item Exercise 1.1 \#6 (ii)
	
	$\tau_2$ consists of $\N,\emptyset$, and every $\{n, n+1, \dots\}$, for $n$ any positive integer. (This is called the \textbf{\textit{final segment topology}}.)
	
\end{enumerate}

\item Let $\tau$ be the finite-closed topology on a set $X$.  If $\tau$ is also the discrete topology, prove that the set $X$ is finite.

\SPECIAL{$T_0$-space and the Sierpinsi Space}

\item A topological space $(X,\tau)$ is said to be a \textbf{$T_0$-space} if for each pair of distinct points $a,b$ in $X$, either there exist an open seet containing $a$ and not $b$, or there exists an open set containing $b$ and not $a$.

\begin{enumerate}[label=(\roman*)]
	\item Prove that every $T_1$-space is a $T_0$-space.
	\item Which of $(i)-(iv)$ in Exercise 3 above are $T_0$-spaces?
	\item Put a topology $\tau$ on the set $X=\{0,1\}$ so that $(X,\tau)$ will be a $T_0$-space but not a $T_1$-space. [known as the \textbf{Sierpinski space}.]
	\item Prove that each of the topological spaces described in Exercise 1.1 \#6 is a $T_0$-space.
\end{enumerate}

	\SPECIAL{Countable-Closed Topology}

\item Let $X$ be any infinite set.  The \textbf{\textit{countable-closed topology}} is defined to be the topology having as its closed sets $X$ and all countable subsets of $X$.  Prove that this is indeed a topology on $X$.

\item Let $\tau_1$ and $\tau_2$ be two topologies on a set $X$.  Prove each of the following statements.
\begin{enumerate}[label=(\roman*)]
	\item $\tau_3$ is definted by $\tau_3 = \tau_1 \cup \tau_2$, then $\tau_3$ is not necessarily a topology on $X$. 
	\item If $\tau_4$ is defined by $\tau_4 = \tau_1 \cap \tau_2$, then $\tau_4$ is a topology on $X$.
	\item If $(X,\tau_1)$ and $(X,\tau_2)$ are $T_1$-spaces, then $(X,\tau_4)$ is a $T_1$-space.
	\item If $(X,\tau_1)$ and $(X,\tau_2)$ are $T_0$-spaces, then $(X,\tau_4)$ is not necessarily a $T_0$-space.
	\item If $\tau_1, \tau_2, \dots, \tau_n$ are topologies on a set $X$, the $\tau = \bigcap_{i=1}^n \tau_i$ is a toplogy on $X$.
	\item If for each $i \in I$, for some index set $I$, each $\tau_i$ is a topology on the set $X$, then $\tau = \bigcap_{i\in I} \tau_i$ is a topology on $X$.
\end{enumerate}

\SPECIAL{Distinct $T_1$-topologies on a Finite Set}

\item In Wikipedia \url{//enwikipedia.org/wiki/Finite_topological_space}, as we noted in Exercise 1.2 \#7, it says that the number of topologies on a finite set with $n \in \N$ points can be quite large, even for small $n$.  This is also true even for $T_0$-spaces; for $n=5$, ther are 4231 distinct $T_0$-spaces.  Prove, using mathemtaical induction, that as $n$ increases, the number of $T_0$-spaces increases.

\item A topological space $(X,T)$ is said to be a \textbf{\textit{door space}} if every subset of $X$ is either an open set or a closed set (or both).
\begin{enumerate}[label=(\roman*)]
	\item Is a discrete space a door space?
	\item Is an indiscrete space a door space?
	\item If $X$ is an infintie set and $\tau$ is the finite-closed topology, is $(X,\tau)$ a door space?
	\item Let $X$ be the set $\{a,b,c,d\}$.  Identify those topologies $\tau$ on $X$ which make it into a door space.
\end{enumerate}

\SPECIAL{Saturated Sets}

\item A subset $S$ of a topological space $(X,\tau)$ is said to be \textbf{saturated} if it is an intersection of open sets in $(X, \tau)$.
\begin{enumerate}[label=(\roman*)]
	\item Verify that every open set is a saturated set.
	\item Verify that in a $T_1$-space every set is saturated set.
	\item Give an example of a topological space which has atleast one subset which is not saturated.
	\item Is it true that if the topological sapce $(X, \tau)$ is such that every subset is saturated, then $(X,\tau)$ is a $T_1$-space?
\end{enumerate}
\end{enumerate} % Finite Closed Topology Exercises

\chapter{The Euclidean Topology}
\section{Euclidian Space -- Exercises}

\begin{enumerate}

\item Prove that if $a,b \in \R$ with $a < b$ then neither $[a,b)$ nor $(a,b]$ is an open subset of $\R$.  Also show that neither is a closed subset of $\R$.

In the case of $[a,b)$ there is no set $a \in (x,y)$ because $x < a$ implies that $x+\frac{|x-a|}{2}$ would have to be a member of $[a,b)$ which it cannot.  Similarly for $(a,b]$.

\item Prove that the sets $[a,\infty)$ and $(-\infty, a]$ are closed subsets of $\R$.

The composite of $[a,\infty)$ is $(-\infty, a)$ which is open and similarly for $(-\infty, a]$.

\item Show, by example, that the union of an infinite number of closed subsets of $\R$ is not necessarily a closed subset of $\R$.

Define $S_i = [1/i, 1]$ then $\mathcal{S} = \cup_{i=1}^\infty S_i$.  Obviously, given any $n \in \N$ there is a closed set $S_n=[1/n, 1]$ and there exists $(1/(n+1),1) \subseteq \mathcal{S}$ such that $1/n \in  (1/(n+1),1)$ hence $\mathcal{S}$ must be open.

\item Prove each of the following statements.
\begin{enumerate}[label=(\roman*)]
	\item The set $\Z$ of all integers is not an open set of $\R$.
	\item The set $\mathbb{P}$ of all prime numbers is a closed subset of $\R$ but not an open subset of $\R$.
	\item The set $\mathbb{I}$ of all irrational numbers is neither a closed subset nor an open subset of $\R$.
\end{enumerate}

\item If $F$ is a non-empty finite subset of $\R$, show that $F$ is closed in $\R$ but that $F$ is not open in $\R$.

\item if $F$ is non-empty countable subset of $\R$, prove that $F$ is not an open set, but that $F$ may or may not be a closed set depending on the choice of $F$.

\item \begin{enumerate}[label=(\roman*)]

	\item Let $S=\{0,1,1/2,1/3,1/4,1/5, \dots, 1/n, \dots\}$.  Prove that the set $S$ is closed in the euclidean topology on $\R$.
	
	\item Is the set $T=\{1,1/2,1/3,1/4,1/5,\dots,1/n, \dots\}$ closed in $\R$?
	
	\item Is the set $\{\sqrt{2}, 2\sqrt{2},3\sqrt{2},\dots, n\sqrt{2}, \dots\}$ closed in $\R$?

\end{enumerate}
	
	\SPECIAL{$F_\sigma$-Sets and $G_\delta$-sets.}

\item \begin{enumerate}[label=(\roman*)]

	\item Let $(X, \tau)$ be a toplogical space.  A subset $S$ of $X$ is said to be an \DEFINE{$F_\sigma$ set} if it is the union of a countable number of closed sets. Prove that all open intervals $(a,b)$ and all closed intervals $[a,b]$ are $F_\sigma$-sets in $\R$.
	
	\item Let $(X, \tau)$ be topological space.  A subset $T$ of $X$ is said to be a \DEFINE{$G_\delta$-set} if it is the intersection of a countable number of open sets.  Prove that all open intervals $(a,b)$ and all closed intervalse $[a,b]$ are $G_\delta$-sets in $\R$.
	
	\item Prove that the set $\Q$ of rationasl is an $F_\sigma$-set in $\R$.  
	
	\item Verify that the complement of an $F_\sigma$-set is a $G_\delta$-set and the complement of a $G_\delta$-set is an $F_\sigma$-set.

\end{enumerate}

\end{enumerate}  %2.1 Euclidian Topology


\end{document}