\documentclass[10pt,a4paper]{report}
\usepackage[utf8]{inputenc}
\usepackage{amsmath}
\usepackage{amsfonts}
\usepackage{amssymb}
\usepackage{amsthm}
\usepackage{hyperref}

\usepackage{multicol}
\usepackage{fancyhdr}
\usepackage[inline]{enumitem}
\usepackage{tikz}
\usepackage{tikz-cd}
\usetikzlibrary{calc}
\usetikzlibrary{shapes.geometric}
\usepackage[margin=0.5in]{geometry}
\usepackage{xcolor}

\hypersetup{
    colorlinks=true,
    linkcolor=blue,
    filecolor=magenta,      
    urlcolor=cyan,
    pdftitle={Tensors},
    pdfpagemode=FullScreen,
    }

%\urlstyle{same}

\newcommand{\CLASSNAME}{Math 5111 -- Real Analysis II}
\newcommand{\STUDENTNAME}{Paul Carmody}
\newcommand{\ASSIGNMENT}{Homework (not graded) }
\newcommand{\DUEDATE}{May 2025}
\newcommand{\SEMESTER}{Sprint 2025}
\newcommand{\SCHEDULE}{MW 9:30 - 10:45}
\newcommand{\ROOM}{Remote}

\newcommand{\MMN}{M_{m\times n}}
\newcommand{\FF}{\mathcal{F}}

\pagestyle{fancy}
\fancyhf{}
\chead{ \fancyplain{}{\CLASSNAME} }
%\chead{ \fancyplain{}{\STUDENTNAME} }
\rhead{\thepage}
\newcommand{\LET}{\text{Let }}
%\newcommand{\IF}{\text{if }}
\newcommand{\AND}{\text{ and }}
\newcommand{\OR}{\text{ or }}
\newcommand{\FORSOME}{\text{ for some }}
\newcommand{\FORALL}{\text{ for all }}
\newcommand{\WHERE}{\text{ where }}
\newcommand{\WTS}{\text{ WTS }}
\newcommand{\WLOG}{\text{ WLOG }}
\newcommand{\BS}{\backslash}
\newcommand{\DEFINE}[1]{\textbf{\emph{#1}}}
\newcommand{\IF}{$(\Rightarrow)$}
\newcommand{\ONLYIF}{$(\Leftarrow)$}
\newcommand{\ITH}{\textsuperscript{th} }
\newcommand{\FST}{\textsuperscript{st} }
\newcommand{\SND}{\textsuperscript{nd} }
\newcommand{\TRD}{\textsuperscript{rd} }
\newcommand{\INV}{\textsuperscript{-1} }


%%%%%%%
% derivatives
%%%%%%%

\newcommand{\PART}[2]{\frac{\partial #1}{\partial #2}}
\newcommand{\SPART}[2]{\frac{\partial^2 #1}{\partial #2^2}}
\newcommand{\DERIV}[2]{\frac{d #1}{d #2}}
\newcommand{\LAPLACIAN}[1]{\frac{\partial^2 #1}{\partial x^2} + \frac{\partial^2 #1}{\partial y^2}}

%%%%%%%
% sum, product, union, intersections
%%%%%%%

\newcommand{\SUM}[2]{\underset{#1}{\overset{#2}{\sum}}}
\newcommand{\PROD}[2]{\underset{#1}{\overset{#2}{\prod}}}
\newcommand{\UNION}[2]{\underset{#1}{\overset{#2}{\bigcup}}}
\newcommand{\INTERSECT}[2]{\underset{#1}{\overset{#2}{\bigcap}}}
\newcommand{\FSUM}{\SUM{n=-\infty}{\infty}}
       

%%%%%%%
% supremum and infimum
%%%%%%%

\newcommand{\SUP}[1]{\underset{#1}\sup \,}
\newcommand{\INF}[1]{\underset{#1}\inf \,}
\newcommand{\MAX}[1]{\underset{#1}\max \,}
\newcommand{\MIN}[1]{\underset{#1}\min \,}

%%%%%%%
% infinite sums, limits
%%%%%%%

\newcommand{\SUMK}{\SUM{k=1}{\infty}}
\newcommand{\SUMN}{\SUM{n=1}{\infty}}
\newcommand{\SUMKZ}{\SUM{k=0}{\infty}}
\newcommand{\LIM}[1]{\underset{#1}\lim\,}
\newcommand{\IWOB}[1]{\LIM{#1 \to \infty}}
\newcommand{\LIMK}{\IWOB{k}}
\newcommand{\LIMN}{\IWOB{n}}
\newcommand{\LIMX}{\IWOB{x}}
\newcommand{\NIWOB}{\LIM{n \to \infty}}
\newcommand{\LIMSUPK}{\underset{k\to\infty}\limsup \,}
\newcommand{\LIMSUPN}{\underset{n\to\infty}\limsup \,}
\newcommand{\LIMINFK}{\underset{k\to\infty}\liminf \,}
\newcommand{\LIMINFN}{\underset{n\to\infty}\liminf \,}
\newcommand{\ROOTRULE}[1]{\LIMSUPK \BARS{#1}^{1/k}}

\newcommand{\CUPK}{\bigcup_{k=1}^{\infty}}
\newcommand{\CAPK}{\bigcap_{k=1}^{\infty}}
\newcommand{\CUPN}{\bigcup_{n=1}^{\infty}}
\newcommand{\CAPN}{\bigcap_{n=1}^{\infty}}

%%%%%%%
% number systems (real, rational, etc.)
%%%%%%%

\newcommand{\REALS}{\mathbb{R}}
\newcommand{\RATIONALS}{\mathbb{Q}}
\newcommand{\IRRATIONALS}{\REALS \backslash \RATIONALS}
\newcommand{\INTEGERS}{\mathbb{Z}}
\newcommand{\NUMBERS}{\mathbb{N}}
\newcommand{\COMPLEX}{\mathbb{C}}
\newcommand{\DISC}{\mathbb{D}}
\newcommand{\HPLANE}{\mathbb{H}}

\newcommand{\R}{\mathbb{R}}
\newcommand{\Q}{\mathbb{Q}}
\newcommand{\Z}{\mathbb{Z}}
\newcommand{\N}{\mathbb{N}}
\newcommand{\C}{\mathbb{C}}
\newcommand{\T}{\mathbb{T}}
\newcommand{\COUNTABLE}{\aleph_0}
\newcommand{\UNCOUNTABLE}{\aleph_1}


%%%%%%%
% Arithmetic/Algebraic operators
%%%%%%%


\DeclareMathOperator{\MOD}{mod}
%\newcommand{\MOD}[1]{\mod #1}
\newcommand{\BAR}[1]{\overline{#1}}
\newcommand{\LCM}{\text{ lcm}}
\newcommand{\ZMOD}[1]{\Z/#1\Z}
\DeclareMathOperator{\VAR}{Var}
%%%%%%%
% complex operators
%%%%%%%

\DeclareMathOperator{\RR}{Re}
%\newcommand{\RE}{\text{Re}}
\DeclareMathOperator{\IM}{Im}
%\newcommand{\IM}{\text{Im}}
\newcommand{\CONJ}[1]{\overline{#1}}
\DeclareMathOperator{\LOG}{Log}
%\newcommand{\LOG}{\text{ Log }}
\newcommand{\RES}[2]{\underset{#1}{\text{res}} #2}

%%%%%%%
% Group operators
%%%%%%%

\newcommand{\AUT}{\text{Aut}\,}
\newcommand{\KER}{\text{ker}\,}
\newcommand{\END}{\text{End}}
\newcommand{\HOM}{\text{Hom}}
\newcommand{\CYCLE}[1]{(\begin{array}{cccccccccc}
		#1
	\end{array})}
\newcommand{\SUBGROUP}{\underset{\text{group}}\subseteq}	
%\newcommand{\SUBGROUP}{\subseteq_g}
\newcommand{\SUBRING}{\underset{\text{ring}}\subseteq}
\newcommand{\SUBMOD}{\underset{\text{mod}}\subseteq}
\newcommand{\SUBFIELD}{\underset{\text{field}}\subseteq}
\newcommand{\ISO}{\underset{\text{iso}}\longrightarrow}
\newcommand{\HOMO}{\underset{\text{homo}}\longrightarrow}

%%%%%%%
% grouping (parenthesis, absolute value, square, multi-level brackets).
%%%%%%%

\newcommand{\PAREN}[1]{\left (\, #1 \,\right )}
\newcommand{\BRACKET}[1]{\left \{\, #1 \,\right \}}
\newcommand{\SQBRACKET}[1]{\left [\, #1 \,\right ]}
\newcommand{\ABRACKET}[1]{\left \langle\, #1 \,\right \rangle}
\newcommand{\BARS}[1]{\left |\, #1 \,\right |}
\newcommand{\DBARS}[1]{\left \| \, #1 \,\right \|}
\newcommand{\LBRACKET}[1]{\left \{ #1 \right .} 
\newcommand{\RBRACKET}[1]{\left . #1 \right \]}
\newcommand{\RBAR}[1]{\left . #1 \, \right |}
\newcommand{\LBAR}[1]{\left | \, #1 \right .}
\newcommand{\BLBRACKET}[2]{\BRACKET{\RBAR{#1}#2}}
\newcommand{\GEN}[1]{\ABRACKET{#1}}
\newcommand{\BINDEF}[2]{\LBRACKET{\begin{array}{ll}
     #1\\
     #2
\end{array}}}

%%%%%%%
% Fourier Analysis
%%%%%%%

\newcommand{\ONEOTWOPI}{\frac{1}{2\pi}}
\newcommand{\FHAT}{\hat{f}(n)}
\newcommand{\FINT}{\int_{-\pi}^\pi}
\newcommand{\FINTWO}{\int_{0}^{2\pi}}
\newcommand{\FSUMN}[1]{\SUM{n=-#1}{#1}}
%\newcommand{\FSUM}{\SUMN{\infty}}
\newcommand{\EIN}[1]{e^{in#1}}
\newcommand{\NEIN}[1]{e^{-in#1}}
\newcommand{\INTALL}{\int_{-\infty}^{\infty}}
\newcommand{\FTINT}[1]{\INTALL #1 e^{2\pi inx\xi} dx}
\newcommand{\GAUSS}{e^{-\pi x^2}}

%%%%%%%
% formatting 
%%%%%%%

\newcommand{\LEFTBOLD}[1]{\noindent\textbf{#1}}
\newcommand{\SEQ}[1]{\{#1\,\}}
\newcommand{\WIP}{\footnote{work in progress}}
\newcommand{\QED}{\hfill\square}
\newcommand{\ts}{\textsuperscript}
\newcommand{\HLINE}{\noindent\rule{7in}{1pt}\\}

%%%%%%%
% Mathematical note taking (definitions, theorems, etc.)
%%%%%%%

\newcommand{\REM}{\noindent\textbf{\\Remark: }}
\newcommand{\DEF}{\noindent\textbf{\\Definition: }}
\newcommand{\THE}{\noindent\textbf{\\Theorem: }}
\newcommand{\COR}{\noindent\textbf{\\Corollary: }}
\newcommand{\LEM}{\noindent\textbf{\\Lemma: }}
\newcommand{\PROP}{\noindent\textbf{\\Proposition: }}
\newcommand{\PROOF}{\noindent\textbf{\\Proof: }}
\newcommand{\EXP}{\noindent\textbf{\\Example: }}
\newcommand{\TRICKS}{\noindent\textbf{\\Tricks: }}


%%%%%%%
% text highlighting
%%%%%%%

\newcommand{\B}[1]{\textbf{#1}}
\newcommand{\CAL}[1]{\mathcal{#1}}
\newcommand{\UL}[1]{\underline{#1}}

%%%%%%
% Linear Algebra
%%%%%%

\newcommand{\COLVECTOR}[1]{\PAREN{\begin{array}{c}
#1
\end{array} }}
\newcommand{\TWOXTWO}[4]{\PAREN{ \begin{array}{c c} #1&#2 \\ #3 & #4 \end{array} }}
\newcommand{\THREEXTHREE}[9]{\PAREN{ \begin{array}{c c c} #1&#2&#3 \\ #4 & #5 & #6 \\ #7 & #8 & #9 \end{array} }}
\newcommand{\NXN}{\PAREN{ \begin{array}{c c c c} 
			a_{11} & a_{12} & \cdots & a_{1n} \\
			a_{21} & a_{22} & \cdots & a_{2n} \\
			\vdots & \vdots & \ddots & a_{1n} \\
			a_{n1} & a_{n2} & \cdots & a_{nn} \\
		\end{array} }}
\newcommand{\SLR}{SL_2(\R)}
\newcommand{\GLR}{GL_2(\R)}
\DeclareMathOperator{\TR}{tr}
\DeclareMathOperator{\BIL}{Bil}
\DeclareMathOperator{\SPAN}{span}

%%%%%%%
%  White space
%%%%%%%

\newcommand{\BOXIT}[1]{\noindent\fbox{\parbox{\textwidth}{#1}}}


\newtheorem{theorem}{Theorem}[section]
\newtheorem{corollary}{Corollary}[theorem]
\newtheorem{lemma}[theorem]{Lemma}

\theoremstyle{definition}
\newtheorem{definition}[theorem]{Definition}
\newtheorem{prop}[theorem]{Proposition}

\theoremstyle{remark}
\newtheorem{remark}[theorem]{Remark}
\newtheorem{example}[theorem]{Example}
%\newtheorem*{proof}[theorem]{Proof}



\newcommand{\RED}[1]{\textcolor{red}{#1}}
\newcommand{\BLUE}[1]{\textcolor{blue}{#1}}

\begin{document}

\begin{center}
	\Large{\CLASSNAME -- \SEMESTER} \\
	\large{ w/Professor Perera}
\end{center}
\begin{center}
	\STUDENTNAME \\
	\ASSIGNMENT -- \DUEDATE\\
\end{center} 


\HLINE
\noindent Pg 30, 1\\

Does there exist an infinite $\sigma$-algebra which has only countably many members?\\

\BLUE{First we establish the smallest possible measurable set containing a point, $B_x$.  Then demonstrate that these are unique and distinct (that is, $x \ne y \implies B_x \cap B_y = \emptyset$).  Then assuming that there are countably many $B_x$ show that there are still some missing points not in the $\bigcup_{\forall x} B_x$.
}


\HLINE
\noindent Pg 30, 2\\

Prove an analogue of Theorem 1.8 for $n$ functions.

\HLINE
\noindent Pg 30, 3\\

\noindent Prove that if $f$ is a real function on a measurable space $X$ such that $\{x:f(x) \ge r\}$ is measurable for for every rational $r$, then $f$ is measurable. \\

\BLUE{Let $E_r = \{x:f(x) \ge r\}$ each of which is measurable and $E = \bigcup_{r \in \Q} E_r$.  $E$ is measurable because it is the countable union of measurable sets.  Given any measurable set $I \in \R$ we can clearly see that $f(E) \cap I \ne \emptyset$ because $\Q$ is dense.
\begin{align*}
	f(E) &= f\PAREN{\bigcup_{r\in I\cap \Q} E_r}
\end{align*}We are saying that $I$ is measurable, therefore it can be made up of the union of disjoint measurable sets $A$ and $B$ where $B = I \cap f(E)$ and $A = I\backslash B$.  $f^{-1}(I) = f^{-1}(A\cup B) = f^{-1}(A) \cup f^{-1}(B)$. We can see that $f^{-1}(B)$ is measurable,  How do we show that $f^{-1}(A)$ is?\\
\\
Notice, $E_r \in \mathfrak{M} \implies E_r^c \in \mathfrak{M}$.  Now, we can try to connect $f^{-1}(A)$ to these sets. 
}

\HLINE
\noindent Pg 30, 4\\ \\
Let $\{a_n\}$ and $\{b_n\}$ be sequencesin $[-\infty,\infty]$, and prove the following assertions:
\begin{enumerate}[label=(\alph*)]
	\item $ \LIMSUPN (-a_n) = - \LIMINFN a_n. $
	
	\BLUE{\begin{align*}
			\sup_{n\to\infty} \{ - a_n \} &= - \inf_{n\to \infty} \{a_n\} \\
			\LIMSUPN \{ -a_n\} &= \LIMN \sup_{k>n} \{-a_k \} = \LIMN (-\inf_{k>n} \{a_k \}) = -\LIMINFN \{a_n \}
		\end{align*}	
	}
	
	\item $ \LIMSUPN (a_n+b_n) \le \LIMSUPN a_n + \LIMSUPN b_n$ provided none of the sums is $\infty - \infty$.
	
	\BLUE{\begin{align*}
		\LIMSUPN \{a_n + b_n\} &= \LIMN \sup_{k>n} \{a_k+b_k\}
	\end{align*}keep mind that $a_k + b_k$ could be zero, but $\sup {a_k} + \sup{b_k}$ could only be zero after evaluating both sequences.  Thus,
	\begin{align*}
		\sup_{k > n} \{a_k + b_k\} &\le \sup_{k>n}\{a_k\} + \sup_{k>n}\{b_k\}, \, \forall n \\
		\LIMN \sup_{k > n} \{a_k + b_k\} &\le \LIMN \sup_{k>n}\{a_k\} + \sup_{k>n}\{b_k\}, \, \forall n \\
		\LIMSUPN \{a_n + b_n\} &\le \LIMSUPN \{a_n\} + \LIMSUPN\{b_n\} 
	\end{align*}
	}
	
	\item if $a_n \le b_n$, for all $n$, then 
	\begin{align*}
		\LIMINFN a_n \le \LIMINFN b_n
	\end{align*}
\end{enumerate}Show by example that strick inequalty can hold in (b).\\ \\
\HLINE
\noindent Pg 30, 5\\
\begin{enumerate}[label=(\alph*)]
	\item Suppose $f: X \to [-\infty, \infty]$ and $g; X \to [-\infty,\infty ]$ are measurable.  Prove that the sets 
	\begin{align*}
		\{x:f(x) < g(x)\}, \{x:f(x)=g(x)|
	\end{align*}are measurable.
	
	\BLUE{Given any $y \in X$ let $s_y = \{x: f(x) < g(y)\}$.  $f$ is measurable implies that $s_y$ is measurable.  Now taking the intersection over all $y \in X$, that is 
	\begin{align*}
		\bigcup_{y\in X} s_y = \{x:f(x) < g(x)\} \in \mathfrak{M} 
	\end{align*}
	}
	
	\item Prove that the set of points at which a sequence of measurable real-value functions converges (to a finite limit) is measurable.
	
	\BLUE{Let $f, f_n : X \to [0, \infty] \in \mathfrak{M}$ and let $f_n \to f, \forall x \in X$.  Given any measurable set $p\in X$ the sequence of sets $\{ f_i(p) \}$ are all measurable. Further, since $f_i(x) \to f(x), \forall x\in X$ we can say $f_i(p) \to f(p),\forall x \in p$.  Thus, for any measurable set $P$ and $\exists p \in \mathfrak{M}, n \to f_n(p)=P$.\\
	\\
	Given any measurable set $P \subset [-\infty, \infty]$ then $f_n^{-1}(P) \in \mathfrak{M},\forall n$.  Let $p_n \subset X$ such that $p_n = f_n^{-1}(P)$.  Then, $\cap_n f_n(p_n) = \cap_n f(p_n) = P$.  Thus, $f$ is measurable.
	}
	
\end{enumerate}

\HLINE
\noindent Pg 30, 6\\

Let $X$ be an uncountable set, let $\mathfrak{M}$ be the collection of all sets $E\subset X$ such that either $E$ and $E^c$ is at most countable, and define $\mu(E)=0$ in the first case, $\mu(E)=1$ in the second.  Prove that $\mathfrak{M}$ is a $\sigma$-algebra in $X$ and that $\mu$ is a measure on $\mathfrak{M}$.  Describe the corresponding measurable functions and their integrals.\\

\HLINE
\noindent Pg 30, 7\\

Suppose $f_n: X \to [0,\infty]$ is measurable fo r$n=1,2,\dots, f_1 \ge f_2\ge
f_3 \ge \cdots \ge 0, f_n(x)\to f(x)$ as $n\to \infty$, for every $x \in X$, and $f_1 \in L^1(\mu)$.  Prove that then
\begin{align*}
	\LIMN \int_X f_n d\mu = \int_X f d \mu
\end{align*}and show that this conclusion does \textit{not} follow if the condition ``$f_1 \in L^1(\mu)$" is omitted.

\BLUE{\begin{align*}
	\LIMN \int_X f_n d\mu &= \LIMN \sup \int_X s_n d\mu
\end{align*}where each $s_n$ are the simple function less than $f_n$ for all $x$.  These are all less than $f$ and whose $\LIMSUPN s_n = \LIMSUPN S_n$ of simple functions less than $f$.   thus
\begin{align*}
	\LIMSUPN \int_X s_n d\mu &= \LIMSUPN \int S_n d\mu = \int_x f d\mu
\end{align*}
}

\HLINE
\noindent Pg 30, 8\\

Put $f_n= \chi_E$ if $n$ is odd, $f_n=1-\chi_E$ if $n$ is even.  What is the relevance of this example to Fatou's lemma?\\

\BLUE{\textbf{Fatou's Lemma} given any any $f_n : X \to [0, \infty) \in \mathfrak{M}$ then 
\begin{align*}
	\int_X \LIMINFN f_x d\mu \le \LIMINFN \int_X f_n d\mu 
\end{align*}The sequence of functions does not converge but oscilates.  Thus, LHS goes to zero and the RHS goes to 1.
}


\HLINE
\noindent Pg 30, 9\\

Suppose $\mu$ is a postiive measure on $X,f:X\to [0,\infty]$ is measurable $\int_X f d\mu=c$, where $0 < c<\infty$, and $\alpha$ is constant.  Prove that 
\begin{align*}
	\LIMN \int_X n \log [1+(f/n)^\alpha] d\mu = \LBRACKET{\begin{array}{ll}
		\infty	& \text{if } 0<\alpha<1,\\
		c & \text{if }\alpha = 1, \\	
		0 & \text{if } 1 < \alpha < \infty.
	\end{array} }
\end{align*} \textit{Hint:} If $\alpha \ge 1$, the integrands are dominated by $\alpha f$.  If $\alpha < 1$, Fatous lemma can be applied.


\HLINE
\noindent Pg 30, 10\\

Suppose $\mu(X) < \infty, \{f_n\}$ is a sequence of bounded complex measurable functions on $X$, and $f_n \to f$ uniformly on $X$.  Prove that
\begin{align*}
	\LIMN \int_X f_n d \mu = \int_X f d \mu.
\end{align*}and show that the hypothesis``$\mu(X) < \infty$" cannot be omitted.


\HLINE
\noindent Pg 30, 11\\

Show that 
\begin{align*}
	A = \bigcap_{nj=1}^\infty \bigcup_{k=n}^\infty E^k
\end{align*}in Theorem 1.41 and hence prove the theorme without any reference to integration.


\HLINE
\noindent Pg 30, 12\\

Suppose $f \in L^1(\mu)$.  Prove that to each $\epsilon > 0$ there exists a $\delta > 0$ sucht that $\int_E|f| d\mu < \epsilon$ whenever $\mu(E) < \delta$.


\HLINE
\noindent Pg 30, 13\\

Show that proposition 1.24(c) is also true when $c = \infty$.
\end{document}
