\documentclass[10pt,a4paper]{report}
\usepackage[utf8]{inputenc}
\usepackage{amsmath}
\usepackage{amsfonts}
\usepackage{amssymb}
\usepackage{amsthm}
\usepackage{hyperref}

\usepackage{multicol}
\usepackage{fancyhdr}
\usepackage[inline]{enumitem}
\usepackage{tikz}
\usepackage{tikz-cd}
\usetikzlibrary{calc}
\usetikzlibrary{shapes.geometric}
\usepackage[margin=0.5in]{geometry}
\usepackage{xcolor}

\hypersetup{
    colorlinks=true,
    linkcolor=blue,
    filecolor=magenta,      
    urlcolor=cyan,
    pdftitle={Tensors},
    pdfpagemode=FullScreen,
    }

%\urlstyle{same}

\newcommand{\CLASSNAME}{Math 5111 -- Real Analysis II}
\newcommand{\STUDENTNAME}{Paul Carmody}
\newcommand{\ASSIGNMENT}{Semester Notes }
\newcommand{\DUEDATE}{May 2025}
\newcommand{\SEMESTER}{Sprint 2025}
\newcommand{\SCHEDULE}{MW 9:30 - 10:45}
\newcommand{\ROOM}{Remote}

\newcommand{\MMN}{M_{m\times n}}
\newcommand{\FF}{\mathcal{F}}

\pagestyle{fancy}
\fancyhf{}
\chead{ \fancyplain{}{\CLASSNAME} }
%\chead{ \fancyplain{}{\STUDENTNAME} }
\rhead{\thepage}
\newcommand{\LET}{\text{Let }}
%\newcommand{\IF}{\text{if }}
\newcommand{\AND}{\text{ and }}
\newcommand{\OR}{\text{ or }}
\newcommand{\FORSOME}{\text{ for some }}
\newcommand{\FORALL}{\text{ for all }}
\newcommand{\WHERE}{\text{ where }}
\newcommand{\WTS}{\text{ WTS }}
\newcommand{\WLOG}{\text{ WLOG }}
\newcommand{\BS}{\backslash}
\newcommand{\DEFINE}[1]{\textbf{\emph{#1}}}
\newcommand{\IF}{$(\Rightarrow)$}
\newcommand{\ONLYIF}{$(\Leftarrow)$}
\newcommand{\ITH}{\textsuperscript{th} }
\newcommand{\FST}{\textsuperscript{st} }
\newcommand{\SND}{\textsuperscript{nd} }
\newcommand{\TRD}{\textsuperscript{rd} }
\newcommand{\INV}{\textsuperscript{-1} }

\newcommand{\XXX}{\mathfrak{X}}
\newcommand{\MMM}{\mathfrak{M}}
%\newcommand{\????}{\textfrak{A}}
%\newcommand{\????}{\textgoth{A}}
%\newcommand{\????}{\textswab{A}}

\DeclareMathOperator{\DER}{Der}
\DeclareMathOperator{\SGN}{sgn}

%%%%%%%
% derivatives
%%%%%%%

\newcommand{\PART}[2]{\frac{\partial #1}{\partial #2}}
\newcommand{\SPART}[2]{\frac{\partial^2 #1}{\partial #2^2}}
\newcommand{\DERIV}[2]{\frac{d #1}{d #2}}
\newcommand{\LAPLACIAN}[1]{\frac{\partial^2 #1}{\partial x^2} + \frac{\partial^2 #1}{\partial y^2}}

%%%%%%%
% sum, product, union, intersections
%%%%%%%

\newcommand{\SUM}[2]{\underset{#1}{\overset{#2}{\sum}}}
\newcommand{\PROD}[2]{\underset{#1}{\overset{#2}{\prod}}}
\newcommand{\UNION}[2]{\underset{#1}{\overset{#2}{\bigcup}}}
\newcommand{\INTERSECT}[2]{\underset{#1}{\overset{#2}{\bigcap}}}
\newcommand{\FSUM}{\SUM{n=-\infty}{\infty}}
       

%%%%%%%
% supremum and infimum
%%%%%%%

\newcommand{\SUP}[1]{\underset{#1}\sup \,}
\newcommand{\INF}[1]{\underset{#1}\inf \,}
\newcommand{\MAX}[1]{\underset{#1}\max \,}
\newcommand{\MIN}[1]{\underset{#1}\min \,}

%%%%%%%
% infinite sums, limits
%%%%%%%

\newcommand{\SUMK}{\SUM{k=1}{\infty}}
\newcommand{\SUMN}{\SUM{n=1}{\infty}}
\newcommand{\SUMKZ}{\SUM{k=0}{\infty}}
\newcommand{\LIM}[1]{\underset{#1}\lim\,}
\newcommand{\IWOB}[1]{\LIM{#1 \to \infty}}
\newcommand{\LIMK}{\IWOB{k}}
\newcommand{\LIMN}{\IWOB{n}}
\newcommand{\LIMX}{\IWOB{x}}
\newcommand{\NIWOB}{\LIM{n \to \infty}}
\newcommand{\LIMSUPK}{\underset{k\to\infty}\limsup \,}
\newcommand{\LIMSUPN}{\underset{n\to\infty}\limsup \,}
\newcommand{\LIMINFK}{\underset{k\to\infty}\liminf \,}
\newcommand{\LIMINFN}{\underset{n\to\infty}\liminf \,}
\newcommand{\ROOTRULE}[1]{\LIMSUPK \BARS{#1}^{1/k}}

\newcommand{\CUPK}{\bigcup_{k=1}^{\infty}}
\newcommand{\CAPK}{\bigcap_{k=1}^{\infty}}
\newcommand{\CUPN}{\bigcup_{n=1}^{\infty}}
\newcommand{\CAPN}{\bigcap_{n=1}^{\infty}}

%%%%%%%
% number systems (real, rational, etc.)
%%%%%%%

\newcommand{\REALS}{\mathbb{R}}
\newcommand{\RATIONALS}{\mathbb{Q}}
\newcommand{\IRRATIONALS}{\REALS \backslash \RATIONALS}
\newcommand{\INTEGERS}{\mathbb{Z}}
\newcommand{\NUMBERS}{\mathbb{N}}
\newcommand{\COMPLEX}{\mathbb{C}}
\newcommand{\DISC}{\mathbb{D}}
\newcommand{\HPLANE}{\mathbb{H}}

\newcommand{\R}{\mathbb{R}}
\newcommand{\Q}{\mathbb{Q}}
\newcommand{\Z}{\mathbb{Z}}
\newcommand{\N}{\mathbb{N}}
\newcommand{\C}{\mathbb{C}}
\newcommand{\T}{\mathbb{T}}
\newcommand{\COUNTABLE}{\aleph_0}
\newcommand{\UNCOUNTABLE}{\aleph_1}


%%%%%%%
% Arithmetic/Algebraic operators
%%%%%%%


\DeclareMathOperator{\MOD}{mod}
%\newcommand{\MOD}[1]{\mod #1}
\newcommand{\BAR}[1]{\overline{#1}}
\newcommand{\LCM}{\text{ lcm}}
\newcommand{\ZMOD}[1]{\Z/#1\Z}
\DeclareMathOperator{\VAR}{Var}
%%%%%%%
% complex operators
%%%%%%%

\DeclareMathOperator{\RR}{Re}
%\newcommand{\RE}{\text{Re}}
\DeclareMathOperator{\IM}{Im}
%\newcommand{\IM}{\text{Im}}
\newcommand{\CONJ}[1]{\overline{#1}}
\DeclareMathOperator{\LOG}{Log}
%\newcommand{\LOG}{\text{ Log }}
\newcommand{\RES}[2]{\underset{#1}{\text{res}} #2}

%%%%%%%
% Group operators
%%%%%%%

\newcommand{\AUT}{\text{Aut}\,}
\newcommand{\KER}{\text{ker}\,}
\newcommand{\END}{\text{End}}
\newcommand{\HOM}{\text{Hom}}
\newcommand{\CYCLE}[1]{(\begin{array}{cccccccccc}
		#1
	\end{array})}
\newcommand{\SUBGROUP}{\underset{\text{group}}\subseteq}	
%\newcommand{\SUBGROUP}{\subseteq_g}
\newcommand{\SUBRING}{\underset{\text{ring}}\subseteq}
\newcommand{\SUBMOD}{\underset{\text{mod}}\subseteq}
\newcommand{\SUBFIELD}{\underset{\text{field}}\subseteq}
\newcommand{\ISO}{\underset{\text{iso}}\longrightarrow}
\newcommand{\HOMO}{\underset{\text{homo}}\longrightarrow}

%%%%%%%
% grouping (parenthesis, absolute value, square, multi-level brackets).
%%%%%%%

\newcommand{\PAREN}[1]{\left (\, #1 \,\right )}
\newcommand{\BRACKET}[1]{\left \{\, #1 \,\right \}}
\newcommand{\SQBRACKET}[1]{\left [\, #1 \,\right ]}
\newcommand{\ABRACKET}[1]{\left \langle\, #1 \,\right \rangle}
\newcommand{\BARS}[1]{\left |\, #1 \,\right |}
\newcommand{\DBARS}[1]{\left \| \, #1 \,\right \|}
\newcommand{\LBRACKET}[1]{\left \{ #1 \right .} 
\newcommand{\RBRACKET}[1]{\left . #1 \right \]}
\newcommand{\RBAR}[1]{\left . #1 \, \right |}
\newcommand{\LBAR}[1]{\left | \, #1 \right .}
\newcommand{\BLBRACKET}[2]{\BRACKET{\RBAR{#1}#2}}
\newcommand{\GEN}[1]{\ABRACKET{#1}}
\newcommand{\BINDEF}[2]{\LBRACKET{\begin{array}{ll}
     #1\\
     #2
\end{array}}}

%%%%%%%
% Fourier Analysis
%%%%%%%

\newcommand{\ONEOTWOPI}{\frac{1}{2\pi}}
\newcommand{\FHAT}{\hat{f}(n)}
\newcommand{\FINT}{\int_{-\pi}^\pi}
\newcommand{\FINTWO}{\int_{0}^{2\pi}}
\newcommand{\FSUMN}[1]{\SUM{n=-#1}{#1}}
%\newcommand{\FSUM}{\SUMN{\infty}}
\newcommand{\EIN}[1]{e^{in#1}}
\newcommand{\NEIN}[1]{e^{-in#1}}
\newcommand{\INTALL}{\int_{-\infty}^{\infty}}
\newcommand{\FTINT}[1]{\INTALL #1 e^{2\pi inx\xi} dx}
\newcommand{\GAUSS}{e^{-\pi x^2}}

%%%%%%%
% formatting 
%%%%%%%

\newcommand{\LEFTBOLD}[1]{\noindent\textbf{#1}}
\newcommand{\SEQ}[1]{\{#1\,\}}
\newcommand{\WIP}{\footnote{work in progress}}
\newcommand{\QED}{\hfill\square}
\newcommand{\ts}{\textsuperscript}
\newcommand{\HLINE}{\noindent\rule{7in}{1pt}\\}

%%%%%%%
% Mathematical note taking (definitions, theorems, etc.)
%%%%%%%

\newcommand{\REM}{\noindent\textbf{\\Remark: }}
\newcommand{\DEF}{\noindent\textbf{\\Definition: }}
\newcommand{\THE}{\noindent\textbf{\\Theorem: }}
\newcommand{\COR}{\noindent\textbf{\\Corollary: }}
\newcommand{\LEM}{\noindent\textbf{\\Lemma: }}
\newcommand{\PROP}{\noindent\textbf{\\Proposition: }}
\newcommand{\PROOF}{\noindent\textbf{\\Proof: }}
\newcommand{\EXP}{\noindent\textbf{\\Example: }}
\newcommand{\TRICKS}{\noindent\textbf{\\Tricks: }}


%%%%%%%
% text highlighting
%%%%%%%

\newcommand{\B}[1]{\textbf{#1}}
\newcommand{\CAL}[1]{\mathcal{#1}}
\newcommand{\UL}[1]{\underline{#1}}

%%%%%%
% Linear Algebra
%%%%%%

\newcommand{\COLVECTOR}[1]{\PAREN{\begin{array}{c}
#1
\end{array} }}
\newcommand{\TWOXTWO}[4]{\PAREN{ \begin{array}{c c} #1&#2 \\ #3 & #4 \end{array} }}
\newcommand{\DTWOXTWO}[4]{\BARS{ \begin{array}{c c} #1&#2 \\ #3 & #4 \end{array} }}
\newcommand{\THREEXTHREE}[9]{\PAREN{ \begin{array}{c c c} #1&#2&#3 \\ #4 & #5 & #6 \\ #7 & #8 & #9 \end{array} }}
\newcommand{\DTHREEXTHREE}[9]{\BARS{ \begin{array}{c c c} #1&#2&#3 \\ #4 & #5 & #6 \\ #7 & #8 & #9 \end{array} }}
\newcommand{\NXN}{\PAREN{ \begin{array}{c c c c} 
			a_{11} & a_{12} & \cdots & a_{1n} \\
			a_{21} & a_{22} & \cdots & a_{2n} \\
			\vdots & \vdots & \ddots & a_{1n} \\
			a_{n1} & a_{n2} & \cdots & a_{nn} \\
		\end{array} }}
\newcommand{\SLR}{SL_2(\R)}
\newcommand{\GLR}{GL_2(\R)}
\DeclareMathOperator{\TR}{tr}
\DeclareMathOperator{\BIL}{Bil}
\DeclareMathOperator{\SPAN}{span}

%%%%%%%
%  White space
%%%%%%%

\newcommand{\BOXIT}[1]{\noindent\fbox{\parbox{\textwidth}{#1}}}


\newtheorem{theorem}{Theorem}[section]
\newtheorem{corollary}{Corollary}[theorem]
\newtheorem{lemma}[theorem]{Lemma}

\theoremstyle{definition}
\newtheorem{definition}[theorem]{Definition}
\newtheorem{prop}[theorem]{Proposition}

\theoremstyle{remark}
\newtheorem{remark}[theorem]{Remark}
\newtheorem{example}[theorem]{Example}
%\newtheorem*{proof}[theorem]{Proof}



\newcommand{\RED}[1]{\textcolor{red}{#1}}
\newcommand{\BLUE}[1]{\textcolor{blue}{#1}}

\begin{document}

\begin{center}
	\Large{\CLASSNAME -- \SEMESTER} \\
	\large{ w/Professor Liu}
\end{center}
\begin{center}
	\STUDENTNAME \\
	\ASSIGNMENT -- \DUEDATE\\
\end{center} 

\noindent January 18\ts{th}, 2025\\

\noindent \textbf{Continuity} is a property unlike most that we encounter in understanding functions.  Typically, we start with the domain and apply it to the function and get a result in the range.  A function is continuous, however, when the area around the range has the same properties as the area around the domain.
\begin{enumerate}
    \item  \textbf{continuous functions} defined using epsilon/delta.  Which becomes increasingly difficult when our domain and range go beyond the Real numbers (i.e., multiple dimensions, complex numbers, sets that aren’t compact).
    \item  \textbf{Topological definition of continuous functions} which is that an open set in the range comes from an open set in the domain.  This skips over the concept of measure by simply redefining “open sets” in more abstract terms (an open set is a member of a topology).  A metric topology has a way of measuring distance but it isn’t necessary for this definition.
    \item  Analogously speaking, \textbf{a function is measurable} (a.k.a., continuous) when a measurable (a.k.a., open) set in the range comes from a measurable (a.k.a., open) set in the domain.  The sets are defined by the Lebesgue Outer Measure (basically, ‘not a point’, which is much like but not the same as an open set in topology).
    \item Definitions of "superset" topology and $\sigma$-algebra.  These two concepts are so similar that they might be the same thing.
    \begin{enumerate}
    		\item A \textbf{topology} is defined as the collection where at least the set and the empty set as members, as well as subsets where all intersections (countable) and all unions (even uncountable) are also members.  "Open sets" are simply the members of the topology.
    		\item A $\sigma$-algebra is defined as having the set itself, all compliments of its members (which are subsets), and countable unions of its members.\footnote{$\sigma$ signifies infinite unions and $\delta$ (Not shown) would signify infinite intersectios} The set and $\sigma$-algebra combine to make a \textbf{Measurable Space} and its members are called \textbf{Measurable Sets} (analogous to "open sets"). 
    		\begin{definition}[Defining properties of a $\sigma$-algebra].
    		
    		\begin{enumerate}
    			\item $X \in \MMM$ which implies $\emptyset \in \MMM$.
    			\item $E \in \MMM \implies E^c \in \MMM$.
    			\item Countably Additive: if $E_i$ is a partition (or at least disjoint) then $\mu(\cup E_i) = \sum \mu(E_i)$. 
    		\end{enumerate}
    		\end{definition}
    		\item If $f$ maps from a measurable space to a topological space and each open set in the range comes from a measurable set in the domain (i.e., $f^{-1}(V)$ maps open set $V$ from a measurable set through $f$) then $f$ is said to be \textbf{Measurable Function}.
    		\item These two, $\sigma$-algebra and topology, appear to me to be logically equivalent.
    		
    		\item \BLUE{Put more simply: topology is defined by arbitrary unions and countable intersections of \textit{open sets} and a $\sigma$-algebra is defined by countable unions of \textit{measurable sets}.}
	\end{enumerate}      
    \item  Further, a \textbf{Borel Set} is an element of the smallest possible $\sigma$-algebra generated by the set.  It is, consequently, measurable (in the Lebesgue sense, i.e., ‘not a point’).  And a function is \textbf{Borel Measurable} when Borel Sets are mapped from Borel Sets.

\end{enumerate}

\noindent Rules for Composing different types of these functions.

\begin{tabular}{|l|c|c|c|c|}
\hline 
	& Nomen.\footnote{name of the set} & $f:X \to Y$ & $g:Y \to Z$ & $g\circ f(x) =g(f(x)) = z$ \\
	\hline
	standard & open inteval &continuous & continuous & continuous \\
	\hline
	topology\footnote{analogous to standard} & open &continuous & continuous & continuous \\
	\hline
	measurable & measurable & measurable & continuous & measurable \\
	\hline
	measurable & measurable & measurable & measurable & measurable \\
	\hline 
	Borel Measurable & Borel & Borel measurable  & continuous & ??? \\
	\hline 
	Borel Measurable & Borel & continuous & Borel measurable & ??? \\
	\hline 
	Borel Measurable & Borel & measurable & Borel mapping & measurable \\
	\hline 
	Borel Measurable & Borel & Boreal measurable & Borel mapping & Borel measurable \\
\hline
\end{tabular}

\newpage
\noindent February 7, 2025\\

\begin{definition}[\textbf{Simple Integration}]

Where $s$ is a simple function, that is, $\exists \{s_i\}, i= 1, \dots, n$ and $E_i$ such that $x \in E_i \implies s(x) = s_i$.
\begin{align*}
	\int_X s d\mu &= \sum_{i=1}^n s_i \mu(X \cap E_i)
\end{align*}(think $E_i$ forms a partition on $X$.)
\end{definition}
\begin{theorem}
Given any positive measurable function $f$ there exists a sequence of simple measurable functions $\{s_i\}$ such that $s_i \to f$.
\end{theorem}

\begin{definition}[Upper/Lower Semicontinuous].

A function $f: X \to \R$ is said to be 
\begin{tabular}{c}
	\DEFINE{lower semicontinuous, lsc,} if $\{x \in X\,|\, f(x) > \alpha\}$ is \underline{open}\\
	\DEFINE{upper semicontinuous, usc,} if $\{x \in X\,|\, f(x) < \alpha\}$ is \underline{open}
\end{tabular}\textbf{for all $\alpha \in \R$.}

A function $f: X \to \R$ is said to be f \begin{tabular}{cc}
	\\
	\DEFINE{lower semicontinous at $x_0 \iff$} & $\underset{x \to x_0}\liminf \,f(x) \ge f(x_0)$ \\
	\DEFINE{upper semicontinuous at $x_0 \iff$} & $\underset{x \to x_0}\limsup \,f(x) \le f(x_0)$
\end{tabular}

Note: \begin{tabular}{c}
	"l.s.c. can jump down not up" \\
	"u.s.c. can jump up not down"
\end{tabular}

\end{definition}
\begin{definition}[\textbf{Integration of Positive Function}] Given $f: X \to [0,\infty]\in \mathfrak{M}, E \in \mathfrak{M}(X)$ 
\begin{align*}
	\int_E fd\mu = \sup \int_E s d\mu
\end{align*}supremum over all simple functions $0\le s< f$.
\end{definition}

\begin{theorem}
	Let $E_i$ be a partition on $X$ then
	\begin{align*}
		\int_X f d\mu &= \int_X \sum_i f\chi_{E_i} d\mu 
	\end{align*}
\end{theorem}

\begin{theorem}[Lebesque Monotone Convergence Theorem] Given an increasing sequence of measurable functions $f_n : X \to [0,\infty] \in \mathfrak{M}$ where $f_n \to f$. Then,
\begin{align*}
	\LIMN \int_X f_n d\mu &= \int_X f d\mu 
\end{align*}

\end{theorem}

\begin{theorem}[Fatou's Lemma] Given an increasing sequence of measurable functions $f_n : X \to [0,\infty] \in \mathfrak{M}$ where $f_n \to f$. Then,
\begin{align*}
	\int_X \LIMINFN f_n d\mu &\le \LIMINFN \int_X f_n d\mu 
\end{align*}

\end{theorem}

\begin{theorem}[Lebesque Dominated Convergence Theorem]  Suppose $\{f_n\}$ is a sequence of complex measurable functions on $X$ such that 
\begin{align}
	f(x) &= \lim_{n\to\infty} f_n(x)
\end{align}exists for evey $x \in X$.  If there is a function $g\in L^1(X)$ (Lebesque Measurable) such that
\begin{align}
	|f_n(x)| \le g(x) \; (n=1,2,\dots,; x\in X)
\end{align}then $f \in L^1(X)$ (Lebesque Measurable),
\begin{align}
	\lim_{n\to \infty} \int_X \BARS{f_n-f}d\mu = 0,
\end{align}and
\begin{align}
	\lim_{n\to \infty} \int_x f_n(x)d\mu = \int_x fd\mu. 
\end{align}

\end{theorem}

\large{\textbf{Understanding 'almost everywhere', a.e. $[\mu]$}

\begin{remark}[Measure Zero].

This term is almost euphemistically used and might be thought of as 'countably infinite set of distinct points'.  For example, the set of rational numbers has a measure zero, is countably infinite and even dense, as are many other sets (Cantor, set of algebraic numbers).  Thus giving rise to the real value of \textit{measure} as indicating sets that actually contain real numbers (that is distiinct from rationals, integers and so on) and their neighborhoods.  I have come to think of a 'measurable set' as 'not a point'.
\end{remark}

\begin{definition}[almost everywhere].

If $\mu$ is a measure on a $\sigma$-algebra $\MMM$ and if $E \in \MMM$, the statement 'property $P$ holds \textit{almost everywhere} on $E$' means that $\exists N\in \MMM \to N \subset E, \mu(N)=0$ and $P$ holds for every $x \in E\backslash N$.

\end{definition}

\begin{theorem}
	Suppose $\{f_n\}$ is a sequence of complex measuable functions defined \DEFINE{a.e. on $X$} such that
	\begin{align*}
		\sum_{n=1}^\infty \int_X |f_n|d\mu < \infty
	\end{align*}Then the series
	\begin{align*}
		f(x) = \sum_{n=1}^\infty f_n(x)
	\end{align*}converges for \DEFINE{almost all} $x,f \in L^1(\mu)$, and 
	\begin{align*}
		\int_X fd\mu = \sum_{n=1}^\infty\int_X f_n d\mu 
	\end{align*}
\end{theorem}

\begin{remark}[Local Compact Hausdorff Space].

See page 35.  The primary property that allows for  $\Lambda$ to be a measure on $\R^n$ is independent of measure and independent of geometry (i.e., inner product, orientation, etc.).  In fact, the primary property is \DEFINE{local compactnesss} which is that every open set contains a neighborhood whose closer is compact (think, closed and bounded, finite subcover).

A \DEFINE{Hausdorff Space, (T-4 spaces),} means that given any two points, $A$ and $B$, there exists neighborhoods around each that are disjoint.  That is there exists $r,s \in \R$ such that $B_r(A) \cap B_s(B) = \emptyset$.  (Incidentally, you could just choose $t = \min\{r,s\}$ and have $n$-balls of the same size).

All metric spaces are Hausdorff Spaces.  Thus, the primary property for Lebesque Measure Theory is the separation of points and that they are surround by compact sets.

Properties of compactness.
\begin{itemize}
	\item Closed sets within compact sets are compact.
	\item $A \subset B$ and $B$ has a compact closure than so does $A$.
	\item Compact subsets of Hausdorff spaces are closed.
	\item If $F$ is closed and $K$ is compact in a Hausdorff space then $F \cap K$ is compact.
	\item If $\{K_\alpha\}$ is a collection of disjoint compact subsets in a Hausdorff space, the some finite subcollection of $\{K_\alpha\}$ are disjoint.
	\item $U$ is an open set in a locally compact Hausdorff Space. $K\subset U$ is compact.  Then there exists $V$ ``between" $K$ and $U$ (that is, $K \subset V \subset \CONJ{V} \subset U$).
	\item Compactness is invariant through continuity.
	\item range$(f)$ is compact (for $f\in C_c(X)$).
	\item \textbf{Notation} $K \prec f$ means $K$ is a compact set of $X$, $f \in C_c(X)$ and $0\le f \le 1, \forall x\in X$ and $f(x)=1, \forall x\in K$.  Further,
	\begin{align*}
		 f \prec V
	\end{align*}$V$ is open $0\le f(x) \le 1, \forall x\in X$ and spt$(f)\subset V$.
\end{itemize}
\end{remark}

\begin{definition}[Support of a function].

The \DEFINE{support} of a complex function on a topological space $X$ is the closure of set 
\begin{align*}
	\text{spt}(f) = \CONJ{\{ x: f(x) \ne 0 \}}.
\end{align*} \footnote{Consider this in contrast to the kernel of a linear function and in contrast to the domain.}
The collection of all continuous, complex functions with compact supports is denoted $C_c(X)$ (a vector space).
\end{definition}

\newpage
\textbf{The Riesz Representation Theorem}.  

\begin{theorem}
	Let $X$ be a locally compact Hausdorff space, and let $\Lambda$ be a positive linear function on $C_c(X)$.  Then there exists a $\sigma$-algebra $\MMM$ in $X$ which contains \textbf{\textit{all}} Borel Sets in $X$, and there exists a \textbf{\textit{positive unique measure}} $\mu$ in $\MMM$ which represents $\Lambda$ in the sense that
	\begin{enumerate}[label=(\alph*)]
		\item $\Lambda f = \int_x f d\mu$ for every $f\in C_c(X)$.
		
		and which has the following additional properties;
		
		\item Compact implies finite: $\mu(K)<\infty$ for every compact set $K \subset X$.
		\item Outer Regular: For every $E \in \MMM$, we have
		\begin{align*}
			\mu(E) = \inf \BRACKET{\mu(V): E \subset V \text{ open}}
		\end{align*}
		
		\item Inner Regular: The relation
		\begin{align*}
			\mu(E) = \sup \BRACKET{\mu(K): K \subset E, K \text{ compact}}
		\end{align*}holds for every open set $E$ and every $E \in \MMM$ with $\mu(E) < \infty$.
		
		\item If $E \in \MMM, A \subset E$, and $\mu(E)=0$, then $A \in \MMM$.
		
		
	For the sake of clarity, let us be more explicit about the meaning of the word
``positive" in the hypothesis: $\Lambda$ is assumed to be a linear functional on the complex vector space with the additional property that $\Lambda f$ is a nonnegative real number for every $f$ whose range consists of nonnegative real numbers.

\textbf{Extensions}
	\item If $E \in \MMM$ and $\epsilon > 0$ then there is a closed set $F$ and an open set $V$ such that $F \subset E\subset V$ and $\mu(V-F)<\epsilon$.
	\item $\mu$ is a regular Borel Measure on $X$.
	\item and $\exists A \in F_\sigma$ and $B\in G_\delta$ such that $A \subset E \subset B$, and $\mu(B-A) =0$.
	\end{enumerate}
\end{theorem}
\end{document}
