\documentclass[10pt,a4paper]{report}
\usepackage[utf8]{inputenc}
\usepackage{amsmath}
\usepackage{amsfonts}
\usepackage{amssymb}
\usepackage{amsthm}
\usepackage{hyperref}

\usepackage{multicol}
\usepackage{fancyhdr}
\usepackage[inline]{enumitem}
\usepackage{tikz}
\usepackage{tikz-cd}
\usetikzlibrary{calc}
\usetikzlibrary{shapes.geometric}
\usepackage[margin=0.5in]{geometry}
\usepackage{xcolor}
\usepackage{ esint }

\hypersetup{
    colorlinks=true,
    linkcolor=blue,
    filecolor=magenta,      
    urlcolor=cyan,
    pdftitle={Tensors},
    pdfpagemode=FullScreen,
    }

%\urlstyle{same}

\newcommand{\CLASSNAME}{Math 5230 -- Partial Differential Equations}
\newcommand{\STUDENTNAME}{Paul Carmody}
\newcommand{\ASSIGNMENT}{Homework \#1 }
\newcommand{\DUEDATE}{September 4, 2025}
\newcommand{\SEMESTER}{Fall 2025}
\newcommand{\SCHEDULE}{MW 2:00 -- 3:15}
\newcommand{\ROOM}{Remote}

\newcommand{\MMN}{M_{m\times n}}
\newcommand{\FF}{\mathcal{F}}

\pagestyle{fancy}
\fancyhf{}
\chead{ \fancyplain{}{\CLASSNAME} }
%\chead{ \fancyplain{}{\STUDENTNAME} }
\rhead{\thepage}
\newcommand{\LET}{\text{Let }}
%\newcommand{\IF}{\text{if }}
\newcommand{\AND}{\text{ and }}
\newcommand{\OR}{\text{ or }}
\newcommand{\FORSOME}{\text{ for some }}
\newcommand{\FORALL}{\text{ for all }}
\newcommand{\WHERE}{\text{ where }}
\newcommand{\WTS}{\text{ WTS }}
\newcommand{\WLOG}{\text{ WLOG }}
\newcommand{\BS}{\backslash}
\newcommand{\DEFINE}[1]{\textbf{\emph{#1}}}
\newcommand{\IF}{$(\Rightarrow)$}
\newcommand{\ONLYIF}{$(\Leftarrow)$}
\newcommand{\ITH}{\textsuperscript{th} }
\newcommand{\FST}{\textsuperscript{st} }
\newcommand{\SND}{\textsuperscript{nd} }
\newcommand{\TRD}{\textsuperscript{rd} }
\newcommand{\INV}{\textsuperscript{-1} }


%%%%%%%
% derivatives
%%%%%%%

\newcommand{\PART}[2]{\frac{\partial #1}{\partial #2}}
\newcommand{\SPART}[2]{\frac{\partial^2 #1}{\partial #2^2}}
\newcommand{\DERIV}[2]{\frac{d #1}{d #2}}
\newcommand{\LAPLACIAN}[1]{\frac{\partial^2 #1}{\partial x^2} + \frac{\partial^2 #1}{\partial y^2}}

%%%%%%%
% sum, product, union, intersections
%%%%%%%

\newcommand{\SUM}[2]{\underset{#1}{\overset{#2}{\sum}}}
\newcommand{\PROD}[2]{\underset{#1}{\overset{#2}{\prod}}}
\newcommand{\UNION}[2]{\underset{#1}{\overset{#2}{\bigcup}}}
\newcommand{\INTERSECT}[2]{\underset{#1}{\overset{#2}{\bigcap}}}
\newcommand{\FSUM}{\SUM{n=-\infty}{\infty}}
       

%%%%%%%
% supremum and infimum
%%%%%%%

\newcommand{\SUP}[1]{\underset{#1}\sup \,}
\newcommand{\INF}[1]{\underset{#1}\inf \,}
\newcommand{\MAX}[1]{\underset{#1}\max \,}
\newcommand{\MIN}[1]{\underset{#1}\min \,}

%%%%%%%
% infinite sums, limits
%%%%%%%

\newcommand{\SUMK}{\SUM{k=1}{\infty}}
\newcommand{\SUMN}{\SUM{n=1}{\infty}}
\newcommand{\SUMKZ}{\SUM{k=0}{\infty}}
\newcommand{\LIM}[1]{\underset{#1}\lim\,}
\newcommand{\IWOB}[1]{\LIM{#1 \to \infty}}
\newcommand{\LIMK}{\IWOB{k}}
\newcommand{\LIMN}{\IWOB{n}}
\newcommand{\LIMX}{\IWOB{x}}
\newcommand{\NIWOB}{\LIM{n \to \infty}}
\newcommand{\LIMSUPK}{\underset{k\to\infty}\limsup \,}
\newcommand{\LIMSUPN}{\underset{n\to\infty}\limsup \,}
\newcommand{\LIMINFK}{\underset{k\to\infty}\liminf \,}
\newcommand{\LIMINFN}{\underset{n\to\infty}\liminf \,}
\newcommand{\ROOTRULE}[1]{\LIMSUPK \BARS{#1}^{1/k}}

\newcommand{\CUPK}{\bigcup_{k=1}^{\infty}}
\newcommand{\CAPK}{\bigcap_{k=1}^{\infty}}
\newcommand{\CUPN}{\bigcup_{n=1}^{\infty}}
\newcommand{\CAPN}{\bigcap_{n=1}^{\infty}}

%%%%%%%
% number systems (real, rational, etc.)
%%%%%%%

\newcommand{\REALS}{\mathbb{R}}
\newcommand{\RATIONALS}{\mathbb{Q}}
\newcommand{\IRRATIONALS}{\REALS \backslash \RATIONALS}
\newcommand{\INTEGERS}{\mathbb{Z}}
\newcommand{\NUMBERS}{\mathbb{N}}
\newcommand{\COMPLEX}{\mathbb{C}}
\newcommand{\DISC}{\mathbb{D}}
\newcommand{\HPLANE}{\mathbb{H}}

\newcommand{\R}{\mathbb{R}}
\newcommand{\Q}{\mathbb{Q}}
\newcommand{\Z}{\mathbb{Z}}
\newcommand{\N}{\mathbb{N}}
\newcommand{\C}{\mathbb{C}}
\newcommand{\T}{\mathbb{T}}
\newcommand{\COUNTABLE}{\aleph_0}
\newcommand{\UNCOUNTABLE}{\aleph_1}


%%%%%%%
% Arithmetic/Algebraic operators
%%%%%%%


\DeclareMathOperator{\MOD}{mod}
%\newcommand{\MOD}[1]{\mod #1}
\newcommand{\BAR}[1]{\overline{#1}}
\newcommand{\LCM}{\text{ lcm}}
\newcommand{\ZMOD}[1]{\Z/#1\Z}
\DeclareMathOperator{\VAR}{Var}
%%%%%%%
% complex operators
%%%%%%%

\DeclareMathOperator{\RR}{Re}
%\newcommand{\RE}{\text{Re}}
\DeclareMathOperator{\IM}{Im}
%\newcommand{\IM}{\text{Im}}
\newcommand{\CONJ}[1]{\overline{#1}}
\DeclareMathOperator{\LOG}{Log}
%\newcommand{\LOG}{\text{ Log }}
\newcommand{\RES}[2]{\underset{#1}{\text{res}} #2}

%%%%%%%
% Group operators
%%%%%%%

\newcommand{\AUT}{\text{Aut}\,}
\newcommand{\KER}{\text{ker}\,}
\newcommand{\END}{\text{End}}
\newcommand{\HOM}{\text{Hom}}
\newcommand{\CYCLE}[1]{(\begin{array}{cccccccccc}
		#1
	\end{array})}
\newcommand{\SUBGROUP}{\underset{\text{group}}\subseteq}	
%\newcommand{\SUBGROUP}{\subseteq_g}
\newcommand{\SUBRING}{\underset{\text{ring}}\subseteq}
\newcommand{\SUBMOD}{\underset{\text{mod}}\subseteq}
\newcommand{\SUBFIELD}{\underset{\text{field}}\subseteq}
\newcommand{\ISO}{\underset{\text{iso}}\longrightarrow}
\newcommand{\HOMO}{\underset{\text{homo}}\longrightarrow}

%%%%%%%
% grouping (parenthesis, absolute value, square, multi-level brackets).
%%%%%%%

\newcommand{\PAREN}[1]{\left (\, #1 \,\right )}
\newcommand{\BRACKET}[1]{\left \{\, #1 \,\right \}}
\newcommand{\SQBRACKET}[1]{\left [\, #1 \,\right ]}
\newcommand{\ABRACKET}[1]{\left \langle\, #1 \,\right \rangle}
\newcommand{\BARS}[1]{\left |\, #1 \,\right |}
\newcommand{\DBARS}[1]{\left \| \, #1 \,\right \|}
\newcommand{\LBRACKET}[1]{\left \{ #1 \right .} 
\newcommand{\RBRACKET}[1]{\left . #1 \right \]}
\newcommand{\RBAR}[1]{\left . #1 \, \right |}
\newcommand{\LBAR}[1]{\left | \, #1 \right .}
\newcommand{\BLBRACKET}[2]{\BRACKET{\RBAR{#1}#2}}
\newcommand{\GEN}[1]{\ABRACKET{#1}}
\newcommand{\BINDEF}[2]{\LBRACKET{\begin{array}{ll}
     #1\\
     #2
\end{array}}}

%%%%%%%
% Fourier Analysis
%%%%%%%

\newcommand{\ONEOTWOPI}{\frac{1}{2\pi}}
\newcommand{\FHAT}{\hat{f}(n)}
\newcommand{\FINT}{\int_{-\pi}^\pi}
\newcommand{\FINTWO}{\int_{0}^{2\pi}}
\newcommand{\FSUMN}[1]{\SUM{n=-#1}{#1}}
%\newcommand{\FSUM}{\SUMN{\infty}}
\newcommand{\EIN}[1]{e^{in#1}}
\newcommand{\NEIN}[1]{e^{-in#1}}
\newcommand{\INTALL}{\int_{-\infty}^{\infty}}
\newcommand{\FTINT}[1]{\INTALL #1 e^{2\pi inx\xi} dx}
\newcommand{\GAUSS}{e^{-\pi x^2}}

%%%%%%%
% formatting 
%%%%%%%

\newcommand{\LEFTBOLD}[1]{\noindent\textbf{#1}}
\newcommand{\SEQ}[1]{\{#1\,\}}
\newcommand{\WIP}{\footnote{work in progress}}
\newcommand{\QED}{\hfill\square}
\newcommand{\ts}{\textsuperscript}
\newcommand{\HLINE}{\noindent\rule{7in}{1pt}\\}

%%%%%%%
% Mathematical note taking (definitions, theorems, etc.)
%%%%%%%

\newcommand{\REM}{\noindent\textbf{\\Remark: }}
\newcommand{\DEF}{\noindent\textbf{\\Definition: }}
\newcommand{\THE}{\noindent\textbf{\\Theorem: }}
\newcommand{\COR}{\noindent\textbf{\\Corollary: }}
\newcommand{\LEM}{\noindent\textbf{\\Lemma: }}
\newcommand{\PROP}{\noindent\textbf{\\Proposition: }}
\newcommand{\PROOF}{\noindent\textbf{\\Proof: }}
\newcommand{\EXP}{\noindent\textbf{\\Example: }}
\newcommand{\TRICKS}{\noindent\textbf{\\Tricks: }}


%%%%%%%
% text highlighting
%%%%%%%

\newcommand{\B}[1]{\textbf{#1}}
\newcommand{\CAL}[1]{\mathcal{#1}}
\newcommand{\UL}[1]{\underline{#1}}

%%%%%%
% Linear Algebra
%%%%%%

\newcommand{\COLVECTOR}[1]{\PAREN{\begin{array}{c}
#1
\end{array} }}
\newcommand{\TWOXTWO}[4]{\PAREN{ \begin{array}{c c} #1&#2 \\ #3 & #4 \end{array} }}
\newcommand{\THREEXTHREE}[9]{\PAREN{ \begin{array}{c c c} #1&#2&#3 \\ #4 & #5 & #6 \\ #7 & #8 & #9 \end{array} }}
\newcommand{\NXN}{\PAREN{ \begin{array}{c c c c} 
			a_{11} & a_{12} & \cdots & a_{1n} \\
			a_{21} & a_{22} & \cdots & a_{2n} \\
			\vdots & \vdots & \ddots & a_{1n} \\
			a_{n1} & a_{n2} & \cdots & a_{nn} \\
		\end{array} }}
\newcommand{\SLR}{SL_2(\R)}
\newcommand{\GLR}{GL_2(\R)}
\DeclareMathOperator{\TR}{tr}
\DeclareMathOperator{\BIL}{Bil}
\DeclareMathOperator{\SPAN}{span}

%%%%%%%
%  White space
%%%%%%%

\newcommand{\BOXIT}[1]{\noindent\fbox{\parbox{\textwidth}{#1}}}


\newtheorem{theorem}{Theorem}[section]
\newtheorem{corollary}{Corollary}[theorem]
\newtheorem{lemma}[theorem]{Lemma}

\theoremstyle{definition}
\newtheorem{definition}[theorem]{Definition}
\newtheorem{prop}[theorem]{Proposition}

\theoremstyle{remark}
\newtheorem{remark}[theorem]{Remark}
\newtheorem{example}[theorem]{Example}
%\newtheorem*{proof}[theorem]{Proof}



\newcommand{\RED}[1]{\textcolor{red}{#1}}
\newcommand{\BLUE}[1]{\textcolor{blue}{#1}}

\begin{document}

\begin{center}
	\Large{\CLASSNAME -- \SEMESTER} \\
	\large{ w/Professor XXXX}
\end{center}
\begin{center}
	\STUDENTNAME \\
	\ASSIGNMENT -- \DUEDATE\\
\end{center} 

\begin{description}
	\item \textbf{Part I.}
	\begin{enumerate}
		\item 
		\begin{enumerate}
			\item Consider an initial value problem for th elinear tranport equation wiht a bounded, one-dimensional spatial domain:
			\begin{align*}
				\LBRACKET{\begin{array}{ll}
					u_t+3u_x=0, &0<x<1, t>0,\\
					u(x,0)=g(x),&\\
					u(0,t)=0.&
				\end{array}}
			\end{align*}
		We assume that $g(0)=0$, so that the initial condition and boundary condition agree at the corner $(x,t)=(0,0)$.\\
		Find a formula for $u(x,t)$ using the same method as was seen in classe, ile., use the fact that a certian direction derivative of $u$ is zero.  What do you notice about your solution for large times?
		
		\BLUE{\textbf{From the Lecture:}
		\begin{align*}
			u_t+bu_x &= \ABRACKET{b,1}\cdot\ABRACKET{u_x, u_t}. 
		\end{align*}Define $\hat{b}$ as the vector that satisfies $b\cdot\hat{b}=1$.  Then this can also be written as 
		\begin{align*}
			\hat{b}u_t+u_x &= \ABRACKET{1,\hat{b}}\cdot\ABRACKET{u_x, u_t} 
		\end{align*}and, from the description of the PDE
		\begin{align*}
			\hat{b}u_t+u_x &= \ABRACKET{1,\hat{b}}\cdot\ABRACKET{u_x, u_t} =0
		\end{align*}Thus, as in the lecture which emphasized the $(x,t)$-plane we can see a similar solution in the $(t,x)$-plane.\\
		Recall that we have a function $z(s)$
		\begin{align*}
			z(s) = u(x+sb, t+s)
		\end{align*}Let's reparameterize $z$ with $r = sb$ as
		\begin{align*}
			z(r)=u(x+r,t+r\hat{b})
		\end{align*}Now, we differentiate with respect to $r$.
		\begin{align*}
			\frac{dz(r)}{dr} &= \frac{d}{dr}u(x+r,t+r\hat{b}) \\
			&= \PART{u}{x}\PART{x}{r}(x+r)+\PART{u}{t}\PART{t}{r}(t+r\hat{b}) \\
			&= \PART{u}{x}+\hat{b}\PART{u}{t} \\
			&= u_x+\hat{b}u_t = 0
		\end{align*}As expected $z$ is still constant.  Then,
		\begin{align*}
			z(0) &= z(-x)\\
			u(x,t) &= u(0, t-x\hat{b}) \\
			&= h(t-x\hat{b}).
		\end{align*}Now we have two solutions for $u(x,t)$
		\begin{align*}
			u(x,t) &= g(x-bt) & \text{(from the lecture)}\\
			u(x,t) &= h(t-x\hat{b}) \\
			\therefore g(x-bt) &= h(t-x\hat{b})
		\end{align*}is our only solution.
		}
		
		\item Next, derive a solution formula for the same problem with a more general boundary condition and source term:
		\begin{align*}
			\LBRACKET{\begin{array}{ll}
					u_t+3u_x=f(x,t), &0<x<1, t>0,\\
					u(x,0)=g(x),&\\
					u(0,t)=h(t).&
			\end{array}}
		\end{align*}We assume that $g(0)=h(0)$, so that the initial condition and boundary condition agree at the corner $(x,t)=(0,0)$.
		
		\item Why did we only specify the boundary condition at he left-hand boundary $(x=0)$, not the right-hand boundary $(x=1)$?  In other words, what woudl go wrong if we specified boudnary conditions on both sides, as in 
		\begin{align*}
			\LBRACKET{\begin{array}{ll}
				u_t+3u_x=f(x,t), &0<x<1, t>0,\\
				u(x,0)=g(x),&\\
				u(0,t)=h_0(t),&\\
				u(1,t)=h_1(t),
			\end{array}
			}
		\end{align*}for some give functions $h_0(t)$ and $h_1(t)$?  (We can assume $g(0)=h_0(0)$ and $g(10=h_1(0)$, so that the intial and boundary conditions agree at the corners.)
		
		\end{enumerate}
		
		\item In one space dimensio, Laplace's equation $\Delta u=0$ becomes an ODE
		\begin{align*}
			u''(x)=0
		\end{align*}Describe all solutions to this ODE posed on the real line $(-\infty, \infty)$.  Next, for given constants $c_1,c_2$, find the uniqe solution to $u''(x)=0$ on the interval $[0,1]$ with $u(0)=c_1$ and $u(1)=c_2$.\\
		Do these one-dimensional harmonic functions satisfy the mean value property $u(x)=\fint_{B(x,y)}u(y)dy$?  Why or why not?

		\BLUE{\begin{align*}
			\int u''(x)dx &= u'(x) + c \\
			\int (u'(x) + c)dx &= u(x)+cx+d \\
			\therefore u(x)+cx+d &= 0
		\end{align*}When $u(0)=c_1$ and $u(1) = c_2$ we have
		\begin{align*}
			u(0)+c(0)+d = 0 &\implies c_1+d=0, d=-c_1 \\
			u(1)+c(1)+d = 0 &\implies c_2+c+d=0 \\
			c_1 = c_2+c &\implies c=c_1-c_2
		\end{align*}That is
		\begin{align*}
			u(x) +(c_1-c_2)x-c_1 &=0 \\
			u(x) &= c_1-(c_1-c_2)x
		\end{align*}is our solution.  Do these one-dimensional harmonic functions satisfy the mean value property $u(x)=\fint_{B(x,y)}u(y)dy$?
		\begin{align*}
			B(x,y) &= [a,b]\\
			\fint_{B(x,y)}u(y)dy &= \frac{1}{b-a}\int_a^b u(y)dy \\
			&= \frac{1}{b-a}\int_a^b(c_1-(c_1-c_2)y) dy	\\
			&= \frac{1}{b-a}\SQBRACKET{c_1y - \frac{c_1-c_2}{2}y^2}_a^b \\
			&=\frac{1}{b-a}\SQBRACKET{c_1(b-a) - \frac{c_1-c_2}{2}(b-a)^2} \\
			&= c_1 - \frac{c_1-c_2}{2}(b-a) \\
			&= \frac{2c_1-c_1+c_2}{2} & \text{when }[a,b] = [0,1] \\
			&= \frac{c_1+c_2}{2}
		\end{align*}which is the average.		}
				
		\item Let $z=x+iy$ be a complex variable ($x$ and $y$ are real numbers).  Recall that a complex function $f(z)=u(x,y)+iv(x,y)$ is \textit{complex-differentiable} if $u$ and $v$ are continuously differentibable (as functiosn of $x$ and $y$) and Cauchy-Reimann eqations are satisfied:
		\begin{align*}
			\PART{u}{x}&=\PART{v}{y}\\
			\PART{u}{y}&=-\PART{v}{x}
		\end{align*}If $f$ is complex-differentiable, and in addition the real and imaginary parts $u(x,y)$ and $v(x,y)$ are $C^2$ (twice continuously differentiable) functions, then show that $u$ and $v$ are harmonic.
		
		\BLUE{
		\begin{align*}
			\Delta u &= \frac{\partial^2 u}{\partial x^2}+ \frac{\partial^2 u}{\partial y^2} \\
			&= \PART{}{x}\PAREN{\PART{u}{x}}+\PART{}{y}\PAREN{\PART{u}{y}} \\
			&= \PART{}{x}\PAREN{\PART{v}{y}}-\PART{}{y}\PAREN{\PART{v}{x}}\\
			&= 0 \\
			\Delta v &= \frac{\partial^2 v}{\partial x^2}+ \frac{\partial^2 v}{\partial y^2} \\
			&= \PART{}{x}\PAREN{\PART{v}{x}}+\PART{}{y}\PAREN{\PART{v}{y}} \\
			&= -\PART{}{x}\PAREN{\PART{u}{y}}+\PART{}{y}\PAREN{\PART{u}{x}}\\
			&= 0 
		\end{align*}
		}
		
	\end{enumerate}
	
	\item \textbf{Part II}
	\begin{enumerate}
		\item Write down an explicit formula for a function $u$ solving the initial-value problem
		\begin{align*}
			\LBRACKET{\begin{array}{rl}
				u_t+b\cdot Du+c=0 & \text{in } \R^n\times (0,\infty)\\
				u=g & \text{on } \R^n\times \{t=0\}.
			\end{array}
			}
		\end{align*}Here $c \in \R$ and $b\in\R^n$ are constants.
		\item Prove that Laplace's equation $\Delta u=0$ is rotation invariant; that is, if $O$ is an orthogonal $n \times n$ matrix and we definte
		\begin{align*}
			v(x) := u(Ox) \, (x \in \R^n)
		\end{align*}then $\Delta v=0$.
		\setcounter{enumi}{4}
		\item We say $v \in C^2(\bar{U})$ is \textit{subharmonic} if
		\begin{align*}
			-\Delta u \le 0\, \text{in } U.
		\end{align*}
		\begin{enumerate}[label=(\alph*)]
			\item Prove for subharmonic $v$ that
			\begin{align*}
				v(x) \le \fint_{B(x,r)} v dy \, \text{ for all } B(x,y) \subset U.
			\end{align*}
			\item Prove that therefore $\max_{\bar{U}} v = \max_{\partial U} v$.
			\item[(d)] Prove $v := \BARS{Du}^2$ is subharmonic, wherever $u$ is harmonic.
		
		\end{enumerate}
	\end{enumerate}
\end{description}

\end{document}