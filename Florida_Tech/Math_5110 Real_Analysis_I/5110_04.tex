\documentclass[10pt,a4paper]{report}
\usepackage[utf8]{inputenc}
\usepackage{amsmath}
\usepackage{amsfonts}
\usepackage{amssymb}
\usepackage{amsthm}
\usepackage{hyperref}

\usepackage{multicol}
\usepackage{fancyhdr}
\usepackage[inline]{enumitem}
\usepackage{tikz}
\usepackage{tikz-cd}
\usetikzlibrary{calc}
\usetikzlibrary{shapes.geometric}
\usepackage[margin=0.5in]{geometry}
\usepackage{xcolor}

\hypersetup{
    colorlinks=true,
    linkcolor=blue,
    filecolor=magenta,      
    urlcolor=cyan,
    pdftitle={Tensors},
    pdfpagemode=FullScreen,
    }

%\urlstyle{same}

\newcommand{\CLASSNAME}{Math 5110 -- Real Analysis I}
\newcommand{\STUDENTNAME}{Paul Carmody}
\newcommand{\ASSIGNMENT}{Homework \#4 }
\newcommand{\DUEDATE}{November 25, 2024}
\newcommand{\SEMESTER}{Fall 2024}
\newcommand{\SCHEDULE}{MW 11:00 -- 12:15}
\newcommand{\ROOM}{Remote}

\newcommand{\MMN}{M_{m\times n}}
\newcommand{\FF}{\mathcal{F}}

\pagestyle{fancy}
\fancyhf{}
\chead{ \fancyplain{}{\CLASSNAME} }
%\chead{ \fancyplain{}{\STUDENTNAME} }
\rhead{\thepage}
\newcommand{\LET}{\text{Let }}
%\newcommand{\IF}{\text{if }}
\newcommand{\AND}{\text{ and }}
\newcommand{\OR}{\text{ or }}
\newcommand{\FORSOME}{\text{ for some }}
\newcommand{\FORALL}{\text{ for all }}
\newcommand{\WHERE}{\text{ where }}
\newcommand{\WTS}{\text{ WTS }}
\newcommand{\WLOG}{\text{ WLOG }}
\newcommand{\BS}{\backslash}
\newcommand{\DEFINE}[1]{\textbf{\emph{#1}}}
\newcommand{\IF}{$(\Rightarrow)$}
\newcommand{\ONLYIF}{$(\Leftarrow)$}
\newcommand{\ITH}{\textsuperscript{th} }
\newcommand{\FST}{\textsuperscript{st} }
\newcommand{\SND}{\textsuperscript{nd} }
\newcommand{\TRD}{\textsuperscript{rd} }
\newcommand{\INV}{\textsuperscript{-1} }

\newcommand{\XXX}{\mathfrak{X}}
\newcommand{\MMM}{\mathfrak{M}}
%\newcommand{\????}{\textfrak{A}}
%\newcommand{\????}{\textgoth{A}}
%\newcommand{\????}{\textswab{A}}

\DeclareMathOperator{\DER}{Der}
\DeclareMathOperator{\SGN}{sgn}

%%%%%%%
% derivatives
%%%%%%%

\newcommand{\PART}[2]{\frac{\partial #1}{\partial #2}}
\newcommand{\SPART}[2]{\frac{\partial^2 #1}{\partial #2^2}}
\newcommand{\DERIV}[2]{\frac{d #1}{d #2}}
\newcommand{\LAPLACIAN}[1]{\frac{\partial^2 #1}{\partial x^2} + \frac{\partial^2 #1}{\partial y^2}}

%%%%%%%
% sum, product, union, intersections
%%%%%%%

\newcommand{\SUM}[2]{\underset{#1}{\overset{#2}{\sum}}}
\newcommand{\PROD}[2]{\underset{#1}{\overset{#2}{\prod}}}
\newcommand{\UNION}[2]{\underset{#1}{\overset{#2}{\bigcup}}}
\newcommand{\INTERSECT}[2]{\underset{#1}{\overset{#2}{\bigcap}}}
\newcommand{\FSUM}{\SUM{n=-\infty}{\infty}}
       

%%%%%%%
% supremum and infimum
%%%%%%%

\newcommand{\SUP}[1]{\underset{#1}\sup \,}
\newcommand{\INF}[1]{\underset{#1}\inf \,}
\newcommand{\MAX}[1]{\underset{#1}\max \,}
\newcommand{\MIN}[1]{\underset{#1}\min \,}

%%%%%%%
% infinite sums, limits
%%%%%%%

\newcommand{\SUMK}{\SUM{k=1}{\infty}}
\newcommand{\SUMN}{\SUM{n=1}{\infty}}
\newcommand{\SUMKZ}{\SUM{k=0}{\infty}}
\newcommand{\LIM}[1]{\underset{#1}\lim\,}
\newcommand{\IWOB}[1]{\LIM{#1 \to \infty}}
\newcommand{\LIMK}{\IWOB{k}}
\newcommand{\LIMN}{\IWOB{n}}
\newcommand{\LIMX}{\IWOB{x}}
\newcommand{\NIWOB}{\LIM{n \to \infty}}
\newcommand{\LIMSUPK}{\underset{k\to\infty}\limsup \,}
\newcommand{\LIMSUPN}{\underset{n\to\infty}\limsup \,}
\newcommand{\LIMINFK}{\underset{k\to\infty}\liminf \,}
\newcommand{\LIMINFN}{\underset{n\to\infty}\liminf \,}
\newcommand{\ROOTRULE}[1]{\LIMSUPK \BARS{#1}^{1/k}}

\newcommand{\CUPK}{\bigcup_{k=1}^{\infty}}
\newcommand{\CAPK}{\bigcap_{k=1}^{\infty}}
\newcommand{\CUPN}{\bigcup_{n=1}^{\infty}}
\newcommand{\CAPN}{\bigcap_{n=1}^{\infty}}

%%%%%%%
% number systems (real, rational, etc.)
%%%%%%%

\newcommand{\REALS}{\mathbb{R}}
\newcommand{\RATIONALS}{\mathbb{Q}}
\newcommand{\IRRATIONALS}{\REALS \backslash \RATIONALS}
\newcommand{\INTEGERS}{\mathbb{Z}}
\newcommand{\NUMBERS}{\mathbb{N}}
\newcommand{\COMPLEX}{\mathbb{C}}
\newcommand{\DISC}{\mathbb{D}}
\newcommand{\HPLANE}{\mathbb{H}}

\newcommand{\R}{\mathbb{R}}
\newcommand{\Q}{\mathbb{Q}}
\newcommand{\Z}{\mathbb{Z}}
\newcommand{\N}{\mathbb{N}}
\newcommand{\C}{\mathbb{C}}
\newcommand{\T}{\mathbb{T}}
\newcommand{\COUNTABLE}{\aleph_0}
\newcommand{\UNCOUNTABLE}{\aleph_1}


%%%%%%%
% Arithmetic/Algebraic operators
%%%%%%%


\DeclareMathOperator{\MOD}{mod}
%\newcommand{\MOD}[1]{\mod #1}
\newcommand{\BAR}[1]{\overline{#1}}
\newcommand{\LCM}{\text{ lcm}}
\newcommand{\ZMOD}[1]{\Z/#1\Z}
\DeclareMathOperator{\VAR}{Var}
%%%%%%%
% complex operators
%%%%%%%

\DeclareMathOperator{\RR}{Re}
%\newcommand{\RE}{\text{Re}}
\DeclareMathOperator{\IM}{Im}
%\newcommand{\IM}{\text{Im}}
\newcommand{\CONJ}[1]{\overline{#1}}
\DeclareMathOperator{\LOG}{Log}
%\newcommand{\LOG}{\text{ Log }}
\newcommand{\RES}[2]{\underset{#1}{\text{res}} #2}

%%%%%%%
% Group operators
%%%%%%%

\newcommand{\AUT}{\text{Aut}\,}
\newcommand{\KER}{\text{ker}\,}
\newcommand{\END}{\text{End}}
\newcommand{\HOM}{\text{Hom}}
\newcommand{\CYCLE}[1]{(\begin{array}{cccccccccc}
		#1
	\end{array})}
\newcommand{\SUBGROUP}{\underset{\text{group}}\subseteq}	
%\newcommand{\SUBGROUP}{\subseteq_g}
\newcommand{\SUBRING}{\underset{\text{ring}}\subseteq}
\newcommand{\SUBMOD}{\underset{\text{mod}}\subseteq}
\newcommand{\SUBFIELD}{\underset{\text{field}}\subseteq}
\newcommand{\ISO}{\underset{\text{iso}}\longrightarrow}
\newcommand{\HOMO}{\underset{\text{homo}}\longrightarrow}

%%%%%%%
% grouping (parenthesis, absolute value, square, multi-level brackets).
%%%%%%%

\newcommand{\PAREN}[1]{\left (\, #1 \,\right )}
\newcommand{\BRACKET}[1]{\left \{\, #1 \,\right \}}
\newcommand{\SQBRACKET}[1]{\left [\, #1 \,\right ]}
\newcommand{\ABRACKET}[1]{\left \langle\, #1 \,\right \rangle}
\newcommand{\BARS}[1]{\left |\, #1 \,\right |}
\newcommand{\DBARS}[1]{\left \| \, #1 \,\right \|}
\newcommand{\LBRACKET}[1]{\left \{ #1 \right .} 
\newcommand{\RBRACKET}[1]{\left . #1 \right \]}
\newcommand{\RBAR}[1]{\left . #1 \, \right |}
\newcommand{\LBAR}[1]{\left | \, #1 \right .}
\newcommand{\BLBRACKET}[2]{\BRACKET{\RBAR{#1}#2}}
\newcommand{\GEN}[1]{\ABRACKET{#1}}
\newcommand{\BINDEF}[2]{\LBRACKET{\begin{array}{ll}
     #1\\
     #2
\end{array}}}

%%%%%%%
% Fourier Analysis
%%%%%%%

\newcommand{\ONEOTWOPI}{\frac{1}{2\pi}}
\newcommand{\FHAT}{\hat{f}(n)}
\newcommand{\FINT}{\int_{-\pi}^\pi}
\newcommand{\FINTWO}{\int_{0}^{2\pi}}
\newcommand{\FSUMN}[1]{\SUM{n=-#1}{#1}}
%\newcommand{\FSUM}{\SUMN{\infty}}
\newcommand{\EIN}[1]{e^{in#1}}
\newcommand{\NEIN}[1]{e^{-in#1}}
\newcommand{\INTALL}{\int_{-\infty}^{\infty}}
\newcommand{\FTINT}[1]{\INTALL #1 e^{2\pi inx\xi} dx}
\newcommand{\GAUSS}{e^{-\pi x^2}}

%%%%%%%
% formatting 
%%%%%%%

\newcommand{\LEFTBOLD}[1]{\noindent\textbf{#1}}
\newcommand{\SEQ}[1]{\{#1\,\}}
\newcommand{\WIP}{\footnote{work in progress}}
\newcommand{\QED}{\hfill\square}
\newcommand{\ts}{\textsuperscript}
\newcommand{\HLINE}{\noindent\rule{7in}{1pt}\\}

%%%%%%%
% Mathematical note taking (definitions, theorems, etc.)
%%%%%%%

\newcommand{\REM}{\noindent\textbf{\\Remark: }}
\newcommand{\DEF}{\noindent\textbf{\\Definition: }}
\newcommand{\THE}{\noindent\textbf{\\Theorem: }}
\newcommand{\COR}{\noindent\textbf{\\Corollary: }}
\newcommand{\LEM}{\noindent\textbf{\\Lemma: }}
\newcommand{\PROP}{\noindent\textbf{\\Proposition: }}
\newcommand{\PROOF}{\noindent\textbf{\\Proof: }}
\newcommand{\EXP}{\noindent\textbf{\\Example: }}
\newcommand{\TRICKS}{\noindent\textbf{\\Tricks: }}


%%%%%%%
% text highlighting
%%%%%%%

\newcommand{\B}[1]{\textbf{#1}}
\newcommand{\CAL}[1]{\mathcal{#1}}
\newcommand{\UL}[1]{\underline{#1}}

%%%%%%
% Linear Algebra
%%%%%%

\newcommand{\COLVECTOR}[1]{\PAREN{\begin{array}{c}
#1
\end{array} }}
\newcommand{\TWOXTWO}[4]{\PAREN{ \begin{array}{c c} #1&#2 \\ #3 & #4 \end{array} }}
\newcommand{\DTWOXTWO}[4]{\BARS{ \begin{array}{c c} #1&#2 \\ #3 & #4 \end{array} }}
\newcommand{\THREEXTHREE}[9]{\PAREN{ \begin{array}{c c c} #1&#2&#3 \\ #4 & #5 & #6 \\ #7 & #8 & #9 \end{array} }}
\newcommand{\DTHREEXTHREE}[9]{\BARS{ \begin{array}{c c c} #1&#2&#3 \\ #4 & #5 & #6 \\ #7 & #8 & #9 \end{array} }}
\newcommand{\NXN}{\PAREN{ \begin{array}{c c c c} 
			a_{11} & a_{12} & \cdots & a_{1n} \\
			a_{21} & a_{22} & \cdots & a_{2n} \\
			\vdots & \vdots & \ddots & a_{1n} \\
			a_{n1} & a_{n2} & \cdots & a_{nn} \\
		\end{array} }}
\newcommand{\SLR}{SL_2(\R)}
\newcommand{\GLR}{GL_2(\R)}
\DeclareMathOperator{\TR}{tr}
\DeclareMathOperator{\BIL}{Bil}
\DeclareMathOperator{\SPAN}{span}

%%%%%%%
%  White space
%%%%%%%

\newcommand{\BOXIT}[1]{\noindent\fbox{\parbox{\textwidth}{#1}}}


\newtheorem{theorem}{Theorem}[section]
\newtheorem{corollary}{Corollary}[theorem]
\newtheorem{lemma}[theorem]{Lemma}

\theoremstyle{definition}
\newtheorem{definition}[theorem]{Definition}
\newtheorem{prop}[theorem]{Proposition}

\theoremstyle{remark}
\newtheorem{remark}[theorem]{Remark}
\newtheorem{example}[theorem]{Example}
%\newtheorem*{proof}[theorem]{Proof}



\newcommand{\RED}[1]{\textcolor{red}{#1}}
\newcommand{\BLUE}[1]{\textcolor{blue}{#1}}

\begin{document}

\begin{center}
	\Large{\CLASSNAME -- \SEMESTER} \\
	\large{ w/Professor Liu}
\end{center}
\begin{center}
	\STUDENTNAME \\
	\ASSIGNMENT -- \DUEDATE\\
\end{center} 

\newcommand{\VOL}{\text{ vol}}
\begin{enumerate}[label=\Roman*.]
\item \textit{Exercise 7.2.2.}  Let $A$ be a subset of $\R^n$, and let $B$ be a subset of $\R^m$.  Note that the Cartesian product $\{(a,b): a\in A, b\in B\}$ is then a subset of $\R_{n+m}$.  Show that $m_{n+m}^*(A\times B) \le m_n^*(A)m_m^*(B)$.  (It is in fact true that $m_{n+m}^*(A\times B) = m_n^*(A)m_m^*(B)$. but is substantially harder to prove).

In Exercise 7.2.3-7.2.5, we assume that $\R^n$ is Euclidean space, and we have a notion of measurable set in $\R^n$ (which may or may not coincide with the notion of Lebesgue Measurable set)  and a notion of measure (which may or may not coincide with Lebesque measure) which obeys axioms (i)-(xiii).

\BLUE{Since $m^*(\Omega)$ is defined as \footnote{I've substitute $B$ with $C$ from the definition in the text.}
\begin{align*}
	m^*(\Omega) &= \inf\BRACKET{\sum_{j\in J}\VOL (C_j)\,:\, (C_j)_{j\in J}\text{ covers }\Omega;\, J\text{ at most countable}}.
\end{align*}where $(C_j)_{j\in J}$ is a covering for $\Omega$.  Then there exists boxes $(\alpha_k)_{k \in K}$ and $(\beta_l)_{l\in L}$ which are coverings for $A$ and $B$, respectively. And, clearly,
\begin{align*}
	m^*(A) &\le \sum_{k\in K} \VOL(\alpha_k) \AND m^*(B) \le \sum_{l\in L} \VOL(\beta_l)
\end{align*}define a covering $J$ such that
\begin{align*}
	\delta_{k,l} &= \alpha_k \times \beta_l
\end{align*}$\delta_{k,l}$ is countable as it is a union of two countable sets and it is a covering over $A\times B$.  And, since each $\alpha_k$ and $\beta_l$ is a box, then 
\begin{align*}
	m^*(\delta_{k,l}) &= m^*(\alpha_k)m^*(\beta_l),\,\forall k\in K, \, l \in L \\
	m^*(A \times B) &\le \sum_{k\in K,\, l\in L} m^*(\delta_{k,l}) \\
	\sum_{k\in K,\, l\in L} m^*(\delta_{k,l}) &= \sum_{k\in K,\, l\in L} m^*(\alpha_k)m^*(\beta_l) \\
	&\le \sum_{k\in K} m^*(\alpha_k) \sum_{l \in L} m^*(\beta_l) & \text{Cauchy-Schwarz} \\
	&\le m^*(A)m^*(B)
\end{align*}
}

\newpage
\item Section 7.4, problems 1,4 (only parts (e) and (f)).

\textit{Exercise 7.4.1.}  If $A$ is an open interval in $\R$, show that $m^*(A)=m^*(A\cap (0,\infty))+m^*(A\backslash (0,\infty))$.

\BLUE{From Lemma 7.4.8 (\textit{countable additivity)}, we can see that the two sets are disjoint, i.e., 
\begin{align*}
	&(A\cap (0,\infty)) \,\cap\, (A\backslash (0,\infty)) = \emptyset \\
	\AND A &= (A\cap (0,\infty)) \,\sqcup \, (A\backslash (0,\infty)) \\
	m^*(A)&=m^*(A\cap (0,\infty))+m^*(A\backslash (0,\infty))
\end{align*}
}

\textit{Exercise 7.4.4.}  Prove Lemma 7.4.4. (Hints: for (c) first prove that 

(e) \textit{Every open box, and every closed box, is measurable.}

\BLUE{Let $C$ be an open box, $C=\prod_{i=1}^n (a_i, b_i)$.  Define two half-spaces $A_i = \BRACKET{(x_1,\dots,x_n)\in \R^n\,|\, x_i > a_i) }$ and $B_i = \BRACKET{(x_1,\dots,x_n)\in \R^n\,|\, x_i < b_i) }$ for each $i= 1,\dots, n$.  Given any $x\in C$ where $x=(x_1,\dots, x_n)$ then $x_i \in (A_i \cap B_i)$, therefore $C = \bigcap_{i=1}^n (A_i \cap B_i)$.  Each half-space $A_i$ and $B_i$ is measurable and $C$ is the intersection of finitely many measurable sets and is therefore measurable.
}

(f) \textit{any set $E$ of outer measure zero (i.e., $m^*(E) =0)$ is measurable.}

\BLUE{Let $T \subseteq \R^n$ be any measurable subset.  Let $E$ be a set with $m^*(E)=0$.  Then,
\begin{align*}
	T &= (T\cap E) \bigcup (T\backslash E) \\
	m^*(T) &= m^*\PAREN{(T\cap E) \bigcup (T\backslash E)} \\
	&= m^*\PAREN{(T\cap E)} + m^*\PAREN{(T\backslash E)} 
\end{align*}Given any $x \in T\cap E$ clearly $x \in E$ thus $m^*(T\cap E) = 0$.  Therefore $m^*(T) = m^*\PAREN{(T\backslash E)}$.  Also, $x\in T\backslash E$ indicates that $x \in T$ thus $m^*(T) = m^*\PAREN{(T\backslash E)}$.  This is true for all open sets $T \in \R$ therefore $E$ is measurable.
}

\newpage
\item Let $C$ be a parameterized curve in $\R^2$.  In other words, $C$ is the image for a function $\phi:[a,b]\to \R^2$.  Show that if $\phi$ is continuously differentiable in $[a,b]$, then $C$ has outer measure 0.

\textit{Hint: } Partition $[a,b]$ into $N$ equal subintervals, and use the Mean Value Inequality to show that the image of each subinterval is bounded in terms of $N$, i.e., fits inside an open rectangle of side length that can be explicitly bounded in terms of $N$.  Add up the total 2-dimensional volume of the covering obtained in this way, and show that it can be made arbitrarily small by taking $N$ large.

\textit{Warning:}  If $\phi$ is only continuous, then the result fails. One can construct a continuous $\phi$ such that $$ \phi([a,b])=[0,1]\times[0,1].$$

\BLUE{Divide $[a,b]$ into $n$ equal subintervals.  Each subinterval, $[a_i,b_i]$ has lenght $1/n$ and will have an intermediate value $\zeta_i \in [a_i,b_i]$ such that $\phi'(\zeta_i)/n = \phi(b_i)-\phi(a_i)$.  The volume of each subinterval can be represented by $\phi'(\zeta_i)/n$.  $\phi$ is bounded, and if it is continuously differentiable, $\phi'$ is bounded, too.  Thus, $\LIMN \phi'(\zeta_i)/n=0$ and  $\LIMN \SUM{i=1}{n}\phi'(\zeta_i)/n = 0$.  $m_2^*$ is a measure of area and therefore $m^*(C)= \int_a^b \phi(t) dt = \LIMN \SUM{i=1}{n}\phi'(\zeta_i)/n =0$.
}

\newpage
\item \RED{skip}
\item Suppose $A_i \in \mathcal{M}, A_1 \subset A_2 \subset \cdots \subset A_n \subset A_{n+1} \subset \cdots$.
\begin{enumerate}[label=(\alph*)]
	\item if $m(A_1) < \infty$, show that 
	\begin{align*}
		m\PAREN{\bigcap_{n=1}^\infty A_n} &= \lim_{n\to\infty} m(A_n).
	\end{align*}
	
	\BLUE{Since each $A_i$ is strictly contained in $A_{i+1}$ we can say that
	\begin{align*}
		A_{i-1} \cap A_i &= A_{i-1} \\
		\OR A_1 \cap A_2 &= A_1 \\
		\therefore A_1 \cap A_i &= A_1, \, \forall i=1,\dots\AND\lim_{n\to \infty} m(A_n) = m(A_1) \\
		A_1 &= \bigcap_{n=1}^\infty A_n = A_1\\
		m\PAREN{\bigcap_{n=1}^\infty A_n} &= m(A_1)
	\end{align*}
	}
	
	\item Show by example that if $m(A_1)=\infty$, the above conclusion may be wrong.
	
	\BLUE{Let $A_1$ be an open box with $A_1= \prod_{i=1}^n (a_i, b_i)$ where all $a_i, b_i < \infty$ except $b_1 = \infty$.  Clearly, $A_1$ is a half-space and is therefore measureable.  Let $A_2\supset A_1$ be the same as $A_1$ except that $b_2=\infty$.  In general,let $A_i\supset A_{i-1}$ and be the same as $A_{i-1}$ except for $b_i = \infty$.  We can clearly see that 
	\begin{align*}
		A_1 &\subset A_2 \subset \cdots A_{n-1} \subseteq A_n \subseteq \cdots 
\end{align*}	but we cannot know $\LIMN m(A_n)$.
	}

\end{enumerate}

\item Let $\Omega \subset \R^n$ be measurable.  $f: \Omega \to \R$ is a function.  If $f^2$ is measurable, and the set 
\begin{align*}
	A=\{x\in\Omega\,|\, f(x) > 0\}
\end{align*}is also measurable.  Show that $f$ is measurable.

\BLUE{Let $h=f^2$ which is measurable.  Therefore. given open set $V \in \R^n$ there exists a measurable set $U \in \R^n$ such that $h(U) = V$. Let $g = f|_A$ and choose $V = g(A)$ and $U$ such that $h(U)=V$.  Therefore, $g(f(U))=V$ or $f|_A(f(U))=V$ or $f_A(A)=V$.  $A$ is measurable and is mapped to an open set.  Therefore, $f$ is measurable.
}

\end{enumerate}
\end{document}
