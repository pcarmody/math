\documentclass[12pt,a4paper]{report}
\usepackage[utf8]{inputenc}
\usepackage{amsmath}
\usepackage{amsfonts}
\usepackage{amssymb}
\usepackage{amsthm}
\usepackage{hyperref}

\usepackage{multicol}
\usepackage{fancyhdr}
\usepackage[inline]{enumitem}
\usepackage{tikz}
\usepackage{tikz-cd}
\usetikzlibrary{calc}
\usetikzlibrary{shapes.geometric}
\usepackage[margin=0.5in]{geometry}
\usepackage{xcolor}
\usepackage{listings} 

\hypersetup{
    colorlinks=true,
    linkcolor=blue,
    filecolor=magenta,      
    urlcolor=cyan,
    pdftitle={Tensors},
    pdfpagemode=FullScreen,
    }

%\urlstyle{same}

\newcommand{\CLASSNAME}{Math 5301 -- Numerical Analysis}
\newcommand{\STUDENTNAME}{Paul Carmody}
\newcommand{\ASSIGNMENT}{Midterm }
\newcommand{\DUEDATE}{April 14, 2025}
\newcommand{\SEMESTER}{Spring 2025}
\newcommand{\SCHEDULE}{MW 2:00 to 3:15}
\newcommand{\ROOM}{Crawford 330}

\newcommand{\MMN}{M_{m\times n}}
\newcommand{\FF}{\mathcal{F}}

\pagestyle{fancy}
 	\fancyhf{}
\chead{ \fancyplain{}{\CLASSNAME} }
%\chead{ \fancyplain{}{\STUDENTNAME} }
\rhead{\thepage}
\newcommand{\LET}{\text{Let }}
%\newcommand{\IF}{\text{if }}
\newcommand{\AND}{\text{ and }}
\newcommand{\OR}{\text{ or }}
\newcommand{\FORSOME}{\text{ for some }}
\newcommand{\FORALL}{\text{ for all }}
\newcommand{\WHERE}{\text{ where }}
\newcommand{\WTS}{\text{ WTS }}
\newcommand{\WLOG}{\text{ WLOG }}
\newcommand{\BS}{\backslash}
\newcommand{\DEFINE}[1]{\textbf{\emph{#1}}}
\newcommand{\IF}{$(\Rightarrow)$}
\newcommand{\ONLYIF}{$(\Leftarrow)$}
\newcommand{\ITH}{\textsuperscript{th} }
\newcommand{\FST}{\textsuperscript{st} }
\newcommand{\SND}{\textsuperscript{nd} }
\newcommand{\TRD}{\textsuperscript{rd} }
\newcommand{\INV}{\textsuperscript{-1} }


%%%%%%%
% derivatives
%%%%%%%

\newcommand{\PART}[2]{\frac{\partial #1}{\partial #2}}
\newcommand{\SPART}[2]{\frac{\partial^2 #1}{\partial #2^2}}
\newcommand{\DERIV}[2]{\frac{d #1}{d #2}}
\newcommand{\LAPLACIAN}[1]{\frac{\partial^2 #1}{\partial x^2} + \frac{\partial^2 #1}{\partial y^2}}

%%%%%%%
% sum, product, union, intersections
%%%%%%%

\newcommand{\SUM}[2]{\underset{#1}{\overset{#2}{\sum}}}
\newcommand{\PROD}[2]{\underset{#1}{\overset{#2}{\prod}}}
\newcommand{\UNION}[2]{\underset{#1}{\overset{#2}{\bigcup}}}
\newcommand{\INTERSECT}[2]{\underset{#1}{\overset{#2}{\bigcap}}}
\newcommand{\FSUM}{\SUM{n=-\infty}{\infty}}
       

%%%%%%%
% supremum and infimum
%%%%%%%

\newcommand{\SUP}[1]{\underset{#1}\sup \,}
\newcommand{\INF}[1]{\underset{#1}\inf \,}
\newcommand{\MAX}[1]{\underset{#1}\max \,}
\newcommand{\MIN}[1]{\underset{#1}\min \,}

%%%%%%%
% infinite sums, limits
%%%%%%%

\newcommand{\SUMK}{\SUM{k=1}{\infty}}
\newcommand{\SUMN}{\SUM{n=1}{\infty}}
\newcommand{\SUMKZ}{\SUM{k=0}{\infty}}
\newcommand{\LIM}[1]{\underset{#1}\lim\,}
\newcommand{\IWOB}[1]{\LIM{#1 \to \infty}}
\newcommand{\LIMK}{\IWOB{k}}
\newcommand{\LIMN}{\IWOB{n}}
\newcommand{\LIMX}{\IWOB{x}}
\newcommand{\NIWOB}{\LIM{n \to \infty}}
\newcommand{\LIMSUPK}{\underset{k\to\infty}\limsup \,}
\newcommand{\LIMSUPN}{\underset{n\to\infty}\limsup \,}
\newcommand{\LIMINFK}{\underset{k\to\infty}\liminf \,}
\newcommand{\LIMINFN}{\underset{n\to\infty}\liminf \,}
\newcommand{\ROOTRULE}[1]{\LIMSUPK \BARS{#1}^{1/k}}

\newcommand{\CUPK}{\bigcup_{k=1}^{\infty}}
\newcommand{\CAPK}{\bigcap_{k=1}^{\infty}}
\newcommand{\CUPN}{\bigcup_{n=1}^{\infty}}
\newcommand{\CAPN}{\bigcap_{n=1}^{\infty}}

%%%%%%%
% number systems (real, rational, etc.)
%%%%%%%

\newcommand{\REALS}{\mathbb{R}}
\newcommand{\RATIONALS}{\mathbb{Q}}
\newcommand{\IRRATIONALS}{\REALS \backslash \RATIONALS}
\newcommand{\INTEGERS}{\mathbb{Z}}
\newcommand{\NUMBERS}{\mathbb{N}}
\newcommand{\COMPLEX}{\mathbb{C}}
\newcommand{\DISC}{\mathbb{D}}
\newcommand{\HPLANE}{\mathbb{H}}

\newcommand{\R}{\mathbb{R}}
\newcommand{\Q}{\mathbb{Q}}
\newcommand{\Z}{\mathbb{Z}}
\newcommand{\N}{\mathbb{N}}
\newcommand{\C}{\mathbb{C}}
\newcommand{\T}{\mathbb{T}}
\newcommand{\COUNTABLE}{\aleph_0}
\newcommand{\UNCOUNTABLE}{\aleph_1}


%%%%%%%
% Arithmetic/Algebraic operators
%%%%%%%


\DeclareMathOperator{\MOD}{mod}
%\newcommand{\MOD}[1]{\mod #1}
\newcommand{\BAR}[1]{\overline{#1}}
\newcommand{\LCM}{\text{ lcm}}
\newcommand{\ZMOD}[1]{\Z/#1\Z}
\DeclareMathOperator{\VAR}{Var}
%%%%%%%
% complex operators
%%%%%%%

\DeclareMathOperator{\RR}{Re}
%\newcommand{\RE}{\text{Re}}
\DeclareMathOperator{\IM}{Im}
%\newcommand{\IM}{\text{Im}}
\newcommand{\CONJ}[1]{\overline{#1}}
\DeclareMathOperator{\LOG}{Log}
%\newcommand{\LOG}{\text{ Log }}
\newcommand{\RES}[2]{\underset{#1}{\text{res}} #2}

%%%%%%%
% Group operators
%%%%%%%

\newcommand{\AUT}{\text{Aut}\,}
\newcommand{\KER}{\text{ker}\,}
\newcommand{\END}{\text{End}}
\newcommand{\HOM}{\text{Hom}}
\newcommand{\CYCLE}[1]{(\begin{array}{cccccccccc}
		#1
	\end{array})}
\newcommand{\SUBGROUP}{\underset{\text{group}}\subseteq}	
%\newcommand{\SUBGROUP}{\subseteq_g}
\newcommand{\SUBRING}{\underset{\text{ring}}\subseteq}
\newcommand{\SUBMOD}{\underset{\text{mod}}\subseteq}
\newcommand{\SUBFIELD}{\underset{\text{field}}\subseteq}
\newcommand{\ISO}{\underset{\text{iso}}\longrightarrow}
\newcommand{\HOMO}{\underset{\text{homo}}\longrightarrow}

%%%%%%%
% grouping (parenthesis, absolute value, square, multi-level brackets).
%%%%%%%

\newcommand{\PAREN}[1]{\left (\, #1 \,\right )}
\newcommand{\BRACKET}[1]{\left \{\, #1 \,\right \}}
\newcommand{\SQBRACKET}[1]{\left [\, #1 \,\right ]}
\newcommand{\ABRACKET}[1]{\left \langle\, #1 \,\right \rangle}
\newcommand{\BARS}[1]{\left |\, #1 \,\right |}
\newcommand{\DBARS}[1]{\left \| \, #1 \,\right \|}
\newcommand{\LBRACKET}[1]{\left \{ #1 \right .} 
\newcommand{\RBRACKET}[1]{\left . #1 \right \]}
\newcommand{\RBAR}[1]{\left . #1 \, \right |}
\newcommand{\LBAR}[1]{\left | \, #1 \right .}
\newcommand{\BLBRACKET}[2]{\BRACKET{\RBAR{#1}#2}}
\newcommand{\GEN}[1]{\ABRACKET{#1}}
\newcommand{\BINDEF}[2]{\LBRACKET{\begin{array}{ll}
     #1\\
     #2
\end{array}}}

%%%%%%%
% Fourier Analysis
%%%%%%%

\newcommand{\ONEOTWOPI}{\frac{1}{2\pi}}
\newcommand{\FHAT}{\hat{f}(n)}
\newcommand{\FINT}{\int_{-\pi}^\pi}
\newcommand{\FINTWO}{\int_{0}^{2\pi}}
\newcommand{\FSUMN}[1]{\SUM{n=-#1}{#1}}
%\newcommand{\FSUM}{\SUMN{\infty}}
\newcommand{\EIN}[1]{e^{in#1}}
\newcommand{\NEIN}[1]{e^{-in#1}}
\newcommand{\INTALL}{\int_{-\infty}^{\infty}}
\newcommand{\FTINT}[1]{\INTALL #1 e^{2\pi inx\xi} dx}
\newcommand{\GAUSS}{e^{-\pi x^2}}

%%%%%%%
% formatting 
%%%%%%%

\newcommand{\LEFTBOLD}[1]{\noindent\textbf{#1}}
\newcommand{\SEQ}[1]{\{#1\,\}}
\newcommand{\WIP}{\footnote{work in progress}}
\newcommand{\QED}{\hfill\square}
\newcommand{\ts}{\textsuperscript}
\newcommand{\HLINE}{\noindent\rule{7in}{1pt}\\}

%%%%%%%
% Mathematical note taking (definitions, theorems, etc.)
%%%%%%%

\newcommand{\REM}{\noindent\textbf{\\Remark: }}
\newcommand{\DEF}{\noindent\textbf{\\Definition: }}
\newcommand{\THE}{\noindent\textbf{\\Theorem: }}
\newcommand{\COR}{\noindent\textbf{\\Corollary: }}
\newcommand{\LEM}{\noindent\textbf{\\Lemma: }}
\newcommand{\PROP}{\noindent\textbf{\\Proposition: }}
\newcommand{\PROOF}{\noindent\textbf{\\Proof: }}
\newcommand{\EXP}{\noindent\textbf{\\Example: }}
\newcommand{\TRICKS}{\noindent\textbf{\\Tricks: }}


%%%%%%%
% text highlighting
%%%%%%%

\newcommand{\B}[1]{\textbf{#1}}
\newcommand{\CAL}[1]{\mathcal{#1}}
\newcommand{\UL}[1]{\underline{#1}}

%%%%%%
% Linear Algebra
%%%%%%

\newcommand{\COLVECTOR}[1]{\PAREN{\begin{array}{c}
#1
\end{array} }}
\newcommand{\TWOXTWO}[4]{\PAREN{ \begin{array}{c c} #1&#2 \\ #3 & #4 \end{array} }}
\newcommand{\THREEXTHREE}[9]{\PAREN{ \begin{array}{c c c} #1&#2&#3 \\ #4 & #5 & #6 \\ #7 & #8 & #9 \end{array} }}
\newcommand{\NXN}{\PAREN{ \begin{array}{c c c c} 
			a_{11} & a_{12} & \cdots & a_{1n} \\
			a_{21} & a_{22} & \cdots & a_{2n} \\
			\vdots & \vdots & \ddots & a_{1n} \\
			a_{n1} & a_{n2} & \cdots & a_{nn} \\
		\end{array} }}
\newcommand{\SLR}{SL_2(\R)}
\newcommand{\GLR}{GL_2(\R)}
\DeclareMathOperator{\TR}{tr}
\DeclareMathOperator{\BIL}{Bil}
\DeclareMathOperator{\SPAN}{span}

%%%%%%%
%  White space
%%%%%%%

\newcommand{\BOXIT}[1]{\noindent\fbox{\parbox{\textwidth}{#1}}}


\newtheorem{theorem}{Theorem}[section]
\newtheorem{corollary}{Corollary}[theorem]
\newtheorem{lemma}[theorem]{Lemma}

\theoremstyle{definition}
\newtheorem{definition}[theorem]{Definition}
\newtheorem{prop}[theorem]{Proposition}

\theoremstyle{remark}
\newtheorem{remark}[theorem]{Remark}
\newtheorem{example}[theorem]{Example}
%\newtheorem*{proof}[theorem]{Proof}



\newcommand{\RED}[1]{\textcolor{red}{#1}}
\newcommand{\BLUE}[1]{\textcolor{blue}{#1}}
\definecolor{mygreen}{RGB}{28,172,0} % color values Red, Green, Blue
\definecolor{mylilas}{RGB}{170,55,241}

\begin{document}
\lstset{language=Matlab,%
    %basicstyle=\color{red},
    breaklines=true,%
    morekeywords={matlab2tikz},
    keywordstyle=\color{blue},%
    morekeywords=[2]{1}, keywordstyle=[2]{\color{black}},
    identifierstyle=\color{black},%
    stringstyle=\color{mylilas},
    commentstyle=\color{mygreen},%
    showstringspaces=false,%without this there will be a symbol in the places where there is a space
    numbers=left,%
    numberstyle={\tiny \color{black}},% size of the numbers
    numbersep=9pt, % this defines how far the numbers are from the text
    emph=[1]{for,end,break},emphstyle=[1]\color{red}, %some words to emphasise
    %emph=[2]{word1,word2}, emphstyle=[2]{style},    
}

\begin{center}
	\Large{\CLASSNAME -- \SEMESTER} \\
	\large{ w/Professor Du}
\end{center}
\begin{center}
	\STUDENTNAME \\
	\ASSIGNMENT -- \DUEDATE\\
\end{center} 
\HLINE

\textbf{Assignment:} Consider 1D Poisson Equation $-\Delta x = \sin (\pi x)$, over the region $(0, \pi/2)$ with boundary conditions $u(0)=0$ and $u(\pi/2)=1$.  Using central difference scheme and a mesh of 128, obtain a linear system of $Au = f$ for the problem, then the solve the system using the following methods until a relative residual of $10^{-4}$ is reached. For all methods below, plot the analytical solution, the numerical solution and the error distribution.  Use zero vectors as your initial guess.

\begin{enumerate}[label=(\alph*)]
\item Gauss-Seidel Method.  Plot the rate of convergence and compare it with analysis.
\item Conjugate Gradient Method. Compare the rate of convergence with that obtained in (a).  Do not use the CG solver provided by MatLab.
\item Conjugage Gradient Method Preconditioned by Cholesky Decomposition.  Compare the rate of convergence with that obtained by (b).

\end{enumerate}
\HLINE
%
%
%In the report, please include basic introduction of the methods, all related computer code and plots, as well as all the necessary analysis
%
%
\begin{center}
	\Large{Analytical and several Numerical Solutions to Poisson's Equation}\\
\end{center}

\noindent\textbf{Introduction:}\\

	We will begin by describing the equation and providing an analytical solution.  Then solve this equation using Jacobi's Method and the Steepest Descent Method.  Comparisons will be made as to accurancy and rate of convergence.\\
	
\noindent\textbf{Poison's Equation and an Analytical Solution:}\\

As described in the assignment we will focus our attenion on this Poisson equation and initial conditions.
\begin{align*}
	-\Delta x &= \sin(\pi x)\\
	u(0) = 0 &\AND u(\pi/2)=1.
\end{align*}When we solve this problem analtyically we get
\begin{align*}
	u'(x) &= \int -\sin(\pi x) dx \\
	&= \frac{1}{\pi}\cos(\pi x)+C \\
	u(x) &= \int \PAREN{\frac{1}{\pi}\cos(\pi x)+C} dx \\
	&= \frac{1}{\pi^2}sin(\pi x) + Cx + D \\
	u(0) = 0 &= \frac{1}{\pi^2}sin(\pi 0) + Cx + D \\
	D &= 0 \\
	u(\pi/2) =1 &= \frac{1}{\pi^2}sin(\pi^2/2) + C(\pi/2) \\
	C &= \frac{2}{\pi}\PAREN{1-\frac{1}{•\pi^2}}
\end{align*}

\noindent\textbf{Centered Difference Scheme}\\

We will attempt to approximate the curve of  the solution at particular points $u_i$ by calculating a slope at a point by using the preceeding point $u_{i-1}$ and succeeding point $u_{i+1}$.  
\begin{align*}
	\frac{u_{i+1}-2u_i+u_{i-1}}{h^2} \approx -\sin (\pi x)
\end{align*}where $h=(\pi/2)/129=\pi/258$ (we use $N+1$ as we start with the left boundary 0).  This can be reduced to 
\begin{align*}
	-u_{i+1}+2u_i-u_{i-1} = h^2\sin (\pi x).
\end{align*}This expands to a linear function over the a matrix $A$ and vector $u=\{ u_i \}$ reflecting the left hand side and the value to our function on the right with $f=\{ f_i \}, f_i=\sin(\pi x_i)$ or
\begin{align*}
	Au &= h^2 f\\
	\PAREN{\begin{array}{cccccc}
		2 & -1 & 0 & 0 &\cdots & 0\\
		-1 & 2 & -1 & 0 &\cdots & 0\\
		0 & -1 & 2 & -1 &\cdots & 0\\
		0 & 0 &-1 & 2 & \cdots & 0\\
		\vdots & \vdots & \vdots & \vdots &\ddots & \vdots\\
		0 & 0 & 0 & 0 & \cdots & 2
	\end{array} } \COLVECTOR{ u_1 \\ u_2 \\ u_3 \\ u_4 \\ \vdots \\ u_i } &= h^2 \COLVECTOR{ f_1 \\ f_2 \\ f_3 \\ f_4 \\ \vdots \\ f_i } \\
	u &= h^2 A^{-1} f
\end{align*}remembering the boundary conditions.  Since $A$ is tri-diagonal we can use several methods and compare the cost and efficiency.  From here we use several methods meant to study the efficiecny and optimization of this method.\\

\noindent\textbf{Gauss-Seidel Method.}\\

Since $A$ is a tri-diagonal matrix it can be divided into three separate matrices that add up.  Let $D$ be zero everywhere except the diagonal where it will hold the values of $A_{ii}$ (namely all 2s).  By applying this iteratively we get
\begin{align*}
	u_i^{[k+1]} &= \frac{1}{2}(u_{i-1}^{[k]}+u_{i+1}^{[k]} - h^2 f_i)
\end{align*}Here is the MatLab code used to generate the graphs that follow:
	\lstinputlisting{midterm_GS.m}
	
%From this we generate the following graphs.%4452

%\includegraphics[scale=0.5]{midterm_1.png} 

\noindent\textbf{Conjugate Gradient Method}\\

This technique for approximating our solution to the Poisson equation is to make each iteration in the direction of greatest change.  That is, 
\begin{align*}
	\nabla \phi(u_{k-1}) = Au_{k-1}-f \equiv -r_{k-1}
\end{align*}where $\phi: \R^m \to \R$ of the form
\begin{align*}
	\phi(u) &= \frac{1}{2}u^TAu-u^Tf
\end{align*}which is a quadratic function in $u$ and can be mapped with local extrema either as a top, a bowl or a saddle point, all based on the eigenvalues of $A$ (negative, positive, or neither, respectively).  Thus, when $A$ is SPD we can expect $r_{k}$ to progress ever closer towards the extremum. \\
\\The source code for the Conjugage Gradient Method differsX from that of the Conjugate Gradient Method Preconditioned by Cholesky Decomposition by only a few lines of code.  The MatLab source code for both will be found in the next section. \\

%This is the source code
%	\lstinputlisting{ml04d_1.m}

%From this we generate the following graphs.

%\includegraphics[scale=0.5]{midterm_1.png} 


\noindent\textbf{Conjugage Gradient Method Preconditioned by Cholesky Decomposition}\\

The primary differences with the Cholesky Decomposition is the use of modified version of the $A$ matrix.  This factorized matrix, $M$, to simplify the $M^{-1}$ process as
\begin{align*}
	Au =f \iff M^{-1}Au = M^{-1}Au= M^{-1}f
\end{align*}which improves the efficiency of the process.  It can be shown that this has the same Condition Number as $A$. Some initialization is necessary to start with the preconditioning settting (found in lines 8 through 12). Then, when the new resolution is calculated, we utilze the preconditioning to recalculate our adjustment variable $\beta$ (found in lines 36 through 40).
\begin{align*}
	r_{k+1} &= r_k^T M^{-1}r \\
	\beta &= r_{k+1}/r_k \\
	p &= r_k^TM^{-1}r _+ \beta p
\end{align*}

This is the source code
	\lstinputlisting{midterm_PCG.m}

From this we generate the following graphs.

\includegraphics[scale=0.5]{midterm_1.png} 
\\
We can see from this graph that the Conjugate Gradient Method (the blue circles and no lines) and the Conjugate Gradient Method w/Preconditioning (the black lines between) follow the same path.  The resulting error distribution indicates the greater accuracy of the Conjugate Gradient Method over the Gauss-Seidel method as well as greater efficiency. \\

\newpage
\noindent\textbf{Analyzing the various solutions}.\\

The following graph shows the rate of convergence of the Conjugate Gradient Method.  The Congjugate Gradient Methdo w/Preconditioning used a single iteration before converging.  It should be noted that the Gauss-Seidel method maximized its interation count before achieving the tolerance level of $10^{-4}$.\\  

\includegraphics[scale=0.5]{midterm_2.png} 



\end{document}
