\documentclass[10pt,a4paper]{report}
\usepackage[utf8]{inputenc}
\usepackage{amsmath}
\usepackage{amsfonts}
\usepackage{amssymb}
\usepackage{amsthm}
\usepackage{hyperref}

\usepackage{multicol}
\usepackage{fancyhdr}
\usepackage[inline]{enumitem}
\usepackage{tikz}
\usepackage{tikz-cd}
\usetikzlibrary{calc}
\usetikzlibrary{shapes.geometric}
\usepackage[margin=0.5in]{geometry}
\usepackage{xcolor}

\hypersetup{
    colorlinks=true,
    linkcolor=blue,
    filecolor=magenta,      
    urlcolor=cyan,
    pdftitle={Tensors},
    pdfpagemode=FullScreen,
    }

%\urlstyle{same}

\newcommand{\CLASSNAME}{Math 5050 -- Special Topics: Manifolds}
\newcommand{\STUDENTNAME}{Paul Carmody}
\newcommand{\ASSIGNMENT}{Assignment 4 }
\newcommand{\DUEDATE}{March 27, 2025}
\newcommand{\SEMESTER}{Spring 2025}
\newcommand{\SCHEDULE}{MW 12:30 - 1:45}
\newcommand{\ROOM}{Remote}

\newcommand{\MMN}{M_{m\times n}}
\newcommand{\FF}{\mathcal{F}}
\newcommand{\RANGE}{\text{range}}

\pagestyle{fancy}
\fancyhf{}
\chead{ \fancyplain{}{\CLASSNAME} }
%\chead{ \fancyplain{}{\STUDENTNAME} }
\rhead{\thepage}
\newcommand{\LET}{\text{Let }}
%\newcommand{\IF}{\text{if }}
\newcommand{\AND}{\text{ and }}
\newcommand{\OR}{\text{ or }}
\newcommand{\FORSOME}{\text{ for some }}
\newcommand{\FORALL}{\text{ for all }}
\newcommand{\WHERE}{\text{ where }}
\newcommand{\WTS}{\text{ WTS }}
\newcommand{\WLOG}{\text{ WLOG }}
\newcommand{\BS}{\backslash}
\newcommand{\DEFINE}[1]{\textbf{\emph{#1}}}
\newcommand{\IF}{$(\Rightarrow)$}
\newcommand{\ONLYIF}{$(\Leftarrow)$}
\newcommand{\ITH}{\textsuperscript{th} }
\newcommand{\FST}{\textsuperscript{st} }
\newcommand{\SND}{\textsuperscript{nd} }
\newcommand{\TRD}{\textsuperscript{rd} }
\newcommand{\INV}{\textsuperscript{-1} }

\newcommand{\XXX}{\mathfrak{X}}
\newcommand{\MMM}{\mathfrak{M}}
%\newcommand{\????}{\textfrak{A}}
%\newcommand{\????}{\textgoth{A}}
%\newcommand{\????}{\textswab{A}}

\DeclareMathOperator{\DER}{Der}
\DeclareMathOperator{\SGN}{sgn}

%%%%%%%
% derivatives
%%%%%%%

\newcommand{\PART}[2]{\frac{\partial #1}{\partial #2}}
\newcommand{\SPART}[2]{\frac{\partial^2 #1}{\partial #2^2}}
\newcommand{\DERIV}[2]{\frac{d #1}{d #2}}
\newcommand{\LAPLACIAN}[1]{\frac{\partial^2 #1}{\partial x^2} + \frac{\partial^2 #1}{\partial y^2}}

%%%%%%%
% sum, product, union, intersections
%%%%%%%

\newcommand{\SUM}[2]{\underset{#1}{\overset{#2}{\sum}}}
\newcommand{\PROD}[2]{\underset{#1}{\overset{#2}{\prod}}}
\newcommand{\UNION}[2]{\underset{#1}{\overset{#2}{\bigcup}}}
\newcommand{\INTERSECT}[2]{\underset{#1}{\overset{#2}{\bigcap}}}
\newcommand{\FSUM}{\SUM{n=-\infty}{\infty}}
       

%%%%%%%
% supremum and infimum
%%%%%%%

\newcommand{\SUP}[1]{\underset{#1}\sup \,}
\newcommand{\INF}[1]{\underset{#1}\inf \,}
\newcommand{\MAX}[1]{\underset{#1}\max \,}
\newcommand{\MIN}[1]{\underset{#1}\min \,}

%%%%%%%
% infinite sums, limits
%%%%%%%

\newcommand{\SUMK}{\SUM{k=1}{\infty}}
\newcommand{\SUMN}{\SUM{n=1}{\infty}}
\newcommand{\SUMKZ}{\SUM{k=0}{\infty}}
\newcommand{\LIM}[1]{\underset{#1}\lim\,}
\newcommand{\IWOB}[1]{\LIM{#1 \to \infty}}
\newcommand{\LIMK}{\IWOB{k}}
\newcommand{\LIMN}{\IWOB{n}}
\newcommand{\LIMX}{\IWOB{x}}
\newcommand{\NIWOB}{\LIM{n \to \infty}}
\newcommand{\LIMSUPK}{\underset{k\to\infty}\limsup \,}
\newcommand{\LIMSUPN}{\underset{n\to\infty}\limsup \,}
\newcommand{\LIMINFK}{\underset{k\to\infty}\liminf \,}
\newcommand{\LIMINFN}{\underset{n\to\infty}\liminf \,}
\newcommand{\ROOTRULE}[1]{\LIMSUPK \BARS{#1}^{1/k}}

\newcommand{\CUPK}{\bigcup_{k=1}^{\infty}}
\newcommand{\CAPK}{\bigcap_{k=1}^{\infty}}
\newcommand{\CUPN}{\bigcup_{n=1}^{\infty}}
\newcommand{\CAPN}{\bigcap_{n=1}^{\infty}}

%%%%%%%
% number systems (real, rational, etc.)
%%%%%%%

\newcommand{\REALS}{\mathbb{R}}
\newcommand{\RATIONALS}{\mathbb{Q}}
\newcommand{\IRRATIONALS}{\REALS \backslash \RATIONALS}
\newcommand{\INTEGERS}{\mathbb{Z}}
\newcommand{\NUMBERS}{\mathbb{N}}
\newcommand{\COMPLEX}{\mathbb{C}}
\newcommand{\DISC}{\mathbb{D}}
\newcommand{\HPLANE}{\mathbb{H}}

\newcommand{\R}{\mathbb{R}}
\newcommand{\Q}{\mathbb{Q}}
\newcommand{\Z}{\mathbb{Z}}
\newcommand{\N}{\mathbb{N}}
\newcommand{\C}{\mathbb{C}}
\newcommand{\T}{\mathbb{T}}
\newcommand{\COUNTABLE}{\aleph_0}
\newcommand{\UNCOUNTABLE}{\aleph_1}


%%%%%%%
% Arithmetic/Algebraic operators
%%%%%%%


\DeclareMathOperator{\MOD}{mod}
%\newcommand{\MOD}[1]{\mod #1}
\newcommand{\BAR}[1]{\overline{#1}}
\newcommand{\LCM}{\text{ lcm}}
\newcommand{\ZMOD}[1]{\Z/#1\Z}
\DeclareMathOperator{\VAR}{Var}
%%%%%%%
% complex operators
%%%%%%%

\DeclareMathOperator{\RR}{Re}
%\newcommand{\RE}{\text{Re}}
\DeclareMathOperator{\IM}{Im}
%\newcommand{\IM}{\text{Im}}
\newcommand{\CONJ}[1]{\overline{#1}}
\DeclareMathOperator{\LOG}{Log}
%\newcommand{\LOG}{\text{ Log }}
\newcommand{\RES}[2]{\underset{#1}{\text{res}} #2}

%%%%%%%
% Group operators
%%%%%%%

\newcommand{\AUT}{\text{Aut}\,}
\newcommand{\KER}{\text{ker}\,}
\newcommand{\END}{\text{End}}
\newcommand{\HOM}{\text{Hom}}
\newcommand{\CYCLE}[1]{(\begin{array}{cccccccccc}
		#1
	\end{array})}
\newcommand{\SUBGROUP}{\underset{\text{group}}\subseteq}	
%\newcommand{\SUBGROUP}{\subseteq_g}
\newcommand{\SUBRING}{\underset{\text{ring}}\subseteq}
\newcommand{\SUBMOD}{\underset{\text{mod}}\subseteq}
\newcommand{\SUBFIELD}{\underset{\text{field}}\subseteq}
\newcommand{\ISO}{\underset{\text{iso}}\longrightarrow}
\newcommand{\HOMO}{\underset{\text{homo}}\longrightarrow}

%%%%%%%
% grouping (parenthesis, absolute value, square, multi-level brackets).
%%%%%%%

\newcommand{\PAREN}[1]{\left (\, #1 \,\right )}
\newcommand{\BRACKET}[1]{\left \{\, #1 \,\right \}}
\newcommand{\SQBRACKET}[1]{\left [\, #1 \,\right ]}
\newcommand{\ABRACKET}[1]{\left \langle\, #1 \,\right \rangle}
\newcommand{\BARS}[1]{\left |\, #1 \,\right |}
\newcommand{\DBARS}[1]{\left \| \, #1 \,\right \|}
\newcommand{\LBRACKET}[1]{\left \{ #1 \right .} 
\newcommand{\RBRACKET}[1]{\left . #1 \right \]}
\newcommand{\RBAR}[1]{\left . #1 \, \right |}
\newcommand{\LBAR}[1]{\left | \, #1 \right .}
\newcommand{\BLBRACKET}[2]{\BRACKET{\RBAR{#1}#2}}
\newcommand{\GEN}[1]{\ABRACKET{#1}}
\newcommand{\BINDEF}[2]{\LBRACKET{\begin{array}{ll}
     #1\\
     #2
\end{array}}}

%%%%%%%
% Fourier Analysis
%%%%%%%

\newcommand{\ONEOTWOPI}{\frac{1}{2\pi}}
\newcommand{\FHAT}{\hat{f}(n)}
\newcommand{\FINT}{\int_{-\pi}^\pi}
\newcommand{\FINTWO}{\int_{0}^{2\pi}}
\newcommand{\FSUMN}[1]{\SUM{n=-#1}{#1}}
%\newcommand{\FSUM}{\SUMN{\infty}}
\newcommand{\EIN}[1]{e^{in#1}}
\newcommand{\NEIN}[1]{e^{-in#1}}
\newcommand{\INTALL}{\int_{-\infty}^{\infty}}
\newcommand{\FTINT}[1]{\INTALL #1 e^{2\pi inx\xi} dx}
\newcommand{\GAUSS}{e^{-\pi x^2}}

%%%%%%%
% formatting 
%%%%%%%

\newcommand{\LEFTBOLD}[1]{\noindent\textbf{#1}}
\newcommand{\SEQ}[1]{\{#1\,\}}
\newcommand{\WIP}{\footnote{work in progress}}
\newcommand{\QED}{\hfill\square}
\newcommand{\ts}{\textsuperscript}
\newcommand{\HLINE}{\noindent\rule{7in}{1pt}\\}

%%%%%%%
% Mathematical note taking (definitions, theorems, etc.)
%%%%%%%

\newcommand{\REM}{\noindent\textbf{\\Remark: }}
\newcommand{\DEF}{\noindent\textbf{\\Definition: }}
\newcommand{\THE}{\noindent\textbf{\\Theorem: }}
\newcommand{\COR}{\noindent\textbf{\\Corollary: }}
\newcommand{\LEM}{\noindent\textbf{\\Lemma: }}
\newcommand{\PROP}{\noindent\textbf{\\Proposition: }}
\newcommand{\PROOF}{\noindent\textbf{\\Proof: }}
\newcommand{\EXP}{\noindent\textbf{\\Example: }}
\newcommand{\TRICKS}{\noindent\textbf{\\Tricks: }}


%%%%%%%
% text highlighting
%%%%%%%

\newcommand{\B}[1]{\textbf{#1}}
\newcommand{\CAL}[1]{\mathcal{#1}}
\newcommand{\UL}[1]{\underline{#1}}

%%%%%%
% Linear Algebra
%%%%%%

\newcommand{\COLVECTOR}[1]{\PAREN{\begin{array}{c}
#1
\end{array} }}
\newcommand{\TWOXTWO}[4]{\PAREN{ \begin{array}{c c} #1&#2 \\ #3 & #4 \end{array} }}
\newcommand{\DTWOXTWO}[4]{\BARS{ \begin{array}{c c} #1&#2 \\ #3 & #4 \end{array} }}
\newcommand{\THREEXTHREE}[9]{\PAREN{ \begin{array}{c c c} #1&#2&#3 \\ #4 & #5 & #6 \\ #7 & #8 & #9 \end{array} }}
\newcommand{\DTHREEXTHREE}[9]{\BARS{ \begin{array}{c c c} #1&#2&#3 \\ #4 & #5 & #6 \\ #7 & #8 & #9 \end{array} }}
\newcommand{\NXN}{\PAREN{ \begin{array}{c c c c} 
			a_{11} & a_{12} & \cdots & a_{1n} \\
			a_{21} & a_{22} & \cdots & a_{2n} \\
			\vdots & \vdots & \ddots & a_{1n} \\
			a_{n1} & a_{n2} & \cdots & a_{nn} \\
		\end{array} }}
\newcommand{\SLR}{SL_2(\R)}
\newcommand{\GLR}{GL_2(\R)}
\DeclareMathOperator{\TR}{tr}
\DeclareMathOperator{\BIL}{Bil}
\DeclareMathOperator{\SPAN}{span}

%%%%%%%
%  White space
%%%%%%%

\newcommand{\BOXIT}[1]{\noindent\fbox{\parbox{\textwidth}{#1}}}


\newtheorem{theorem}{Theorem}[section]
\newtheorem{corollary}{Corollary}[theorem]
\newtheorem{lemma}[theorem]{Lemma}

\theoremstyle{definition}
\newtheorem{definition}[theorem]{Definition}
\newtheorem{prop}[theorem]{Proposition}

\theoremstyle{remark}
\newtheorem{remark}[theorem]{Remark}
\newtheorem{example}[theorem]{Example}
%\newtheorem*{proof}[theorem]{Proof}



\newcommand{\RED}[1]{\textcolor{red}{#1}}
\newcommand{\BLUE}[1]{\textcolor{blue}{#1}}

\begin{document}

\begin{center}
	\Large{\CLASSNAME -- \SEMESTER} \\
	\large{ w/Professor Berchenko-Kogan}
\end{center}
\begin{center}
	\STUDENTNAME \\
	\ASSIGNMENT -- \DUEDATE\\
\end{center} 

\noindent Section 4 problems:

\begin{description}
	\item Within the section: 4.3 (p.37): \textbf{(A basis for 3-covectors).}  Let $x^1, x^2,x^3, x^4$ be the coordinates in $\R^4$ and $p$ a point in $\R^4$.  Write down a basis for the vector space $A_3(T_p(\R^4))$.
	
	\BLUE{\begin{align*}
		\Phi =\{\; &dx_p^i\wedge dx_p^j \wedge dx_p^k\;: i<j<k \le 4\;\}\\ 
		 \{\; &dx_p^1\wedge dx_p^2 \wedge dx_p^3, \\
		   &dx_p^1\wedge dx_p^2 \wedge dx_p^4,\\
		   &dx_p^1\wedge dx_p^3 \wedge dx_p^4,\\
		   &dx_p^2\wedge dx_p^3 \wedge dx_p^4 \;\} \\
		   \BARS{\Phi} = &\binom{4}{3} = 4
	\end{align*}
	}
	
	\item Within the section: 4.4 (p.38), \textbf{Wedge product of a 2-form with a 1-form.}  Let $\omega$ be a 2-form and $\tau$ be a 1-form on on $\R^3$.  If $X,Y,Z$ are vector fields on $M$, find an explicit formula for $(\omega\wedge\tau)(X,Y,Z)$ in terms of the values of $\omega$ and $\tau$ on the vector fields $X,Y,Z$
	
	\BLUE{
	\begin{align*}
		(\omega \wedge \tau)(X,Y,Z) &= (\omega\otimes\tau)(X,Y,Z) - (\tau\otimes\omega)(X,Y,Z) \\
		&= \omega(X)\tau(Y,Z) - \tau(X,Y)\omega(Z)\\
		(\omega \wedge \tau)(X,Y,Z) &= \frac{1}{1!2!}A(\omega \otimes\tau)(X,Y,Z)) \\
		&= \frac{1}{2}\PAREN{ \omega(X,Y)\tau(Z) +
		\omega(Y,Z)\tau(X) +
		\omega(Z,X)\tau(Y) -
		\omega(Z,Y)\tau(X) -
		\omega(Y,X)\tau(Z) -
		\omega(X,Z)\tau(Y) }\\
		&= \omega(X,Y)\tau(Z) +
		\omega(Y,Z)\tau(X) +
		\omega(Z,X)\tau(Y)
	\end{align*}
	}
	
	\item Within the section: 4.9 (p.40) \textbf{A closed 1-form on the punctured plane.}  Define a 1-form on $\omega$ on $\R^2-\{0\}$ by 
	\begin{align*}
		\omega = \frac{1}{x^2+y^2}
(-ydx-xdy).	
\end{align*}Show that $\omega$ is closed.

\BLUE{\begin{align*}
	d\omega = &\PART{\omega}{x}dx+\PART{\omega}{y}dy \\
	= &\PAREN{\frac{-2x}{(x^2+y^2)^2}(-ydx-xdy)+\frac{1}{x^2+y^2}(-yd^2x-dy)}dx+ \\
	&\; \PAREN{\frac{-2y}{(x^2+y^2)^2}(-ydx-xdy)+\frac{1}{x^2+y^2}(-dx-xd^2y)}dy\\
%	= &\frac{2x^2}{(x^2+y^2)^2}dy +\frac{2y^2}{(x^2+y^2)^2}dx 	\\
	= &\PAREN{\frac{-2x}{(x^2+y^2)^2}(-xdydx)+\frac{1}{x^2+y^2}(-dydx)}+ \\
	&\; \PAREN{\frac{-2y}{(x^2+y^2)^2}(-ydxdy)+\frac{1}{x^2+y^2}(-dxdy)}\\
%	= &\frac{2x^2}{(x^2+y^2)^2}dy +\frac{2y^2}{(x^2+y^2)^2}dx \\
	= &\PAREN{\frac{2x^2}{(x^2+y^2)^2}(dydx)+\frac{1}{x^2+y^2}(-dydx)}+ \\
	&\; \PAREN{\frac{2y^2}{(x^2+y^2)^2}(dxdy)+\frac{1}{x^2+y^2}(-dxdy)} \\
	= &\frac{2x^2-x^2-y^2}{(x^2+y^2)^2}(dydx)+ \frac{2y^2-x^2-y^2}{(x^2+y^2)^2}(dxdy) \\
	&= \frac{2x^2-x^2-y^2+2y^2-x^2-y^2}{(x^2+y^2)^2} dxdy\\
	&=0
\end{align*}
}
\end{description}

End of the section: 1 through 6.

\begin{enumerate}[label=4.\arabic*]
\item \textbf{A 1-form on $\R^3$}.

Let $\omega$ be the 1-form $zdx-dz$ and let $X$ be the vector $y\partial/\partial x +  x\partial/\partial y$ on $\R^3$.  Computer $\omega(X)$ and $d(\omega)$.

\BLUE{\begin{align*}
	\omega(X) &= (zdx-dz)\PAREN{y\partial/\partial x +  x\partial/\partial y} \\
	&= (zdx-dz)\PAREN{y\partial/\partial x} + (zdx-dz)\PAREN{ x\partial/\partial y} \\
	&= zy \PART{}{x}dx - y\PART{}{x}dz + zx\PART{}{y}dx -x\PART{}{y}dz &\text{recall } \PART{}{x^i}dx^j = \delta_i^j \\
	&= zy \\ \\
	d(\omega) &= d(zdx-dz) = d(zdx)-d^2z = dz\wedge dx + z\wedge d^2x = dz \wedge dx
\end{align*}
}

\item \textbf{A 2-form on $\R^3$}
\newcommand{\AAA}{\textbf{a}}
\newcommand{\BBB}{\textbf{b}}
At each point $p \in \R^3$, define a bilinear function $\omega_p$ on $T_p(\R^3)$ by
\begin{align*}
	\omega_p(\AAA,\BBB) &= \omega_p\PAREN{\COLVECTOR{a^1\\a^2\\a^3},\COLVECTOR{b^1\\b^2\\b^3}} = p^3 \det\TWOXTWO{a^1}{b^1}{a^2}{b^2},
\end{align*}for tangent vectors $\AAA,\BBB \in T_p(\R^3)$, where $p^3$ is the third component of $p=(p^1,p^2,p^3)$.  Since $\omega_p$ is an alternaing bilinear function on $T_p(\R^3)$, $\omega$ is a 2-form on $\R^3$.  Write $\omega$ in terms of the standard basis $dx^i\wedge dx^j$ at each point.

\BLUE{\begin{align*}
	\omega(p) &= c_{xy}(p)(dx\wedge dy)+c_{yz}(p)(dy\wedge dz)+c_xz(p)(dx\wedge dz) \\
	c_{xy}(p) &= \omega_p(e_x,e_y) = p^3 \TWOXTWO{\PART{}{x}}{0}{0}{\PART{}{y}} = p^3\PAREN{\PART{}{x}\PART{}{y}-0} \\
	c_{yz}(p) &= \omega_p(e_y,e_z) = p^3 \TWOXTWO{0}{\PART{}{y}}{0}{0} = 0  \\
	c_{xz}(p) &= \omega_p(e_x,e_z) = p^3 \TWOXTWO{\PART{}{x}}{0}{0}{0} = 0 
\end{align*}notice that $(dx\wedge dy)(a,b) = dx(a)dy(b)-dy(a)dx(b)= a^1b^2-a^2b^1 = \det\TWOXTWO{a^1}{b^1}{a^2}{b^2}$ Thus.
\begin{align*}
	\omega &= p^3\, dx\wedge dy
\end{align*}
}

\item \textbf{Exterior Calculus}.

Suppose the standard coordinates on $\R^2$ are called $r$ and $\theta$ (this $\R^2$ is the $(r,\theta)$-plane, not the $(x,y)$-plane).  If $x=r\cos\theta$ and $y=r\sin\theta$, calculate $dx,dy,$ and $dx\wedge dy$ in of $dr$ and $d\theta$.

\BLUE{\begin{align*}
	dx &=  \cos \theta dr -r\sin \theta d\theta\\
	dy &=  \sin \theta dr + r\cos \theta d\theta\\
	dx \wedge dy &= 	(\cos \theta dr -r\sin \theta d\theta)\wedge(\sin \theta dr + r\cos \theta d\theta) \\
	&= (\cos \theta dr)\wedge(\sin \theta dr + r\cos \theta d\theta) -(r\sin \theta d\theta)\wedge(\sin \theta dr + r\cos \theta d\theta)\\
	&= (\cos \theta dr)\wedge(\sin \theta dr) +(\cos \theta dr)\wedge( r\cos \theta d\theta) -(r\sin \theta d\theta)\wedge(\sin \theta dr) + (r\sin \theta d\theta)\wedge(r\cos \theta d\theta)\\
	&= 0 +(\cos \theta dr)\wedge( r\cos \theta d\theta) -(r\sin \theta d\theta)\wedge(\sin \theta dr) + 0 \\
	&= (\cos \theta dr)\wedge( r\cos \theta d\theta) +(\sin \theta dr)\wedge(r\sin \theta d\theta) \\
	&= (r\cos^2 \theta) (dr\wedge d\theta) +(r\sin^2 \theta) (dr\wedge d\theta) \\
	&= r(dr\wedge d\theta)
\end{align*}
}

\item \textbf{Exterior Calculus}.

Suppose the standard coordiantes on $\R^3$ are called $\rho, \phi,$ and $\theta$.  If $x=\rho\sin\phi\cos\theta$, $y=\rho\sin\phi\sin\theta$, and $z=\rho\cos\phi$, calculate $dx,dy,dz$, and $dx\wedge dy\wedge dz$ in terms of $d\rho, d\phi$, and $d\theta$.

\BLUE{\begin{align*}
	dx &= \sin\phi \cos \theta \,d\rho + \rho \cos \phi \cos \theta \,d\phi - \rho \sin\phi \sin \theta\,  d\theta \\
	dy &= \sin\phi\sin\theta\,d\rho + \rho\cos\phi\sin\theta\,d\phi + \rho\sin\phi\cos\theta\,d\theta \\
	dz &= \cos\theta\, d\rho-\rho\sin\phi\, d\phi
%	dx \wedge dy \wedge dz &= (\sin\phi \cos \theta \,d\rho + \rho \cos \phi \cos \theta \,d\phi - \rho \sin\phi \sin \theta\,  d\theta)\wedge(\sin\phi\sin\theta\,d\rho + \rho\cos\phi\sin\theta\,d\phi + \rho\sin\phi\cos\theta\,d\theta)\wedge(\cos\theta\, d\rho-\rho\sin\theta\, d\theta 
\end{align*}We will attempt to cancel out any terms which have a $dx^i\wedge dx^i$ by simplifying $dx, dy$, and $dz$ in the following manner
\begin{align*}
	dx \wedge dy \wedge dz &= (x_1\,d\rho + x_2\,d\phi+x_3\,d\theta)\wedge(y_1\,d\rho + y_2\,d\phi+y_3\,d\theta)\wedge(z_1\,d\rho + z_2\,d\phi+z_3\,d\theta) \\ 
	&= (x_1\,d\rho\wedge y_2\,d\phi\wedge z_3\,d\theta)+(x_1\,d\rho\wedge y_3\,d\theta\wedge z_2\,d\phi) \\
	&+ (x_2\,d\phi\wedge y_1\,d\rho \wedge z_3\,d\theta) + (x_2\,d\phi\wedge y_3\,d\theta\wedge z_2\,d\phi) \\
	&+ (x_3\,d\theta\wedge y_1\,d\rho \wedge z_2\,d\phi) + (x_3\,d\theta\wedge y_2\,d\phi\wedge z_1\,d\rho) \\
	&= (x_1y_2z_3+x_1y_3z_2 + x_2y_1z_3 + x_2y_3z_2 + x_3y_1z_2 + x_3y_2z_1)(d\rho\wedge d\phi\wedge d\theta) \\
	&= \DTHREEXTHREE{x_1}{y_1}{z_1}{x_2}{y_2}{z_2}{x_3}{y_3}{z_3}(d\rho\wedge d\phi\wedge d\theta)
\end{align*}Solving for the determinant by expanding the bottom row
\begin{align*}
	\DTHREEXTHREE{\sin\phi \cos \theta}{\rho \cos \phi \cos \theta }{- \rho \sin\phi \sin \theta}{\sin\phi\sin\theta}{ \rho\cos\phi\sin\theta} {\rho\sin\phi\cos\theta}{\cos\phi}{-\rho\sin\phi}{} &= \rho^2\DTHREEXTHREE{\sin\phi \cos \theta}{ \cos \phi \cos \theta }{-  \sin\phi \sin \theta}{\sin\phi\sin\theta}{\cos\phi\sin\theta} {\sin\phi\cos\theta}{\cos\phi}{-\sin\phi}{} \\
	&= \rho^2\sin\phi\DTHREEXTHREE{\sin\phi \cos \theta}{ \cos \phi \cos \theta }{- \sin \theta}{\sin\phi\sin\theta}{\cos\phi\sin\theta} {\cos\theta}{\cos\phi}{-\sin\phi}{}\\
	&= \rho^2\sin\phi \PAREN{\cos\phi\DTWOXTWO{\cos\phi\cos\theta}{-\sin\theta}{\cos\phi\sin\theta}{\cos\theta} +\sin\phi\DTWOXTWO{\sin\phi\cos\theta}{-\sin\theta}{\sin\phi\sin\theta}{\cos\theta} }\\
	&= \rho^2\sin\phi \PAREN{\cos^2\phi\DTWOXTWO{\cos\theta}{-\sin\theta}{\sin\theta}{\cos\theta} +\sin^2\phi\DTWOXTWO{\cos\theta}{-\sin\theta}{\sin\theta}{\cos\theta} }\\
	&= \rho^2\sin\phi \PAREN{\cos^2\phi +\sin^2\phi} \\
	&= \rho^2\sin\phi
\end{align*}That is
\begin{align*}
	dx\wedge dy\wedge dx &= (\rho^2\sin\phi) \,dr\wedge d\phi \wedge d\theta
\end{align*}
}

\item \textbf{Wedge Product}. Let $\alpha$ be a 1-form and $\beta$ a 2-form on $\R^3$.  Then 
\begin{align*}
	\alpha &= a_1dx^1+a_2dx^2+a_3dx^3 \\
	\beta &= b_1 dx^2\wedge dx^3 + b_2 dx^3 \wedge dx^1 + b_3dx^1\wedge dx^2
\end{align*}Simplify the expression $\alpha \wedge \beta$ as much as possible.

\BLUE{The resulting expression $\alpha\wedge\beta\in \Omega^3(\R^3)$.  The $\dim(\Omega^3(\R^3)) = 1$.  Thus, there will be one term of the form $dx^1\wedge dx^2\wedge dx^3$.  Further by distributing the terms of $\alpha$ across the terms of $\beta$ and ignoring any terms where any two elements are equal, i.e., $dx^i\wedge dx^i =0$.  We will then have
\begin{align*}
	\alpha \wedge \beta &= a_1\,dx^1\wedge(b_1\,dx^2\wedge dx^3)+a_2\,dx^2\wedge(b_2\,dx^3\wedge dx^1)+a_3\,dx^3(b_3\,dx^1\wedge dx^2) \\
	&= (a_1b_1+a_2b_2+a_3b_3)\, dx^1\wedge dx^2\wedge dx^3
\end{align*}
}

\item \textbf{Wedge product and cross product}

The correspondence between differential forms and vector fields on an open subset of $\R^3$ in Subsection 4.6 also makes sense pointwise.  let $V$ be a vector space of dimension 3 with basis $e_1, e_2, e_3$,  and dual basis $\alpha^1, \alpha^2, \alpha^3$.  To a 1-covector $\alpha=a_1\alpha^1+a_2\alpha^2+a_3\alpha^3$ on $V$, we associate the vector $v_\alpha = \ABRACKET{a_1,a_2,a_3} \in \R^3$.  To the 2-covector
\begin{align*}
	\gamma = c_1 \alpha^2\wedge \alpha^3 + c_2\alpha^3\wedge\alpha^1 + c_3\alpha^1\wedge\alpha^2
\end{align*}on $V$, we associate the vector $v_\gamma=\ABRACKET{c_1,c_2,c_3}\in \R^3$.  Show that under the correspodence, the wedge product of 1-covectors correponds to the cross product of vectors $\R^3$:  if $\alpha = a_1\alpha^1+a_2\alpha^2+a_3\alpha^3$ and $\beta = b_1\alpha^1+b_2\alpha^2+b_3\alpha^3$, then $v_{\alpha\wedge\beta}= v_\alpha \times v_\beta$.

\BLUE{First the cross product\begin{align*}
	v_\alpha \times v_\beta &= \DTHREEXTHREE{i}{j}{k}{a_1}{a_2}{a_3}{b_1}{b_2}{b_3}\\
	&= i\DTWOXTWO{a_2}{a_3}{b_2}{b_3} - j\DTWOXTWO{a_1}{a_3}{b_1}{b_3} + k\DTWOXTWO{a_1}{a_2}{b_1}{b_2} \\
	&=i(a_2b_3-a_3b_2)+j(a_3b_1-a_1b_3)+k(a_1b_2-a_2b_1)
\end{align*}then the tensor product and ignoring any $a^i\wedge a^i$ terms when we expand them
\begin{align*}
	\alpha \wedge \beta &= (a_1\alpha^1 + a_2\alpha^2 + a_3\alpha^3)\wedge(b_1\alpha^1 + b_2\alpha^2 + b_3\alpha^3) \\
	&= a_1\alpha^1\wedge b_2\alpha^2 + a_1\alpha^1\wedge b_3\alpha^3 + a_2\alpha^2\wedge b_1\alpha^1 + a_2\alpha^2\wedge b_3\alpha^3 + a_3\alpha^3\wedge b_1\alpha^1 + a_3\alpha^3\wedge b_2\alpha^2 \\
	&= \underbrace{(a_1b_2-a_2b_1)}_{k\ITH \text{ of }v_\alpha\times v_\beta}\alpha^1\wedge\alpha^2 + \underbrace{(a_2b_3-a_2b_3)}_{i\ITH \text{ of }v_\alpha\times v_\beta}\alpha^2\wedge\alpha^3 + \underbrace{(a_1b_3-a_3b_1)}_{j\ITH \text{ of }v_\alpha\times v_\beta}\alpha^1\wedge\alpha^3
\end{align*}
}

\end{enumerate}

\end{document}
