\documentclass[12pt,a4paper]{report}
\usepackage[utf8]{inputenc}
\usepackage{amsmath}
\usepackage{amsfonts}
\usepackage{amssymb}
\usepackage{amsthm}
\usepackage{hyperref}

\usepackage{multicol}
\usepackage{fancyhdr}
\usepackage[inline]{enumitem}
\usepackage{tikz}
\usepackage{tikz-cd}
\usetikzlibrary{calc}
\usetikzlibrary{shapes.geometric}
\usetikzlibrary{positioning}
\usepackage[margin=0.5in]{geometry}
\usepackage{xcolor}

\hypersetup{
    colorlinks=true,
    linkcolor=blue,
    filecolor=magenta,      
    urlcolor=cyan,
    pdftitle={Tensors},
    pdfpagemode=FullScreen,
    }

%\urlstyle{same}

\newcommand{\CLASSNAME}{Math XXXX -- Independent Study: Manifolds}
\newcommand{\STUDENTNAME}{Paul Carmody}
\newcommand{\ASSIGNMENT}{\textit{An Introduction to Lie Algebras}}
\newcommand{\DUEDATE}{August, 2025}
\newcommand{\PROFESSOR}{Professor Berchenko-Kogan}
\newcommand{\SEMESTER}{Summer 2025}
\newcommand{\SCHEDULE}{TBD}
\newcommand{\ROOM}{Remote}

\newcommand{\MMN}{M_{m\times n}}
\newcommand{\FF}{\mathcal{F}}

\pagestyle{fancy}
\fancyhf{}
\chead{ \fancyplain{}{\CLASSNAME} }
%\chead{ \fancyplain{}{\STUDENTNAME} }
\rhead{\thepage}
\newcommand{\LET}{\text{Let }}
%\newcommand{\IF}{\text{if }}
\newcommand{\AND}{\text{ and }}
\newcommand{\OR}{\text{ or }}
\newcommand{\FORSOME}{\text{ for some }}
\newcommand{\FORALL}{\text{ for all }}
\newcommand{\WHERE}{\text{ where }}
\newcommand{\WTS}{\text{ WTS }}
\newcommand{\WLOG}{\text{ WLOG }}
\newcommand{\BS}{\backslash}
\newcommand{\DEFINE}[1]{\textbf{\emph{#1}}}
\newcommand{\IF}{$(\Rightarrow)$}
\newcommand{\ONLYIF}{$(\Leftarrow)$}
\newcommand{\ITH}{\textsuperscript{th} }
\newcommand{\FST}{\textsuperscript{st} }
\newcommand{\SND}{\textsuperscript{nd} }
\newcommand{\TRD}{\textsuperscript{rd} }
\newcommand{\INV}{\textsuperscript{-1} }

\newcommand{\XXX}{\mathfrak{X}}
\newcommand{\MMM}{\mathfrak{M}}
%\newcommand{\????}{\textfrak{A}}
%\newcommand{\????}{\textgoth{A}}
%\newcommand{\????}{\textswab{A}}

\DeclareMathOperator{\DER}{Der}
\DeclareMathOperator{\SGN}{sgn}

%%%%%%%
% derivatives
%%%%%%%

\newcommand{\PART}[2]{\frac{\partial #1}{\partial #2}}
\newcommand{\SPART}[2]{\frac{\partial^2 #1}{\partial #2^2}}
\newcommand{\DERIV}[2]{\frac{d #1}{d #2}}
\newcommand{\LAPLACIAN}[1]{\frac{\partial^2 #1}{\partial x^2} + \frac{\partial^2 #1}{\partial y^2}}

%%%%%%%
% sum, product, union, intersections
%%%%%%%

\newcommand{\SUM}[2]{\underset{#1}{\overset{#2}{\sum}}}
\newcommand{\PROD}[2]{\underset{#1}{\overset{#2}{\prod}}}
\newcommand{\UNION}[2]{\underset{#1}{\overset{#2}{\bigcup}}}
\newcommand{\INTERSECT}[2]{\underset{#1}{\overset{#2}{\bigcap}}}
\newcommand{\FSUM}{\SUM{n=-\infty}{\infty}}
       

%%%%%%%
% supremum and infimum
%%%%%%%

\newcommand{\SUP}[1]{\underset{#1}\sup \,}
\newcommand{\INF}[1]{\underset{#1}\inf \,}
\newcommand{\MAX}[1]{\underset{#1}\max \,}
\newcommand{\MIN}[1]{\underset{#1}\min \,}

%%%%%%%
% infinite sums, limits
%%%%%%%

\newcommand{\SUMK}{\SUM{k=1}{\infty}}
\newcommand{\SUMN}{\SUM{n=1}{\infty}}
\newcommand{\SUMKZ}{\SUM{k=0}{\infty}}
\newcommand{\LIM}[1]{\underset{#1}\lim\,}
\newcommand{\IWOB}[1]{\LIM{#1 \to \infty}}
\newcommand{\LIMK}{\IWOB{k}}
\newcommand{\LIMN}{\IWOB{n}}
\newcommand{\LIMX}{\IWOB{x}}
\newcommand{\NIWOB}{\LIM{n \to \infty}}
\newcommand{\LIMSUPK}{\underset{k\to\infty}\limsup \,}
\newcommand{\LIMSUPN}{\underset{n\to\infty}\limsup \,}
\newcommand{\LIMINFK}{\underset{k\to\infty}\liminf \,}
\newcommand{\LIMINFN}{\underset{n\to\infty}\liminf \,}
\newcommand{\ROOTRULE}[1]{\LIMSUPK \BARS{#1}^{1/k}}

\newcommand{\CUPK}{\bigcup_{k=1}^{\infty}}
\newcommand{\CAPK}{\bigcap_{k=1}^{\infty}}
\newcommand{\CUPN}{\bigcup_{n=1}^{\infty}}
\newcommand{\CAPN}{\bigcap_{n=1}^{\infty}}

%%%%%%%
% number systems (real, rational, etc.)
%%%%%%%

\newcommand{\REALS}{\mathbb{R}}
\newcommand{\RATIONALS}{\mathbb{Q}}
\newcommand{\IRRATIONALS}{\REALS \backslash \RATIONALS}
\newcommand{\INTEGERS}{\mathbb{Z}}
\newcommand{\NUMBERS}{\mathbb{N}}
\newcommand{\COMPLEX}{\mathbb{C}}
\newcommand{\DISC}{\mathbb{D}}
\newcommand{\HPLANE}{\mathbb{H}}

\newcommand{\R}{\mathbb{R}}
\newcommand{\Q}{\mathbb{Q}}
\newcommand{\Z}{\mathbb{Z}}
\newcommand{\N}{\mathbb{N}}
\newcommand{\C}{\mathbb{C}}
\newcommand{\T}{\mathbb{T}}
\newcommand{\COUNTABLE}{\aleph_0}
\newcommand{\UNCOUNTABLE}{\aleph_1}


%%%%%%%
% Arithmetic/Algebraic operators
%%%%%%%


\DeclareMathOperator{\MOD}{mod}
%\newcommand{\MOD}[1]{\mod #1}
\newcommand{\BAR}[1]{\overline{#1}}
\newcommand{\LCM}{\text{ lcm}}
\newcommand{\ZMOD}[1]{\Z/#1\Z}
\DeclareMathOperator{\VAR}{Var}
%%%%%%%
% complex operators
%%%%%%%

\DeclareMathOperator{\RR}{Re}
%\newcommand{\RE}{\text{Re}}
\DeclareMathOperator{\IM}{Im}
%\newcommand{\IM}{\text{Im}}
\newcommand{\CONJ}[1]{\overline{#1}}
\DeclareMathOperator{\LOG}{Log}
%\newcommand{\LOG}{\text{ Log }}
\newcommand{\RES}[2]{\underset{#1}{\text{res}} #2}

%%%%%%%
% Group operators
%%%%%%%

\newcommand{\AUT}{\text{Aut}\,}
\newcommand{\KER}{\text{ker}\,}
\newcommand{\END}{\text{End}}
\newcommand{\HOM}{\text{Hom}}
\newcommand{\CYCLE}[1]{(\begin{array}{cccccccccc}
		#1
	\end{array})}
\newcommand{\SUBGROUP}{\underset{\text{group}}\subseteq}	
%\newcommand{\SUBGROUP}{\subseteq_g}
\newcommand{\SUBRING}{\underset{\text{ring}}\subseteq}
\newcommand{\SUBMOD}{\underset{\text{mod}}\subseteq}
\newcommand{\SUBFIELD}{\underset{\text{field}}\subseteq}
\newcommand{\ISO}{\underset{\text{iso}}\longrightarrow}
\newcommand{\HOMO}{\underset{\text{homo}}\longrightarrow}

%%%%%%%
% grouping (parenthesis, absolute value, square, multi-level brackets).
%%%%%%%

\newcommand{\PAREN}[1]{\left (\, #1 \,\right )}
\newcommand{\BRACKET}[1]{\left \{\, #1 \,\right \}}
\newcommand{\SQBRACKET}[1]{\left [\, #1 \,\right ]}
\newcommand{\ABRACKET}[1]{\left \langle\, #1 \,\right \rangle}
\newcommand{\BARS}[1]{\left |\, #1 \,\right |}
\newcommand{\DBARS}[1]{\left \| \, #1 \,\right \|}
\newcommand{\LBRACKET}[1]{\left \{ #1 \right .} 
\newcommand{\RBRACKET}[1]{\left . #1 \right \]}
\newcommand{\RBAR}[1]{\left . #1 \, \right |}
\newcommand{\LBAR}[1]{\left | \, #1 \right .}
\newcommand{\BLBRACKET}[2]{\BRACKET{\RBAR{#1}#2}}
\newcommand{\GEN}[1]{\ABRACKET{#1}}
\newcommand{\BINDEF}[2]{\LBRACKET{\begin{array}{ll}
     #1\\
     #2
\end{array}}}

%%%%%%%
% Fourier Analysis
%%%%%%%

\newcommand{\ONEOTWOPI}{\frac{1}{2\pi}}
\newcommand{\FHAT}{\hat{f}(n)}
\newcommand{\FINT}{\int_{-\pi}^\pi}
\newcommand{\FINTWO}{\int_{0}^{2\pi}}
\newcommand{\FSUMN}[1]{\SUM{n=-#1}{#1}}
%\newcommand{\FSUM}{\SUMN{\infty}}
\newcommand{\EIN}[1]{e^{in#1}}
\newcommand{\NEIN}[1]{e^{-in#1}}
\newcommand{\INTALL}{\int_{-\infty}^{\infty}}
\newcommand{\FTINT}[1]{\INTALL #1 e^{2\pi inx\xi} dx}
\newcommand{\GAUSS}{e^{-\pi x^2}}

%%%%%%%
% formatting 
%%%%%%%

\newcommand{\LEFTBOLD}[1]{\noindent\textbf{#1}}
\newcommand{\SEQ}[1]{\{#1\,\}}
\newcommand{\WIP}{\footnote{work in progress}}
\newcommand{\QED}{\hfill\square}
\newcommand{\ts}{\textsuperscript}
\newcommand{\HLINE}{\noindent\rule{7in}{1pt}\\}

%%%%%%%
% Mathematical note taking (definitions, theorems, etc.)
%%%%%%%

\newcommand{\REM}{\noindent\textbf{\\Remark: }}
\newcommand{\DEF}{\noindent\textbf{\\Definition: }}
\newcommand{\THE}{\noindent\textbf{\\Theorem: }}
\newcommand{\COR}{\noindent\textbf{\\Corollary: }}
\newcommand{\LEM}{\noindent\textbf{\\Lemma: }}
\newcommand{\PROP}{\noindent\textbf{\\Proposition: }}
\newcommand{\PROOF}{\noindent\textbf{\\Proof: }}
\newcommand{\EXP}{\noindent\textbf{\\Example: }}
\newcommand{\TRICKS}{\noindent\textbf{\\Tricks: }}


%%%%%%%
% text highlighting
%%%%%%%

\newcommand{\B}[1]{\textbf{#1}}
\newcommand{\CAL}[1]{\mathcal{#1}}
\newcommand{\UL}[1]{\underline{#1}}

%%%%%%
% Linear Algebra
%%%%%%

\newcommand{\COLVECTOR}[1]{\PAREN{\begin{array}{c}
#1
\end{array} }}
\newcommand{\TWOXTWO}[4]{\PAREN{ \begin{array}{c c} #1&#2 \\ #3 & #4 \end{array} }}
\newcommand{\DTWOXTWO}[4]{\BARS{ \begin{array}{c c} #1&#2 \\ #3 & #4 \end{array} }}
\newcommand{\THREEXTHREE}[9]{\PAREN{ \begin{array}{c c c} #1&#2&#3 \\ #4 & #5 & #6 \\ #7 & #8 & #9 \end{array} }}
\newcommand{\DTHREEXTHREE}[9]{\BARS{ \begin{array}{c c c} #1&#2&#3 \\ #4 & #5 & #6 \\ #7 & #8 & #9 \end{array} }}
\newcommand{\NXN}{\PAREN{ \begin{array}{c c c c} 
			a_{11} & a_{12} & \cdots & a_{1n} \\
			a_{21} & a_{22} & \cdots & a_{2n} \\
			\vdots & \vdots & \ddots & a_{1n} \\
			a_{n1} & a_{n2} & \cdots & a_{nn} \\
		\end{array} }}
\newcommand{\SLR}{SL_2(\R)}
\newcommand{\GLR}{GL_2(\R)}
\DeclareMathOperator{\TR}{tr}
\DeclareMathOperator{\BIL}{Bil}
\DeclareMathOperator{\SPAN}{span}

%%%%%%%
%  White space
%%%%%%%

\newcommand{\BOXIT}[1]{\noindent\fbox{\parbox{\textwidth}{#1}}}


\newtheorem{theorem}{Theorem}[section]
\newtheorem{corollary}{Corollary}[theorem]
\newtheorem{lemma}[theorem]{Lemma}

\theoremstyle{definition}
\newtheorem{definition}[theorem]{Definition}
\newtheorem{prop}[theorem]{Proposition}

\theoremstyle{remark}
\newtheorem{remark}[theorem]{Remark}
\newtheorem{example}[theorem]{Example}
%\newtheorem*{proof}[theorem]{Proof}



\newcommand{\RED}[1]{\textcolor{red}{#1}}
\newcommand{\BLUE}[1]{\textcolor{blue}{#1}}
\newcommand{\GL}{\operatorname{gl}}
\newcommand{\SL}{\operatorname{sl}}
%\newcommand{\TR}{\operatorname{tr}}
\newcommand{\AD}{\operatorname{ad}}
\newcommand{\LB}[2]{\left [ #1,#2 \right ]}
\newcommand{\IMG}{\operatorname{im}}
\newcommand{\IDER}{\operatorname{IDer}}
%\newcommand{\SPAN}{\operatorname{span}}
\newcommand{\RANK}{\operatorname{rank}}

\begin{document}

\begin{center}
	\Large{\CLASSNAME -- \SEMESTER} \\
	\large{ w/\PROFESSOR}
\end{center}
\begin{center}
	\STUDENTNAME \\
	\ASSIGNMENT -- \DUEDATE\\
\end{center} 

\chapter{Introduction}

\begin{definition}[Lie Bracket]

We define the Lie Bracket, $[\cdot, \cdot]$ as a bilinear operation
\begin{align*}
	[\cdot,\cdot] : L &\times L \to L
\end{align*}with the following properties
\begin{align*}
	&[x,x] =0 & (L1)\\
	&[x, [y,z]] + [y, [z,x]] + [z, [x,y]] =0 & (L2)
\end{align*}
\end{definition}

\begin{definition}[Derivation of $A$]

Given an algebra $A$ over a field $F$, a \DEFINE{derivation} $D: A \to A$ is defined by 
\begin{align*}
	D(ab) = aD(b)+D(A)b, \forall a,b, \in A
\end{align*}We denote $\DER A$ as the set of all derivations of $A$.\\ \\

We definte the \DEFINE{inner derivation} of the Lie Algebra $L$, denoted as $\IDER L$ as the set of all $\AD x: L \to L$ which are derivations.
\end{definition}


\section{Exercises}

\begin{enumerate}[label=\textit{1.\arabic*}]

\item (Pg 2.)  \begin{enumerate}
	\item Show that $[v,0] = 0 = [0,v]$ for all $ v \in L$.
	
	\BLUE{\begin{align*}
		[v, v] &= 0 \\
		[v, v] - [v, 0] &= 0 - [v, 0] \\
		[v-v, v-0] &= [0,v] \\
		[0,v] &= [v, 0]
	\end{align*}	but $[0,v] = -[v, 0]$ for all $v$ therefore $[0,v] = 0$.
	}
	
	\item Suppose that $x,y \in L$ satisfy $[x,y] \ne 0$.  Show that $x$ and $y$ are linearly independent on $F$.
	
	\BLUE{Want to show that $ax + by=0$ implies that $a, b = 0$.
	\begin{align*}
		\LET ax + by &= 0 \\
		by &= -ax \implies y = cx, \text{ for some } c\\
		[x,y] &= [x,cx] = c[x,x] = 0
	\end{align*}but $[x,y]\ne 0$ therefore $c = 0$ and $x,y$ are linearly independent.
	}
\end{enumerate}

\item (Pg 2.) Convince yourself that $\wedge$ is bilinear. Then check that the Jacobi Identity holds. \textit{Hint: } if $x \cdot y$ denotes the dot product  of $x,y \in \R^3$, then 
\begin{align*}
	x\wedge(y\wedge z) = (x\cdot{z})y-(x\cdot y)z,\, \forall x,y,z \in \R^3.
\end{align*}

\BLUE{\textbf{$wedge$ is bilinear.}\\
Given $x=(x_1,x_2, x_3)$ and $y=(y_1, y_2, y_3)$ we have 
\begin{align*}
	x \wedge y &= ( x_2y_3-x_3y_2, x_3y_1-x_1y_3, x_1y_2-x_2y_1) \\
	(x+(0,b,0)) \wedge y &= ( (x_2+b)y_3-(x_3+0)y_2, (x_3+0)y_1-(x_1+0)y_3, (x_1+0)y_2-(x_2+b)y_1) \\
	&= ( x_2y_3-x_3y_2, x_3y_1-x_1y_3, x_1y_2-x_2y_1) + ( by_3, 0 , -by_1) \\
	&= x \wedge y + (0,b,0)\wedge y
\end{align*}Therefore additive on the left for the middle coordinate.  Each argument is independent of coordinate so is true for $(a, 0,0)$ and $(0,0,c)$ and can be easily seen when used on the write (e.g., $x \wedge (y+(0,b,0)$).\\
\textbf{The Jacobi Identity}\\
Want to show 
\begin{align}
	x\wedge(y\wedge z)+y\wedge (z \wedge x) + z\wedge(y \wedge x) &= 0
\end{align} from the hint
\newcommand{\ABC}[3]{ (#1 \cdot #3)#2-(#1 \cdot #2)#3}
\begin{align*}
	x\wedge(y\wedge z) &= \ABC{x}{y}{z} 
\end{align*}and from (1)
\begin{align*}
	x\wedge(y\wedge z)+y\wedge (z \wedge x) + z\wedge(y \wedge x) &= \ABC{x}{y}{z} \\&+\ABC{y}{z}{x} \\&+ \ABC{z}{x}{y} \\
	&= \PAREN{(x\cdot z)-(z\cdot x)}y\\&+\PAREN{-(x\cdot y)+(y\cdot x)}z \\&+ \PAREN{-(y\cdot z)+(z\cdot y)}x  \\
	&= 0
\end{align*}
}

\item (Pg 2.) Supoose that $V$ is a finite-dimensional vector space over $F$.  Write $\GL(V)$ for the set of all linear maps from $V$ to $V$.  This is again a vector space over $F$, and it becomes a Lie algebra, known as the \textit{general linear algebra}, if we define the Lie bracket $[--,--]$ by
\begin{align*}
	[x,y] := x \circ y -y \circ x,\, \forall x,y \in \GL(V),
\end{align*}where $\circ$ deontes the composition of maps. Check that the Jacobi Identity holds.

\newcommand{\GLLB}[2]{(#1\circ #2-#2\circ #1)}

\BLUE{Given $R,S,T \in \GL(V)$ there exists matrix $A,B,C \in \mathcal{M}_{n\times n}(F)$ where $n = \dim V$ and $Rx = Ax,\, Sx = Bx, \, Tx = Cx, \, \forall x \in V$.  Further remember that $R \circ S = AB$ (similar for the other two transormations) for all $x \in v$.  Then
\begin{align*}
	\LB{R}{\LB{S}{T}} + \LB{S}{\LB{T}{R}}+\LB{T}{\LB{R}{S}} &= \GLLB{R}{\GLLB{S}{T}} \\&+ \GLLB{S}{\GLLB{T}{R}}\\&+\GLLB{T}{\GLLB{R}{S}} \\
	&= (A(BC-CB)-(BC-CB)A)\\&+(B(CA-AC)-(CA-AC)B)\\&+(C(AB-BC)-(AB-BA)C)
\end{align*}by rearranging the terms we can see that they all cancel out.  Most notably this is done \textit{without commuting}.  It is important to remember that, in general, $R\circ S \ne S \circ R$.
}

\item Let $b(n,F)$ be the upper triangular matrices in $\GL(n, F)$.  (A matrix $x$ is said to be upper triangular if $x_{ij}=0$ whnever $i> j$.)  This is a Lie algebra with the same Lie bracket as $\GL(n,F)$.\\

Similarly, let $n(n,F)$ be the strictly upper triangular matrices in $\GL(n,F)$.  (A matrix $x$ i said to be strictly upper triangular if $x_{ij}=0$ whenever $i \ge j$.)  Again this is a Lie algebra with the same Lie bracket as $\GL(n,F)$.\\

Verify these assertions.

\BLUE{Let $b(n,F) = \{A \in \GL(n,F)\,|\, A=[x_{ij}],\,i>j \to x_{ij}=0\}$. Define \begin{align*}
	[x,y] := x \circ y -y \circ x,\, \forall x,y \in b(n,F),
\end{align*}\\
The only question that needs to be answered is ... Given $S, T \in \b(n,F)$ is $S\circ T \in b(n,F)$.  Let $A,B \in \mathcal{M}_{n\times n}(F)$ and $T(x) = Ax, S(x)=Bx, \forall x \in F$.  Then $(T\circ S)(x) = ABx$.  Is $AB \in b(n,F)$.
\begin{align*}
	\LET A &= [a_{ij}]\AND B= [b_ij]\\
	AB &= \SQBRACKET{x_{ij} = \sum_{k=1}^n  a_{ik}b_{k_j}}
\end{align*}If $i>j$ then $x_{ij}$
}

\item (Pg 4) Find $Z(L)$ when $L=\SL(2,F)$.  You should find the answer depends on the characteristic of $F$.

\RED{Let $\SL(n,F)$ be the subspace of $GL(n,F)$ consisting of all matrices whose trace is zero, i.e., $\SL(n,F) = \BRACKET{A \in \mathcal{M}_{n\times n}(F)\,\LBAR{ \SUM{i=1}{n} a_{ii} = 0}}$.  This is known as \textit{Special Linear Algebra} on square matrices.\\
\textbf{When is $\SUM{i=1}{n} a_{ii} = 0$ for all $a_{ii} \in F$? OR $a_{11}+a_{22}=0$?}. \\
Notice, for example, that on the discrete field $
F=\Z/\Z5$, $2+3=0$. Thus, when $L=\SL(2,\Z/\Z p)$ where $p$ is prime, $Z(L)$ will have elements where $a_{11}+a_{22} = p$.  
}

\item (Pg 5.) Show that if $\varphi : L_1\to L_2$ is a homormorphism, then the kernel of $\varphi, \ker \varphi$, is an ideal of $L_1$, and the image of $\varphi, \IMG \varphi$, is a Lie subalgebra of $L_2$.

\BLUE{\textbf{Show that the kernel is an ideal.}  Let $h,k \in \ker\varphi$ such that $h\ne k$.  Then $\varphi(k) =\varphi(h) = 0$.  \begin{align*}
	\varphi(a-b) &= \varphi(a)-\varphi(b) = 0\\
	\therefore a-b &\in \ker \varphi
\end{align*}which makes it a group under addition.  Now we need to show that it is closed under multiplication, that is, $ra \in \ker \varphi$ for all $r \in L$.  Let $r\in L$ then 
\begin{align*}
	\varphi(ra) &= \varphi(r)\varphi(a) = 0\\
	\therefore ra &\in \ker \varphi
\end{align*}\\
\textbf{Show that the image is a subalgebra.}  We need to show three things:
\begin{description}
	\item \textbf{Closed under addition (group condition).}\\
	Let $u,v \in \IMG \varphi$ then there exists $x,y\in L_1$ such that $\varphi(x)=u, \varphi(y)=v$.  \\Then $\varphi(x+y)= \varphi(x)+\varphi(y)=u+v \in \IMG \varphi$.\\
	Therefore closed under addition.
	\item \textbf{closed under scalar multipication (ring condition).}\\
	Let $r,a \in \IMG \varphi$.  Then there exists $x,y \in L_1$ such that $\varphi(x)=r, \varphi(y)=x$.\\
	Then $\varphi(xy)=\varphi(x)\varphi(y)=ra \in \IMG \varphi$\\
	Therefore closed under scalar multiplication.
	\item \textbf{closed under Lie bracket (subalgebra condition).}\\
	Let $u,v \in \IMG \varphi$ then there exists $x,y\in L_1$ such that $\varphi(x)=u, \varphi(y)=v$.  \\Then 
	\begin{align*}
		\varphi(\LB{x+y}{x+y}) &= \varphi\PAREN{ \LB{x}{x} + \LB{x}{y} + \LB{y}{x}+\LB{y}{y} }\\
		&= \varphi\PAREN{ \LB{x}{y} + \LB{y}{x} } \\
		&= \varphi\PAREN{ \LB{x}{y} } + \varphi\PAREN{\LB{y}{x} } \\
		\varphi\PAREN{ \LB{x}{y} } &= - \varphi\PAREN{\LB{y}{x} } \\
		[ \varphi(x+y), \varphi(x+y) ] &= [\varphi(x)+\varphi(y), \varphi(x)+\varphi(y)] \\
		&= [u+v, u+v]\\
		&= \LB{u}{u} + \LB{u}{v} + \LB{v}{u}+\LB{v}{v} \\
		&= \LB{u}{v} + \LB{v}{u} \\
		\LB{u}{v} &= - \LB{v}{u}
	\end{align*}therefore closed under Lie Bracket.
\end{description}
}

\item (Pg 6.) Let $L$ be a Lie algebra.  Show that the Lie bracket is associative, this is $\LB{x}{\LB{y}{z}} = \LB{\LB{x}{y}}{z}$ for all $x,y,z \in L$, if and only if for all $a,b \in L$ the commutator $[a,b]$ lies in $Z(L)$.

\item (Pg 6) Let $D$ and $E$ be derivations on algebra $A$.
\begin{enumerate}[label=(\roman*)]
	\item Show that $\LB{D}{E}=D \circ E-E\circ D$ is also a derivation.
	
	\BLUE{\begin{align*}
		(D \circ E)(ab) &= D\PAREN{aE(b)-E(a)b} \\
		&=D(aE(b))-D(E(a)b)\\
		&= aD(E(b))-D(a)E(b)-E(a)D(b)+D(E(a))b \\
		&= aD(E(b))+D(E(a))b-D(a)E(b)-E(a)D(b) 
	\end{align*}We can switch $D$ and $E$ to computer $E \circ D$
	\begin{align*}
		(E\circ D)(ab) &= aE(D(b))+E(D(a))b-E(a)D(b)-D(a)E(b) 
	\end{align*}taking the difference
	\begin{align*}
		(D \circ E)(ab)-(E\circ D)(ab) &= aD(E(b))+D(E(a))b-D(a)E(b)-E(a)D(b) \\&-\PAREN{aE(D(b))+E(D(a))b-E(a)D(b)-D(a)E(b)}
	\end{align*}
	\begin{align*}
		\LB{D}{E}(ab) &=a\LB{D}{E}(b)-\LB{D}{E}(a)b \\
		&= a(D \circ E)(b)-((D\circ E)(a))b -\PAREN{a(E\circ D)(b)-(E\circ D)(a)b}\\
		\LB{D}{E}(ab) &= (D \circ E)(ab)-(E\circ D)(ab) \\
		&= D(E(ab))-E(D(ab))\\
		&= D\PAREN{aE(b)-E(a)b}-E\PAREN{aD(b)-D(a)b} \\
		&= D(aE(b))-D(E(a)b)-E(aD(b))+E(D(a)b) \\
		&= aD(E(b))-E(b)D(a)\\&-E(a)D(b)+D(E(a))b\\&-aE(D(b))+E(a)D(b)\\&+D(a)E(b)-E(D(a))b\\
		&= a(D(E(b))-E(D(b))-(E(b))D(a)
	\end{align*}
	}
	
	\item Show that $D\circ E$ need not be a derivation.  (see example).

\end{enumerate}

\item (Pg 7.)  Let $L_1$ and $L_2$ be Lie algebras.  Show that $L_1$ is isomorphic to $L_2$ if and only if there is a basis $B_1$ of $L_1$ and a basis $B_2$ of $L_2$ such that the structure constants of $L_1$ with respect to $B_1$ are equal to the structure constants of $L_2$ with respect to $B_2$.

\BLUE{\IF Assuming that $L_1 \ISO L_2$.  Define $f:L_1 \to L_2$ to be that isomorphism. Let $B_1=(x_1,\dots,x_n)$ be the basis vectors for $L_1$.  Then, 
\begin{align*}
	f(\LB{x_i}{x_j}) 
	&= f\PAREN{\sum_{k=1}^na_{ij}^k x_k} \\
	&= \sum_{k=1}^na_{ij}^k f(x_k) & (1.6)
\end{align*}since $f$ is isomorphic, it is also injective and surjective.  Thus, each $f(x_k)$ is unique.  Further, given any $i,j \in [1,\dots,n]$ we know that $x_i,x_j$ are linearly independent.  Thus, 
\begin{align*}
	0&= Ax_i+Bx_j \implies A=B=0 \AND \\
	f(0) = 0 &= f(Ax_i+Bx_j)= Af(x_i)+Bf(x_j)
\end{align*}therefore, $f(x_i), f(x_j)$ are linearly independent and thus, form a basis.  From (1.6) we see that it has the same Structure Constants.
}

\item (Pg 7.)  Let $L$ be a Lie algebra with basis $(x_1, \dots, x_n)$.  What condition does the Jacobi identity impose on the structure constants $a_{ij}^k$?

\BLUE{We have three brackets for the Jacobi Identity that start with 
\begin{align*}
	\LB{x_i}{x_j} &= \sum_{k=1}^n a_{ij}^k x_k\\
	\LB{x_e}{x_f} &= \sum_{k=1}^n a_{ef}^k x_k\\
	\LB{x_b}{x_c} &= \sum_{k=1}^n a_{bc}^k x_k\\
	\LB{x_i}{\LB{x_e}{x_f}} &= \LB{x_i}{\sum_{k=1}^n a_{ef}^k x_k}\\
	&= \sum_{k=1}^n a_{ef}^k\LB{x_i}{ x_k}\\
	&= \sum_{k=1}^n a_{ef}^k\sum_{l=1}^n a_{ik}^l x_l
\end{align*}Since, the $x_i$ are linearly independeent we can examining each element $l$ independently that is 
\begin{align*}
	\LB{x_i}{\LB{x_e}{x_f}}_l &= \sum_{k=1}^n a_{ef}^ka_{ik}^l x_l
\end{align*}cycling through the other terms of the Jacobi identity we get
\begin{align*}
	\LB{x_e}{\LB{x_f}{x_i}}_l &= \sum_{k=1}^n a_{fi}^ka_{ek}^l x_l\\
	\LB{x_f}{\LB{x_i}{x_e}}_l &= \sum_{k=1}^n a_{ei}^ka_{fk}^l x_l
\end{align*}The Jacobi Identity means that the sum of the coefficiints of these terms must be zero that is
\begin{align*}
	0 &= \sum_{k=1}^n a_{ef}^ka_{ik}^l + \sum_{k=1}^n a_{fi}^ka_{ek}^lg+\sum_{k=1}^n a_{ei}^ka_{fk}^l
\end{align*}
}

\item (Pg 8.) Let $L_1$ and $L_2$ be two abelian Lie algebras.  Show that $L_1$ and $L_2$ are isomorphic if and only if they have the same dimension.

\BLUE{If $L_1$ and $L_2$ are abelian then since $[x,y] = -[y,x]$ then $[x,y]=0$ for all $x,y, \in L_1$ or $L_2$.  Consequently, these are vector spaces that are isomorphic to the each other and, hence, have the same dimension.
}

\item Find the structure constants of $\SL(2,F)$ with respect to the basis given by the matrices
\begin{align*}
	e=\TWOXTWO{0}{1}{0}{0},f=\TWOXTWO{0}{0}{1}{0}, h=\TWOXTWO{1}{0}{0}{-1}.
\end{align*}

\BLUE{The Lie Bracket for $\SL(2,F)$ is $\LB{X}{Y} = XY-YX$.  Thus,
\begin{align*}
	[e,f] &= \TWOXTWO{0}{1}{0}{0}\TWOXTWO{0}{0}{1}{0} - \TWOXTWO{0}{0}{1}{0} \TWOXTWO{0}{1}{0}{0}\\
	&= \TWOXTWO{1}{0}{0}{0}-\TWOXTWO{0}{0}{0}{1} \\
	&= \TWOXTWO{1}{0}{0}{-1} \\
	&= h\\
	[f,h] &= \TWOXTWO{0}{0}{1}{0}\TWOXTWO{1}{0}{0}{-1} - \TWOXTWO{1}{0}{0}{-1}\TWOXTWO{0}{0}{1}{0} \\
	&= \TWOXTWO{0}{0}{1}{0}-\TWOXTWO{0}{0}{-1}{0}\\
	&= \TWOXTWO{0}{0}{2}{0} \\
	&= 2f \\
	[h,e] &= \TWOXTWO{0}{1}{0}{0}\TWOXTWO{1}{0}{0}{-1}-\TWOXTWO{1}{0}{0}{-1} \TWOXTWO{0}{1}{0}{0}\\
	&= \TWOXTWO{0}{-1}{0}{0}-\TWOXTWO{0}{1}{0}{0}\\
	&= \TWOXTWO{0}{-2}{0}{0} \\
	&= -2e
\end{align*}Thus, 
\begin{align*}
	a_{ii}^k &= 0, \forall k=1,2,3\\
	[e,f] &= a_{12}^1e + a_{12}^2f+a_{12}^3h = h \to a_{12}^3=1 \\
	[f,h] &= a_{23}^1e + a_{23}^2f+a_{23}^3h = 2f \to a_{23}^2 = 2\\
	[h,e] &= a_{31}^1e + a_{31}^2f+a_{31}^3h = -2e \to  a_{31}^1=-2
\end{align*}all else are zero.
}

\item Prove $\SL(2,\C)$ has no non-trivial ideals.
\newcommand{\ASYM}[3]{\THREEXTHREE{0}{#1}{#2}{-#1}{0}{#3}{-#2}{-#3}{0}}

\item Let $L$ by the 3-dimensional \textit{complex} Lie algebra with basis $(x,y,z)$ and Lie bracket defined by 
\begin{align*}
	\LB{x}{y} =z,\, \LB{y}{z} = x,\, \LB{z}{x}=y
\end{align*}(Here $L$ is the``complexification" of the 3-dimensional real Lie algebra $\R_\wedge^3$.)
\begin{enumerate}[label=(\roman*)]
	\item Show that $L$ is isomorphic to the Lie subalgebra of $\GL(3,\C)$ consistent for all $3 \times 3$ antisymmetric matrices with entries in $\C$.
	
	\BLUE{Let $U = \{A = \GL(3,N): A$ is an anti-symmetric matrix $\}$.  Thus for any $A \in U$ there exists $a,b,c \in \C$ such that
	\begin{align*}
		X &= \ASYM{a}{b}{c} 
\end{align*}	which have three linearly independent elements
\begin{align*}
	x &= \THREEXTHREE{0}{1}{0}{-1}{0}{0}{0}{0}{0} \\
	y &= \THREEXTHREE{0}{0}{1}{0}{0}{0}{-1}{0}{0} \\
	z &= \THREEXTHREE{0}{0}{0}{0}{0}{-1}{0}{1}{0} \\
\end{align*}Verify
\begin{align*}
		[x,y] &= xy-yx\\
		&= \THREEXTHREE{0}{1}{0}{-1}{0}{0}{0}{0}{0}\THREEXTHREE{0}{0}{1}{0}{0}{0}{-1}{0}{0}-\THREEXTHREE{0}{0}{1}{0}{0}{0}{-1}{0}{0}\THREEXTHREE{0}{1}{0}{-1}{0}{0}{0}{0}{0} \\
		&= \THREEXTHREE{0}{0}{0}{0}{0}{-1}{0}{0}{0}-\THREEXTHREE{0}{0}{0}{0}{0}{0}{0}{-1}{0} \\
		&= \THREEXTHREE{0}{0}{0}{0}{0}{-1}{0}{1}{0}\\
		&= z\\
\end{align*}
\begin{align*} 
		[y,z] &= yz-zy \\
		&= \THREEXTHREE{0}{0}{1}{0}{0}{0}{-1}{0}{0} \THREEXTHREE{0}{0}{0}{0}{0}{-1}{0}{1}{0}- \THREEXTHREE{0}{0}{0}{0}{0}{-1}{0}{1}{0}\THREEXTHREE{0}{0}{1}{0}{0}{0}{-1}{0}{0} \\
	&= \THREEXTHREE{0}{1}{0}{0}{0}{0}{0}{0}{0}-\THREEXTHREE{0}{0}{0}{1}{0}{0}{0}{0}{0}\\
	&= \THREEXTHREE{0}{1}{0}{-1}{0}{0}{0}{0}{0}\\
	&= x
\end{align*}
\begin{align*} 
	[z,x] &= \THREEXTHREE{0}{0}{0}{0}{0}{-1}{0}{1}{0}\THREEXTHREE{0}{1}{0}{-1}{0}{0}{0}{0}{0}-\THREEXTHREE{0}{1}{0}{-1}{0}{0}{0}{0}{0}\THREEXTHREE{0}{0}{0}{0}{0}{-1}{0}{1}{0}\\
	&= \THREEXTHREE{0}{0}{0}{0}{0}{0}{-1}{0}{0}-\THREEXTHREE{0}{0}{-1}{0}{0}{0}{0}{0}{0}\\
	&= \THREEXTHREE{0}{0}{1}{0}{0}{0}{-1}{0}{0}\\
	&= y
\end{align*}
	}
	
	\item Find an explicit isomorphism $\SL(2,\C) \ISO L$.
\end{enumerate}

\item Let $S$ be an $n \times n$ matrix with entries in a field $F$.  Define
\begin{align*}
	\GL_S(n,F) = \{x \in \GL(n,F)\,:\, x^tS=-Sx\}.
\end{align*}
\begin{enumerate}[label=(\roman*)]
	\item Show that $\GL_S(n,F)$ is a Lie subalgebra of $\mathfrak{gl}(n,F)$.
	
	\BLUE{\begin{description}
		\item Additive Group\\
			Let $x,y \in \GL_S(n,F)$, then
			\begin{align*}
				(x+y)^tS &= x^tS+y^tS = -Sx-Sy = -S(x+y)
			\end{align*}
		\item Multicative property. \\
		Let $x \in \GL_S(n,F)$ then $x^tS=-Sx$ and $rx^tS=-Sxr$ for all $r, \in F$
		\item Lie Bracket\\
		Let $x,y \in \GL_S(n,F)$ then
		\begin{align*}
			[x,y] &= xy-yx \\
			[x,y]^tS &= (xy-yx)^tS \\
			&= (xy)^tS-(yx^t)S \\
			&= y^tx^tS-x^ty^tS \\
			&= -y^tSx+x^tSy\\
			&= Syx-Sxy \\
			&= S(yx-xy)\\
			&= -S[x,y]
		\end{align*}
	\end{description}
	}
	
	\item Find $\GL_S(2,\R)$ if $S = \TWOXTWO{0}{1}{0}{0}$.
	
	\BLUE{Let $x \in \GL_S(2,\R)$ and
	\begin{align*}
		x &= \TWOXTWO{a}{b}{c}{d}\\
		x^tS &= \TWOXTWO{a}{c}{b}{d}\TWOXTWO{0}{1}{0}{0} = \TWOXTWO{0}{a}{0}{b} \\
		Sx &= \TWOXTWO{0}{1}{0}{0}\TWOXTWO{a}{c}{b}{d} = \TWOXTWO{b}{d}{0}{0} \\
		0 = x^tS+Sx &= \TWOXTWO{0}{a}{0}{b}+\TWOXTWO{b}{d}{0}{0}=\TWOXTWO{b}{a+d}{0}{b} \\
		x &= \TWOXTWO{a}{0}{c}{-a} \\
		&= a\TWOXTWO{1}{0}{0}{-1} + c\TWOXTWO{0}{0}{1}{0}
	\end{align*}
	}
	
	\item Does there exist a matrix $S$ such that $\GL_S(2,\R)$ is equal to the set of all diagonal matrices in $\GL(2,\R)$.
	
	\BLUE{Let $A \in \GL(2,\R)$ be a diagonal matrix.  
	\begin{align*}
		\LET A &= \TWOXTWO{a}{0}{0}{b} \\
		\LET S &= \TWOXTWO{u}{v}{w}{z} \\
		A^tS + SA &= AS+SA \to AS = -SA\\
		au &= -ua \AND bz = -zb
	\end{align*}No, no such $S$ exists.
	}
	
	\item Find a matrix $S$ such that $\GL_S(3,\R)$ is isomorphic to the Lie algebra $\R_\wedge^3$ defined in $\S1.2$, Example 1.
	
	\textit{Hint:} Part (i) of Exercise 1.14 is relevant.
	
	\BLUE{Let $x,y,z$ be a basis of $\R^3$.  We want to find $\phi: \R^3 \to \R_\wedge^3$.
	}
	
	\BLUE{Let $X,Y \in \GL_S(3,\R)$ and $\phi: \GL_S(3,\R) \to \R_\wedge^3$ such that 
	\begin{align*}
		\phi([X,Y]) &= [\phi(X),\phi(Y)] = \phi(X) \wedge \phi(Y) \\
		\phi(XY-YX) &= \phi(X) \wedge \phi(Y)
	\end{align*} Notice that 
	\begin{align*}
		(XY)^tS &= Y^tX^tS= -Y^tSX=SYX \\
		\AND [X,Y]^tS &= (XY - YS)^tS \\
		&= (XY)^tS-(YX)^tS \\
		&= SYX-SXY\\
		&=S(YX-XY) \\
		&= -S[X,Y]
	\end{align*}
	\begin{align*}
		\phi(X^tS) &= \phi(-SX) = -\phi(S)\phi(X)
	\end{align*}
	}
\end{enumerate}

\item Show, by giving an example, that if $F$ is a field of characteristic 2, there are algebras over $F$ which statisfy (L1') and (L2) but are not Lie algebras.

\item Let $V$ be an $n$-dimensional complex vector space and let $L=\GL(V)$.  Suppose that $x \in L$ is diagonalisable, with eigenvalues $\lambda_1, \dots, \lambda_n$.  Show that $\operatorname{ad} x \in \GL(L)$ is also diagonalisable and that its eigenvalues are $\lambda_i-\lambda_j$ for $1\le i,j\le n$.

\item Let $L$ be a Lie algebra.  We saw in $\S1.6$, Example 1.2(2) that the maps $\operatorname{ad} x: L \to L$ for $x \in L$ are derivations of $L$; these are known as \textit{inner derivations}.  Show that if $\operatorname{IDER} L$ is the set of inner derivations of $L$, then $\operatorname{IDER} L$ is an ideal of $\operatorname{DER} L$.

\item Let $A$ be an algebra and let $\delta : A \to A$ be a derivation.  Prove that $\delta$ satisfies the Leibniz rule
\begin{align*}
	\delta^n(xy) = \sum_{r=0}^n\binom{n}{r}\delta^r(x)\delta^{n-r}(y),\, \forall x,y \in A.
\end{align*}

\BLUE{This resembles the binomial theorem
\begin{align*}
	(a+b)^n &= \sum_{r=0}^n \binom{n}{r}a^rb^{n-r}
\end{align*}Consider an inductive proof and consider $\delta^0(x) = x$\\
\textbf{Show true for $n=1$.}
\begin{align*}
	\delta(xy) &= \binom{1}{0} \delta^0(x)\delta(y)+\binom{1}{1}\delta(x)\delta^0(y) \\
	&= x\delta(y)+\delta(x)y
\end{align*}which is the Liebniz rule.  \\
\textbf{Show true for $n+1$}.  Now, assuming that this is true for some number $n$, we must show that it is also true for $n+1$.  Thus, starting with $n$ we'll calculate $\delta(\delta^n(xy)) = \delta^{n+1}(xy)$.
\begin{align*}
	\delta^n(xy) &= \sum_{r=0}^n\binom{n}{r}\delta^r(x)\delta^{n-r}(y),\, \forall x,y \in A. \\
	\delta(\delta^n(xy)) &= \delta\PAREN{\sum_{r=0}^n\binom{n}{r}\delta^r(x)\delta^{n-r}(y)} \\
	&= \sum_{r=0}^n\binom{n}{r}\delta\PAREN{\delta^r(x)\delta^{n-r}(y) } &(*)
\end{align*}Let us focus on the term in the summation
\begin{align*}
	\delta\PAREN{\delta^r(x)\delta^{n-r}(y) } &= \delta^r(x)\delta(\delta^{n-r}(y))+\delta(\delta^r(x))\delta^{n-r}(y) \\
	&= \delta^r(x)\delta^{n-r+1}(y) + \delta^{r+1}(x)\delta^{n-r}(y).
\end{align*}Thus, 
\begin{align*}
	\sum_{r=0}^n\binom{n}{r}\delta\PAREN{\delta^r(x)\delta^{n-r}(y) } &= \sum_{r=0}^n\binom{n}{r}\PAREN{\delta^r(x)\delta^{n-r+1}(y) + \delta^{r+1}(x)\delta^{n-r}(y) } \\
	&= \sum_{r=0}^n\PAREN{\binom{n}{r}+\binom{n}{r-1}}\delta^r(x)\delta^{n-r+1}(y) 
\end{align*}
when $r=0$ we have
\begin{align*}
	r=0 & \to x\delta^{n+1}(y) + \delta(x)\delta^{n}(y) \\
	r=n & \to \delta^n(x)\delta(y) + \delta^{n+1}(x)y
\end{align*}From combinatorics we have the identity
\begin{align*}
	\binom{n+1}{k} = \binom{n}{k}+\binom{n}{k-1}
\end{align*}and we have 
\begin{align*}
	\delta^{n+1}(xy) &= x\delta^{n+1}(y) + \delta(x)\delta^{n}(y)\\&+ \sum_{r=0}^n\binom{n+1}{r}\delta^r(x)\delta^{n-r+1}(y) \\&+\delta^n(x)\delta(y) + \delta^{n+1}(x)y \\
	&= \sum_{r=0}^{n+1}\binom{n+1}{r}\delta^r(x)\delta^{n-r+1}(y)
\end{align*}Thus, by Mathematical Induction, our assertion is true \qed
}

\end{enumerate}

\chapter{Ideals and Homomorphisms}

\textbf{Operations that work on Ideals}

\begin{description}
\item Addition: $I+J=\{x+y\,:\, x\in I, y\in J\}$ is an ideal.
\item Lie Bracket: $[I,J]=\SPAN\{[x,y]\,|\,x\in I, y\in J\}$ is an ideal.
\item Quotient: $L/I = \{z+I\,:\, z\in L\}$ is a quotient algebra.
\end{description}

\textbf{Notes:}

\begin{description}
	\item Correspondence: $L \supset J \supset I$, where $I,J$ are ideals of $L$.  Then, $J/I$ is an ideal of $L/I$.  
	
	Also, if $K$ is an ideal of $L/I$ and $J=\{z \in L : z+I\in K\}$ (i.e., $J$ is the set of cosets of $K$ in $I$) then $J$ is an ideal of $L$ and $J \supset I$.
\end{description}

\section{Exercises}

\begin{enumerate}[label=2.\arabic*]

\item (Pg. 11) Show that $I+J$ is an ideal of $L$ where
\begin{align*}
	I+J &=\{x+y\,:\, x \in I, y\in J\}.
\end{align*}

\BLUE{Let $z \in L$ and $x, y \in I+J$ then there exists $x_I, y_I \in I$ and $x_J, y_J \in J$ such that $x=x_I+x_J$ and $y=y_I+y_J$  then from (L2) we have
\begin{align*}
	[\underbrace{[y,x]}_{\in I+J}, z] &= \underbrace{[x, [y,z]]}_{\in I} + \underbrace{[y,[z,x]]}_{\in J}  \in I+J
\end{align*}
}

\item (Pg. 12) Show that $\SL(2, \C)' = \SL(2, \C)$.

\BLUE{Let $L = \SL(2, \C)$ and $X \in [L, L]$.  Then, there exist $A,B \in L$ such that $[A,B] = X$ thus
\begin{align*}
	X = [A,B] &= AB - BA
\end{align*}$AB \in L$ and $BA \in L$ therefore $X \in L$.
}

\item (Pg. 13) \begin{enumerate}[label=(\roman*)]
	\item Show that the Lie Bracket defined in $L/I$ is bilinear and satisfies the axioms (L1) and (L2).
	
	\BLUE{Define the Lie Bracket of two cosets as \begin{align*}
		[w+I, z+I] &= [w,z]+I,, \forall w,z \in L
	\end{align*}where the bracket on the right side is the Lie Bracket defined for $L$.  Thus, let $a,b \in L$ then we have 
	\begin{align*}
		[a+w+I, b+z+I] &= [a+w,b+z]+I \\
		&= [a,b]+[a,z]+[w,b]+[w,z]+I
	\end{align*}the four Lie Brackets add up to a single element in $L$ and is therefore true.  Thus, this Lie Bracket is bilinear.
	}	
	
	\item Show that the linear transformation $\pi : L \to L/I$ which takes an element $z \in L$ to its coset $z+I$ is a homomorphism of a Lie Algebras.
	
	\BLUE{Need to show that 
	\begin{align*}
		\pi([x,y]) &= [\pi(x),\pi(y)]
	\end{align*}	I prefer to call elements of $L/I$ equivalence classes.  That is $L/I$ is partitioned into equivalence classes (cosets) and its elements are these subsets.  The proper notation for sucn and element would be $[x] \in L/I$ where $x$ is a representative element of the equivalence class containing $x$.  Thus $\pi(x) = [x] = \{x+I\}$.
	\begin{align*}
		\pi(x) &= [x] = \{x+I\}\\
		[\pi(x),\pi(y)] &= \SQBRACKET{[x],[y]} \\
		&= \SQBRACKET{\BRACKET{x+I}, \BRACKET{y+I}} \\
		&= [x,y] +I \\
		&= \SQBRACKET{[x,y]}
\end{align*}	or the equivalence class of the Lie Bracket of the left hand side.
	}

\end{enumerate}

\item (Pg. 14) Show that if $L$ is a Lie Algebra then $L/Z(L)$ is isomorphic to a subalgebra of $\GL(L)$.

\BLUE{$Z(L) = \{x \in L: [x,y]=0$ for all $y\in L\}$.  Therefore, $[x]\in L/Z(L) = \{y\in L : y=x+z, z \in Z(L) \}$.  $Z(L)$ is an ideal.  Thus, $[x] = x +Z(L)$.  Let $\varphi : L/Z(L) \to \GL(L)$ be a homomorphism.  Then $x,y \in Z(L)$ implies that $\varphi([x,y]) = \ker \varphi$.  From the first isomorphism theorem, $L/\ker \varphi = L/Z(L) \cong \IM \varphi$. 
}

\item Show that if $z \in L'$ then $\TR \AD z = 0$.

\BLUE{The thing to remember is that every $z \in L'$ is a linear combination of Lie Brackets.  Thus
\begin{align*}
	z &= \sum_k [x_k, y_k] \\
	\TR \AD z &= \sum_k \TR \AD ([x_k, y_k]) \\
	\text{or each } \TR \AD ([x_k, y_k])&=0, \forall k
\end{align*}That is, 
\begin{align*}
	\AD ([x_k,y_k]) &= \AD x_k\circ \AD y_k - \AD y_k\circ \AD x_k = 0 \\
	\therefore \TR \AD z &= 0
\end{align*}
}

\item Suppose $L_1$ and $L_2$ are Lie algebras.  let $L := \{(x_1,x_2)\,:\, x_i \in L_i\}$ be the direct sum of their underlying vector spaces, e.g., $L = L_1 \oplus L_2$.  Show that if we define
\begin{align*}
	\LB{(x_1,x_2)}{(y_1,y_2)} := \PAREN{\LB{x_1}{y_1},\LB{x_2}{y_2}}
\end{align*}then $L$ becomes a Lie algebra, the \textit{direct sum} of $L_1$ and $L_2$, $L = L_1 \oplus L_2$.  
\begin{enumerate}[label=(\roman*)]

	\item Prove that $\GL(2,\C)$ is isomorphic to the direct sum of $\SL(2,\C) \oplus \C$, the 1-dimensional complex abelien Lie algebra.
	
	\BLUE{Let $\varphi : \GL(2,\C) \to \SL(2,\C)\oplus \C$ be a surjective transformation.  Then
	\begin{align*}
		\dim \GL(2,\C) &= \dim \ker \varphi + \dim \operatorname{range} \varphi \\		
		\dim \ker \varphi &= \dim \GL(2,\C) -\dim (\SL(2,\C)\oplus \C) \\
		&= n^2 - n^2 = 0
\end{align*}	The dimension of the kernel of $\varphi$ is 0 therefore $\varphi$ is a bijection implying an isomorphsim.
	}
	
	\item Show that if $L=L_1 \oplus L_2$ then $Z(L)=Z(L_1)\oplus Z(L_2)$ and $L'=L_1'\oplus L_2'$.  Formulate a general version for a direct sum $L_1\oplus \cdots \oplus L_k$.
	
	\BLUE{\textbf{1: Show $Z(L)=Z(L_1)\oplus Z(L_2)$.} \\For any $u \in L$ there exists $u_1 \in L_1$ and $u_2\in L_2$ such that $u=(u_1, u_2)$.  If $z \in Z(L)$ then $[z,u]=0$.
	\begin{align*}
		[z, u] &= [(z_1,z_2), (u_1,u_2)] \\
		&= \PAREN{[z_1,u_1], [z_2,u_2]} \\
		\therefore [z_1,u_1] &=0 \AND [z_2,u_2]=0
	\end{align*}for any $u$. Thus, $z_1 \in Z(L_1)$ and $z_2 \in Z(L_2)$.\\
	\textbf{2: Show $L'=L_1'\oplus L_2'$}.  \\Let $z \in L$ then there exists a linear combination of commutators $[x_k,y_k]$ equal to zero
	\begin{align*}
		z &= \sum_k [x_k,y_k]
	\end{align*}There exist $a_k,b_k \in L_1$ and $c_k,d_k \in L_2$ such that $x_k = (a_k, c_k)$ and $y_k=(b_k, d_k)$. then 
	\begin{align*}
		z &= \sum_k [(a_k, c_k),(b_k, d_k)] \\
		&= \sum_k ([a_k,b_k],[c_k,d_k]) \\
		&= \PAREN{\sum_k [a_k,b_k], \sum_k [c_k,d_k]} \\
		&\in L_1 \oplus L_2
	\end{align*}Thus
	\begin{align*}
		L &= \bigoplus_k L_k \implies Z(L) = \bigoplus_k Z(L_k) \AND L' = \bigoplus_k L_k'
\end{align*}		
	}
	
	\item Are the summands in the direct sum decomposition of a Lie Algebra uniquely determined?  \textit{Hint: } If you think that the answer is yes, now might be a good time to read $\S 16.4$ in Appendix A on the ``diagonal fallacy".  The next question looks at this point in more detail.

\end{enumerate}

\item Suppose $L=L_1 \oplus L_2$ is the direct sum of two Lie algebras.
\begin{enumerate}[label=(\roman*)]

	\item Show that $\{(x_1,0):x_1 \in L_1\}$ is an ideal of $L$ isomorphic to $L_1$ and that $\{(0,x_2):x_2\in L_2\}$ is an ideal of $L$ isomorphic to $L_2$.  Show that the projections $p_1(x_1,x_2) = x_1$ and $p_2(x_1,x_2)=x_2$ are Lie algebra homomorphisms.
	
	\BLUE{\textbf{Show the $L_1$ isomorphism.}\\Let $u=(u_1, u_2) \in L$.  Then $N_1 = \{(x_1,0):x_1 \in L_1\}$ and $ x=(x_1, x_2)\in N_1$ then $[u,x] = [(u_1, u_2), (x_1, 0)] = ([u_1, x_1],[u_2,0]) =([u_1, x_1],0) \in N_1$ and therefore an ideal.  Also, allow $\varphi : N_1 \to L_1$.  Let $a,b \in \ker \varphi$.  Then $\varphi(a+b)= \varphi(a)+\varphi(b) = (0,0)$ implies that $a_1 =b_1$ or $a=b$.  Thus, $\varphi$ is an isomorphism. \\
	\textbf{A similar argument for the $L_2$ isomorphism.}\\
	\textbf{Proejctions:}\\
	Given any $x,y \in L$
	\begin{align*}
		p_1([x,y]) &= p_1([x_1,y_1],[x_2,y_2]) \\
		&= [x_1,y_1] 
	\end{align*}thus $p_1([x,y]) \in L_1$.  A similar argument for $L_2$.
	}
	
	Now suppose that $L_1$ and $L_2$ do not have any non-trivial proper ideals.
	
	\item Let $J$ be a proper ideal of $L$. Show that $J \cap L_1 =0$ and $J\cap L_2= 0$, then the projection $p_1:J \to L_1$ and $p_2: J\to L_2$ are isomorphisms.
	
	\item Deduce that if $L_1$ and $L_2$ are not isomorphic as Lie algebras, then $L_1\oplus L_2$ has only two non-trivial proper ideals.
	
	\item Assume that the ground field is infintie.  Show that if $L_1 \cong L_2$ and  $L_1$ is 1-dimensional, then $L_1\oplus L_2$ has infinitely many different ideals.

\end{enumerate}

\item Let $L_1$ and $L_2$ be Lie algebras, and let $\varphi : L_1\to L_2$ be a surjective Lie algebra homomorphism.  True or False:
\begin{enumerate}[label=(\alph*)]

	\item $\varphi(L_1') = L_2'$;
	
	\BLUE{Let $x,y \in L_1$ then $[x,y] \in L_1'$.  Then $\varphi([x,y]) = [\varphi(x),\varphi(y)] \in L_2'$.  Therefore $\varphi(L_1') \subseteq L_2'$.\\
	Then, we know that given any $u,v \in L_2$ there is $[u,v]\in \L_2'$ and there exist $x,y \in L_1$ such that $\varphi(x) = u, \varphi(y)=v$.  Thus, $[u,v]=[\varphi(x),\varphi(y)]=\varphi([x,y]$ and $L_2' \subseteq \varphi(L_1')$.\\
	\textbf{TRUE}
	}
	
	\item $\varphi(Z(L_1))=Z(L_2)$;
	
	\BLUE{Let $u \in Z(L_1)$.  Then for any $x\in L_1$ we have $\varphi(0)=0$ or $\varphi(u)=0$ therefore $\varphi(Z(L_1)) \subseteq Z(L_2)$\\
	Given any $v \in Z(L_2)$ then any $y\in L_2$ implies that $[v,y]=0$ and there exist $w,z \in L_1$ such that $\varphi(w)=v$ and $\varphi(z)=y$.  Then $[v,y]=[\varphi(w),\varphi(z)]=\varphi([w,z]) \in Z(L_1)$ because $w \in Z(L_1)$.  Therefore, $Z(L_2) \subseteq \varphi(Z(L_1))$.
	}
	
	\item $h \in L_1$ and $\AD_h$ is diagonalisable then $\AD_{\varphi(h)}$ is diagonalisable.
	
	\BLUE{Notice that
	\begin{align*}
		\AD_h(x) &= [h,x] \\
		\varphi(\AD_h(x)) &= \varphi([h,x]) \\
		&= [\varphi(h), \varphi(x)] \\
		&= \AD_{\varphi(h)}(\varphi(x))
\end{align*}	
	}

\end{enumerate}

\item For each pair of the following Lie algebras over $\R$, decide whether or not they are isomorphic:
\begin{enumerate}[label=(\roman*)]

	\item the Lie algebra $R_\wedge^3$ where the Lie bracket is given by the vector product;
	
	\BLUE{In other words, compare the wedge with the cross product.  Define a Lie Algebra $\R^3_\times = \R^3$ with $[x,y] = x \times y$ for all $x,y\in \R^3_\times$.  Let $\varphi: \R^3_\vee \to \R^3_\times$.  Then, $u,v \in \R^3_\vee$ and $x,y \in \R^3_\times$ such that $\varphi(u)=x, \varphi(v)=y$.
	\begin{align*}
		\varphi([u,v]) &= [\varphi(u),\varphi(v)] =[x,y] \\
		\varphi(u_2v_3-u_3v_2, u_3v_1-u_1v_3, u_2v_1-v_1u_2) &= \PAREN{x_2y_3-x_3y_2, x_3y_1-x_1y_3, x_2y_1-x_1y_2 }
\end{align*}	which is true if $\varphi$ is the identity
	}
	
	\item the upper triangular $2 \times 2$ matices over $\R$;
	
	\item the strict upper triangular $3 \times 3$ matrices over $\R$;
	
	\item $L=\{ x \in \GL(3,\R):x^t=-x\}$.
	
	\textit{Hint:} Use Exercises 1.15 and 2.8.

\end{enumerate}

\item Let $F$ be a field.  Show that the derived algebra of $\GL(n,F)$ is $\SL(n,F)$

\BLUE{\textbf{Show that $\GL(n,F)' \subseteq \SL(n,F)$} Let $u \in \GL(n,F)'$ then there exist $x,y \in \GL(n,F)$ such that $u =[x,y]$ \begin{align*}
	\TR u &= \TR [x,y] \\
	&= \TR(xy - yx) \\
	&= \TR(xy)-\TR(yx) \\
	&= 0 \\
	\therefore u \in \SL(n,F)
\end{align*}\\
\textbf{Show that $\GL(n,F)'\supseteq\SL(n,F)$} Let $u \in \SL(n, F)$ then $\TR u=0$
}

\item In Exercise 1.15, we defined the Lie Algebra $\GL_S(n,F)$ over a field $F$ where $S$ is an $n\times n$ matrix with entries in $F$.

Suppose that $T \in \GL(n,F)$ is another $n\times n$ matrix such that $T=P^tSP$ for some invertible $n\times n$ matrix $P\in \GL(n,F)$  (Equivalently, the bilinear forms defined by $S$ and $T$ are congruent.)  Show that the Lie algebras $\GL_S(n,F)$ and $\GL_T(n,F)$ are isomorphic.

\item Let $S$ be an $n \times n$ invertible matrix with entries in $\C$.  Show that if $x \in \GL_S(n,\C)$, then $\TR x=0$

\item Let $I$ be an ideal of a Lie Algebra $L$.  Let $B$ be the centraliser of $I$ in $L$; that is 
\begin{align*}
	B = C_L(I) = \{x \in L : [x,a]=0, \, \forall a \in I\}
\end{align*}Show that $B$ is an ideal of $L$.  Now suppose that
\begin{enumerate}
	\item $Z(I)=0$, and
	\item if $D:I\to I$ is a derivation, then $D=\AD x$ for some $x \in I$.
	
	Show that $L=I\oplus B$.
	
	\item Recall that if $L$ is  Lie algebra, we defined $L'$ to be the subspace spanned by the commutators $[x,y]$ for $x,y\in L$.  The purpose of this execise, which may safely be skipped on first reading, is to show that the \textit{set} of commutators may not even be a vector space (and so certainly not an ideal of $L$.).
	
	Let $\R[x,y]$ denote the ring of all real polynomials in two variables.  Let $L$ be the set of all matrices of the form 
	\begin{align*}
		A\PAREN{f(x), g(y), h(x,y)} = \THREEXTHREE{0}{f(x)}{h(x,y)}{0}{0}{g(y)}{0}{0}{0}.
	\end{align*}\begin{enumerate}[label=(\roman*)]
		\item Prove $L$ is a Lie algebra with usual commutator bracket. (In contrast to all the Lie algebras seen so fro, $L$ is infinite-dimensional.)
		\item Prove that
		\begin{align*}
			\SQBRACKET{A\PAREN{f_1(x), g_1(y), h_1(x,y)}, A\PAREN{(f_2(x), g_2(y), h_2(x,y)}} = A(0,0,f_1(x)g_2(x)-f_2(x)g_1(y)).
		\end{align*}Hence describe $L'$.
		
		\item Show that if $h(x,y)=s^2+xy+y^2$, then $A(0,m0,h(x,y))$ is not a commutator.
	
	\end{enumerate}
\end{enumerate}


\end{enumerate}

\chapter{Low Dimensional Lie Aglebras}

\begin{description}
	\item \textbf{Dimension 1}

	Given a single basis vector $e$ then by definition the Lie bracket must be $[e,e]=0$	making the entire 1-dimensional vector space Abelian.
	\begin{align*}
		\mathfrak{g} \cong \R
	\end{align*}
	
	\item \textbf{Dimension 2}
	
	Given a basis $e_1, e_2$ .
	\begin{enumerate}
		\item \textbf{Abelian}
		
			The Lie Breacket must be $[e_1, e_2]=0$ for all elements of the vector space (plane).
		
		\item \textbf{Non-Abelian (solvable)}
		
			The Lie Bracket must be $[e_1,e_2] = e_2$.  This algebra is solvable but not nilpotent (i.e., $A^n=0$ for some $n$).
			
	\end{enumerate}
	\item \textbf{Dimension 3}
	
	Given a basis $e_1, e_2, e_3$.
	
	\begin{enumerate}
		\item \textbf{Abelian}
		
		All Lie brackets are zero.
		
		\item \textbf{Heisenberg Algebra}	
		
		The Lie bracket is $[e_1, e_2] = e_3$.  This becomes nilpotent.
		
		\item \textbf{Solvable (non-nilpotent)}	
	
		\begin{description}
			\item \textbf{Type 1:}
				\begin{align*}
					[e_1, e_2] = e_3 \AND [e_1, e_3] = e_2
				\end{align*}
			\item \textbf{Type 2:}
				\begin{align*}
					[e_1, e_2] = -e_3 \AND [e_1, e_3] = e_2
				\end{align*}
		\end{description}			
	
		\item \textbf{Simple Lie Algebras}
		
		\begin{enumerate}
			\item $\mathfrak{so}(3)$
				\begin{align*}
					[e_1, e_2] = e_3, \; [e_2, e_3] = e_1 \AND [e_3, e_1]=e_2
				\end{align*}
			\item $\mathfrak{sl}(2, \R)$
			
			Consider the basis $h, e, f$
			\begin{align*}
				[h,e] = 2e,\;[h,f]=-2f \AND [e,f]=h
			\end{align*}
		\end{enumerate}
		
		
	\end{enumerate}
\end{description}

\section{Exercises}


\begin{enumerate}[label=3.\arabic*.]

\item Let $V$ be a vector space and let $\varphi$ be an endomorpism of $V$.  Let $L$ have underlying vector space $V \oplus \operatorname{span}\{x\}$.  Show that if we define the Lie bracket on $L$ by $[y,z]=0$ and $[x,y]=\varphi(y)$ for $y,z\in V$, then $L$ is a Lie algebra and $\dim L' = \operatorname{rank} \varphi$.  (For a more general construction, see Exercise 3.9 below.)

\BLUE{If $u_1, u_2 \in L$ then there exist $y_1, y_2 \in V$ and $x_1, x_1 \in \operatorname{Span}({x})$ such that $u_1 = (y_1, x_1)$ and $u_2 = (y_2, x_2)$ then
\begin{align*}
	[u_1, u_2] &= [(y_1, x_1), (y_2, x_2)] \\ &= [y_1, x_1]+[y_1, x_2] + [y_2, x_1] + [y_2, x_2] \\
	&= 2\varphi(y_1)+2\varphi(y_2)
\end{align*}The only elements of $L$ that are non-zero are the ones indicated by $x$ and hence $\varphi$.  Therefore, $\dim L' = \operatorname{rank}\varphi$, i.e., the number of linearly independent rows of its matrix of transformation.
}

\item With the notation of $\S 3.2.3$, show that the Lie algebra $L_\mu$ is isomorphic to $L_\nu$ if and only if either $\mu =\nu$ or $\mu=\nu^{-1}$.

\BLUE{From 3.1, let $V_\mu, V_\ne$ be spanned by $[y_\mu, z_\mu], [y_\nu, z_\nu]$ and $x_\mu, x\nu$ be the such that $L_\mu = V_\mu\oplus \SPAN\{x\},L_\nu = V_\nu\oplus \SPAN\{x\}$.  We can see that the Lie bracket for each $L_\mu$ is $[x_\mu,y_\nu] = y$ (from the text) and $[y_mu, z_\mu] = 0$, from the lemma.  This Lie Bracket is precisely the Lie bracket from 3.1 for both $\mu, \nu$.  There rank must each be 2, therefore an isomorphisms exists $\varphi: L_\mu \to L_\nu$.
}

\BLUE{Let $\varphi: L_\mu \to L_\nu$ be this isomorphism.  Thus, given any $u \in L_\mu$, $v \in L_\nu$
\begin{align*}
	\varphi(u) &= A_\mu u \AND \varphi^{-1}(v) = A^{-1}v \\
	A_\mu &= \TWOXTWO{1}{0}{0}{\mu}\\
	I &= \TWOXTWO{a}{b}{c}{d}\TWOXTWO{1}{0}{0}{\mu} \\
	&= \TWOXTWO{a}{b\mu}{c}{d\mu}\\
	&= \TWOXTWO{1}{0}{0}{d\mu} \\
	\therefore d &= \mu^{-1}
\end{align*}$A^{-1}$ is the $\AD x_\nu$.
}

\item Find out where each of the following 3-dimensional complex Lie algebras appears in our classification:
\begin{enumerate}[label=(\roman*)]

	\item $\GL_S(3,\C)$, where $S=\THREEXTHREE{1}{0}{0}{0}{1}{0}{0}{0}{-1}$;
	
	\BLUE{Remember that 
	\begin{align*}
	\GL_S(n,F) = \{u \in \GL(n,F)\,:\, u^tS=-Su\}.
	\end{align*}Let's determine a basis for $\GL_S(3,\C)$.  Let $u \in \GL_S(3,\C)$  then 
	\begin{align*}
		u^tS &= -Su \\
		\LET u &= \THREEXTHREE{a}{b}{c}{d}{e}{f}{g}{h}{k} \\
		-Su &= \THREEXTHREE{-1}{0}{0}{0}{1}{0}{0}{0}{1}\THREEXTHREE{a}{b}{c}{d}{e}{f}{g}{h}{k} \\
		&= \THREEXTHREE{-a}{-b}{-c}{d}{e}{f}{g}{h}{k} \\
		u^tS &= \THREEXTHREE{a}{d}{g}{b}{e}{h}{c}{f}{k}\THREEXTHREE{1}{0}{0}{0}{1}{0}{0}{0}{-1} \\
		&= \THREEXTHREE{a}{d}{-g}{b}{e}{-h}{c}{f}{-k}
	\end{align*}Which leads us to the following conclusions:
	\begin{align*}
		a &= k = 0\\
		-b &= d \AND b=d \to b=d=0 \\
		g &= c \\
		e &= e \\
		f &= -h \AND f=h \to f=h=0
	\end{align*}Thus, 
	\begin{align*}
		u &= \THREEXTHREE{0}{0}{c}{0}{e}{0}{c}{0}{0} \\
		&= c\THREEXTHREE{0}{0}{1}{0}{0}{0}{1}{0}{0}+e\THREEXTHREE{0}{0}{0}{0}{1}{0}{0}{0}{0}
	\end{align*}$\GL_S(3,\C)$ is 2 dimensional. This is not Abelian therefore it is non-abelian solvable.
	}
	
	\item the Lie subalgebra of $\GL(3,\C)$ spanned by the matrices
	\begin{align*}
		\mu &= \THREEXTHREE{\lambda}{0}{0}{0}{\mu}{0}{0}{0}{\nu}, v= \THREEXTHREE{0}{0}{1}{0}{0}{0}{0}{0}{0}, w=\THREEXTHREE{0}{0}{0}{0}{0}{1}{0}{0}{0}
	\end{align*}where $\lambda, \mu, \nu$ are fixed complex numbers;
	
	\BLUE{\begin{align*}
		\mu v &= \THREEXTHREE{\lambda}{0}{0}{0}{\mu}{0}{0}{0}{\nu}\THREEXTHREE{0}{0}{1}{0}{0}{0}{0}{0}{0} \\
		&= \THREEXTHREE {0}{0}{\lambda}{0}{0}{0}{0}{0}{0} = \lambda v \\
		v w &= \THREEXTHREE{0}{0}{1}{0}{0}{0}{0}{0}{0}\THREEXTHREE{0}{0}{0}{0}{0}{1}{0}{0}{0} = 0\\
	\mu w &= \THREEXTHREE{\lambda}{0}{0}{0}{\mu}{0}{0}{0}{\nu}\THREEXTHREE{0}{0}{0}{0}{0}{1}{0}{0}{0} \\
	&= \THREEXTHREE{0}{0}{0}{0}{0}{\nu}{0}{0}{0} = \nu w
\end{align*}	
	}
	
	\item $\BRACKET{\PAREN{\begin{array}{cccc}
		0&a&b&0 \\
		0&0&c&0\\
		0&0&0&0\\
		0&0&0&0\\	
	\end{array}
	}: a,b,c \in \C}$;
	
	\item $\BRACKET{\PAREN{\begin{array}{cccc}
		0&0&a&b \\
		0&0&0&c\\
		0&0&0&0\\
		0&0&0&0\\	
	\end{array}
	}: a,b,c \in \C}$.
\end{enumerate}

\item Suppose $L$ is a vector space with basis $x,y$ and that a bilinear operation $[-,-]$ on $L$ is defined such that $[u,u]=0$ for aqll $u \in L$.  Show that the Jacobi Identity holds and hence $L$ is a Lie algebra.

\item Show that over $\R$ the Lie algebras $\SL(2,\R)$ and $\R^3_\wedge$ are not isomorphic.  \textit{Hint}: Prove tht there is no non-zero $x\in \R^3_\wedge$ such that the map $\AD x$ is diagonalisable.

\item Show that over $\R$ there are exactly two non-isomophic 3-dimensional Lie algrebras with $L'=L$.

\item Let $L$ be a non-abelian Lie algebra.  Show that $\dim Z(L) \le \dim L -2$.

\item let $L$ be the 3-dimensional Heisenberg Lie algebra defined over field $F$.  Show that $\DER L$ is 6-dimensional.  Identify the inner derivations (as defined in Exercise 1.18) and show that he quotient $\DER L/\IDER L$ is isomorphic to $\GL(2,F)$.

\item Suppose that $I$ is an ideal of a Lie algebra $L$ and that there is a subalgebra $S$ of $L$ such that $L=S\oplus I$.

\begin{enumerate}[label=(\roman*)]

	\item Show that the map $\theta : S \to \GL(I)$ defined by $\theta(s)x=[s,x]$ is a Lie algebra homomorphism from $S$ into $\DER I$.
	
	We say that $L$ is \textit{semidirect product} of $I$ by $S$.  (the reader may have seen the analogous construction for groups.)
	
	\item Show conversely that given Lie aglebras $S$ and $I$ and a Lie algebra homomorphism $\theta : S \to \DER I$, the vector space $S \oplus I$ may be made into a Lie algebra by defining
	\begin{align*}
		[(s_1,x_1),(s_2,x_2)] = ([s_1,s_2],[x_1,x_2]+\theta(s_1)x_2-\theta(x_2)x_1)
	\end{align*}for $s_1,s_2 \in S$, and $x_1,x_2 \in I$, and that this Lie algebra is a semidirect product of $I$ by $S$.  (The direct sum construction introduced in Exercise 2.6 is the special case where $\theta(s)=0$ for all $s \in S$.)
	
	\item Show that the Lie algebras in Exercise 3.1 may be constructed as semidirect products.
	
	\item Investigate necessary and sufficient condtiions for two semidirect products to be isomorphic.

\end{enumerate}

\item Find, up to isomorphism, all Lie algebras with a 1-dimensional derived algebra.

\end{enumerate}

\chapter{Solvable Lie Algebras}

\begin{remark}[\textbf{MathGPT: Describe the difference between an Ideal and a derived Lie algebra.}].

\noindent
Solution:
\begin{enumerate}[label=Step \arabic*:]
	\item \textbf{Define an ideal of a Lie Algebra.}
	
	An ideal of a Lie algebra is a subalgebra that interacts with the Lie bracket in a specific way.  For a Lie algebra $\mathfrak{g}$ over a field $F$, a subsapce $\mathfrak{i} \subset \mathfrak{g}$ is an ideal if and only $x \in \mathfrak{g}$ and $y \in \mathfrak{i}$, the Lie bracket $[x,y]$ is in $i$.  This condition is expressed as $[\mathfrak{g}, \mathfrak{i}]$.  Ideals are analogous to normal subgroups in group theory and are essential for construcitng quotient Lie algebras.
	
	\item \textbf{Define a Derived Lie Algebra}
	
	The derived Lie algebra, also known as the commutator subalgebra, of a Lie algebra $\mathfrak{g}$ is denoted $\mathfrak{g}'$ or $[\mathfrak{g}, \mathfrak{g}]$.  It is the subspace spanned by all Lie brackets of elements in $\mathfrak{g}$, as shown using the Jacobi identity.
	
	\item \textbf{Difference in Generality.}
	
	An ideal is a type of subspace with a specific property related to the Lie bracket, while the derived Lie algebra is a specific construction that results in an ideal.  Every derived Lie algebra is an ideal, but not every ideal is a derived algebra.
	
	\item An ideal $\mathfrak{i}$ is defined by the property $[\mathfrak{g},\mathfrak{i}] \subseteq \mathfrak{i}$ and can be any subspace satisfying this condition.  The derived Lie algebra $\mathfrak{g}'$ is defined as the set of all possible Lie brackets of elements within $\mathfrak{g}$ and is uniquely determined by $\mathfrak{g}$.
	
	\item \textbf{Difference in Purpose/Role}
	
	Ideals are used to define quotient Lie algebras and help understand the structure of Lie algebras,  similar to normal subgroups in group theory.  The derived Lie algebra $\mathfrak{g}'$ measures how "non-abelian" a Lie algebra is.  If $\mathfrak{g}'=\{0\}$, then $\mathfrak{g}$ is abelian.  The derrived series of all Lie algebra is used to defien solvable and nilpotent Lie algebras.
	
	\item \textbf{Examples}
	
	Consider the Lie algebra $\mathfrak{gl}(n,F)$ of all $n \times n$ matrices over a field $F$, with the Lie bracket $[A,B]=AB-BA$.
	\begin{itemize}
		\item \textbf{Ideal: }The set of scalar matrices, $\mathfrak{s}=\{cI\,|\, c\in F\}$, forms an ideal of $\mathfrak{gl}(n,F)$.  This ideal is not the derived Lie algebras.
		\item \textbf{Derived Lie Algebra:} the derived Lie algebra $\mathfrak{gl}(n,F)$ is $\mathfrak{sl}(n,F)$, the Lie algebra of $n\times n$ matrices with trace zero.  This is because the trace of a commutator is always zero, and any matricx with trace zero can be written as a commutator.  $\mathfrak{sl}(n,F)$ is an ideal of $\mathfrak{gl}(n,F)$.
	\end{itemize}
	
	\textbf{Final Answer}  In summary, while the derived Lie algebra is always an ideal, an ideal is a more general concept. The derived Lie algebra is a specific ideal that captures the "commutativity" properties of the Lie algebra itself.

\end{enumerate}

\end{remark}

\HLINE
\newcommand{\SLC}{\mathfrak{sl}(n,\C)}
\newcommand{\SOC}{\mathfrak{so}(n,\C)}
\newcommand{\SPC}{\mathfrak{sp}(2n,\C)}
\begin{remark}{\textbf{Simple Algebras:} Special Linear, Special Orthogonal and Symplectic Lie Algebras}
\begin{description}
	\item \textbf{Special Linear, $\mathfrak{sl}(n,\C)$}
	
	Description:
	
	Teh Lie algebra $\SLC$ consists of all $n \times n$ complex matrices with trace zero, $\TR A = 0$.  The Lie bracket is defined by the commutator $[A,B]=AB-BA$.
	
	Dimensions:
	
	The dimensions of $\SLC$ is $n^2-1$, as the trace condition imposes one linear constraint on the $n^2$ entries of a matrix.
	
	Importance:
	\begin{itemize}
		\item If $L = \mathfrak{gl}(n,\C)$ then $L'= \SLC$.
		\item Fundamental Representation: It is the Lie algebra of the specal linear group $SL(n,\C)$ which acts on $\C^n$.
		\item Simple Lie Algebra: For $n \ge 2$, it is a simple Lie algebar, correspoidng ot the $A$-series of the Carten classification.
		\item Physics: It appears in quantum mechanics and quantum field theory,  describing symmetries.
	\end{itemize}
	
	\item \textbf{The Special Orthogonal Lie Algebra, $\SOC$}
	
	Description:
	
	The Lie Algebra $\SOC$ consists of $n \times n$ complex skew-symmetric matrices.  The Lie bracket is the commutator $[A,B]=AB-BA$.
	
	Dimension:
	
	The dimesion is $\frac{n(n-1)}{2}$, as skew-symmetric matrices have z ero diagonal entries and $A_{aj}=-A_{ji}$ for $i \ne j$.
	
	Impotrance:
	
	\begin{itemize}
		\item Orthogonal Group:  It si the Lie algebra of the special orthogoanl gorup $SO(n,\C)$, preserving a symmetric bilinear form.
		
		\item Simple Lie Algebra: For $n\ge 3$ and $n \ne 4$, it is simple, corresponging to the $B$-series  and $D$-series in the Cartan classification.
		
		\item Geometry and Physics:  It is crucial in geometery and physics, related to rotattions and Lorentz transformations.
	\end{itemize}
	
	\item \textbf{The Symplectic Lie Algebra, $\SPC$.}
	
	Description:
	
	The Lie algebra $\SPC$ consists $2n\times 2n$ complex matrices $A$ staisfying $A^TJ+JA = 0$ where $J$ is a standard skew-symmetric matrix.
	
	Dimension:
	
	The dimension is $n(2n+1)$, derived fromt the conditiosn on the matrix blocks $P,Q,R,S$.
	
	Importance:
	\begin{itemize}
		\item Symplectic Group:  It is the Lie algebra of the symplectic group $Sp(2n,\C)$, preserving skew-symmetric bilinear form.
		\item Simple Lie Algebra: For $n\ge 1$, it is simple, corresponding to the $C$-series in the Cartan classification.
		
		\item Hamiltonian Mechanics and Quantum Mechancis:  It is fundamental in Hamiltonian mechanics and quantum mechancis, realted to symplectic manifolds and canonical transformations.
	\end{itemize}
\end{description}
\end{remark}
\section{Exercises}

\newcommand{\PPP}[1]{^{(#1)}}
\begin{enumerate}[label=4.\arabic*]

\item  Suppose that $\varphi:L_1 \to L_2$ is a surjective homomorphism of Lie algebras.  Show that 
\begin{align*}
	\varphi\PAREN{L_1\PPP{k}}=(L_2)\PPP{k}.
\end{align*}

\BLUE{A proof by induction.  The initial case\begin{align*}
	\varphi(L_1') &= \varphi([L_1,L_1]) \\
	&= [\varphi(L_1),\varphi(L_1)]
\end{align*}Since $\varphi$ is surjective $\varphi(L_1)=L_2$ hence
\begin{align*}
	\varphi(L_1') &= [L_2,L_2] = L_2'
\end{align*}The inductive case.
\begin{align*}
	\text{assume } \varphi(L_1\PPP{k}) &= (L_2)\PPP{k} \\
	\varphi(L_1\PPP{k+1}) &= \varphi([L_1\PPP{k},L_1\PPP{k}]) \\
	&= \SQBRACKET{\varphi(L_1\PPP{k}),\varphi(L_1\PPP{k})} \\
	&= [ (L_2)\PPP{k}, (L_2)\PPP{k}] \\
	&= (L_2)\PPP{k+1}
\end{align*}
}

\item Let $x \in \mathfrak{gl}(2\ell, \C)$.  Show that $x$ belongs to $\mathfrak{sp}(2\ell,\C)$ if and only if it is of the form 
\begin{align*}
	x = \TWOXTWO{m}{p}{q}{-m}
\end{align*}where $p$ and $q$ are symmetric.  Hence find the dimension of $\mathfrak{sp}(2\ell, \C)$.  (See Exercise 12.1 for the other families)

\item Use lemma 4.4 to show that if $L$ is a Lie algebra then $L$ is solvable if and only if $\AD L$ is a solvable subalgebra of $\GL(L)$.  Show that this result also holds if we replace "solvable" with "nilpotent".

\item Let $L=\mathfrak{n}(n,F)$, the Lie algebra of strictly upper triangular $n \times n$ matrices over a field $F$.  Show that $L^k$ has a basis consisting of all the matrix units $e_{ij}$ with $j-i>k$. Hence show that $L$ is nilpotent.  What is the smallest $m$ such that $L^m = 0$?

\BLUE{Let $L \in \mathfrak{n}(n,F)$.\begin{align*}
	L^1 = L' &= [L,L]
\end{align*}given any $A,B \in L$, $[A,B] = AB-BA$.  Given any $i,j < n$, first we can see that $A_{ij}=0$ whenever $i\ge j$ (the diagonal and lower triangle of the matrix).  Further, 
\begin{align*}
	AB &= \SQBRACKET{ \sum_{k=1}^n A_{ik}B_{kj} } \\
	&= \SQBRACKET{\sum_{k=j}^n A_{ik}B_{kj}} \\
	\text{or } i<j+1 &\implies AB_{ij} = BA_{ij} = 0 
\end{align*}that is, the elements along the diagonal one row up (or one column to the right) are all zero.  Similarly, let $C=AB$ then for some $D \in L$ we get a similar argument, that is $i<j+2$ implies that $DC_{ij}=CD_{ij}=0$.  $C \in [L,L]$ and $D\in [L, L']=L^2$, this process can be repeated as long as $j-i<k$, that is $k=n-1$.  When $k=n-1$ we have a zero indicating that $L$ is nilpotent at $m=n-1$
}
\item Let $L=\mathfrak{b}(n,F)$ be the Lie algebra of upper triangular $n\times n$ matrices over a field $F$.
\begin{enumerate}[label=(\roman*)]

	\item Show that $L'=\mathfrak{n}(n,F)$.
	
	\BLUE{Let $A,B \in \mathfrak{b}(n,F)$.  Looking only at the elements on the diagonal, we can see that, 
	\begin{align*}
		(AB)_{ii} &= \sum_{k=1}^n A_{ik}B_{ki} = \sum_{k=1}^n B_{ik}A_{ki} = (BA)_{ii} \\
		\therefore (AB)_{ii}-(BA)_{ii} &= 0 \implies [A,B] \in \mathfrak{n}(n, F)
	\end{align*}
	}
	
	\item More generally, show that $L\PPP{k}$ has a basis consisting of all the matrix units $e_{ij}$ with $j-i\ge 2^{k-1}$.  (the commutator formula for the $e_{ij}$ given in $\S1.2$ will be helpful.)
	
	\item Hence show that $L$ is solvable.  What is the smallest $m$ such that $L\PPP{m}=0$?
	
	\item Show that if $n\ge 2$ then $L$ is not nilpotent.

\end{enumerate}

\item Show that a Lie algebra is semisimple if and only if it has no non-zero abelian ideals.  (this was the original definition of semisimplicity given Wilhelm Killing.)

\item Prove directly that $\SLC$ is a simple Lie algebra $n \ge 2$.

\item Let $L$ be a Lie algebra over a field $F$ such that $[[a,b],b]=0$ for all $a,b \in L$, (or equivalently, $(\AD b)^2=0$ for all $b \in L$).

\begin{enumerate}[label=(\roman*)]

	\item Suppose the characteristic of $F$ is not 3.  Show that then $L^4 = 0$.  \textit{Hint:}  Show first that the Lie brackets $[[x,y],z]$ are alterenating; that is, 
	\begin{align*}
		[[x,y],z]=-[[y,x],z],\; [[x,y],z]=-[[x,z],y]
	\end{align*}for all $x,y,z \in L$.

\end{enumerate}

\item The purpose of this exercise is to give some idea why the families of Lie algebra are given the names that we have used.  We shall not need to refer to this exercise later;  some basic group theory is needed $\dots$

\item Let $F$ be a field.  Exercise 2.11 shows that if $S,T \in \GL(n,F)$ are congruent matrices (that is, there exists an invertible matrix $P$ such that $T=P^tSP$), then $\GL_s(n,F) \cong \GL_T(n,F)$.  Does the converse hold when $F = \C$?  For a challenge, think about other fields.
\end{enumerate}
\end{document}
