 \documentclass[10pt,a4paper]{report}
\usepackage[utf8]{inputenc}
\usepackage{amsmath}
\usepackage{amsfonts}
\usepackage{amssymb}
\usepackage{amsthm}
\usepackage{hyperref}

\usepackage{multicol}
\usepackage{fancyhdr}
\usepackage[inline]{enumitem}
\usepackage{tikz}
\usepackage{tikz-cd}
\usetikzlibrary{calc}
\usetikzlibrary{shapes.geometric}
\usepackage[margin=0.5in]{geometry}
\usepackage{xcolor}

\usepackage{pgfplots}
\hypersetup{
    colorlinks=true,
    linkcolor=blue,
    filecolor=magenta,      
    urlcolor=cyan,
    pdftitle={Tensors},
    pdfpagemode=FullScreen,
    }

%\urlstyle{same}

\newcommand{\CLASSNAME}{Math 5050 -- Special Topics: Manifolds}
\newcommand{\STUDENTNAME}{Paul Carmody}
\newcommand{\ASSIGNMENT}{Assignment 7 }
\newcommand{\DUEDATE}{April 30, 2025}
\newcommand{\SEMESTER}{Spring 2025}
\newcommand{\SCHEDULE}{MW 12:30 - 1:45}
\newcommand{\ROOM}{Remote}

\newcommand{\MMN}{M_{m\times n}}
\newcommand{\FF}{\mathcal{F}}
\newcommand{\RANGE}{\text{range}}

\pagestyle{fancy}
\fancyhf{}
\chead{ \fancyplain{}{\CLASSNAME} }
%\chead{ \fancyplain{}{\STUDENTNAME} }
\rhead{\thepage}
\newcommand{\LET}{\text{Let }}
%\newcommand{\IF}{\text{if }}
\newcommand{\AND}{\text{ and }}
\newcommand{\OR}{\text{ or }}
\newcommand{\FORSOME}{\text{ for some }}
\newcommand{\FORALL}{\text{ for all }}
\newcommand{\WHERE}{\text{ where }}
\newcommand{\WTS}{\text{ WTS }}
\newcommand{\WLOG}{\text{ WLOG }}
\newcommand{\BS}{\backslash}
\newcommand{\DEFINE}[1]{\textbf{\emph{#1}}}
\newcommand{\IF}{$(\Rightarrow)$}
\newcommand{\ONLYIF}{$(\Leftarrow)$}
\newcommand{\ITH}{\textsuperscript{th} }
\newcommand{\FST}{\textsuperscript{st} }
\newcommand{\SND}{\textsuperscript{nd} }
\newcommand{\TRD}{\textsuperscript{rd} }
\newcommand{\INV}{\textsuperscript{-1} }


%%%%%%%
% derivatives
%%%%%%%

\newcommand{\PART}[2]{\frac{\partial #1}{\partial #2}}
\newcommand{\SPART}[2]{\frac{\partial^2 #1}{\partial #2^2}}
\newcommand{\DERIV}[2]{\frac{d #1}{d #2}}
\newcommand{\LAPLACIAN}[1]{\frac{\partial^2 #1}{\partial x^2} + \frac{\partial^2 #1}{\partial y^2}}

%%%%%%%
% sum, product, union, intersections
%%%%%%%

\newcommand{\SUM}[2]{\underset{#1}{\overset{#2}{\sum}}}
\newcommand{\PROD}[2]{\underset{#1}{\overset{#2}{\prod}}}
\newcommand{\UNION}[2]{\underset{#1}{\overset{#2}{\bigcup}}}
\newcommand{\INTERSECT}[2]{\underset{#1}{\overset{#2}{\bigcap}}}
\newcommand{\FSUM}{\SUM{n=-\infty}{\infty}}
       

%%%%%%%
% supremum and infimum
%%%%%%%

\newcommand{\SUP}[1]{\underset{#1}\sup \,}
\newcommand{\INF}[1]{\underset{#1}\inf \,}
\newcommand{\MAX}[1]{\underset{#1}\max \,}
\newcommand{\MIN}[1]{\underset{#1}\min \,}

%%%%%%%
% infinite sums, limits
%%%%%%%

\newcommand{\SUMK}{\SUM{k=1}{\infty}}
\newcommand{\SUMN}{\SUM{n=1}{\infty}}
\newcommand{\SUMKZ}{\SUM{k=0}{\infty}}
\newcommand{\LIM}[1]{\underset{#1}\lim\,}
\newcommand{\IWOB}[1]{\LIM{#1 \to \infty}}
\newcommand{\LIMK}{\IWOB{k}}
\newcommand{\LIMN}{\IWOB{n}}
\newcommand{\LIMX}{\IWOB{x}}
\newcommand{\NIWOB}{\LIM{n \to \infty}}
\newcommand{\LIMSUPK}{\underset{k\to\infty}\limsup \,}
\newcommand{\LIMSUPN}{\underset{n\to\infty}\limsup \,}
\newcommand{\LIMINFK}{\underset{k\to\infty}\liminf \,}
\newcommand{\LIMINFN}{\underset{n\to\infty}\liminf \,}
\newcommand{\ROOTRULE}[1]{\LIMSUPK \BARS{#1}^{1/k}}

\newcommand{\CUPK}{\bigcup_{k=1}^{\infty}}
\newcommand{\CAPK}{\bigcap_{k=1}^{\infty}}
\newcommand{\CUPN}{\bigcup_{n=1}^{\infty}}
\newcommand{\CAPN}{\bigcap_{n=1}^{\infty}}

%%%%%%%
% number systems (real, rational, etc.)
%%%%%%%

\newcommand{\REALS}{\mathbb{R}}
\newcommand{\RATIONALS}{\mathbb{Q}}
\newcommand{\IRRATIONALS}{\REALS \backslash \RATIONALS}
\newcommand{\INTEGERS}{\mathbb{Z}}
\newcommand{\NUMBERS}{\mathbb{N}}
\newcommand{\COMPLEX}{\mathbb{C}}
\newcommand{\DISC}{\mathbb{D}}
\newcommand{\HPLANE}{\mathbb{H}}

\newcommand{\R}{\mathbb{R}}
\newcommand{\Q}{\mathbb{Q}}
\newcommand{\Z}{\mathbb{Z}}
\newcommand{\N}{\mathbb{N}}
\newcommand{\C}{\mathbb{C}}
\newcommand{\T}{\mathbb{T}}
\newcommand{\COUNTABLE}{\aleph_0}
\newcommand{\UNCOUNTABLE}{\aleph_1}


%%%%%%%
% Arithmetic/Algebraic operators
%%%%%%%


\DeclareMathOperator{\MOD}{mod}
%\newcommand{\MOD}[1]{\mod #1}
\newcommand{\BAR}[1]{\overline{#1}}
\newcommand{\LCM}{\text{ lcm}}
\newcommand{\ZMOD}[1]{\Z/#1\Z}
\DeclareMathOperator{\VAR}{Var}
%%%%%%%
% complex operators
%%%%%%%

\DeclareMathOperator{\RR}{Re}
%\newcommand{\RE}{\text{Re}}
\DeclareMathOperator{\IM}{Im}
%\newcommand{\IM}{\text{Im}}
\newcommand{\CONJ}[1]{\overline{#1}}
\DeclareMathOperator{\LOG}{Log}
%\newcommand{\LOG}{\text{ Log }}
\newcommand{\RES}[2]{\underset{#1}{\text{res}} #2}

%%%%%%%
% Group operators
%%%%%%%

\newcommand{\AUT}{\text{Aut}\,}
\newcommand{\KER}{\text{ker}\,}
\newcommand{\END}{\text{End}}
\newcommand{\HOM}{\text{Hom}}
\newcommand{\CYCLE}[1]{(\begin{array}{cccccccccc}
		#1
	\end{array})}
\newcommand{\SUBGROUP}{\underset{\text{group}}\subseteq}	
%\newcommand{\SUBGROUP}{\subseteq_g}
\newcommand{\SUBRING}{\underset{\text{ring}}\subseteq}
\newcommand{\SUBMOD}{\underset{\text{mod}}\subseteq}
\newcommand{\SUBFIELD}{\underset{\text{field}}\subseteq}
\newcommand{\ISO}{\underset{\text{iso}}\longrightarrow}
\newcommand{\HOMO}{\underset{\text{homo}}\longrightarrow}

%%%%%%%
% grouping (parenthesis, absolute value, square, multi-level brackets).
%%%%%%%

\newcommand{\PAREN}[1]{\left (\, #1 \,\right )}
\newcommand{\BRACKET}[1]{\left \{\, #1 \,\right \}}
\newcommand{\SQBRACKET}[1]{\left [\, #1 \,\right ]}
\newcommand{\ABRACKET}[1]{\left \langle\, #1 \,\right \rangle}
\newcommand{\BARS}[1]{\left |\, #1 \,\right |}
\newcommand{\DBARS}[1]{\left \| \, #1 \,\right \|}
\newcommand{\LBRACKET}[1]{\left \{ #1 \right .} 
\newcommand{\RBRACKET}[1]{\left . #1 \right \]}
\newcommand{\RBAR}[1]{\left . #1 \, \right |}
\newcommand{\LBAR}[1]{\left | \, #1 \right .}
\newcommand{\BLBRACKET}[2]{\BRACKET{\RBAR{#1}#2}}
\newcommand{\GEN}[1]{\ABRACKET{#1}}
\newcommand{\BINDEF}[2]{\LBRACKET{\begin{array}{ll}
     #1\\
     #2
\end{array}}}

%%%%%%%
% Fourier Analysis
%%%%%%%

\newcommand{\ONEOTWOPI}{\frac{1}{2\pi}}
\newcommand{\FHAT}{\hat{f}(n)}
\newcommand{\FINT}{\int_{-\pi}^\pi}
\newcommand{\FINTWO}{\int_{0}^{2\pi}}
\newcommand{\FSUMN}[1]{\SUM{n=-#1}{#1}}
%\newcommand{\FSUM}{\SUMN{\infty}}
\newcommand{\EIN}[1]{e^{in#1}}
\newcommand{\NEIN}[1]{e^{-in#1}}
\newcommand{\INTALL}{\int_{-\infty}^{\infty}}
\newcommand{\FTINT}[1]{\INTALL #1 e^{2\pi inx\xi} dx}
\newcommand{\GAUSS}{e^{-\pi x^2}}

%%%%%%%
% formatting 
%%%%%%%

\newcommand{\LEFTBOLD}[1]{\noindent\textbf{#1}}
\newcommand{\SEQ}[1]{\{#1\,\}}
\newcommand{\WIP}{\footnote{work in progress}}
\newcommand{\QED}{\hfill\square}
\newcommand{\ts}{\textsuperscript}
\newcommand{\HLINE}{\noindent\rule{7in}{1pt}\\}

%%%%%%%
% Mathematical note taking (definitions, theorems, etc.)
%%%%%%%

\newcommand{\REM}{\noindent\textbf{\\Remark: }}
\newcommand{\DEF}{\noindent\textbf{\\Definition: }}
\newcommand{\THE}{\noindent\textbf{\\Theorem: }}
\newcommand{\COR}{\noindent\textbf{\\Corollary: }}
\newcommand{\LEM}{\noindent\textbf{\\Lemma: }}
\newcommand{\PROP}{\noindent\textbf{\\Proposition: }}
\newcommand{\PROOF}{\noindent\textbf{\\Proof: }}
\newcommand{\EXP}{\noindent\textbf{\\Example: }}
\newcommand{\TRICKS}{\noindent\textbf{\\Tricks: }}


%%%%%%%
% text highlighting
%%%%%%%

\newcommand{\B}[1]{\textbf{#1}}
\newcommand{\CAL}[1]{\mathcal{#1}}
\newcommand{\UL}[1]{\underline{#1}}

%%%%%%
% Linear Algebra
%%%%%%

\newcommand{\COLVECTOR}[1]{\PAREN{\begin{array}{c}
#1
\end{array} }}
\newcommand{\TWOXTWO}[4]{\PAREN{ \begin{array}{c c} #1&#2 \\ #3 & #4 \end{array} }}
\newcommand{\THREEXTHREE}[9]{\PAREN{ \begin{array}{c c c} #1&#2&#3 \\ #4 & #5 & #6 \\ #7 & #8 & #9 \end{array} }}
\newcommand{\NXN}{\PAREN{ \begin{array}{c c c c} 
			a_{11} & a_{12} & \cdots & a_{1n} \\
			a_{21} & a_{22} & \cdots & a_{2n} \\
			\vdots & \vdots & \ddots & a_{1n} \\
			a_{n1} & a_{n2} & \cdots & a_{nn} \\
		\end{array} }}
\newcommand{\SLR}{SL_2(\R)}
\newcommand{\GLR}{GL_2(\R)}
\DeclareMathOperator{\TR}{tr}
\DeclareMathOperator{\BIL}{Bil}
\DeclareMathOperator{\SPAN}{span}

%%%%%%%
%  White space
%%%%%%%

\newcommand{\BOXIT}[1]{\noindent\fbox{\parbox{\textwidth}{#1}}}


\newtheorem{theorem}{Theorem}[section]
\newtheorem{corollary}{Corollary}[theorem]
\newtheorem{lemma}[theorem]{Lemma}

\theoremstyle{definition}
\newtheorem{definition}[theorem]{Definition}
\newtheorem{prop}[theorem]{Proposition}

\theoremstyle{remark}
\newtheorem{remark}[theorem]{Remark}
\newtheorem{example}[theorem]{Example}
%\newtheorem*{proof}[theorem]{Proof}



\newcommand{\RED}[1]{\textcolor{red}{#1}}
\newcommand{\BLUE}[1]{\textcolor{blue}{#1}}

\begin{document}

\begin{center}
	\Large{\CLASSNAME -- \SEMESTER} \\
	\large{ w/Professor Berchenko-Kogan}
\end{center}
\begin{center}
	\STUDENTNAME \\
	\ASSIGNMENT -- \DUEDATE\\
\end{center} 

\noindent\textbf{Exercise 8.3: (The Differential of a Map)}.  Check that $F_*(X_p)$ is a derivation at $F(p)$ and that $F_*:T_pN\to T_{F(p)}M$ is a linear map.

\BLUE{Let $f$ be a germ at $F(p)$. Then
\begin{align*}
	(F_*(X_p))f = X_p(f \circ F)\in \R, \text{ for } f \in C^\infty_{F(p)}(M)
\end{align*}Need to show that $F_*$ has the Liebniz Condition that is, given $f,g$ of the same germ at $F(p)$ then
\begin{align*}
	(F_*(X_p))fg &= X_p(fg \circ F) \\
	&= X_p((f \circ F)(g\circ F)) \\
	&= (g \circ F)X_p(f \circ F) + (f\circ F)X_p(g\circ F) \\
	&= (g \circ F)(F_*(X_p))f + (f\circ F)(F_*(X_p))g 
\end{align*}
}

\begin{enumerate}[label=8.\arabic*.]
\setcounter{enumi}{1}
	\item \textbf{Differential of a linear map}.  Let $L : \R^n \to \R^m$ be a linear map.  For any $p\in \R^n$ there is a canonical identification $T_p(\R^n) \overset{\sim}\to\R^n$ given by 
	\begin{align*}
		\sum a^i \RBAR{\PART{}{x^i}}_p \mapsto \textbf{a} = \ABRACKET{a^1, \dots, a^n}
	\end{align*}Show that the differential $L_{*,p}: T_p(\R^n) \to T_{L(p)}(\R^m)$ is the map $L: \R^n \to \R^m$ itself, with the identification of the tangent spaces as above.
	
	\BLUE{Let $[L_k^j]$ be the Jacobian of $L$.  From equation 8.2 of the test \begin{align*}
		L_{*,p}\PAREN{\RBAR{\PART{}{x^j}}_p} &= \sum_{k} L_k^j\RBAR{\PART{}{x^k}}_{F(p)}  = \PART{F^i}{x^j}(p)		
	\end{align*}applying any vector $v = \SUM{i=1}{n} a_i \PART{}{x^i}$ we have
	\begin{align*}
	 	L_{*,p}(v) &= \sum_{i=1}^n a_i L_{*.p}\PAREN{\RBAR{ \PART{}{x^i}}_p} \\
	 	&= \sum_{i=1}^n a_i \sum_{k=1}^m L_k^i\RBAR{\PART{}{x^k}}_{F(p)}  \\
	 	&= \sum_{k=1}^m \PAREN{ \sum_{i=1}^n a_i L_k^i\RBAR{\PART{}{x^k}}_{F(p)}}  \\
	 	&= \sum_{k=1}^m L^k(v) \\
	 	&= L(v)
	\end{align*}
	}
	
	\item \textbf{Differental on a map}
	
	Fix a real number $\alpha$ and define $F: \R^2\to \R^2$ by 
	\begin{align*}
		\SQBRACKET{\begin{array}{c}
		u \\v
		\end{array}
		} = (u,v) = F(x,y) = \SQBRACKET{\begin{array}{cc}
		\cos \alpha & -\sin \alpha \\
		\sin \alpha & \cos \alpha		
		\end{array}
		}\SQBRACKET{\begin{array}{c}
		x \\y		
		\end{array}
		}
	\end{align*}Let $X = -y\PART{}{x}+x\PART{}{y}$ be a vector field on $\R^2$.  If $p = (x,y) \in \R^2$ and $F_*(X_p)=\RBAR{\PAREN{a\PART{}{u}+ b\PART{}{v}}}_{F(p)}$, find $a$ and $b$ in terms of $x,y$, and $\alpha$.
	
	\BLUE{Since $F$ is linear its Jacobian is constant and at $p$ we have 
	\begin{align*}
		p &= (x,y) = \SQBRACKET{\begin{array}{c}
			-y\\x
		\end{array} }\\
		F_*(X_p) &= \SQBRACKET{\begin{array}{cc}
		\cos \alpha & -\sin \alpha \\
		\sin \alpha & \cos \alpha		
		\end{array}
		}\SQBRACKET{\begin{array}{c}
		-y \\x		
		\end{array}
		} \\
		&= \SQBRACKET{\begin{array}{c}
		 -y\cos \alpha -x\sin\alpha \\
		 -y\sin \alpha + x\cos \alpha
		\end{array}
		} \\
		&= (-y\cos \alpha -x\sin\alpha)\PART{}{u} + (-y\sin \alpha + x\cos \alpha)\PART{}{V}
\end{align*}	therefore $a = -y\cos \alpha -x\sin\alpha$ and $b=-y\sin \alpha + x\cos \alpha$
	}
	
	\item \textbf{Transition matrix for coordinate vectors}
	
	Let $x,y$ be the standared coordinates on $\R^2$, and let $U$ be the open set
	\begin{align*}
		U = \R^2-\{(x,0)|x \ge 0\}.
	\end{align*}On $U$ the polar coordinates $r,\theta$ are uniquely defined by
	\begin{align*}
		x &= r\cos \theta \\
		y &= r\sin \theta, r >0, 0<\theta< 2\pi
	\end{align*}find $\PART{}{r}$ and $\PART{}{\theta}$ in terms of $\PART{}{x}$ and $\PART{}{y}.$
	
	\BLUE{Given any $f: U \to \R$  We have 
	\begin{align*}
		\PART{f}{r} &= \PART{f}{x}\PART{x}{r} + \PART{f}{y}\PART{y}{r} = \cos \theta \PART{f}{x} + \sin \theta \PART{f}{y} \\
		\PART{f}{\theta} &= \PART{f}{x}\PART{x}{\theta} + \PART{f}{y}\PART{y}{\theta} = -r\sin \theta \PART{f}{x} + r\cos \theta \PART{f}{y} 
\end{align*}	that is the expression may be expressed as 
\begin{align*}
		\PART{}{r} &= \cos \theta \PART{}{x} + \sin \theta \PART{}{y} \\
		\PART{}{\theta} &= -r\sin \theta \PART{}{x} + r\cos \theta \PART{}{y} 
\end{align*}
	}
\end{enumerate}
\end{document}
