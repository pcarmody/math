\documentclass[10pt,a4paper]{report}
\usepackage[utf8]{inputenc}
\usepackage{amsmath}
\usepackage{amsfonts}
\usepackage{amssymb}
\usepackage{amsthm}
\usepackage{hyperref}

\usepackage{multicol}
\usepackage{fancyhdr}
\usepackage[inline]{enumitem}
\usepackage{tikz}
\usepackage{tikz-cd}
\usetikzlibrary{calc}
\usetikzlibrary{shapes.geometric}
\usepackage[margin=0.5in]{geometry}
\usepackage{xcolor}

\hypersetup{
    colorlinks=true,
    linkcolor=blue,
    filecolor=magenta,      
    urlcolor=cyan,
    pdftitle={Tensors},
    pdfpagemode=FullScreen,
    }

%\urlstyle{same}

\newcommand{\CLASSNAME}{Math 5050 -- Special Topics: Manifolds}
\newcommand{\STUDENTNAME}{Paul Carmody}
\newcommand{\ASSIGNMENT}{Assignment 2 }
\newcommand{\DUEDATE}{Februaray 20, 2025}
\newcommand{\SEMESTER}{Spring 2025}
\newcommand{\SCHEDULE}{MW 12:30 - 1:45}
\newcommand{\ROOM}{Remote}

\newcommand{\MMN}{M_{m\times n}}
\newcommand{\FF}{\mathcal{F}}
\newcommand{\RANGE}{\text{range}}

\pagestyle{fancy}
\fancyhf{}
\chead{ \fancyplain{}{\CLASSNAME} }
%\chead{ \fancyplain{}{\STUDENTNAME} }
\rhead{\thepage}
\newcommand{\LET}{\text{Let }}
%\newcommand{\IF}{\text{if }}
\newcommand{\AND}{\text{ and }}
\newcommand{\OR}{\text{ or }}
\newcommand{\FORSOME}{\text{ for some }}
\newcommand{\FORALL}{\text{ for all }}
\newcommand{\WHERE}{\text{ where }}
\newcommand{\WTS}{\text{ WTS }}
\newcommand{\WLOG}{\text{ WLOG }}
\newcommand{\BS}{\backslash}
\newcommand{\DEFINE}[1]{\textbf{\emph{#1}}}
\newcommand{\IF}{$(\Rightarrow)$}
\newcommand{\ONLYIF}{$(\Leftarrow)$}
\newcommand{\ITH}{\textsuperscript{th} }
\newcommand{\FST}{\textsuperscript{st} }
\newcommand{\SND}{\textsuperscript{nd} }
\newcommand{\TRD}{\textsuperscript{rd} }
\newcommand{\INV}{\textsuperscript{-1} }

\newcommand{\XXX}{\mathfrak{X}}
\newcommand{\MMM}{\mathfrak{M}}
%\newcommand{\????}{\textfrak{A}}
%\newcommand{\????}{\textgoth{A}}
%\newcommand{\????}{\textswab{A}}

\DeclareMathOperator{\DER}{Der}
\DeclareMathOperator{\SGN}{sgn}

%%%%%%%
% derivatives
%%%%%%%

\newcommand{\PART}[2]{\frac{\partial #1}{\partial #2}}
\newcommand{\SPART}[2]{\frac{\partial^2 #1}{\partial #2^2}}
\newcommand{\DERIV}[2]{\frac{d #1}{d #2}}
\newcommand{\LAPLACIAN}[1]{\frac{\partial^2 #1}{\partial x^2} + \frac{\partial^2 #1}{\partial y^2}}

%%%%%%%
% sum, product, union, intersections
%%%%%%%

\newcommand{\SUM}[2]{\underset{#1}{\overset{#2}{\sum}}}
\newcommand{\PROD}[2]{\underset{#1}{\overset{#2}{\prod}}}
\newcommand{\UNION}[2]{\underset{#1}{\overset{#2}{\bigcup}}}
\newcommand{\INTERSECT}[2]{\underset{#1}{\overset{#2}{\bigcap}}}
\newcommand{\FSUM}{\SUM{n=-\infty}{\infty}}
       

%%%%%%%
% supremum and infimum
%%%%%%%

\newcommand{\SUP}[1]{\underset{#1}\sup \,}
\newcommand{\INF}[1]{\underset{#1}\inf \,}
\newcommand{\MAX}[1]{\underset{#1}\max \,}
\newcommand{\MIN}[1]{\underset{#1}\min \,}

%%%%%%%
% infinite sums, limits
%%%%%%%

\newcommand{\SUMK}{\SUM{k=1}{\infty}}
\newcommand{\SUMN}{\SUM{n=1}{\infty}}
\newcommand{\SUMKZ}{\SUM{k=0}{\infty}}
\newcommand{\LIM}[1]{\underset{#1}\lim\,}
\newcommand{\IWOB}[1]{\LIM{#1 \to \infty}}
\newcommand{\LIMK}{\IWOB{k}}
\newcommand{\LIMN}{\IWOB{n}}
\newcommand{\LIMX}{\IWOB{x}}
\newcommand{\NIWOB}{\LIM{n \to \infty}}
\newcommand{\LIMSUPK}{\underset{k\to\infty}\limsup \,}
\newcommand{\LIMSUPN}{\underset{n\to\infty}\limsup \,}
\newcommand{\LIMINFK}{\underset{k\to\infty}\liminf \,}
\newcommand{\LIMINFN}{\underset{n\to\infty}\liminf \,}
\newcommand{\ROOTRULE}[1]{\LIMSUPK \BARS{#1}^{1/k}}

\newcommand{\CUPK}{\bigcup_{k=1}^{\infty}}
\newcommand{\CAPK}{\bigcap_{k=1}^{\infty}}
\newcommand{\CUPN}{\bigcup_{n=1}^{\infty}}
\newcommand{\CAPN}{\bigcap_{n=1}^{\infty}}

%%%%%%%
% number systems (real, rational, etc.)
%%%%%%%

\newcommand{\REALS}{\mathbb{R}}
\newcommand{\RATIONALS}{\mathbb{Q}}
\newcommand{\IRRATIONALS}{\REALS \backslash \RATIONALS}
\newcommand{\INTEGERS}{\mathbb{Z}}
\newcommand{\NUMBERS}{\mathbb{N}}
\newcommand{\COMPLEX}{\mathbb{C}}
\newcommand{\DISC}{\mathbb{D}}
\newcommand{\HPLANE}{\mathbb{H}}

\newcommand{\R}{\mathbb{R}}
\newcommand{\Q}{\mathbb{Q}}
\newcommand{\Z}{\mathbb{Z}}
\newcommand{\N}{\mathbb{N}}
\newcommand{\C}{\mathbb{C}}
\newcommand{\T}{\mathbb{T}}
\newcommand{\COUNTABLE}{\aleph_0}
\newcommand{\UNCOUNTABLE}{\aleph_1}


%%%%%%%
% Arithmetic/Algebraic operators
%%%%%%%


\DeclareMathOperator{\MOD}{mod}
%\newcommand{\MOD}[1]{\mod #1}
\newcommand{\BAR}[1]{\overline{#1}}
\newcommand{\LCM}{\text{ lcm}}
\newcommand{\ZMOD}[1]{\Z/#1\Z}
\DeclareMathOperator{\VAR}{Var}
%%%%%%%
% complex operators
%%%%%%%

\DeclareMathOperator{\RR}{Re}
%\newcommand{\RE}{\text{Re}}
\DeclareMathOperator{\IM}{Im}
%\newcommand{\IM}{\text{Im}}
\newcommand{\CONJ}[1]{\overline{#1}}
\DeclareMathOperator{\LOG}{Log}
%\newcommand{\LOG}{\text{ Log }}
\newcommand{\RES}[2]{\underset{#1}{\text{res}} #2}

%%%%%%%
% Group operators
%%%%%%%

\newcommand{\AUT}{\text{Aut}\,}
\newcommand{\KER}{\text{ker}\,}
\newcommand{\END}{\text{End}}
\newcommand{\HOM}{\text{Hom}}
\newcommand{\CYCLE}[1]{(\begin{array}{cccccccccc}
		#1
	\end{array})}
\newcommand{\SUBGROUP}{\underset{\text{group}}\subseteq}	
%\newcommand{\SUBGROUP}{\subseteq_g}
\newcommand{\SUBRING}{\underset{\text{ring}}\subseteq}
\newcommand{\SUBMOD}{\underset{\text{mod}}\subseteq}
\newcommand{\SUBFIELD}{\underset{\text{field}}\subseteq}
\newcommand{\ISO}{\underset{\text{iso}}\longrightarrow}
\newcommand{\HOMO}{\underset{\text{homo}}\longrightarrow}

%%%%%%%
% grouping (parenthesis, absolute value, square, multi-level brackets).
%%%%%%%

\newcommand{\PAREN}[1]{\left (\, #1 \,\right )}
\newcommand{\BRACKET}[1]{\left \{\, #1 \,\right \}}
\newcommand{\SQBRACKET}[1]{\left [\, #1 \,\right ]}
\newcommand{\ABRACKET}[1]{\left \langle\, #1 \,\right \rangle}
\newcommand{\BARS}[1]{\left |\, #1 \,\right |}
\newcommand{\DBARS}[1]{\left \| \, #1 \,\right \|}
\newcommand{\LBRACKET}[1]{\left \{ #1 \right .} 
\newcommand{\RBRACKET}[1]{\left . #1 \right \]}
\newcommand{\RBAR}[1]{\left . #1 \, \right |}
\newcommand{\LBAR}[1]{\left | \, #1 \right .}
\newcommand{\BLBRACKET}[2]{\BRACKET{\RBAR{#1}#2}}
\newcommand{\GEN}[1]{\ABRACKET{#1}}
\newcommand{\BINDEF}[2]{\LBRACKET{\begin{array}{ll}
     #1\\
     #2
\end{array}}}

%%%%%%%
% Fourier Analysis
%%%%%%%

\newcommand{\ONEOTWOPI}{\frac{1}{2\pi}}
\newcommand{\FHAT}{\hat{f}(n)}
\newcommand{\FINT}{\int_{-\pi}^\pi}
\newcommand{\FINTWO}{\int_{0}^{2\pi}}
\newcommand{\FSUMN}[1]{\SUM{n=-#1}{#1}}
%\newcommand{\FSUM}{\SUMN{\infty}}
\newcommand{\EIN}[1]{e^{in#1}}
\newcommand{\NEIN}[1]{e^{-in#1}}
\newcommand{\INTALL}{\int_{-\infty}^{\infty}}
\newcommand{\FTINT}[1]{\INTALL #1 e^{2\pi inx\xi} dx}
\newcommand{\GAUSS}{e^{-\pi x^2}}

%%%%%%%
% formatting 
%%%%%%%

\newcommand{\LEFTBOLD}[1]{\noindent\textbf{#1}}
\newcommand{\SEQ}[1]{\{#1\,\}}
\newcommand{\WIP}{\footnote{work in progress}}
\newcommand{\QED}{\hfill\square}
\newcommand{\ts}{\textsuperscript}
\newcommand{\HLINE}{\noindent\rule{7in}{1pt}\\}

%%%%%%%
% Mathematical note taking (definitions, theorems, etc.)
%%%%%%%

\newcommand{\REM}{\noindent\textbf{\\Remark: }}
\newcommand{\DEF}{\noindent\textbf{\\Definition: }}
\newcommand{\THE}{\noindent\textbf{\\Theorem: }}
\newcommand{\COR}{\noindent\textbf{\\Corollary: }}
\newcommand{\LEM}{\noindent\textbf{\\Lemma: }}
\newcommand{\PROP}{\noindent\textbf{\\Proposition: }}
\newcommand{\PROOF}{\noindent\textbf{\\Proof: }}
\newcommand{\EXP}{\noindent\textbf{\\Example: }}
\newcommand{\TRICKS}{\noindent\textbf{\\Tricks: }}


%%%%%%%
% text highlighting
%%%%%%%

\newcommand{\B}[1]{\textbf{#1}}
\newcommand{\CAL}[1]{\mathcal{#1}}
\newcommand{\UL}[1]{\underline{#1}}

%%%%%%
% Linear Algebra
%%%%%%

\newcommand{\COLVECTOR}[1]{\PAREN{\begin{array}{c}
#1
\end{array} }}
\newcommand{\TWOXTWO}[4]{\PAREN{ \begin{array}{c c} #1&#2 \\ #3 & #4 \end{array} }}
\newcommand{\DTWOXTWO}[4]{\BARS{ \begin{array}{c c} #1&#2 \\ #3 & #4 \end{array} }}
\newcommand{\THREEXTHREE}[9]{\PAREN{ \begin{array}{c c c} #1&#2&#3 \\ #4 & #5 & #6 \\ #7 & #8 & #9 \end{array} }}
\newcommand{\DTHREEXTHREE}[9]{\BARS{ \begin{array}{c c c} #1&#2&#3 \\ #4 & #5 & #6 \\ #7 & #8 & #9 \end{array} }}
\newcommand{\NXN}{\PAREN{ \begin{array}{c c c c} 
			a_{11} & a_{12} & \cdots & a_{1n} \\
			a_{21} & a_{22} & \cdots & a_{2n} \\
			\vdots & \vdots & \ddots & a_{1n} \\
			a_{n1} & a_{n2} & \cdots & a_{nn} \\
		\end{array} }}
\newcommand{\SLR}{SL_2(\R)}
\newcommand{\GLR}{GL_2(\R)}
\DeclareMathOperator{\TR}{tr}
\DeclareMathOperator{\BIL}{Bil}
\DeclareMathOperator{\SPAN}{span}

%%%%%%%
%  White space
%%%%%%%

\newcommand{\BOXIT}[1]{\noindent\fbox{\parbox{\textwidth}{#1}}}


\newtheorem{theorem}{Theorem}[section]
\newtheorem{corollary}{Corollary}[theorem]
\newtheorem{lemma}[theorem]{Lemma}

\theoremstyle{definition}
\newtheorem{definition}[theorem]{Definition}
\newtheorem{prop}[theorem]{Proposition}

\theoremstyle{remark}
\newtheorem{remark}[theorem]{Remark}
\newtheorem{example}[theorem]{Example}
%\newtheorem*{proof}[theorem]{Proof}



\newcommand{\RED}[1]{\textcolor{red}{#1}}
\newcommand{\BLUE}[1]{\textcolor{blue}{#1}}

\begin{document}

\begin{center}
	\Large{\CLASSNAME -- \SEMESTER} \\
	\large{ w/Professor Berchenko-Kogan}
\end{center}
\begin{center}
	\STUDENTNAME \\
	\ASSIGNMENT -- \DUEDATE\\
\end{center} 

\noindent Section 3: 1, 2, 3, 7, 8, 9\\

\begin{enumerate}[label=3.\arabic*.]

	\item \textbf{Tensor Product of covectors}
	
	Let $e_1, \dots, e_n$ be a basis for a vector space $V$ and let $\alpha^1, \dots, \alpha^n$ be its dual basis in $V^\vee$.  Suppose $g_{ij}\in \R^{n\times m}$ is an $n \times m$ matrix Define a bilinear function $f: V\times V \to \R$ by 
	\begin{align*}
		f(v,w) = \sum_{i\le i,j, n} g_{ij}v^i w^j
	\end{align*}for $v=\sum v^j e_i$ and $w=\sum w^j e_j$ in $V$.  Describe $f$ in terms of the tensor products of $\alpha^i$ and $\alpha^j, 1 \le i,j \le n$.
	
	\BLUE{\begin{align*}
		\alpha^i(e_j) &= \delta^j_i & (1)\\
		\alpha^i(v) &= \alpha^i\PAREN{\sum_{j=1}^n v^je_j} \\
		 & =\sum_{j=1}^n \alpha^i(v^je_j) & \alpha^i \text{ is linear } \\
		 & =\sum_{j=1}^n v^j\alpha^i(e_j) & v^j \text{ is a scalar }\\
		 & =\sum_{j=1}^n v^j\delta^i_j=v^i & \text{apply (1)}\\
		(\alpha^i\otimes\alpha^j)(v,w) &= \alpha^i(v)\alpha^j(w) = v^iw^j\\
		\therefore \sum_{i\le i,j, n} g_{ij}v^i w^j &= \sum_{i\le i,j, n} g_{ij}\,(\alpha^i\otimes\alpha^j)(v,w) 
	\end{align*}
	}
	
	\item \textbf{Hyperplanes}
	
	\begin{enumerate}[label=(\alph*)]
	
		\item Let $V$ be a vector space of dimension $n$ and $f:V \to \R$ a nonzero linear functional.  Show that $\dim\ker f = n-1$.  A linear subspace of $V$ of dimesnion $n-1$ is called a \textit{hyperplane} in $V$.
		
		\BLUE{\begin{align*}
%			\ker f &= \{x\in X: f(x) = 0\} \\
			\dim V &= \dim \text{range} (f) + \dim \ker (f) \\
			\dim \ker (f) &= \dim V - \dim \text{range} (f) \\
			&= n-1
		\end{align*}
		}
		
		\item Show that a nonzero linear functional on a vector space $V$ is determined up to a multiplicative constant by its kernel, a hyperplane in $V$.  In other words, if $f$ and $g: V\to \R$ are nonzero linear functionals and $\ker f = \ker g$, then $g=cf$ for some constant $c \in \R$.
		
		\BLUE{\begin{align*}
			\LET v = (y+z) &\in V \AND f(y) \in \RANGE(f), z \in \ker(f) \\
			 u = (x +w) &\in V \AND g(x) \in \RANGE(g), z \in \ker(g) \\
			 g(v) &= g(y)+g(z) = g(y) \in \RANGE(f)
		\end{align*}
		}
		
	\end{enumerate}
		
	\item \textbf{A basis for $k$-tensors}
		
	Let $V$ be a vector space of dimension $n$ with basis $e_i,\dots,e_n$.  Let $\alpha^1,\dots, \alpha^n$ be the dual basis in $V^\vee$  Show that a basis for the space $L_k(V)$ of $k$-linear functions on $V$ is $\{\alpha^{i_1}\otimes \cdots \otimes \alpha^{i_k}\}$ for all multi-indices $(i_1,\dots, i_k)$ (not just the strictly ascending multi-indices as for $A_k(L)$).  In particular, this show that $\dim L_k(V) = n^k$.  (This problem generalizes Problem 3.1..)
	
	\RED{We need to show three things:
	\begin{enumerate}
		\item That span$\{\alpha^{i_1}\otimes \cdots \otimes \alpha^{i_k}\} = L_k(V)$.
		\item that $\{\alpha^{i_1}\otimes \cdots \otimes \alpha^{i_k}\}$ is linearly independent.
		\item that this is independend of order.
	\end{enumerate}		
	\textbf{$\bullet$ Proving the Span} A proof by mathematical induction.  Constructing a bilinear function from functionals is simply done with $(f \otimes g)(v,w)= f(v)g(w)$.  Then
	}
		
	\setcounter{enumi}{6}
	\item \textbf{Transformation rule for a wedge product of covectors}
		
		Suppose two set so of covectors on a vector space $V$.  $\beta^1,\dots, \beta^k$ and $\gamma^i,\dots,\gamma^k$, are related by 
		\begin{align*}
			\beta^i=\sum_{j=1}^k a^i_j\gamma^i, \, i=1,\dots,k
		\end{align*}for a $k \times k$ matrix $A=[a_j^i]$.  Show that
		\begin{align*}
			\beta^1 \wedge\cdots\wedge \beta^k = (\det A)\gamma^1\wedge\dots\wedge\gamma^k.
		\end{align*}
		
	\BLUE{\begin{align*}
		\LET \beta, \gamma &\in \mathcal{M}_{n \times n}(V^\vee) \\
		\beta &= \SQBRACKET{ \beta^i } \AND \beta(v_1,\dots,v_k) = \SQBRACKET{ \beta^i }(v_1,\dots,v_n) = \SQBRACKET{\beta^i(v_j)}\\
		\gamma &= \SQBRACKET{ \gamma^i } \AND \gamma(v_1,\dots,v_k) = \SQBRACKET{ \gamma^i }(v_1,\dots,v_n) = \SQBRACKET{\gamma^i(v_j)}\\
		A &= [ a^i_j ] \\
		(\beta^1 \wedge \cdots \wedge \beta^k)(v_1, \dots, v_k) &= \det [ \beta^i(v_j) ] = \det \beta(v_1,\dots,v_k) \\
		(\gamma^1 \wedge \cdots \wedge \gamma^k)(v_1, \dots, v_k) &= \det [ \gamma^i (v_j) ] = \det \gamma(v_1,\dots,v_k)\\
	\end{align*}we can see that 
	\begin{align*}
		\beta^i=\sum_{j=1}^k a^i_j\gamma^i &\implies \beta = A \cdot \gamma \AND \beta(v_1,\dots,v_k) = A \cdot \gamma(v_1,\dots,v_k)\\
		\det\beta &= \det (A \cdot \gamma) = \det A \cdot \det \gamma \\
		\det\beta(v_1,\dots,v_k) &=  \det A \cdot \det \gamma(v_1,\dots,v_k) \\
		(\beta^1 \wedge \cdots \wedge \beta^k)(v_1, \dots, v_k) &=  \det A (\gamma^1 \wedge \cdots \wedge \gamma^k)(v_1, \dots, v_k) \\
		\beta^1 \wedge \cdots \wedge \beta^k &=  \det A (\gamma^1 \wedge \cdots \wedge \gamma^k)
	\end{align*}
	}
	
	\item \textbf{Transformation rule for $k$-covectors}
		
	Let $f$ be a $k$-covector on a vector space $V$.  Suppose two sets of vectors $u_1, \dots, u_k$ and $v_1, \dots, v_k$ in $V$ are related by 
		\begin{align*}
			u_j =\sum_{i=1}^k a_j^iv_i,\, j=1,\dots,k,
		\end{align*}for $k\times k$ matrix $A=[a_j^i]$.  Show that 
		\begin{align*}
			f(u_1,\dots,u_k)=(\det A)f(v_1,\dots, v_k).
		\end{align*}
		
	\BLUE{\begin{align*}
			f(u_1,\dots,u_k) &= f\PAREN{\sum_{i_1=1}^k a_1^{i_1}v_{i_1}, \sum_{i_2=1}^k a_2^{i_2}v_{i_2}, \dots, \sum_{{i_k}=1}^k a_k^{i_k}v_{i_k}} \\
			&= \sum_{i_1=1}^k a_1^{i_1} \sum_{i_2=1}^k a_2^{i_2} \cdots \sum_{{i_k}=1}^k a_k^{i_k} f(v_{i_1},v_{i_2}, \dots,v_{i_k})
		\end{align*}	
		}
	
	\item \textbf{Vanishing of a covector of top degree}
	
	Let $V$ be a vector space of dimension $n$.  Prove that if an $n$-covector $\omega$ vanishes on a basis $e_1,\dots,e_n$ for $V$.  then $\omega$ is the zero covector on $V$.

\end{enumerate}

\end{document}
