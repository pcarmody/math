\documentclass[10pt,a4paper]{report}
\usepackage[utf8]{inputenc}
\usepackage{amsmath}
\usepackage{amsfonts}
\usepackage{amssymb}
\usepackage{amsthm}
\usepackage{hyperref}

\usepackage{multicol}
\usepackage{fancyhdr}
\usepackage[inline]{enumitem}
\usepackage{tikz}
\usepackage{tikz-cd}
\usetikzlibrary{calc}
\usetikzlibrary{shapes.geometric}
\usepackage[margin=0.5in]{geometry}
\usepackage{xcolor}

\hypersetup{
    colorlinks=true,
    linkcolor=blue,
    filecolor=magenta,      
    urlcolor=cyan,
    pdftitle={Tensors},
    pdfpagemode=FullScreen,
    }

%\urlstyle{same}

\newcommand{\CLASSNAME}{Math 5050 -- Special Topics: Manifolds}
\newcommand{\STUDENTNAME}{Paul Carmody}
\newcommand{\ASSIGNMENT}{Homework \#1 }
\newcommand{\DUEDATE}{January 28, 2025}
\newcommand{\SEMESTER}{Spring 2025}
\newcommand{\SCHEDULE}{MW 9:30 - 10:45}
\newcommand{\ROOM}{Remote}

\newcommand{\MMN}{M_{m\times n}}
\newcommand{\FF}{\mathcal{F}}

\pagestyle{fancy}
\fancyhf{}
\chead{ \fancyplain{}{\CLASSNAME} }
%\chead{ \fancyplain{}{\STUDENTNAME} }
\rhead{\thepage}
\newcommand{\LET}{\text{Let }}
%\newcommand{\IF}{\text{if }}
\newcommand{\AND}{\text{ and }}
\newcommand{\OR}{\text{ or }}
\newcommand{\FORSOME}{\text{ for some }}
\newcommand{\FORALL}{\text{ for all }}
\newcommand{\WHERE}{\text{ where }}
\newcommand{\WTS}{\text{ WTS }}
\newcommand{\WLOG}{\text{ WLOG }}
\newcommand{\BS}{\backslash}
\newcommand{\DEFINE}[1]{\textbf{\emph{#1}}}
\newcommand{\IF}{$(\Rightarrow)$}
\newcommand{\ONLYIF}{$(\Leftarrow)$}
\newcommand{\ITH}{\textsuperscript{th} }
\newcommand{\FST}{\textsuperscript{st} }
\newcommand{\SND}{\textsuperscript{nd} }
\newcommand{\TRD}{\textsuperscript{rd} }
\newcommand{\INV}{\textsuperscript{-1} }

\newcommand{\XXX}{\mathfrak{X}}
\newcommand{\MMM}{\mathfrak{M}}
%\newcommand{\????}{\textfrak{A}}
%\newcommand{\????}{\textgoth{A}}
%\newcommand{\????}{\textswab{A}}

\DeclareMathOperator{\DER}{Der}
\DeclareMathOperator{\SGN}{sgn}

%%%%%%%
% derivatives
%%%%%%%

\newcommand{\PART}[2]{\frac{\partial #1}{\partial #2}}
\newcommand{\SPART}[2]{\frac{\partial^2 #1}{\partial #2^2}}
\newcommand{\DERIV}[2]{\frac{d #1}{d #2}}
\newcommand{\LAPLACIAN}[1]{\frac{\partial^2 #1}{\partial x^2} + \frac{\partial^2 #1}{\partial y^2}}

%%%%%%%
% sum, product, union, intersections
%%%%%%%

\newcommand{\SUM}[2]{\underset{#1}{\overset{#2}{\sum}}}
\newcommand{\PROD}[2]{\underset{#1}{\overset{#2}{\prod}}}
\newcommand{\UNION}[2]{\underset{#1}{\overset{#2}{\bigcup}}}
\newcommand{\INTERSECT}[2]{\underset{#1}{\overset{#2}{\bigcap}}}
\newcommand{\FSUM}{\SUM{n=-\infty}{\infty}}
       

%%%%%%%
% supremum and infimum
%%%%%%%

\newcommand{\SUP}[1]{\underset{#1}\sup \,}
\newcommand{\INF}[1]{\underset{#1}\inf \,}
\newcommand{\MAX}[1]{\underset{#1}\max \,}
\newcommand{\MIN}[1]{\underset{#1}\min \,}

%%%%%%%
% infinite sums, limits
%%%%%%%

\newcommand{\SUMK}{\SUM{k=1}{\infty}}
\newcommand{\SUMN}{\SUM{n=1}{\infty}}
\newcommand{\SUMKZ}{\SUM{k=0}{\infty}}
\newcommand{\LIM}[1]{\underset{#1}\lim\,}
\newcommand{\IWOB}[1]{\LIM{#1 \to \infty}}
\newcommand{\LIMK}{\IWOB{k}}
\newcommand{\LIMN}{\IWOB{n}}
\newcommand{\LIMX}{\IWOB{x}}
\newcommand{\NIWOB}{\LIM{n \to \infty}}
\newcommand{\LIMSUPK}{\underset{k\to\infty}\limsup \,}
\newcommand{\LIMSUPN}{\underset{n\to\infty}\limsup \,}
\newcommand{\LIMINFK}{\underset{k\to\infty}\liminf \,}
\newcommand{\LIMINFN}{\underset{n\to\infty}\liminf \,}
\newcommand{\ROOTRULE}[1]{\LIMSUPK \BARS{#1}^{1/k}}

\newcommand{\CUPK}{\bigcup_{k=1}^{\infty}}
\newcommand{\CAPK}{\bigcap_{k=1}^{\infty}}
\newcommand{\CUPN}{\bigcup_{n=1}^{\infty}}
\newcommand{\CAPN}{\bigcap_{n=1}^{\infty}}

%%%%%%%
% number systems (real, rational, etc.)
%%%%%%%

\newcommand{\REALS}{\mathbb{R}}
\newcommand{\RATIONALS}{\mathbb{Q}}
\newcommand{\IRRATIONALS}{\REALS \backslash \RATIONALS}
\newcommand{\INTEGERS}{\mathbb{Z}}
\newcommand{\NUMBERS}{\mathbb{N}}
\newcommand{\COMPLEX}{\mathbb{C}}
\newcommand{\DISC}{\mathbb{D}}
\newcommand{\HPLANE}{\mathbb{H}}

\newcommand{\R}{\mathbb{R}}
\newcommand{\Q}{\mathbb{Q}}
\newcommand{\Z}{\mathbb{Z}}
\newcommand{\N}{\mathbb{N}}
\newcommand{\C}{\mathbb{C}}
\newcommand{\T}{\mathbb{T}}
\newcommand{\COUNTABLE}{\aleph_0}
\newcommand{\UNCOUNTABLE}{\aleph_1}


%%%%%%%
% Arithmetic/Algebraic operators
%%%%%%%


\DeclareMathOperator{\MOD}{mod}
%\newcommand{\MOD}[1]{\mod #1}
\newcommand{\BAR}[1]{\overline{#1}}
\newcommand{\LCM}{\text{ lcm}}
\newcommand{\ZMOD}[1]{\Z/#1\Z}
\DeclareMathOperator{\VAR}{Var}
%%%%%%%
% complex operators
%%%%%%%

\DeclareMathOperator{\RR}{Re}
%\newcommand{\RE}{\text{Re}}
\DeclareMathOperator{\IM}{Im}
%\newcommand{\IM}{\text{Im}}
\newcommand{\CONJ}[1]{\overline{#1}}
\DeclareMathOperator{\LOG}{Log}
%\newcommand{\LOG}{\text{ Log }}
\newcommand{\RES}[2]{\underset{#1}{\text{res}} #2}

%%%%%%%
% Group operators
%%%%%%%

\newcommand{\AUT}{\text{Aut}\,}
\newcommand{\KER}{\text{ker}\,}
\newcommand{\END}{\text{End}}
\newcommand{\HOM}{\text{Hom}}
\newcommand{\CYCLE}[1]{(\begin{array}{cccccccccc}
		#1
	\end{array})}
\newcommand{\SUBGROUP}{\underset{\text{group}}\subseteq}	
%\newcommand{\SUBGROUP}{\subseteq_g}
\newcommand{\SUBRING}{\underset{\text{ring}}\subseteq}
\newcommand{\SUBMOD}{\underset{\text{mod}}\subseteq}
\newcommand{\SUBFIELD}{\underset{\text{field}}\subseteq}
\newcommand{\ISO}{\underset{\text{iso}}\longrightarrow}
\newcommand{\HOMO}{\underset{\text{homo}}\longrightarrow}

%%%%%%%
% grouping (parenthesis, absolute value, square, multi-level brackets).
%%%%%%%

\newcommand{\PAREN}[1]{\left (\, #1 \,\right )}
\newcommand{\BRACKET}[1]{\left \{\, #1 \,\right \}}
\newcommand{\SQBRACKET}[1]{\left [\, #1 \,\right ]}
\newcommand{\ABRACKET}[1]{\left \langle\, #1 \,\right \rangle}
\newcommand{\BARS}[1]{\left |\, #1 \,\right |}
\newcommand{\DBARS}[1]{\left \| \, #1 \,\right \|}
\newcommand{\LBRACKET}[1]{\left \{ #1 \right .} 
\newcommand{\RBRACKET}[1]{\left . #1 \right \]}
\newcommand{\RBAR}[1]{\left . #1 \, \right |}
\newcommand{\LBAR}[1]{\left | \, #1 \right .}
\newcommand{\BLBRACKET}[2]{\BRACKET{\RBAR{#1}#2}}
\newcommand{\GEN}[1]{\ABRACKET{#1}}
\newcommand{\BINDEF}[2]{\LBRACKET{\begin{array}{ll}
     #1\\
     #2
\end{array}}}

%%%%%%%
% Fourier Analysis
%%%%%%%

\newcommand{\ONEOTWOPI}{\frac{1}{2\pi}}
\newcommand{\FHAT}{\hat{f}(n)}
\newcommand{\FINT}{\int_{-\pi}^\pi}
\newcommand{\FINTWO}{\int_{0}^{2\pi}}
\newcommand{\FSUMN}[1]{\SUM{n=-#1}{#1}}
%\newcommand{\FSUM}{\SUMN{\infty}}
\newcommand{\EIN}[1]{e^{in#1}}
\newcommand{\NEIN}[1]{e^{-in#1}}
\newcommand{\INTALL}{\int_{-\infty}^{\infty}}
\newcommand{\FTINT}[1]{\INTALL #1 e^{2\pi inx\xi} dx}
\newcommand{\GAUSS}{e^{-\pi x^2}}

%%%%%%%
% formatting 
%%%%%%%

\newcommand{\LEFTBOLD}[1]{\noindent\textbf{#1}}
\newcommand{\SEQ}[1]{\{#1\,\}}
\newcommand{\WIP}{\footnote{work in progress}}
\newcommand{\QED}{\hfill\square}
\newcommand{\ts}{\textsuperscript}
\newcommand{\HLINE}{\noindent\rule{7in}{1pt}\\}

%%%%%%%
% Mathematical note taking (definitions, theorems, etc.)
%%%%%%%

\newcommand{\REM}{\noindent\textbf{\\Remark: }}
\newcommand{\DEF}{\noindent\textbf{\\Definition: }}
\newcommand{\THE}{\noindent\textbf{\\Theorem: }}
\newcommand{\COR}{\noindent\textbf{\\Corollary: }}
\newcommand{\LEM}{\noindent\textbf{\\Lemma: }}
\newcommand{\PROP}{\noindent\textbf{\\Proposition: }}
\newcommand{\PROOF}{\noindent\textbf{\\Proof: }}
\newcommand{\EXP}{\noindent\textbf{\\Example: }}
\newcommand{\TRICKS}{\noindent\textbf{\\Tricks: }}


%%%%%%%
% text highlighting
%%%%%%%

\newcommand{\B}[1]{\textbf{#1}}
\newcommand{\CAL}[1]{\mathcal{#1}}
\newcommand{\UL}[1]{\underline{#1}}

%%%%%%
% Linear Algebra
%%%%%%

\newcommand{\COLVECTOR}[1]{\PAREN{\begin{array}{c}
#1
\end{array} }}
\newcommand{\TWOXTWO}[4]{\PAREN{ \begin{array}{c c} #1&#2 \\ #3 & #4 \end{array} }}
\newcommand{\DTWOXTWO}[4]{\BARS{ \begin{array}{c c} #1&#2 \\ #3 & #4 \end{array} }}
\newcommand{\THREEXTHREE}[9]{\PAREN{ \begin{array}{c c c} #1&#2&#3 \\ #4 & #5 & #6 \\ #7 & #8 & #9 \end{array} }}
\newcommand{\DTHREEXTHREE}[9]{\BARS{ \begin{array}{c c c} #1&#2&#3 \\ #4 & #5 & #6 \\ #7 & #8 & #9 \end{array} }}
\newcommand{\NXN}{\PAREN{ \begin{array}{c c c c} 
			a_{11} & a_{12} & \cdots & a_{1n} \\
			a_{21} & a_{22} & \cdots & a_{2n} \\
			\vdots & \vdots & \ddots & a_{1n} \\
			a_{n1} & a_{n2} & \cdots & a_{nn} \\
		\end{array} }}
\newcommand{\SLR}{SL_2(\R)}
\newcommand{\GLR}{GL_2(\R)}
\DeclareMathOperator{\TR}{tr}
\DeclareMathOperator{\BIL}{Bil}
\DeclareMathOperator{\SPAN}{span}

%%%%%%%
%  White space
%%%%%%%

\newcommand{\BOXIT}[1]{\noindent\fbox{\parbox{\textwidth}{#1}}}


\newtheorem{theorem}{Theorem}[section]
\newtheorem{corollary}{Corollary}[theorem]
\newtheorem{lemma}[theorem]{Lemma}

\theoremstyle{definition}
\newtheorem{definition}[theorem]{Definition}
\newtheorem{prop}[theorem]{Proposition}

\theoremstyle{remark}
\newtheorem{remark}[theorem]{Remark}
\newtheorem{example}[theorem]{Example}
%\newtheorem*{proof}[theorem]{Proof}



\newcommand{\RED}[1]{\textcolor{red}{#1}}
\newcommand{\BLUE}[1]{\textcolor{blue}{#1}}
\newcommand{\ts}{\textsuperscript}
\newcommand{\HLINE}{\noindent\rule{7in}{1pt}\\}

\begin{document}

\begin{center}
	\Large{\CLASSNAME -- \SEMESTER} \\
	\large{ w/Professor Berchenko-Kogan}
\end{center}
\begin{center}
	\STUDENTNAME \\
	\ASSIGNMENT -- \DUEDATE\\
\end{center} 

\noindent Section 1 problems 1, 3, 4, 5, 8.

\noindent \textbf{1.1.  A function that is $C^2$ but not $C^3$.}

\noindent Let $g: \R \to \R$ be the function in example 1.2(iii).  Show that the function $h(x) = \int_0^x g(t) dt$ is $C^2$ but not $C^3$ at $x=0$.\\

\BLUE{\begin{align*}
	h(x) &= \int_0^x g(t) dt \\
	h'(x) &= g(x) = \frac{3}{4}x^\frac{4}{3} \\
	h''(x) &= g'(x) = x^\frac{1}{3} \\
	h'''(x) &= \frac{1}{3}x^\frac{-2}{3}
\end{align*}which is NOT continuous at $x=0$.\\
}

\noindent \textbf{1.3.  A diffeomorphism of open interval in $\R$}

\noindent Let $U \subset \R^n$ and $V \subset \R$ be open subsets.  A $C^\infty$ map $F:U \to V$ is called a \textit{differomorphism} if it is bijective and has  $C^\infty$ inverse $F^{-1}: V\to U$.

\begin{enumerate}[label=(\alph*)]
\item Show that the function $f: ] -\pi/2, /pi/2[ \to \R,f(x)=\tan x,$ is a diffeomorphism.

\BLUE{\begin{align*}
	f(x) &= \tan x \\
	f^{-1}(x) &= \tan^{-1} x \\
	f^{-1}(x)' &= \frac{1}{1+x^2} \\
	f^{-1}(x)'' &= \frac{-2x}{(1+x^2)^2} 
\end{align*}It is clear that further differentiaion will increase the power of the denominator and the number of terms indefinitely. Thus $f^{-1} \in C^\infty$.
}

\item Let $a,b$ be real numbers of $a < b$.  Find a linear function $h:(a,b)\to (-1,1)$, thus proving that any two finite open intervals are diffeomorphic.

The composite $f \circ h:(a,b)\to \R$ is then a diffeomorphism of an open interval with $\R$.

\BLUE{We must a function of the form $h(x)=mx+c$ and find both the slope $m$ and the $y$-intercept in terms of $a$ and $b$.
\begin{align*}
	h(a) &= -1 \AND h(b) = a \\
	m &= \frac{b-a}{1-(-1)} = \frac{b-a}{2} \\
	-1 &= ma + c \\
	1 &= mb + c \\
	0 &= m(a+b) + c = \frac{b-a}{2}(b+a) + c\\
	c &= -\frac{b^2-a^2}{2} \\
	h(x) &= \frac{b-a}{2}x - \frac{b^2-a^2}{2}
\end{align*}
}

\item The exponent function exp$:\R \to ]0,\infty[$ is a differomorphism.  Use it to show that for any real numbers $a$ and $b$, the intervals $\R,]a,\infty[$, and $],-\infty,b[$ are diffeomorphic.

\BLUE{Goal: find a map $f: ]-\infty, b[ \to ]a, \infty[$ which has the form $f(x) = ce ^{-x}$ try
\begin{align*}
	f(a) &= b \implies ce^{-a} = b,\, c=be^a\\
	f(x) &= be^{x-a}
\end{align*}$f \in C^\infty$ as is $f^{-1}$.  This also maps onto $\R$ quite well.
}

\end{enumerate}

\noindent \textbf{1.4. A differomorphism of an open cube with $\R^n$}

\noindent Show that the map
\begin{align*}
	f:\PAREN{-\frac{\pi}{2}, \frac{\pi}{2}}^n \to \R^n,\, f(x_1,\dots, x_n) = (\tan x_1, \dots, \tan x_n),
\end{align*}is a differomorphism.

\BLUE{From 1.3. we can see that $f(x^i) = \tan x^i$ and $\PART{f}{x^i} = \frac{d \tan x^i}{d x^i}$.  Thus, each $f(x^i) \in C^\infty$ for all $i=1,\dots,n$ and $f \in C^\infty$.  The same can be said for $f^{-1}$, thus $f$ is a diffeomorphism.\\
}

\noindent \textbf{1.5. A diffeomorphism of an open ball with $\R^n$}

\noindent Let $\textbf{0} = (0,0)$ be the origin and $B(\textbf{0}, 1)$ the open unit disk in $\R^2$.  To find a diffeomorphism between $B(\textbf{0},1)$ and $\R^2$, we identify $\R^2$ with the $xy$-plane in $R^3$ and introduce the lower open hemisphere
\begin{align*}
	S: x^2+y^2+(z-1)^2=1, \, z<1
\end{align*}in $\R^3$ as an intermediate space (Figure 1.4). First note that the map
\begin{align*}
	f: B(\textbf{0}, 1) \to S, \, (a,b) \mapsto(a,b,1-\sqrt{1-a^2-b^2}),
\end{align*}is a bijection.

\begin{enumerate}[label=(\alph*)]
	\item The \textit{stereographic projection} $g: S \to \R^2$ from $(0,0,1)$ is the map that sends a point $(a,b,c) \in S$ to the intersection of the line through $(0,0,1)$ and $(a,b,c)$ with the $xy$-plane.
	
	Show that it is given by 
	\begin{align*}
		(a,b,c)\mapsto (u,v)=\PAREN{\frac{a}{1-c},\frac{b}{1-c}}, \, c=1-\sqrt{1-a^2-b^2},
	\end{align*}with inverse
	\begin{align*}
		(u,v) \mapsto\PAREN{\frac{u}{\sqrt{1+u^2+v^2}}, \frac{v}{\sqrt{1+u^2+v^2}}, 1-\frac{1}{\sqrt{1+u^2+v^2}}}.
	\end{align*}
	
	\BLUE{We have a line in space with two points $(0,0,1)$ and $(a,b,c)$ and we want to find the point where on the $xy$-plane or where $z=0$.  
	\begin{align*}
		a^2+b^2+(c-1)^2 &= 1 \implies c = 1-\sqrt{1-a^2-b^2}
	\end{align*}remembering the symmetric form for the equation of a line
	\begin{align*}
%		(x,y,z) &= (0,0,1)+t(a,b,c) \text{ solving for }t\\
		\frac{x}{a} &= \frac{y}{b} = \frac{z+1}{1-c}
	\end{align*}When this line intersects with the $xy$-plane at $(u,v,0)$ we have
	\begin{align*}
		\frac{u}{a} &= \frac{1}{1-c} \implies u = \frac{a}{1-c} \\
		\frac{v}{b}  &= \frac{1}{1-c} \implies v = \frac{b}{1-c} \\
		(a,b,c) &\mapsto\PAREN{\frac{a}{1-c}, \frac{b}{1-c}}
	\end{align*}the inverse would be a line from $(0,0,1)$ to $(u,v,0)$ through a point on the hemisphere $(p,q,r)$ by using similar triangles and the radius of the sphere $R$
	\begin{align*}
		\frac{p}{R} &= \frac{u}{\sqrt{1+u^2+v^2}}, \, \frac{q}{R} = \frac{v}{\sqrt{1+u^2+v^2}} \\
		\frac{1-r}{R} &= \frac{1}{\sqrt{1+u^2+v^2}} \\
		R = 1 &\implies (p,q,r) = \PAREN{\frac{u}{\sqrt{1+u^2+v^2}}, \frac{v}{\sqrt{1+u^2+v^2}}, 1-\frac{1}{\sqrt{1+u^2+v^2}}}
	\end{align*}
	}
	
	\item Composing the two maps $f$ and $g$ gives the map
	\begin{align*}
		h=g \circ f: B(\textbf{0}, 1) \to \R^2: \, h(a,b) = \PAREN{\frac{a}{\sqrt{1-a^2-b^2}}, \frac{b}{\sqrt{1-a^2-b^2}}}.
	\end{align*}Find a formula for $h^{-1}(u,v)=(f^{-1}\circ g^{-1})(u,v)$ and conclude that $h$ is a differomorphism of the open disk $B(\textbf{0},1)$ with $\R^2$.
	
	\BLUE{$f^{-1}$ accepts a point on the hemisphere $S$ and projects it down to a point on the disc.  $f^{-1}(x,y,z) = (x,y)$ thus
	\begin{align*}
		f^{-1}\circ g^{-1} (u,v) &= f^{-1}\PAREN{\frac{u}{\sqrt{1+u^2+v^2}}, \frac{v}{\sqrt{1+u^2+v^2}}, 1-\frac{1}{\sqrt{1+u^2+v^2}}} \\
		&= \PAREN{\frac{u}{\sqrt{1+u^2+v^2}}, \frac{v}{\sqrt{1+u^2+v^2}}}
	\end{align*}	
	}
	
	\item Generalize part (b) to $\R^n$. 
	
	\BLUE{Changing $S$ from a hemisphere to a half-hypersphere with radius=1 in $\R^n$ and still moving its center up one axis, $x^i$, by 1, all other parameters will be equally effected as the $x$ and $y$ coordinates where $x^i$ will respond like the $z$.  Thus,
	\begin{align*}
		h(x) &= (h^1(x), h^2(x), \dots, h^k(x), \dots, h^n(x)) \\
		h^k(x) &=  \BINDEF{ \frac{x^k}{1+\sqrt{\sum_{j=1}^n (x^j)^2}} & k \ne i} {1-\frac{1}{1+\sqrt{\sum_{j=1}^n (x^j)^2}}}
	\end{align*}
	}
	
\end{enumerate}

\noindent \textbf{1.8. Bijective $C^\infty$ maps.}

\noindent Define $f: \R \to \R$ by $f(x)=x^3$.  Show that $f$ is a bijective $C^\infty$ map, but that $f^{-1}$ is not $C^\infty$.  (This example shows that a bijective $C^\infty$ map need not have a $C^\infty$ inverse.  In complex analysis, the situation is quite different: a bijective holomorphic map $f: \C \to \C$ necessarily has a holomorphic inverse.)

\BLUE{\begin{align*}
	f^{-1}(x) &= x^{1/3} \\
	(f^{-1}(x))' &= \frac{1}{3}x^{-2/3}
\end{align*}which is not continuous at zero and therefore $f^{-1} \not \in C^\infty$.
}

\end{document}
