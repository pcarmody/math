\documentclass[12pt,a4paper]{report}
\usepackage[utf8]{inputenc}
\usepackage{amsmath}
\usepackage{amsfonts}
\usepackage{amssymb}
\usepackage{amsthm}
\usepackage{hyperref}

\usepackage{multicol}
\usepackage{fancyhdr}
\usepackage[inline]{enumitem}
\usepackage{tikz}
\usepackage{tikz-cd}
\usetikzlibrary{calc}
\usetikzlibrary{shapes.geometric}
\usetikzlibrary{positioning}
\usepackage[margin=0.5in]{geometry}
\usepackage{xcolor}

\hypersetup{
    colorlinks=true,
    linkcolor=blue,
    filecolor=magenta,      
    urlcolor=cyan,
    pdftitle={Tensors},
    pdfpagemode=FullScreen,
    }

%\urlstyle{same}

\newcommand{\CLASSNAME}{Math 5050 -- Special Topics: Manifolds}
\newcommand{\STUDENTNAME}{Paul Carmody}
\newcommand{\ASSIGNMENT}{Section 8: The Tangent Space }
\newcommand{\DUEDATE}{May 30, 2025}
\newcommand{\PROFESSOR}{Professor Berchenko-Kogan}
\newcommand{\SEMESTER}{Fall 2025}
\newcommand{\SCHEDULE}{TBD}
\newcommand{\ROOM}{Remote}

\newcommand{\MMN}{M_{m\times n}}
\newcommand{\FF}{\mathcal{F}}

\pagestyle{fancy}
\fancyhf{}
\chead{ \fancyplain{}{\CLASSNAME} }
%\chead{ \fancyplain{}{\STUDENTNAME} }
\rhead{\thepage}
\newcommand{\LET}{\text{Let }}
%\newcommand{\IF}{\text{if }}
\newcommand{\AND}{\text{ and }}
\newcommand{\OR}{\text{ or }}
\newcommand{\FORSOME}{\text{ for some }}
\newcommand{\FORALL}{\text{ for all }}
\newcommand{\WHERE}{\text{ where }}
\newcommand{\WTS}{\text{ WTS }}
\newcommand{\WLOG}{\text{ WLOG }}
\newcommand{\BS}{\backslash}
\newcommand{\DEFINE}[1]{\textbf{\emph{#1}}}
\newcommand{\IF}{$(\Rightarrow)$}
\newcommand{\ONLYIF}{$(\Leftarrow)$}
\newcommand{\ITH}{\textsuperscript{th} }
\newcommand{\FST}{\textsuperscript{st} }
\newcommand{\SND}{\textsuperscript{nd} }
\newcommand{\TRD}{\textsuperscript{rd} }
\newcommand{\INV}{\textsuperscript{-1} }


%%%%%%%
% derivatives
%%%%%%%

\newcommand{\PART}[2]{\frac{\partial #1}{\partial #2}}
\newcommand{\SPART}[2]{\frac{\partial^2 #1}{\partial #2^2}}
\newcommand{\DERIV}[2]{\frac{d #1}{d #2}}
\newcommand{\LAPLACIAN}[1]{\frac{\partial^2 #1}{\partial x^2} + \frac{\partial^2 #1}{\partial y^2}}

%%%%%%%
% sum, product, union, intersections
%%%%%%%

\newcommand{\SUM}[2]{\underset{#1}{\overset{#2}{\sum}}}
\newcommand{\PROD}[2]{\underset{#1}{\overset{#2}{\prod}}}
\newcommand{\UNION}[2]{\underset{#1}{\overset{#2}{\bigcup}}}
\newcommand{\INTERSECT}[2]{\underset{#1}{\overset{#2}{\bigcap}}}
\newcommand{\FSUM}{\SUM{n=-\infty}{\infty}}
       

%%%%%%%
% supremum and infimum
%%%%%%%

\newcommand{\SUP}[1]{\underset{#1}\sup \,}
\newcommand{\INF}[1]{\underset{#1}\inf \,}
\newcommand{\MAX}[1]{\underset{#1}\max \,}
\newcommand{\MIN}[1]{\underset{#1}\min \,}

%%%%%%%
% infinite sums, limits
%%%%%%%

\newcommand{\SUMK}{\SUM{k=1}{\infty}}
\newcommand{\SUMN}{\SUM{n=1}{\infty}}
\newcommand{\SUMKZ}{\SUM{k=0}{\infty}}
\newcommand{\LIM}[1]{\underset{#1}\lim\,}
\newcommand{\IWOB}[1]{\LIM{#1 \to \infty}}
\newcommand{\LIMK}{\IWOB{k}}
\newcommand{\LIMN}{\IWOB{n}}
\newcommand{\LIMX}{\IWOB{x}}
\newcommand{\NIWOB}{\LIM{n \to \infty}}
\newcommand{\LIMSUPK}{\underset{k\to\infty}\limsup \,}
\newcommand{\LIMSUPN}{\underset{n\to\infty}\limsup \,}
\newcommand{\LIMINFK}{\underset{k\to\infty}\liminf \,}
\newcommand{\LIMINFN}{\underset{n\to\infty}\liminf \,}
\newcommand{\ROOTRULE}[1]{\LIMSUPK \BARS{#1}^{1/k}}

\newcommand{\CUPK}{\bigcup_{k=1}^{\infty}}
\newcommand{\CAPK}{\bigcap_{k=1}^{\infty}}
\newcommand{\CUPN}{\bigcup_{n=1}^{\infty}}
\newcommand{\CAPN}{\bigcap_{n=1}^{\infty}}

%%%%%%%
% number systems (real, rational, etc.)
%%%%%%%

\newcommand{\REALS}{\mathbb{R}}
\newcommand{\RATIONALS}{\mathbb{Q}}
\newcommand{\IRRATIONALS}{\REALS \backslash \RATIONALS}
\newcommand{\INTEGERS}{\mathbb{Z}}
\newcommand{\NUMBERS}{\mathbb{N}}
\newcommand{\COMPLEX}{\mathbb{C}}
\newcommand{\DISC}{\mathbb{D}}
\newcommand{\HPLANE}{\mathbb{H}}

\newcommand{\R}{\mathbb{R}}
\newcommand{\Q}{\mathbb{Q}}
\newcommand{\Z}{\mathbb{Z}}
\newcommand{\N}{\mathbb{N}}
\newcommand{\C}{\mathbb{C}}
\newcommand{\T}{\mathbb{T}}
\newcommand{\COUNTABLE}{\aleph_0}
\newcommand{\UNCOUNTABLE}{\aleph_1}


%%%%%%%
% Arithmetic/Algebraic operators
%%%%%%%


\DeclareMathOperator{\MOD}{mod}
%\newcommand{\MOD}[1]{\mod #1}
\newcommand{\BAR}[1]{\overline{#1}}
\newcommand{\LCM}{\text{ lcm}}
\newcommand{\ZMOD}[1]{\Z/#1\Z}
\DeclareMathOperator{\VAR}{Var}
%%%%%%%
% complex operators
%%%%%%%

\DeclareMathOperator{\RR}{Re}
%\newcommand{\RE}{\text{Re}}
\DeclareMathOperator{\IM}{Im}
%\newcommand{\IM}{\text{Im}}
\newcommand{\CONJ}[1]{\overline{#1}}
\DeclareMathOperator{\LOG}{Log}
%\newcommand{\LOG}{\text{ Log }}
\newcommand{\RES}[2]{\underset{#1}{\text{res}} #2}

%%%%%%%
% Group operators
%%%%%%%

\newcommand{\AUT}{\text{Aut}\,}
\newcommand{\KER}{\text{ker}\,}
\newcommand{\END}{\text{End}}
\newcommand{\HOM}{\text{Hom}}
\newcommand{\CYCLE}[1]{(\begin{array}{cccccccccc}
		#1
	\end{array})}
\newcommand{\SUBGROUP}{\underset{\text{group}}\subseteq}	
%\newcommand{\SUBGROUP}{\subseteq_g}
\newcommand{\SUBRING}{\underset{\text{ring}}\subseteq}
\newcommand{\SUBMOD}{\underset{\text{mod}}\subseteq}
\newcommand{\SUBFIELD}{\underset{\text{field}}\subseteq}
\newcommand{\ISO}{\underset{\text{iso}}\longrightarrow}
\newcommand{\HOMO}{\underset{\text{homo}}\longrightarrow}

%%%%%%%
% grouping (parenthesis, absolute value, square, multi-level brackets).
%%%%%%%

\newcommand{\PAREN}[1]{\left (\, #1 \,\right )}
\newcommand{\BRACKET}[1]{\left \{\, #1 \,\right \}}
\newcommand{\SQBRACKET}[1]{\left [\, #1 \,\right ]}
\newcommand{\ABRACKET}[1]{\left \langle\, #1 \,\right \rangle}
\newcommand{\BARS}[1]{\left |\, #1 \,\right |}
\newcommand{\DBARS}[1]{\left \| \, #1 \,\right \|}
\newcommand{\LBRACKET}[1]{\left \{ #1 \right .} 
\newcommand{\RBRACKET}[1]{\left . #1 \right \]}
\newcommand{\RBAR}[1]{\left . #1 \, \right |}
\newcommand{\LBAR}[1]{\left | \, #1 \right .}
\newcommand{\BLBRACKET}[2]{\BRACKET{\RBAR{#1}#2}}
\newcommand{\GEN}[1]{\ABRACKET{#1}}
\newcommand{\BINDEF}[2]{\LBRACKET{\begin{array}{ll}
     #1\\
     #2
\end{array}}}

%%%%%%%
% Fourier Analysis
%%%%%%%

\newcommand{\ONEOTWOPI}{\frac{1}{2\pi}}
\newcommand{\FHAT}{\hat{f}(n)}
\newcommand{\FINT}{\int_{-\pi}^\pi}
\newcommand{\FINTWO}{\int_{0}^{2\pi}}
\newcommand{\FSUMN}[1]{\SUM{n=-#1}{#1}}
%\newcommand{\FSUM}{\SUMN{\infty}}
\newcommand{\EIN}[1]{e^{in#1}}
\newcommand{\NEIN}[1]{e^{-in#1}}
\newcommand{\INTALL}{\int_{-\infty}^{\infty}}
\newcommand{\FTINT}[1]{\INTALL #1 e^{2\pi inx\xi} dx}
\newcommand{\GAUSS}{e^{-\pi x^2}}

%%%%%%%
% formatting 
%%%%%%%

\newcommand{\LEFTBOLD}[1]{\noindent\textbf{#1}}
\newcommand{\SEQ}[1]{\{#1\,\}}
\newcommand{\WIP}{\footnote{work in progress}}
\newcommand{\QED}{\hfill\square}
\newcommand{\ts}{\textsuperscript}
\newcommand{\HLINE}{\noindent\rule{7in}{1pt}\\}

%%%%%%%
% Mathematical note taking (definitions, theorems, etc.)
%%%%%%%

\newcommand{\REM}{\noindent\textbf{\\Remark: }}
\newcommand{\DEF}{\noindent\textbf{\\Definition: }}
\newcommand{\THE}{\noindent\textbf{\\Theorem: }}
\newcommand{\COR}{\noindent\textbf{\\Corollary: }}
\newcommand{\LEM}{\noindent\textbf{\\Lemma: }}
\newcommand{\PROP}{\noindent\textbf{\\Proposition: }}
\newcommand{\PROOF}{\noindent\textbf{\\Proof: }}
\newcommand{\EXP}{\noindent\textbf{\\Example: }}
\newcommand{\TRICKS}{\noindent\textbf{\\Tricks: }}


%%%%%%%
% text highlighting
%%%%%%%

\newcommand{\B}[1]{\textbf{#1}}
\newcommand{\CAL}[1]{\mathcal{#1}}
\newcommand{\UL}[1]{\underline{#1}}

%%%%%%
% Linear Algebra
%%%%%%

\newcommand{\COLVECTOR}[1]{\PAREN{\begin{array}{c}
#1
\end{array} }}
\newcommand{\TWOXTWO}[4]{\PAREN{ \begin{array}{c c} #1&#2 \\ #3 & #4 \end{array} }}
\newcommand{\THREEXTHREE}[9]{\PAREN{ \begin{array}{c c c} #1&#2&#3 \\ #4 & #5 & #6 \\ #7 & #8 & #9 \end{array} }}
\newcommand{\NXN}{\PAREN{ \begin{array}{c c c c} 
			a_{11} & a_{12} & \cdots & a_{1n} \\
			a_{21} & a_{22} & \cdots & a_{2n} \\
			\vdots & \vdots & \ddots & a_{1n} \\
			a_{n1} & a_{n2} & \cdots & a_{nn} \\
		\end{array} }}
\newcommand{\SLR}{SL_2(\R)}
\newcommand{\GLR}{GL_2(\R)}
\DeclareMathOperator{\TR}{tr}
\DeclareMathOperator{\BIL}{Bil}
\DeclareMathOperator{\SPAN}{span}

%%%%%%%
%  White space
%%%%%%%

\newcommand{\BOXIT}[1]{\noindent\fbox{\parbox{\textwidth}{#1}}}


\newtheorem{theorem}{Theorem}[section]
\newtheorem{corollary}{Corollary}[theorem]
\newtheorem{lemma}[theorem]{Lemma}

\theoremstyle{definition}
\newtheorem{definition}[theorem]{Definition}
\newtheorem{prop}[theorem]{Proposition}

\theoremstyle{remark}
\newtheorem{remark}[theorem]{Remark}
\newtheorem{example}[theorem]{Example}
%\newtheorem*{proof}[theorem]{Proof}



\newcommand{\RED}[1]{\textcolor{red}{#1}}
\newcommand{\BLUE}[1]{\textcolor{blue}{#1}}

\begin{document}

\begin{center}
	\Large{\CLASSNAME -- \SEMESTER} \\
	\large{ w/\PROFESSOR}
\end{center}
\begin{center}
	\STUDENTNAME \\
	\ASSIGNMENT -- \DUEDATE\\
\end{center} 

\noindent\textbf{Pg. 88: Exercise 8.3 (The Differential of a mpa).}  Check that $F_*(X_p)$ is a derivation at $F(p)$ and that $F_*: T_p N\to T_{F(p)} M$ is a linear map.\\

\BLUE{Let $[f] \in C^\infty_{F(p)}(M)$  and  $f, g \in [f]$. Then,
\begin{align*}
	(F_*(X_p))f\cdot g &=  X_p(f\cdot g \circ F)\\
	&=  X_p\PAREN{(f\circ F)\cdot(g \circ F) }\\
	&= f\cdot X_p(g \circ F) + g\cdot X_p(f\circ F)\\
	&= f\cdot (F_*(X_p))g+g\cdot(F_*(X_p))f
\end{align*}hence $F_*$ obeys the Liebniz Rule.
\begin{align*}
	F_*(aX_p +b)f &= (aX_p+b)(f \circ F)\\
	&= a(X_p)(f\circ F)+b(f \circ F)\\
	&= a(F_*(X_p))f+(F_*(b))f
\end{align*}
}

\HLINE
\noindent\textbf{Pg. 92: Exercise 8.14 (The Velocity Vector vs the Calculus Derivative).}  Let $c:(a,b) \to \R$ be a curve with target space $\R$.  Verify that $c'(t) = \dot{c}(t)d/dx|_{c(t)}$.

\HLINE
\noindent \textbf{\\\large{Problems}}

\begin{enumerate}[label=8.\arabic*.]

\item \textbf{Differential of a map.}

Let $F: \R^2 \to \R^3$ be the map
\begin{align*}
	(u,v,w) = F(x,y) = (x,y,xy).
\end{align*}Let $p =(x,y)\in \R^2$.  Compute $F_*\PAREN{\RBAR{\PART{}{x}}_p}$ as a linear combination of $\PART{}{u}, \PART{}{v}$, and $\PART{}{w}$ at $F(p)$.

\BLUE{Let $F(x,y) = (F^1(x,y), F^2(x,y), F^3(x,y))$ then
\begin{align*}
	J_F &= \SQBRACKET{\begin{array}{cc}
		\PART{F^1}{x} & \PART{F^1}{y} \\
		\PART{F^2}{x} & \PART{F^2}{y} \\
		\PART{F^3}{x} & \PART{F^3}{y}
	\end{array}
	} = \SQBRACKET{\begin{array}{cc}
		1 & 0 \\
		0 & 1 \\
		y & x
	\end{array}
	}\\
	F_*\PAREN{\RBAR{\PART{}{x}}_p} &= F_*(1,0) \\
	&= \SQBRACKET{\begin{array}{cc}
		1 & 0 \\
		0 & 1 \\
		y & x
	\end{array}
	} \SQCOLVECTOR{1 \\ 0}\\
	&= \CYCLE{ 1 & 0 & y} \\
	&= \PART{}{u} + v\PART{}{w}
\end{align*}
}

\item \textbf{Differential of a linear map}

Let $L: \R^n\to \R^m$ be a linear map.  For any $p \in \R^n$, there is a canonical identification $T_p(\R^n)\ISO \R^n$ given by 
\begin{align*}
	\sum a^i \RBAR{\PART{}{x^i}}_p \mapsto \textbf{a}=\ABRACKET{a^1,\cdots a^n}.
\end{align*}Show that the differential $L_{*,p}:T_p(\R^n)\to T_{L(p)}(\R^m)$ is the map $L: \R^n\to\R^m$ itself, with the identificaton of the tangent spaces as above.

\BLUE{Given $p=(a^1, \dots, a^n)$ and $L: N \to M$ and $L(p) = \ABRACKET{a^1, \dots, a^n}$. Given any $f: M \to \R$ then
\begin{align*}
	(L_{*,p}(X_p))f(p) &= X_p(f \circ L)(p) \\
	&= \sum_{i=1}^n \RBAR{\PART{f(L(p))}{x^i}}_p \\
	&= \sum_{i=1}^n \RBAR{\PART{f(\ABRACKET{a^1,\dots,a^n})}{x^i}}_p \\
	&= \sum_{i=1}^n \RBAR{\PART{f(p)}{x^i}}_p \\
	&= \sum_{i=1}^n \RBAR{\PART{}{x^i}}_p f(p)\\
	&= L_p f(p)
\end{align*}
}

\item \textbf{Differential of a map}

Fix a real number $\alpha$ and define $F: \R^2 \to \R^2$ by 
\begin{align*}
	\SQCOLVECTOR{
		u \\ v	
	} = (u,v) = F(x,y) = \SQTWOXTWO{\cos \alpha}{-\sin \alpha }{ \sin \alpha }{ \cos \alpha} \SQCOLVECTOR{x\\y}
\end{align*}Let $X = -y\PART{}{x}+x\PART{}{y}$ be a vector field on $\R^2$. If $p=(x,y)\in\R^2$ and $F_*(X_p) = \RBAR{\PAREN{ a\PART{}{u}+b\PART{}{v}}}_{F(p)}$, find $a$ and $b$ in terms of $x,y, $ and $\alpha$.

\BLUE{Remember that $F_*$ is linear then\begin{align}
	(F_*(X_p))f &= -yF_*\PAREN{\PART{}{x}}+xF_*\PAREN{\PART{}{y}}
\end{align}The Jacobian is 
\begin{align*}
	J_{F_*} &= \SQTWOXTWO{\cos \alpha}{-\sin \alpha }{ \sin \alpha }{ \cos \alpha} \\
	F_*\PAREN{\PART{}{x}} &= J_{F_*} \SQCOLVECTOR{ 1 \\ 0 } = \SQCOLVECTOR{\cos \alpha \\ \sin\alpha}= \cos \alpha \PART{}{u} + \sin\alpha\PART{}{v}\\
	F_*\PAREN{\PART{}{y}} &= J_{F_*} \SQCOLVECTOR{ 0 \\ 1 } = \SQCOLVECTOR{-\sin \alpha \\ \cos\alpha}= -\sin \alpha \PART{}{u} + \cos\alpha\PART{}{v}
\end{align*}From (1) we have
\begin{align*}
	(F_*(X_p))f &= -yF_*\PAREN{\PART{}{x}}+xF_*\PAREN{\PART{}{y}} \\
	&= -y \PAREN{\cos \alpha \PART{}{u} + \sin\alpha\PART{}{v}} +x \PAREN{-\sin \alpha \PART{}{u} + \cos\alpha\PART{}{v}} \\
	&= -(y\cos \alpha + \sin \alpha)\PART{}{u} + (-y\sin\alpha+x\cos\alpha)\PART{}{v}
\end{align*}
}

\item \textbf{Transition matrix for coordinate vectors}

Let $x,y$ be the standard coordinates on $\R^2$, and let $U$ be the ope set
\begin{align*}
	U = \R^2-\BRACKET{(x,0)\,|\, x \ge 0}.
\end{align*}On $U$ the polar coordinates $r,\theta$ are uniquley defined by 
\begin{align*}
	x &= r\cos \theta\\
	y &= r \sin, r > 0, 0 < \theta < 2 \pi
\end{align*}Find $\PART{}{r}$ and $\PART{}{\theta}$ in terms of $\PART{}{x}$ and $\PART{}{y}$.

\BLUE{\begin{align*}
	F: U &\to \R^2 \\
	F(r,\theta) &= (r\cos \theta, r\sin \theta) \\
	J_F &= \SQTWOXTWO{\PART{x}{r}}{\PART{x}{\theta}}{\PART{y}{r}}{\PART{y}{\theta}} = \SQTWOXTWO{ \cos \theta}{-r\sin\theta}{\sin\theta}{r\cos \theta} \\
	F\PAREN{\PART{}{r}} &= J_F\SQCOLVECTOR{1 \\ 0} = \SQCOLVECTOR{\cos\theta \\ \sin \theta}\\
	&= \cos\theta \PART{}{x} + \sin\theta \PART{}{y}\\
	F\PAREN{\PART{}{\theta}} &= J_F\SQCOLVECTOR{0 \\ 1} = \SQCOLVECTOR{-r\sin\theta \\ r\cos \theta}\\
	&= -r\sin\theta \PART{}{x} + r\cos\theta \PART{}{y}
\end{align*}$F$ is bijective, hence, $J_F^{-1}$ exists and is 
\begin{align*}
	J_F^{-1} &= \SQTWOXTWO{\PART{r}{x}}{\PART{r}{y}}{\PART{\theta}{x}}{\PART{\theta}{y}} = \frac{1}{\det J_F }\SQTWOXTWO{r\cos\theta}{\sin\theta}{-r\sin\theta}{\cos\theta} \\
	\det J_F &= r\cos^2\theta +r\sin^\theta = r\\
	J_F^{-1} &= \SQTWOXTWO{\cos\theta}{\frac{\sin\theta}{r}}{-\sin\theta}{\frac{\cos\theta}{r}} 
\end{align*}Then we can write $F^{-1}(x,y)$ in terms of $J_F^{-1}$.
}

\item \textbf{Velocity of a curve in local coordinates}

Prove Proposition 8.15.

\item \textbf{Velocity vector}

Let $p=(x,y)$ be a point in $\R^2$.  Then
\begin{align*}
	c_p(t) = \SQTWOXTWO{\cos 2t}{-\sin 2t}{\sin 2t}{\cos 2t} \SQCOLVECTOR{x\\y}, t \in \R
\end{align*}is a curve with initail point $p$ in $\R^2$.  compute the veolocity vector $c_p'(0)$.

\BLUE{\begin{align*}
	c_p(t) &= \SQTWOXTWO{\cos 2t}{-\sin 2t}{\sin 2t}{\cos 2t} \SQCOLVECTOR{x\\y} \\
	&= \SQCOLVECTOR{x\cos 2t-y\sin 2t\\x\sin 2t+y\cos 2t} \\
	\dot{c}^1(t) &= -2x\sin 2t-2y\cos 2t\\
	\dot{c}^2(t) &= 2x\cos 2t -2y\sin 2t \\
	c'(0) &= \SQCOLVECTOR{-2x\cos 0-2y\sin 0\\2x\cos 0-2y \sin 0} = \SQCOLVECTOR{-2y\\2x}
\end{align*}
}

\item \textbf{Tangent space to a product}

If $M$ and $N$ are manifolds, let $\pi_1 : M \times N \to M$ and $\pi_2 : M \times N \to N$ be the two projections.  Prove that for $(p,q) \in M \times N$,
\begin{align*}
	(\pi_1, \pi_2) : T_{(p,q)}(M \times N)\to T_pM \times T_qN
\end{align*}is an isomorphism.

\item \textbf{Differentials of multiplication and inverse}

Let $G$ be a Lie group with multiplication map $\mu:G \times G \to G$, inverse map $\iota : G \to G$, and identity element $e$.
\begin{enumerate}[label=(\alph*)]

\item Show that the differential at the identity of the multiplication map $\mu$ is addition:
\begin{align*}
	\mu_{*,(e,e)} : T_eG \times T_e G &\to T_e G,\\
	\mu_{*,(e,e)}(X_e, Y_e) &= X_e + Y_e.
\end{align*}(\textit{Hint:}  First, computer $\mu_{*,(e,e)}(X_e, 0)$ and $\mu_{*,(e,e)}(0,Y_e)$ using Proposition 8.18)

\BLUE{Want to show that if $F=\mu$ and $p=(e,e)$.  Thus when we write $F_*(X_p)f = X_p(f\circ F)$ we mean $\mu_{*,(e,e)}(X_{(e,e)})f = X_{(e,e)}(f\circ \mu)$
}

\item Show that the differential at the identity of $\iota$ is the negative:
\begin{align*}
	\iota_{*,(e,e)}: T_e G &\to T_e G,\\
	\iota_{*,(e,e)}(X_e) = -X_e.
\end{align*}(\textit{Hint:} Take the differential of $\mu(c(t), (t\circ c)(t)) = e.$).

\end{enumerate}

\item \textbf{Transforming vectors to coordinate vectors}

Let $X_1, \dots, X_n$ be $n$ vector fields on an open subset $U$ of a manifold of dimensions $n$.  Suppose that at $p\in U$, the vectors $(X_1)_p, \dots, (X_n)_p$ are linearly independent.  Show that there is a chart $(V, x^1, \dots, x^n)$ about $p$ such that $(X_i)_p = \PAREN{\PART{}{x^i}}_p$ for $i = 1, \dots, n$.

\item \textbf{Local maxima}

A real-valued function $f: M \to \R$ on a manifold is said to have \textit{local maximum} at $p \in M$ if there is a neighborhood $U$ of $p$ such that $f(p) \ge f(q)$ for all $q \in U$.

\begin{enumerate}[label=(\alph*)]

\item Prove that if a differentiable functions $f: I \to \R$ defined on an open interval $I$ has a local maximum at $p \in I$, then $f'(p)=0$.
\item Prove that a local maximum of a $C^\infty$ fucntion $f: M \to \R$ is a critical pint of $f$.  (\textit{Hint: }Let $X_p$ be a tangent vector in $T_pM$ and let $c(t)$ be a curve in $M$ starting at $p$ with initial vector $X_p$.  Then $f \circ c$ is a real-valued functions with a local maximum at 0. Apply (a).)

\end{enumerate}

\end{enumerate}
\end{document}
