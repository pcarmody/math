\documentclass[11pt]{amsart}

\usepackage{amsthm, amssymb,amsmath}
\usepackage{graphicx}

\theoremstyle{definition}  % Heading is bold, text is roman
\newtheorem{theorem}{Theorem}
\newtheorem{definition}{Definition}
\newtheorem{example}{Example}

\newcommand{\ojo}[1]{{\sffamily\bfseries\boldmath[#1]}}

\newcommand{\Z}{\mathbb{Z}}
\newcommand{\N}{\mathbb{N}}
\newcommand{\Q}{\mathbb{Q}}
\newcommand{\R}{\mathbb{R}}
\newcommand{\C}{\mathbb{C}}

\newcommand{\nullspace}{\mathrm{null}}
\newcommand{\rank}{\mathrm{rank}}


\oddsidemargin 0pt
\evensidemargin 0pt
\marginparwidth 0pt
\marginparsep 10pt
\topmargin -10pt
\headsep 10pt
\textheight 8.4in
\textwidth 7in

%\input{../header}


\begin{document}

%\homework{}{Homework III}

\noindent
{\bf 1.} Suppose that $V$ is a finite dimensional vector space. Show that any linear transformation on a subspace of $V$ can be extended to a 
linear transformation on $V$. In other words, show that if $U$ is a subspace of $V$ and $S \in \mathcal{L}(U,W)$, then there is 
$T \in \mathcal{L}(V,W)$ such that $Tu = Su$ for all $u \in U$. \\

\vskip 0.1cm
\noindent
{\bf 2.} Let $V = \mathcal{M}_{n \times n}(F)$, and let $B$ be fixed matrix in $V$. Show that $T \, : \, V \mapsto V$ defined by
$T(A) = AB-BA$ is a linear transformation. \\

\vskip 0.1cm
\noindent
{\bf 3.a)} Recall that $\C$ is a real vector space. Find $T \, : \, \C \mapsto \C$ which is a $\R$-linear transformation which is not a 
$\C$-linear transformation. \\
{\bf b)} Find a linear transformation $T \, : \, V \mapsto V$ where the range and nullspace of $T$ are identical. \\
{\bf c)} Find $T$ and $U$ on $\R^2$ such that $TU = 0$ but $UT \neq 0$. \\

\vskip 0.1cm
\noindent
{\bf 4.} Let $T$ and $U$ be two linear operators on $\R^2$ defined by $T(x_1, x_2) = (x_2,x_1)$ and $U(x_1,x_2) = (x_1,0)$. \\
{\bf a)} Give a geometric interpretation for $T$ and $U$. \\
{\bf b)} Give rules for $U+T$, $UT$, $TU$, $T^2$, and $U^2$. \\





\vfill
\eject
\noindent {\it Extra Questions}\\
{\bf 1.} Let $A$ be an $m \times n$ matrix over $F$ of rank $k$. Show that there exist a $m \times k$ matrix $B$ and a
$k \times n$ matrix $C$, both with rank $k$, where $A = BC$. Conclude that $A$ has rank $1$ if and only if $A = xy^t$ where
$x \in F^m$ and $y \in F^n$. \\


\vskip 0.1cm
\noindent 
{\bf 2.} Let $W$ be the vector space of $2 \times 2$ complex Hermitian matrices. Note that $W$ is a vector space over $\R$ but not
over $\C$. Let $T \, : \, \R^4 \mapsto W$ be the map defined by 
$$ (x,y,z,t) \mapsto \left( \begin{array}{cc} t +x & y + iz \\ y - iz & t -x \end{array} \right).$$
Show that $T$ is an isomorphism. \\



\vskip 0.1cm
\noindent
{\bf 3.} We will consider the vector space $ V = \mathcal{P}^{(n)}(\R)$ of polynomials at most degree $n$. Let 
$$[x]_k := x(x-1)(x-2) \cdots (x-k+1)$$
for $k \geq 1$ and $[x]_0 = 1$. \\
{\bf a)} Show that $([x]_0, [x]_1, [x]_2, \ldots, [x]_n)$ is a basis of $V$. [Hint: argue that   
$[x]_k = x^k + a(k,k-1) x^{k-1} + \cdots + a(k,1) x + a(k,0)$ where $a(k,j)$ are integers. Construct the $(n+1) \times (n+1)$ matrix
which expresses each $[x]_k$ in the basis $(1,x,x^2, \ldots, x^n)$. Show that this matrix is invertible].\\
{\bf b)} Now prove that $x^k = \sum_{j=0}^k S(k,j) [x]_j$ where $S(k,j)$ are integers. \\
{\bf c)} Show that $S(k,0) = 0 $ for $k \geq 1$. Also show that $S(k,k) = 1$ for $k\geq 0 $. \\
{\bf d)} Prove that if $ 1 \leq j \leq k-1$ then
$$ S(k,j) = j S(k-1, j) + S(k-1, j-1).$$ 

\vskip 0.1cm
\noindent
The above exercise shows that $S(k,j)$ are nonnegative integers. They are called {\it Stirling numbers of the second kind}. 
\end{document}