\documentclass[11pt]{amsart}

\usepackage{amsthm, amssymb,amsmath}
\usepackage{graphicx}

\usepackage{enumitem}

\theoremstyle{definition}  % Heading is bold, text is roman
\newtheorem{theorem}{Theorem}
\newtheorem{definition}{Definition}
\newtheorem{example}{Example}

\newcommand{\ojo}[1]{{\sffamily\bfseries\boldmath[#1]}}

\newcommand{\Z}{\mathbb{Z}}
\newcommand{\N}{\mathbb{N}}
\newcommand{\Q}{\mathbb{Q}}
\newcommand{\R}{\mathbb{R}}
\newcommand{\C}{\mathbb{C}}

\newcommand{\nullspace}{\mathrm{null}}
\newcommand{\rank}{\mathrm{rank}}


\oddsidemargin 0pt
\evensidemargin 0pt
\marginparwidth 0pt
\marginparsep 10pt
\topmargin -10pt
\headsep 10pt
\textheight 8.4in
\textwidth 7in

%\input{../header}
\newcommand{\range}{\mathrm{range}}
\newcommand{\NULL}{\mathrm{null}}
\newcommand{\IP}[1]{\left \langle\, #1 \,\right \rangle}
\newcommand{\LL}{\mathcal{L}}
\newcommand{\MM}{\mathcal{M}}
\newcommand{\PP}{\mathcal{P}}
\newcommand{\MATRIX}{\mathcal{M}_{n\times n}}

\begin{document}

%\homework{}{Homework IX}
\begin{center}
\Large{Math 725 -- Advanced Linear Algebra}\\
\large{Paul Carmody}\\
All about Matrices/Transformations
\end{center}

\vskip 1.0 cm

Important terms:
\begin{enumerate}
	\item minimum polynomial: The polynomial, $p$, with lowest degree such that $p(T)=0, \, \forall x\in F$.
	\begin{enumerate}[label=\roman*)]
		\item The roots of the minimum polynomial are eigenvalues.
		\item If the roots have singular multiplicity, then the matrix is diagonalizable.
	
	\end{enumerate}
		
	\item characteristic polynomial
	
		$\det(xI-T)$ forms a polynomial.
		\begin{enumerate}[label=\roman*)]
			\item the characteristic polynomial is divided by the minimum polynomial.
			\item the characteristic polynomial and the minimum polynomial have the same roots, i.e, the same eigenvalues
			\item if all of the factors of the characteristic polynomial are simple (i.e., have degree one) then it is the minimum polynomial.
		\end{enumerate}
		
	\item triangularizable: a matrix that has zeros below the diagonal.  All matrices over the complex numbers are triangulizable.
	
	\item diagonalizable: a matrix that has zeros everywhere except the diagonal.
	\item \textbf{Inner Product Space} defines an inner product.  The primary ability of the Inner Product is define orthogonality, orthonormal basis and norm.  An inner product $\IP{}$ has the following properties.
	\begin{enumerate}[label=\roman*)]
		\item $\IP{u+v,w} = \IP{u,w}+\IP{v,w}$
		\item $\IP{cv,w} = c\IP{v,w}$ and $\IP{v,dw}=\overline{d}\IP{w,w}$
		\item $\IP{w,v} = \overline{\IP{v,w}}$
		\item $\IP{v,v} > 0$ and $\IP{v,v}=0$ if $v=0$
	\end{enumerate}
	\item norm: is a function $||\cdot||: F \to \R$ with the following properties:
	\begin{enumerate}[label=\roman*)]
		\item $||\cdot||\ge 0$.
		\item $||cv||=|c|\, ||v||$.
		\item $||v+w||\le ||v||+||w||$ (triangular inequality).
	\end{enumerate}
	\item orthogonal: $u,v$ are orthogonal is $\IP{u,v}=0$ and $u\ne 0$ and $v \ne 0$.\\
	if $Q$ is an \textit{orthogonal matrix} if $QQ^T=I$.
	\item orthonomal. $u,v$ are said to be orthonormal if they both have length one.\\
		Every finite dimensional inner product space has an orthonormal basis.  Every linear operator $T$ has an upper triangular matrix $[T]_B^B$ w.r.t. an orthonormal basis.
	\item orthogonal compliment.  Given any set $S \subseteq V$ then $S^\perp=\{v \in V\,:\, \IP{v,w}=0,\, \forall w \in S\}$\\
	Any subspace $W \subseteq V$, then $V=W\oplus W^\perp$.
	\item adjoint: $T^* \in \LL(W,V) \to \IP{Tv,w}_W=\IP{v,T^*w}_V$
	
	Properties (analogous to complex arithmetic):
	\begin{enumerate}[label=\roman*)]
		\item Additive: $(S+T)^*=S^*+T^*,\, \forall S,T \in \LL(V,W)$.
		\item Scalar Multiplication: $(\lambda T)^* = \overline{\lambda}T^*,\, \forall T \in \LL(V,W) \,\&\, \lambda \in F$
		\item Multiplication anti-commutative: $(S \circ T)^* = T^*\circ S^*,\, \forall T \in \LL(U,V) \, \& \, S \in \LL(V,W)$
		\item Inverse: $(T^*)^*=T, \, \forall T \in \LL(V,V)$
		\item If $T=U_1+iU_2$ then 
		\begin{enumerate}[label=\alph*)]
			\item $U_1=\frac{1}{2}(T+T^*), \,U_1^*=U_1$
			\item $U_2=\frac{1}{2}(T-T^*), \,U_2^*=U_2$
			\item Note: $U_1,\, U_2$ "look" like real numbers.
		\end{enumerate}
	\end{enumerate}
	Matrices: \\
	Given orthonormal bases $B, B'$ on $V, W$, respectively. then $[T]_{B'}^B = A, [T^*]_B^{B'}=A^* \implies A^*= \overline{A^T}$, i.e, if $W=V$ then $T$ is an operator and $A$ is Hermitian/Symmetric.
	\item self-adjoint: $T=T^*$
	\begin{enumerate}[label=\alph*)]
		\item All $\lambda \in \R$ for eigenvalues of $T$.
		\item $\IP{Tv,v} \in \R$ even if $V$ is complex.
		\item if $\IP{Tv,v}=0,\,\forall v \in V$ then $T=0$.
	\end{enumerate}
	\item normal: If $TT^*=T^*T$ then $T$ is said to be normal.  Self-adjoint implies Normal but not visa versa.
	\begin{enumerate}[label=\alph*)]
		\item If $Tv=\lambda v$ then $T^*v= \overline{\lambda}v$.
		\item $T$ normal $\iff$ diagonalizable w.r.t. orthonormal basis.
		\item $[T]_B^B$ is hermitian, i.e, $A=[T]_B^B =\overline{A^T}=A^*$
		\item $\exists Q \to QQ^*=I$ and $A=Q^*\Lambda Q$ where $\Lambda$ is a diagonal matrix consisting of eigenvalues and $Q$ is a matrix consisting of orthonormal column eigenvectors (i.e., unitary).
		\item iff $||Tv||=||T^*v||,\, \forall v \in V$.
		\item eigenvectors from different eigenvalues are orthogonal to each other.
	\end{enumerate}
	\item unitary: a matrix made up of orthonormal column vectors.
	\begin{itemize}
		\item the conjugate transpose is the inverse
		\item determinant is one
	\end{itemize}
	\item (semi-)positive definite when $\IP{Tv,v} > 0,\,\forall v\ne 0$ semi- implies $\IP{Tv.v} \ge 0,\,\forall v\ne 0$.  The following are equivalent
	\begin{enumerate}[label=\roman*)]
		\item $T$ is (semi-)positive definite.
		\item eigenvalues of $T$ are (semi-)positive.
		\item $\exists R \in \LL(V) \implies T=RR^*$.
	\end{enumerate}
	\item Theorem: Let $f(x_1,\dots,x_n)$ be a polynomial in $\R$ coefficients with degree $2d$ and $X^T$ be the set of possible terms whose exponents that added up to less than or equal to $2d$.  Then $f$ is a Sum of Squares if and only if there exists a positive semi-definite matrix $A$ such that $f=X^TAX$.\\
	
	This is useful in positive-definite multi-variable polynomials.  These all have even valued degrees (hence the $2d$ term), then bodes the question are they the sum of squares polynomials (i.e., equal to $\sum p_i^2$ where $p_i$ are multivariable polynomials).\\
	
	\item Definition: The \textbf{\textit{Singular Value Decomposition}}, $\sigma_1, \sigma_2,\dots, \sigma_r$ of $A \in \MM_{n\times n}(\R)$ are the positive square roots of eigenvalues $\sigma_i = \sqrt{\lambda_i}$ where $\lambda_i \ne 0$ of the matrix $K = A^TA$ (which is positive definite, hence all $\lambda_i\ge 0$).
	\item Single Value Decomposition Theorem: $A \in \MM_{m\times n}(\R)$ of rank $r$ can be factored as $A=P\Sigma Q^T$ where 
	\begin{itemize}
		\item $P \in \MM_{m\times r}(\R)$ with ornormal columns (i.e, $P^TP=I_r$.
		\item $\Sigma$ is a diagonal matrix made up of $\sigma_1, \sigma_2, \dots, \sigma_r$.
		\item $Q^T \in \MM_{r\times n}(\R)$ orthonormal rows $(Q^TQ=I_r)$.\\
	\end{itemize}
	
	\item Pseudo-Inverse (not square): $A \in \MM_{m\times n}(F)$ and $A=U\Sigma V^T$ then $A^+\in \MM_{n\times m}(F)$ and \[A^+=V\Sigma^{-1}U^T\]
	if $A$ is square then
	\begin{align*}
		A^+ = (A^TA)^{-1}A^T
	\end{align*}
	\item Definition: Given $A \in \MM_{n\times n}(\R)$ then we define the \textbf{\textit{norm}} of $A$ as 
	\begin{align*}
		||A|| := \max_{||x||=1} ||Ax||
	\end{align*}\\
	Remark: if $A$ has an SVD of $\sigma_1 \ge \sigma_2 \ge\cdots \ge \sigma_r > 0$ then $||A|| = \sigma_1$.\\
	\item Eichart-Mintsky-Young:  Let $A=\sum_{j=1}^r \sigma_j u_j v_j^T$ (the Single Value Decomposition of $A$ as the sum of rank 1 matrices).  For each $1\le p \le r$ let $A_p=\sum_{j=1}^p \sigma_j u_j v_j^T$ (a rank $p$ matrix).  Then, 
	\begin{align*}
		|| A - A_p|| = \min_{B \in F^{n\times n}} ||A-B|| \text{ where } \rank (B) \le p
	\end{align*}hence $A_p$ is the \textit{closest} rank $p$ matrix to $A$.
\end{enumerate}

Important theorems:
\begin{itemize}
	\item Fundamental Theorem of Algebra
	
		Most notably that $\C$ is algebraically closed (i.e., all polynomials have a zero).
\end{itemize}


\end{document}