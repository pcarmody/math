\documentclass[10pt,a4paper]{report}
\usepackage[utf8]{inputenc}
\usepackage{amsmath}
\usepackage{amsfonts}
\usepackage{amssymb}
\usepackage{amsthm}
\usepackage{hyperref}

\usepackage{multicol}
\usepackage{fancyhdr}
\usepackage{enumitem}
\usepackage{tikz}
\usepackage{tikz-cd}
\usetikzlibrary{calc}
\usetikzlibrary{shapes.geometric}
\usepackage[margin=0.5in]{geometry}
\usepackage{xcolor}
\DeclareMathOperator{\RANGE}{range}
\DeclareMathOperator{\NULL}{null}

\hypersetup{
    colorlinks=true,
    linkcolor=blue,
    filecolor=magenta,      
    urlcolor=cyan,
    pdftitle={Tensors},
    pdfpagemode=FullScreen,
    }

%\urlstyle{same}

\newcommand{\CLASSNAME}{Functional Analysis}
\newcommand{\STUDENTNAME}{Paul Carmody}
\newcommand{\ASSIGNMENT}{Assignment \#7}
\newcommand{\DUEDATE}{May 16, 2024}
\newcommand{\SEMESTER}{Spring 2024}
\newcommand{\SCHEDULE}{T/Th 9:30 -- 10:45}
\newcommand{\ROOM}{Remote}

\pagestyle{fancy}
\fancyhf{}
\chead{ \fancyplain{}{\CLASSNAME} }
%\chead{ \fancyplain{}{\STUDENTNAME} }
\rhead{\thepage}
\newcommand{\LET}{\text{Let }}
%\newcommand{\IF}{\text{if }}
\newcommand{\AND}{\text{ and }}
\newcommand{\OR}{\text{ or }}
\newcommand{\FORSOME}{\text{ for some }}
\newcommand{\FORALL}{\text{ for all }}
\newcommand{\WHERE}{\text{ where }}
\newcommand{\WTS}{\text{ WTS }}
\newcommand{\WLOG}{\text{ WLOG }}
\newcommand{\BS}{\backslash}
\newcommand{\DEFINE}[1]{\textbf{\emph{#1}}}
\newcommand{\IF}{$(\Rightarrow)$}
\newcommand{\ONLYIF}{$(\Leftarrow)$}
\newcommand{\ITH}{\textsuperscript{th} }
\newcommand{\FST}{\textsuperscript{st} }
\newcommand{\SND}{\textsuperscript{nd} }
\newcommand{\TRD}{\textsuperscript{rd} }
\newcommand{\INV}{\textsuperscript{-1} }

\newcommand{\XXX}{\mathfrak{X}}
\newcommand{\MMM}{\mathfrak{M}}
%\newcommand{\????}{\textfrak{A}}
%\newcommand{\????}{\textgoth{A}}
%\newcommand{\????}{\textswab{A}}

\DeclareMathOperator{\DER}{Der}
\DeclareMathOperator{\SGN}{sgn}

%%%%%%%
% derivatives
%%%%%%%

\newcommand{\PART}[2]{\frac{\partial #1}{\partial #2}}
\newcommand{\SPART}[2]{\frac{\partial^2 #1}{\partial #2^2}}
\newcommand{\DERIV}[2]{\frac{d #1}{d #2}}
\newcommand{\LAPLACIAN}[1]{\frac{\partial^2 #1}{\partial x^2} + \frac{\partial^2 #1}{\partial y^2}}

%%%%%%%
% sum, product, union, intersections
%%%%%%%

\newcommand{\SUM}[2]{\underset{#1}{\overset{#2}{\sum}}}
\newcommand{\PROD}[2]{\underset{#1}{\overset{#2}{\prod}}}
\newcommand{\UNION}[2]{\underset{#1}{\overset{#2}{\bigcup}}}
\newcommand{\INTERSECT}[2]{\underset{#1}{\overset{#2}{\bigcap}}}
\newcommand{\FSUM}{\SUM{n=-\infty}{\infty}}
       

%%%%%%%
% supremum and infimum
%%%%%%%

\newcommand{\SUP}[1]{\underset{#1}\sup \,}
\newcommand{\INF}[1]{\underset{#1}\inf \,}
\newcommand{\MAX}[1]{\underset{#1}\max \,}
\newcommand{\MIN}[1]{\underset{#1}\min \,}

%%%%%%%
% infinite sums, limits
%%%%%%%

\newcommand{\SUMK}{\SUM{k=1}{\infty}}
\newcommand{\SUMN}{\SUM{n=1}{\infty}}
\newcommand{\SUMKZ}{\SUM{k=0}{\infty}}
\newcommand{\LIM}[1]{\underset{#1}\lim\,}
\newcommand{\IWOB}[1]{\LIM{#1 \to \infty}}
\newcommand{\LIMK}{\IWOB{k}}
\newcommand{\LIMN}{\IWOB{n}}
\newcommand{\LIMX}{\IWOB{x}}
\newcommand{\NIWOB}{\LIM{n \to \infty}}
\newcommand{\LIMSUPK}{\underset{k\to\infty}\limsup \,}
\newcommand{\LIMSUPN}{\underset{n\to\infty}\limsup \,}
\newcommand{\LIMINFK}{\underset{k\to\infty}\liminf \,}
\newcommand{\LIMINFN}{\underset{n\to\infty}\liminf \,}
\newcommand{\ROOTRULE}[1]{\LIMSUPK \BARS{#1}^{1/k}}

\newcommand{\CUPK}{\bigcup_{k=1}^{\infty}}
\newcommand{\CAPK}{\bigcap_{k=1}^{\infty}}
\newcommand{\CUPN}{\bigcup_{n=1}^{\infty}}
\newcommand{\CAPN}{\bigcap_{n=1}^{\infty}}

%%%%%%%
% number systems (real, rational, etc.)
%%%%%%%

\newcommand{\REALS}{\mathbb{R}}
\newcommand{\RATIONALS}{\mathbb{Q}}
\newcommand{\IRRATIONALS}{\REALS \backslash \RATIONALS}
\newcommand{\INTEGERS}{\mathbb{Z}}
\newcommand{\NUMBERS}{\mathbb{N}}
\newcommand{\COMPLEX}{\mathbb{C}}
\newcommand{\DISC}{\mathbb{D}}
\newcommand{\HPLANE}{\mathbb{H}}

\newcommand{\R}{\mathbb{R}}
\newcommand{\Q}{\mathbb{Q}}
\newcommand{\Z}{\mathbb{Z}}
\newcommand{\N}{\mathbb{N}}
\newcommand{\C}{\mathbb{C}}
\newcommand{\T}{\mathbb{T}}
\newcommand{\COUNTABLE}{\aleph_0}
\newcommand{\UNCOUNTABLE}{\aleph_1}


%%%%%%%
% Arithmetic/Algebraic operators
%%%%%%%


\DeclareMathOperator{\MOD}{mod}
%\newcommand{\MOD}[1]{\mod #1}
\newcommand{\BAR}[1]{\overline{#1}}
\newcommand{\LCM}{\text{ lcm}}
\newcommand{\ZMOD}[1]{\Z/#1\Z}
\DeclareMathOperator{\VAR}{Var}
%%%%%%%
% complex operators
%%%%%%%

\DeclareMathOperator{\RR}{Re}
%\newcommand{\RE}{\text{Re}}
\DeclareMathOperator{\IM}{Im}
%\newcommand{\IM}{\text{Im}}
\newcommand{\CONJ}[1]{\overline{#1}}
\DeclareMathOperator{\LOG}{Log}
%\newcommand{\LOG}{\text{ Log }}
\newcommand{\RES}[2]{\underset{#1}{\text{res}} #2}

%%%%%%%
% Group operators
%%%%%%%

\newcommand{\AUT}{\text{Aut}\,}
\newcommand{\KER}{\text{ker}\,}
\newcommand{\END}{\text{End}}
\newcommand{\HOM}{\text{Hom}}
\newcommand{\CYCLE}[1]{(\begin{array}{cccccccccc}
		#1
	\end{array})}
\newcommand{\SUBGROUP}{\underset{\text{group}}\subseteq}	
%\newcommand{\SUBGROUP}{\subseteq_g}
\newcommand{\SUBRING}{\underset{\text{ring}}\subseteq}
\newcommand{\SUBMOD}{\underset{\text{mod}}\subseteq}
\newcommand{\SUBFIELD}{\underset{\text{field}}\subseteq}
\newcommand{\ISO}{\underset{\text{iso}}\longrightarrow}
\newcommand{\HOMO}{\underset{\text{homo}}\longrightarrow}

%%%%%%%
% grouping (parenthesis, absolute value, square, multi-level brackets).
%%%%%%%

\newcommand{\PAREN}[1]{\left (\, #1 \,\right )}
\newcommand{\BRACKET}[1]{\left \{\, #1 \,\right \}}
\newcommand{\SQBRACKET}[1]{\left [\, #1 \,\right ]}
\newcommand{\ABRACKET}[1]{\left \langle\, #1 \,\right \rangle}
\newcommand{\BARS}[1]{\left |\, #1 \,\right |}
\newcommand{\DBARS}[1]{\left \| \, #1 \,\right \|}
\newcommand{\LBRACKET}[1]{\left \{ #1 \right .} 
\newcommand{\RBRACKET}[1]{\left . #1 \right \]}
\newcommand{\RBAR}[1]{\left . #1 \, \right |}
\newcommand{\LBAR}[1]{\left | \, #1 \right .}
\newcommand{\BLBRACKET}[2]{\BRACKET{\RBAR{#1}#2}}
\newcommand{\GEN}[1]{\ABRACKET{#1}}
\newcommand{\BINDEF}[2]{\LBRACKET{\begin{array}{ll}
     #1\\
     #2
\end{array}}}

%%%%%%%
% Fourier Analysis
%%%%%%%

\newcommand{\ONEOTWOPI}{\frac{1}{2\pi}}
\newcommand{\FHAT}{\hat{f}(n)}
\newcommand{\FINT}{\int_{-\pi}^\pi}
\newcommand{\FINTWO}{\int_{0}^{2\pi}}
\newcommand{\FSUMN}[1]{\SUM{n=-#1}{#1}}
%\newcommand{\FSUM}{\SUMN{\infty}}
\newcommand{\EIN}[1]{e^{in#1}}
\newcommand{\NEIN}[1]{e^{-in#1}}
\newcommand{\INTALL}{\int_{-\infty}^{\infty}}
\newcommand{\FTINT}[1]{\INTALL #1 e^{2\pi inx\xi} dx}
\newcommand{\GAUSS}{e^{-\pi x^2}}

%%%%%%%
% formatting 
%%%%%%%

\newcommand{\LEFTBOLD}[1]{\noindent\textbf{#1}}
\newcommand{\SEQ}[1]{\{#1\,\}}
\newcommand{\WIP}{\footnote{work in progress}}
\newcommand{\QED}{\hfill\square}
\newcommand{\ts}{\textsuperscript}
\newcommand{\HLINE}{\noindent\rule{7in}{1pt}\\}

%%%%%%%
% Mathematical note taking (definitions, theorems, etc.)
%%%%%%%

\newcommand{\REM}{\noindent\textbf{\\Remark: }}
\newcommand{\DEF}{\noindent\textbf{\\Definition: }}
\newcommand{\THE}{\noindent\textbf{\\Theorem: }}
\newcommand{\COR}{\noindent\textbf{\\Corollary: }}
\newcommand{\LEM}{\noindent\textbf{\\Lemma: }}
\newcommand{\PROP}{\noindent\textbf{\\Proposition: }}
\newcommand{\PROOF}{\noindent\textbf{\\Proof: }}
\newcommand{\EXP}{\noindent\textbf{\\Example: }}
\newcommand{\TRICKS}{\noindent\textbf{\\Tricks: }}


%%%%%%%
% text highlighting
%%%%%%%

\newcommand{\B}[1]{\textbf{#1}}
\newcommand{\CAL}[1]{\mathcal{#1}}
\newcommand{\UL}[1]{\underline{#1}}

%%%%%%
% Linear Algebra
%%%%%%

\newcommand{\COLVECTOR}[1]{\PAREN{\begin{array}{c}
#1
\end{array} }}
\newcommand{\TWOXTWO}[4]{\PAREN{ \begin{array}{c c} #1&#2 \\ #3 & #4 \end{array} }}
\newcommand{\DTWOXTWO}[4]{\BARS{ \begin{array}{c c} #1&#2 \\ #3 & #4 \end{array} }}
\newcommand{\THREEXTHREE}[9]{\PAREN{ \begin{array}{c c c} #1&#2&#3 \\ #4 & #5 & #6 \\ #7 & #8 & #9 \end{array} }}
\newcommand{\DTHREEXTHREE}[9]{\BARS{ \begin{array}{c c c} #1&#2&#3 \\ #4 & #5 & #6 \\ #7 & #8 & #9 \end{array} }}
\newcommand{\NXN}{\PAREN{ \begin{array}{c c c c} 
			a_{11} & a_{12} & \cdots & a_{1n} \\
			a_{21} & a_{22} & \cdots & a_{2n} \\
			\vdots & \vdots & \ddots & a_{1n} \\
			a_{n1} & a_{n2} & \cdots & a_{nn} \\
		\end{array} }}
\newcommand{\SLR}{SL_2(\R)}
\newcommand{\GLR}{GL_2(\R)}
\DeclareMathOperator{\TR}{tr}
\DeclareMathOperator{\BIL}{Bil}
\DeclareMathOperator{\SPAN}{span}

%%%%%%%
%  White space
%%%%%%%

\newcommand{\BOXIT}[1]{\noindent\fbox{\parbox{\textwidth}{#1}}}


\newtheorem{theorem}{Theorem}[section]
\newtheorem{corollary}{Corollary}[theorem]
\newtheorem{lemma}[theorem]{Lemma}

\theoremstyle{definition}
\newtheorem{definition}[theorem]{Definition}
\newtheorem{prop}[theorem]{Proposition}

\theoremstyle{remark}
\newtheorem{remark}[theorem]{Remark}
\newtheorem{example}[theorem]{Example}
%\newtheorem*{proof}[theorem]{Proof}



\newcommand{\RED}[1]{\textcolor{red}{#1}}
\newcommand{\BLUE}[1]{\textcolor{blue}{#1}}
\newcommand{\GREEN}[1]{\textcolor{black!30!green}{#1}}
\newcommand{\ORANGE}[1]{\textcolor{orange}{#1}}
\newcommand{\F}{\textbf{F}}
\newcommand{\NLL}{\mathcal{N}}

\title{Advanced Linear Algebra}
\author{The Unforgetable Someone}
\date{Summer 2023}

\newcommand{\NORM}[1]{\,\left \Vert #1 \right \Vert}
%\newcommand{\HAT}[1]{}
\begin{document}

\begin{center}
	\Large{\CLASSNAME -- \SEMESTER} \\
\end{center}
\begin{center}
	\STUDENTNAME \\
	\ASSIGNMENT -- \DUEDATE\\
\end{center} 


p. 290 \#6, 7 
\begin{enumerate}
	\setcounter{enumi}{5}
	\item Let $X$ and $Y$ be Banach spaces and $T: X \to Y$ an injective bounded linear operator.  Show that $T^{-1}:\mathcal{R}(T) \to X$ is bounded if and only if $\mathcal{R}(T)$ is closed in $Y$.
	
	\begin{itemize}
		\item $(\Rightarrow)$ $T^{-1}:\mathcal{R}(T) \to X$ is bounded. Given any Cauchy sequence $(x_n) \in X$ we know that $\NORM{x_n - x_m} \to 0$ as $n,m \to \infty$.  Further, we know that $\NORM{Tx_n-Tx_m} \le \NORM{T}\NORM{x_n - x_m}$ which implies that $\NORM{Tx_n-Tx_m} \to 0$ as $n,m \to \infty$.  Let $x$ be such that $Tx_n \to x$ as $n \to \infty$.  Clearly, $x \in X$ because $X$ is complete and $Tx \in \mathcal{R}(T)$ and $\NORM{Tx_n-Tx} \le \NORM{T}\NORM{x_n-x} \to 0$ as $n \to \infty$.  Thus, $\mathcal{R}(T)$ is closed.
		\item $(\Leftarrow)$ $\mathcal{R}(T)$ is closed.  Given any sequence $(y_n) \in \mathcal{R}(T)$ we know that it converges. Let $y $ be such that $y_n \to y$ as $n\to \infty$.  Since $T$ is injective then $T^{-1}: \mathcal{R}(T) \to X$ is a function.  Let $x_i = T^{-1}(y_i)$ for all $i \in \N$ and $x = T^{-1}y$.  $\NORM{y_n - y} \to 0$ thus $\NORM{T^{-1}(y_n -y)} = \NORM{T^{-1}(y_n)-T^{-1}(y)}= \NORM{x_n - x} \le \NORM{T^{-1}}\NORM{y_n - y} \to 0$ as $n \to \infty$.
		\begin{align*}
			\NORM{T^{-1}(y_n -y)} &= \NORM{T^{-1}(y_n)-T^{-1}(y)} \\
			&= \NORM{x_n - x} \\
			&\le \NORM{T^{-1}}\NORM{y_n - y} \\
			&\le \NORM{T^{-1}}\NORM{T}\NORM{x_n - x} \to 0 \text{ as } n \to \infty
		\end{align*}$T$ is bounded.
	\end{itemize}
	
	\item Let $T: X \to Y$ be a bounded linear operator, where $X$ and $Y$ are Banach spaces.  If $T$ is bijective, show that there are positive real numbers $a$ and $b$ such that $a \NORM{x} \le \NORM{Tx}\le b  \NORM{x}$ for all $ x \in X$.
	
	$T$ bounded means that there exists $b$ such that $\NORM{Tx} \le b \NORM{x}$ for all $x \in X$.. $T^{-1}$ bounded means that there exists positive number $c$ such that $\NORM{T^{-1}y} \le c \NORM{y}$ for all $y \in Y$.  Let $x$ be such that $Tx = y$.  Then $\NORM{T^{-1}y} = \NORM{x} \le c \NORM{Tx}$.  Let $a = 1/c$ then $a \NORM{x} \le \NORM{Tx} \le b\NORM{x}.$
	
\end{enumerate}
\newpage p. 296 \# 8, 9, 10 
\begin{enumerate}
	\setcounter{enumi}{7}
	\item Let $X$ and $Y$ be normed spaces and let $T: X \to Y$ be a closed linear operator.  
	\begin{enumerate}
		\item Show that the image $A$ of a compact subset $C \subset X$ is closed in Y.

		Let $(x_n) \in C$.  Since $C$ is compact, let $\alpha$ be the ordered set of integers such that $(x_i)_{i\in \alpha}$ converges and let $x_{\alpha_i} \to x$ as $i \to \infty$.  Then $T(x_{\alpha_i}) \in A$ for all $i \in \N$.  Since $T$ is a closed linear operator and $C$ is compact (hence closed) the set $\mathcal{G}(T)=\{ (x,y) \,|\, x \in C, y \in A\}$ must also be closed.  Therefore $((x_{\alpha_i}, Tx_{\alpha_i}))\in\mathcal{G}(T)$ as $i \to \infty$ so must $(x, Tx) \in \mathcal{G}(T)$ which means that $Tx \in A$.  Hence $A$ is closed in $Y$. 
		
		\item Show that the inverse image $B$ of a compact subset $K \subset Y$ is closed in $X$.  (Cf. Def. 2.5-1)
		
		Let $(y_n) \in K$ and let $\alpha\subset \N$ be an ordered set of indices such that $(y_{\alpha_n})$ converges and let $y = (y_{\alpha_n})$. Then $\NORM{(y_{\alpha_n}) - y} \to 0$ as $n \to \infty$.  Thus, $\NORM{T^{-1}y_{\alpha_n} - T^{-1}y} = \NORM{T^{-1}(y_{\alpha_n} - y)} \le \NORM{T^{-1}}\NORM{y_{\alpha_n}-y} \to 0$ as $n \to \infty$.  Sine $T^{-1}$ is closed $T^{-1}y \in B$ and $B$ is closed.
		
	\end{enumerate}
	
	\item If $T: X \to Y$ is a closed linear opearator, where $X$ and $Y$ are normed spaces and $Y$ is compact, show that $T$ is bounded.
	
	Let $(x_n)$ be a sequence in $X$ then since $Y$ is compact $(Tx_n)$ converges and let $y = Tx_n$ as $n \to \infty$ and let $x$ be such that $Tx=y$. Thus $\NORM{Tx_n - y} = \NORM{Tx_n -Tx} \le M\NORM{x_n-x} \to 0$.  Thus, $T$ is bounded.
	
	\item Let $X$ and $Y$ be normed spaces and $X$ compact.  If $T: X \to Y$ is a bijective closed linear operator, show that $T^{-1}$ is bounded.
	
	Let $A \subset X$ be closed and bounded.  $T$ is bijective implies that $T^{-1}TA = A$  thus $(T^{-1})^{-1}(C) = T(C) \subset Y$ which is compact, that implies that $T^{-1}$ is continuous and hence, bounded.

\end{enumerate}
\newpage p. 246 \#2, 3, 4 
\begin{enumerate}
	\setcounter{enumi}{1}
	\item Give a simpler proof of Lemma 4.6-7 for the case that $X$ is a Hilbert space.
	
	Let $\tilde{f}(x) = \delta\ABRACKET{x, x_0}/\NORM{x_0}$.  When $x \in Y, \tilde{f}(x) = 0$ and when $x = x_0, \tilde{f}(x_0) = \delta$.  
	
	\item If a normed space $X$ is reflexive, show that $X'$ is reflexive. 
	
	$X$ reflexive implies that $C_X: X \to X''$ defined as $x \mapsto g_x(f)=f(x)$ is isomorphic.  Let $C_{X'}: X' \to X^{(3)}$ where $X^{(3)}$ is the dual-dual of $X'$.  Let $h \in X^{(3)}$	and define $\tilde{h} \in X'$ by $\tilde{h}(f) = h(C_X(f))$ for all $f \in X$.  Then, for all $g \in X''$, we have $C_{X'}(\tilde{h})(g)= g(\tilde{h})= \tilde{h}(C_X^{-1}(g))=h(g)$.  That is, $C_{X'}(\tilde{h}) = h$ which implies that $C_{X'}$ is surjective and hence bijective, thus an isomorphism.
	
	\item Show that a Banach space $X$ is reflexive if and only if its dual space $X'$ is reflexive.  (\textit{Hint.}  It can be shown that a closed subspace fo a reflexive Banach space is reflexive.  Use this fact, without proving it.)
	
	$(\Rightarrow)$ see exercise 3 
	
	$(\Leftarrow)$ $X'$ is reflexive.  Then the cannoncial embedded mapping, $C: X \to X''$, maps all of $X$ onto $X''$, that is $C(X)$ is a subspace of $X''$ ismorphic to $X$.  Hence, $C(X)$ is a Banach Space and closed.  Thus, by the Hint, $C(X)$ is reflexive and, being isomorphic, makes $X$ reflexive.
\end{enumerate}

\newpage p. 268 \#4, 7
\begin{enumerate}
	\setcounter{enumi}{3}
	\item Show that weak convergence in footnote 6 implies weak$^*$ convergence.  Show that the converse holds if $X$ is reflexive.
	
	Let $(f_n)=f$ be a weak$^*$ convergent sequence of functions in $X'$ and let $X$ be reflexive.  Therefore $\NORM{f_n(x)-f(x)} \to 0$ for each $x \in X$.  If we choose $g_x \in X''$ be associated with $x$.  Then we can say $\NORM{g_x(f_n(x) - f(x))} = \NORM{g_x(f_n)(x) - g_x(f)(x)} \to \NORM{g_x(0)(x)}$ which is true for all $x\in X$. Thus $\NORM{g_x(f_n)-g_x(f)} \to 0$ which is strongly convergent.
	
	\setcounter{enumi}{6}
	\item Let $T_n \in B(X,Y)$, where $X$ is a Banach space.  If $(T_n)$ is strongly operator convergent, show that $(\NORM{T_n})$ is bounded.
	
	Let $x$ be any fixed member or $X$.  Then, $\NORM{T_nx - Tx} \to 0$ as $n \to \infty$.  
	\begin{align*}
		\NORM{T_nx - Tx} &\le \NORM{T_nx} - \NORM{Tx}\\
		&\le \NORM{T_n}\NORM{x} - \NORM{T}\NORM{x} \\
		&\le \PAREN{\NORM{T_n} - \NORM{T}} \NORM{x} \\
		\therefore \NORM{T_n} - \NORM{T} &\to 0
	\end{align*}hence bounded.

\end{enumerate}
\end{document}