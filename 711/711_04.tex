\documentclass[10pt,a4paper]{report}
\usepackage[utf8]{inputenc}
\usepackage{amsmath}
\usepackage{amsfonts}
\usepackage{amssymb}
\usepackage{amsthm}
\usepackage{hyperref}

\usepackage{multicol}
\usepackage{fancyhdr}
\usepackage{enumitem}
\usepackage{tikz}
\usepackage{tikz-cd}
\usetikzlibrary{calc}
\usetikzlibrary{shapes.geometric}
\usepackage[margin=0.5in]{geometry}
\usepackage{xcolor}
\DeclareMathOperator{\RANGE}{range}
\DeclareMathOperator{\NULL}{null}

\hypersetup{
    colorlinks=true,
    linkcolor=blue,
    filecolor=magenta,      
    urlcolor=cyan,
    pdftitle={Tensors},
    pdfpagemode=FullScreen,
    }

%\urlstyle{same}

\newcommand{\CLASSNAME}{Functional Analysis}
\newcommand{\STUDENTNAME}{Paul Carmody}
\newcommand{\ASSIGNMENT}{Assignment \#4}
\newcommand{\DUEDATE}{April 4, 2024}
\newcommand{\SEMESTER}{Spring 2024}
\newcommand{\SCHEDULE}{T/Th 9:30 -- 10:45}
\newcommand{\ROOM}{Remote}

\pagestyle{fancy}
\fancyhf{}
\chead{ \fancyplain{}{\CLASSNAME} }
%\chead{ \fancyplain{}{\STUDENTNAME} }
\rhead{\thepage}
\newcommand{\LET}{\text{Let }}
%\newcommand{\IF}{\text{if }}
\newcommand{\AND}{\text{ and }}
\newcommand{\OR}{\text{ or }}
\newcommand{\FORSOME}{\text{ for some }}
\newcommand{\FORALL}{\text{ for all }}
\newcommand{\WHERE}{\text{ where }}
\newcommand{\WTS}{\text{ WTS }}
\newcommand{\WLOG}{\text{ WLOG }}
\newcommand{\BS}{\backslash}
\newcommand{\DEFINE}[1]{\textbf{\emph{#1}}}
\newcommand{\IF}{$(\Rightarrow)$}
\newcommand{\ONLYIF}{$(\Leftarrow)$}
\newcommand{\ITH}{\textsuperscript{th} }
\newcommand{\FST}{\textsuperscript{st} }
\newcommand{\SND}{\textsuperscript{nd} }
\newcommand{\TRD}{\textsuperscript{rd} }
\newcommand{\INV}{\textsuperscript{-1} }


%%%%%%%
% derivatives
%%%%%%%

\newcommand{\PART}[2]{\frac{\partial #1}{\partial #2}}
\newcommand{\SPART}[2]{\frac{\partial^2 #1}{\partial #2^2}}
\newcommand{\DERIV}[2]{\frac{d #1}{d #2}}
\newcommand{\LAPLACIAN}[1]{\frac{\partial^2 #1}{\partial x^2} + \frac{\partial^2 #1}{\partial y^2}}

%%%%%%%
% sum, product, union, intersections
%%%%%%%

\newcommand{\SUM}[2]{\underset{#1}{\overset{#2}{\sum}}}
\newcommand{\PROD}[2]{\underset{#1}{\overset{#2}{\prod}}}
\newcommand{\UNION}[2]{\underset{#1}{\overset{#2}{\bigcup}}}
\newcommand{\INTERSECT}[2]{\underset{#1}{\overset{#2}{\bigcap}}}
\newcommand{\FSUM}{\SUM{n=-\infty}{\infty}}
       

%%%%%%%
% supremum and infimum
%%%%%%%

\newcommand{\SUP}[1]{\underset{#1}\sup \,}
\newcommand{\INF}[1]{\underset{#1}\inf \,}
\newcommand{\MAX}[1]{\underset{#1}\max \,}
\newcommand{\MIN}[1]{\underset{#1}\min \,}

%%%%%%%
% infinite sums, limits
%%%%%%%

\newcommand{\SUMK}{\SUM{k=1}{\infty}}
\newcommand{\SUMN}{\SUM{n=1}{\infty}}
\newcommand{\SUMKZ}{\SUM{k=0}{\infty}}
\newcommand{\LIM}[1]{\underset{#1}\lim\,}
\newcommand{\IWOB}[1]{\LIM{#1 \to \infty}}
\newcommand{\LIMK}{\IWOB{k}}
\newcommand{\LIMN}{\IWOB{n}}
\newcommand{\LIMX}{\IWOB{x}}
\newcommand{\NIWOB}{\LIM{n \to \infty}}
\newcommand{\LIMSUPK}{\underset{k\to\infty}\limsup \,}
\newcommand{\LIMSUPN}{\underset{n\to\infty}\limsup \,}
\newcommand{\LIMINFK}{\underset{k\to\infty}\liminf \,}
\newcommand{\LIMINFN}{\underset{n\to\infty}\liminf \,}
\newcommand{\ROOTRULE}[1]{\LIMSUPK \BARS{#1}^{1/k}}

\newcommand{\CUPK}{\bigcup_{k=1}^{\infty}}
\newcommand{\CAPK}{\bigcap_{k=1}^{\infty}}
\newcommand{\CUPN}{\bigcup_{n=1}^{\infty}}
\newcommand{\CAPN}{\bigcap_{n=1}^{\infty}}

%%%%%%%
% number systems (real, rational, etc.)
%%%%%%%

\newcommand{\REALS}{\mathbb{R}}
\newcommand{\RATIONALS}{\mathbb{Q}}
\newcommand{\IRRATIONALS}{\REALS \backslash \RATIONALS}
\newcommand{\INTEGERS}{\mathbb{Z}}
\newcommand{\NUMBERS}{\mathbb{N}}
\newcommand{\COMPLEX}{\mathbb{C}}
\newcommand{\DISC}{\mathbb{D}}
\newcommand{\HPLANE}{\mathbb{H}}

\newcommand{\R}{\mathbb{R}}
\newcommand{\Q}{\mathbb{Q}}
\newcommand{\Z}{\mathbb{Z}}
\newcommand{\N}{\mathbb{N}}
\newcommand{\C}{\mathbb{C}}
\newcommand{\T}{\mathbb{T}}
\newcommand{\COUNTABLE}{\aleph_0}
\newcommand{\UNCOUNTABLE}{\aleph_1}


%%%%%%%
% Arithmetic/Algebraic operators
%%%%%%%


\DeclareMathOperator{\MOD}{mod}
%\newcommand{\MOD}[1]{\mod #1}
\newcommand{\BAR}[1]{\overline{#1}}
\newcommand{\LCM}{\text{ lcm}}
\newcommand{\ZMOD}[1]{\Z/#1\Z}
\DeclareMathOperator{\VAR}{Var}
%%%%%%%
% complex operators
%%%%%%%

\DeclareMathOperator{\RR}{Re}
%\newcommand{\RE}{\text{Re}}
\DeclareMathOperator{\IM}{Im}
%\newcommand{\IM}{\text{Im}}
\newcommand{\CONJ}[1]{\overline{#1}}
\DeclareMathOperator{\LOG}{Log}
%\newcommand{\LOG}{\text{ Log }}
\newcommand{\RES}[2]{\underset{#1}{\text{res}} #2}

%%%%%%%
% Group operators
%%%%%%%

\newcommand{\AUT}{\text{Aut}\,}
\newcommand{\KER}{\text{ker}\,}
\newcommand{\END}{\text{End}}
\newcommand{\HOM}{\text{Hom}}
\newcommand{\CYCLE}[1]{(\begin{array}{cccccccccc}
		#1
	\end{array})}
\newcommand{\SUBGROUP}{\underset{\text{group}}\subseteq}	
%\newcommand{\SUBGROUP}{\subseteq_g}
\newcommand{\SUBRING}{\underset{\text{ring}}\subseteq}
\newcommand{\SUBMOD}{\underset{\text{mod}}\subseteq}
\newcommand{\SUBFIELD}{\underset{\text{field}}\subseteq}
\newcommand{\ISO}{\underset{\text{iso}}\longrightarrow}
\newcommand{\HOMO}{\underset{\text{homo}}\longrightarrow}

%%%%%%%
% grouping (parenthesis, absolute value, square, multi-level brackets).
%%%%%%%

\newcommand{\PAREN}[1]{\left (\, #1 \,\right )}
\newcommand{\BRACKET}[1]{\left \{\, #1 \,\right \}}
\newcommand{\SQBRACKET}[1]{\left [\, #1 \,\right ]}
\newcommand{\ABRACKET}[1]{\left \langle\, #1 \,\right \rangle}
\newcommand{\BARS}[1]{\left |\, #1 \,\right |}
\newcommand{\DBARS}[1]{\left \| \, #1 \,\right \|}
\newcommand{\LBRACKET}[1]{\left \{ #1 \right .} 
\newcommand{\RBRACKET}[1]{\left . #1 \right \]}
\newcommand{\RBAR}[1]{\left . #1 \, \right |}
\newcommand{\LBAR}[1]{\left | \, #1 \right .}
\newcommand{\BLBRACKET}[2]{\BRACKET{\RBAR{#1}#2}}
\newcommand{\GEN}[1]{\ABRACKET{#1}}
\newcommand{\BINDEF}[2]{\LBRACKET{\begin{array}{ll}
     #1\\
     #2
\end{array}}}

%%%%%%%
% Fourier Analysis
%%%%%%%

\newcommand{\ONEOTWOPI}{\frac{1}{2\pi}}
\newcommand{\FHAT}{\hat{f}(n)}
\newcommand{\FINT}{\int_{-\pi}^\pi}
\newcommand{\FINTWO}{\int_{0}^{2\pi}}
\newcommand{\FSUMN}[1]{\SUM{n=-#1}{#1}}
%\newcommand{\FSUM}{\SUMN{\infty}}
\newcommand{\EIN}[1]{e^{in#1}}
\newcommand{\NEIN}[1]{e^{-in#1}}
\newcommand{\INTALL}{\int_{-\infty}^{\infty}}
\newcommand{\FTINT}[1]{\INTALL #1 e^{2\pi inx\xi} dx}
\newcommand{\GAUSS}{e^{-\pi x^2}}

%%%%%%%
% formatting 
%%%%%%%

\newcommand{\LEFTBOLD}[1]{\noindent\textbf{#1}}
\newcommand{\SEQ}[1]{\{#1\,\}}
\newcommand{\WIP}{\footnote{work in progress}}
\newcommand{\QED}{\hfill\square}
\newcommand{\ts}{\textsuperscript}
\newcommand{\HLINE}{\noindent\rule{7in}{1pt}\\}

%%%%%%%
% Mathematical note taking (definitions, theorems, etc.)
%%%%%%%

\newcommand{\REM}{\noindent\textbf{\\Remark: }}
\newcommand{\DEF}{\noindent\textbf{\\Definition: }}
\newcommand{\THE}{\noindent\textbf{\\Theorem: }}
\newcommand{\COR}{\noindent\textbf{\\Corollary: }}
\newcommand{\LEM}{\noindent\textbf{\\Lemma: }}
\newcommand{\PROP}{\noindent\textbf{\\Proposition: }}
\newcommand{\PROOF}{\noindent\textbf{\\Proof: }}
\newcommand{\EXP}{\noindent\textbf{\\Example: }}
\newcommand{\TRICKS}{\noindent\textbf{\\Tricks: }}


%%%%%%%
% text highlighting
%%%%%%%

\newcommand{\B}[1]{\textbf{#1}}
\newcommand{\CAL}[1]{\mathcal{#1}}
\newcommand{\UL}[1]{\underline{#1}}

%%%%%%
% Linear Algebra
%%%%%%

\newcommand{\COLVECTOR}[1]{\PAREN{\begin{array}{c}
#1
\end{array} }}
\newcommand{\TWOXTWO}[4]{\PAREN{ \begin{array}{c c} #1&#2 \\ #3 & #4 \end{array} }}
\newcommand{\THREEXTHREE}[9]{\PAREN{ \begin{array}{c c c} #1&#2&#3 \\ #4 & #5 & #6 \\ #7 & #8 & #9 \end{array} }}
\newcommand{\NXN}{\PAREN{ \begin{array}{c c c c} 
			a_{11} & a_{12} & \cdots & a_{1n} \\
			a_{21} & a_{22} & \cdots & a_{2n} \\
			\vdots & \vdots & \ddots & a_{1n} \\
			a_{n1} & a_{n2} & \cdots & a_{nn} \\
		\end{array} }}
\newcommand{\SLR}{SL_2(\R)}
\newcommand{\GLR}{GL_2(\R)}
\DeclareMathOperator{\TR}{tr}
\DeclareMathOperator{\BIL}{Bil}
\DeclareMathOperator{\SPAN}{span}

%%%%%%%
%  White space
%%%%%%%

\newcommand{\BOXIT}[1]{\noindent\fbox{\parbox{\textwidth}{#1}}}


\newtheorem{theorem}{Theorem}[section]
\newtheorem{corollary}{Corollary}[theorem]
\newtheorem{lemma}[theorem]{Lemma}

\theoremstyle{definition}
\newtheorem{definition}[theorem]{Definition}
\newtheorem{prop}[theorem]{Proposition}

\theoremstyle{remark}
\newtheorem{remark}[theorem]{Remark}
\newtheorem{example}[theorem]{Example}
%\newtheorem*{proof}[theorem]{Proof}



\newcommand{\RED}[1]{\textcolor{red}{#1}}
\newcommand{\BLUE}[1]{\textcolor{blue}{#1}}
\newcommand{\GREEN}[1]{\textcolor{black!30!green}{#1}}
\newcommand{\ORANGE}[1]{\textcolor{orange}{#1}}
\newcommand{\F}{\textbf{F}}

\title{Advanced Linear Algebra}
\author{The Unforgetable Someone}
\date{Summer 2023}

\newcommand{\NORM}[1]{\,\left \Vert #1 \right \Vert}
\begin{document}

\begin{center}
	\Large{\CLASSNAME -- \SEMESTER} \\
\end{center}
\begin{center}
	\STUDENTNAME \\
	\ASSIGNMENT -- \DUEDATE\\
\end{center} 

\noindent p. 175 \#2, 4, 5, 10. \\

\noindent \textbf{\#2}.  Show that if the orthogonal dimension of Hilbert Space $H$ is finite, it equals the dimension of $H$ regarded as a vector space; conversely, if the latter is finite, show that so is the former.\\
\\
Assuming that the Hilbert Dimension of HIlbert space $H$ is finite, $\dim H = n$.  Let the orthonormal family $(e_\alpha)_{\alpha \in A}\in H$ for some set $A$.  Further let there exists a countable subset of $A'$ such $\ABRACKET{e_{\alpha_j},e_{\alpha_k}} = \delta_{jk}$ for all $j,k \in [1..n]$ and that $\SPAN\{e_{\alpha_k}\}$ is dense and equal to $H$.  Further, if $y \perp \SPAN\{e_{\alpha_k}\}$ then $y = 0$. Therefore, given any $x \in H$, $x = \SUM{k=1}{n} \ABRACKET{x,e_{\alpha_k}}e_{\alpha_k}$.  We can see, then, that every $x\in H$ is a linear combination of $\{e_{\alpha_k}\}$.  Hence $\{ e_{\alpha_k}\}$ forms a basis.  There must be $n$ elements in $A'$, hence the vector space dimension is the same as the Hilbert space dimension.\\
\\
Assuming that we have a finite dimensional vector space $X$.  Then there exists an orthonormal basis $\{e_k\} \in X$.  Define an Inner Product on $X, \ABRACKET{\cdot,\cdot}$.  Clearly, $\ABRACKET{e_j,e_k} = \delta_{jk}$.  Also, given any $x \in X$ such that $\ABRACKET{x, e_k}=0$ for all $k\in [1,n]$ we can see that $x=0$ as all $e_k$ are linearly independent from each other.  Hence, $\SPAN{e_k}$ is dense in $X$.  $X$ must be a Hilbert space with dimension $n$.\\
\\
\noindent \textbf{\#4} Derive from (3) the following formula (which is often called the \textit{Parseval relation}).
\begin{align*}
	\ABRACKET{x,y} &= \sum_k \ABRACKET{x,e_k}\CONJ{\ABRACKET{y,e_k}}
\end{align*}
Given an orthonormal basis $\{e_k\}_{k=1}^\infty$ on $H$ we can define $x \in H$ as
\begin{align*}
	x &= \sum_k \ABRACKET{x, e_k}e_k\\
	\NORM{x}^2 &= \BARS{ \ABRACKET{x,x} }\\
	&= \BARS{ \ABRACKET{\sum_k \ABRACKET{x, e_k}e_k,\sum_j \ABRACKET{x, e_j}e_j} } \\
	&= \sum_k \sum_j \BARS{\ABRACKET{\ABRACKET{x, e_k}e_k, \ABRACKET{x, e_j}e_j}}\\
	&= \sum_k \sum_j \BARS{\ABRACKET{x, e_k}\CONJ{\ABRACKET{x, e_j}}\ABRACKET{e_k,e_j}}\\
	&= \sum_k \sum_j \BARS{\ABRACKET{x, e_k}\CONJ{\ABRACKET{x, e_j}}\delta_{jk}}\\
	 &= \sum_k \BARS{\ABRACKET{x, e_k}\CONJ{\ABRACKET{x, e_j}}}
\end{align*}replacing the right $x$ in the Inner Product with $y$ and we get
\begin{align*}
	\ABRACKET{x,y} &= \sum_k \ABRACKET{x,e_k}\CONJ{\ABRACKET{y,e_k}}
\end{align*}


\noindent \#5 Show that an orthonormal family $(e_\kappa), \kappa \in I$, in a Hilbert Space $H$ is total if and only if the relation in Prob. 4 holds for every $x$ and $y$ in $H$. \\
\\
$(\Rightarrow)$ Let an orthonormal family $(e_\kappa), \kappa \in I$, in a Hilbert Space $H$ be total.  Let $x,y \in H$ we know that we can represent them as $x = \SUM{k}{} \ABRACKET{x, e_\kappa}e_\kappa$ and $y = \SUM{\iota}{} \ABRACKET{y, e_\kappa}e_\iota$ where $\kappa, \iota \in I$.  Thus 
\begin{align*}
	\ABRACKET{x,y} &= \ABRACKET{\sum_\kappa \ABRACKET{x, e_\kappa}e_\kappa,\, \sum_{\iota} \ABRACKET{y, e_\iota}e_\iota} \\
	&= \sum_\kappa \sum_\iota \ABRACKET{ \ABRACKET{x, e_\kappa}e_\kappa,\, \ABRACKET{y, e_\iota}e_\iota} \\
	&= \sum_\kappa \sum_\iota \ABRACKET{x, e_\kappa}\CONJ{\ABRACKET{y, e_\iota}}\ABRACKET{ e_\kappa,\, e_\iota} \\
	&= \sum_\kappa \sum_\iota \ABRACKET{x, e_\kappa}\CONJ{\ABRACKET{y, e_\iota}}\delta_{\kappa\iota} \\
	&= \sum_\kappa \ABRACKET{x, e_\kappa}\CONJ{\ABRACKET{y, e_\kappa}}
\end{align*}$x,y$ are arbitrary therefore true for all elements of $H$.\\
\\
$(\Leftarrow)$Assuming that this is true for all $x,y \in H$ and we have an orthonormal set $(e_\kappa) \in H$ where $\kappa \in I$.  We can see that the same steps can be executed in reverse indicating that all $x \in H$ can be represented as $x = \SUM{k}{} \ABRACKET{x, e_\kappa}e_\kappa$.  This indicates that $\SPAN\{e_\kappa\} = H$.  Let $z \in H$ such that $z \in (e_\kappa)^\perp$ then $\ABRACKET{z, e_\kappa} = 0$ for all $\kappa \in I$.  We know that $z = \SUM{\kappa}{} \ABRACKET{z,e_\kappa}e_\kappa$. Therefore $z = 0$ and $(e_\kappa)^\perp = \{0\}$, hence, $\SPAN\{e_\kappa\}$ must be dense.\\

\noindent \textbf{\#10}  Let $M$ be a subset of a Hilbert space $H$, and let $v,w \in H$.  Suppose that $\ABRACKET{v,x}=\ABRACKET{w,x}$ for all $x \in M$ implies $v=w$.  If this holds for all $v,w \in H$ show that $M$ is total in $H$.

\begin{align*}
	\text{let } z &\in M^\perp, \, \ABRACKET{ z, x} = 0, \,  \forall x \in M\\
	\ABRACKET{v,x} &= \ABRACKET{w,x} + \ABRACKET{z,x} \\
	&= \ABRACKET{w+z, x}\\
	v &= w+z \\
	v-w &= z
\end{align*}$v-w = 0$ thus $z=0$.  Since $z$ was arbitrary $M^\perp = \{0\}$ and therefore $\SPAN\{M\}$ is dense in $H$.

\newpage
\noindent p. 194 \#2, 4, 5, 7.

\noindent \#2 \textbf{(Space $\ell^2$)} Show that every bounded linear functional $f$ on $\ell^2$ can be represented in the form
\begin{align*}
	f(x) &= \sum_{j=1}^\infty \xi_j\CONJ{\zeta_j} & [z=(\zeta_j)\in \ell^2].
\end{align*}

Given $x = (\xi_n) \in \ell^2$. Let $f$ be a bounded linear functional.  Then, by the Reisz Representation Theorem
\begin{align*}
	f: \ell^2 &\to K & K = \R \text{ or } \C\\
	f(x) &= \ABRACKET{x,z} &\text{ for some } z = (\zeta_n) \in \ell^2 \\ 
	&= \ABRACKET{(\xi_n),(\zeta_m)} \\
	&= \sum_{k=1}^\infty \xi_k \CONJ{\zeta_k}
\end{align*}\\

\noindent \#4 Consider Prob. 3.  If the mapping $X \to X'$ given by $z \mapsto f$ is surjective, show that $X$ must be a Hilbert space.\\

Given $z$ then $f$ is a bounded linear functional on $X$ with $\NORM{z}=\NORM{f}$.
\begin{align*}
	z &\mapsto f(x)=\ABRACKET{x,z}
\end{align*}

\noindent \#5 Show that the dual space of the real space $\ell^2$ is $\ell^2$. (Use 3.8-1.)\\
\\
Let $f \in \ell^{2'}$.  Then there exists $z = (\zeta_n) \in \ell^2$ such that $f(x) = \ABRACKET{x,z}$ for all $x \in \ell^2$.  Hence, for $x=(\xi_n)$ we have
\begin{align*}
	f(x) &= \ABRACKET{x,z} \\
	&= \ABRACKET{(\xi_n),(\zeta_m)} \\
	&= \sum_{k=1}^\infty \xi_k \zeta_k \\
	&= \ABRACKET{(\zeta_m), (\xi_n)} \\
	&= \ABRACKET{z,x}
\end{align*}
\\
\\

\noindent \#7 Show that the dual space $H'$ of a Hilbert space $H$ is a Hilbert space with inner product $\ABRACKET{\cdot, \cdot}_1$ defined by
\begin{align*}
	\ABRACKET{f_x,f_v}_1 &= \CONJ{\ABRACKET{z,v}}= \ABRACKET{v,z},
\end{align*}where $f_z(x) = \ABRACKET{x,z}$, etc.

\end{document}