\documentclass[10pt,a4paper]{report}
\usepackage[utf8]{inputenc}
\usepackage{amsmath}
\usepackage{amsfonts}
\usepackage{amssymb}
\usepackage{amsthm}
\usepackage{hyperref}

\usepackage{multicol}
\usepackage{fancyhdr}
\usepackage{enumitem}
\usepackage{tikz}
\usepackage{tikz-cd}
\usetikzlibrary{calc}
\usetikzlibrary{shapes.geometric}
\usepackage[margin=0.5in]{geometry}
\usepackage{xcolor}
\DeclareMathOperator{\RANGE}{range}
\DeclareMathOperator{\NULL}{null}

\hypersetup{
    colorlinks=true,
    linkcolor=blue,
    filecolor=magenta,      
    urlcolor=cyan,
    pdftitle={Tensors},
    pdfpagemode=FullScreen,
    }

%\urlstyle{same}

\newcommand{\CLASSNAME}{Functional Analysis}
\newcommand{\STUDENTNAME}{Paul Carmody}
\newcommand{\ASSIGNMENT}{Assignment \#2}
\newcommand{\DUEDATE}{March 1, 2024}
\newcommand{\SEMESTER}{Spring 2024}
\newcommand{\SCHEDULE}{T/Th 9:30 -- 10:45}
\newcommand{\ROOM}{Remote}

\pagestyle{fancy}
\fancyhf{}
\chead{ \fancyplain{}{\CLASSNAME} }
%\chead{ \fancyplain{}{\STUDENTNAME} }
\rhead{\thepage}
\newcommand{\LET}{\text{Let }}
%\newcommand{\IF}{\text{if }}
\newcommand{\AND}{\text{ and }}
\newcommand{\OR}{\text{ or }}
\newcommand{\FORSOME}{\text{ for some }}
\newcommand{\FORALL}{\text{ for all }}
\newcommand{\WHERE}{\text{ where }}
\newcommand{\WTS}{\text{ WTS }}
\newcommand{\WLOG}{\text{ WLOG }}
\newcommand{\BS}{\backslash}
\newcommand{\DEFINE}[1]{\textbf{\emph{#1}}}
\newcommand{\IF}{$(\Rightarrow)$}
\newcommand{\ONLYIF}{$(\Leftarrow)$}
\newcommand{\ITH}{\textsuperscript{th} }
\newcommand{\FST}{\textsuperscript{st} }
\newcommand{\SND}{\textsuperscript{nd} }
\newcommand{\TRD}{\textsuperscript{rd} }
\newcommand{\INV}{\textsuperscript{-1} }

\newcommand{\XXX}{\mathfrak{X}}
\newcommand{\MMM}{\mathfrak{M}}
%\newcommand{\????}{\textfrak{A}}
%\newcommand{\????}{\textgoth{A}}
%\newcommand{\????}{\textswab{A}}

\DeclareMathOperator{\DER}{Der}
\DeclareMathOperator{\SGN}{sgn}

%%%%%%%
% derivatives
%%%%%%%

\newcommand{\PART}[2]{\frac{\partial #1}{\partial #2}}
\newcommand{\SPART}[2]{\frac{\partial^2 #1}{\partial #2^2}}
\newcommand{\DERIV}[2]{\frac{d #1}{d #2}}
\newcommand{\LAPLACIAN}[1]{\frac{\partial^2 #1}{\partial x^2} + \frac{\partial^2 #1}{\partial y^2}}

%%%%%%%
% sum, product, union, intersections
%%%%%%%

\newcommand{\SUM}[2]{\underset{#1}{\overset{#2}{\sum}}}
\newcommand{\PROD}[2]{\underset{#1}{\overset{#2}{\prod}}}
\newcommand{\UNION}[2]{\underset{#1}{\overset{#2}{\bigcup}}}
\newcommand{\INTERSECT}[2]{\underset{#1}{\overset{#2}{\bigcap}}}
\newcommand{\FSUM}{\SUM{n=-\infty}{\infty}}
       

%%%%%%%
% supremum and infimum
%%%%%%%

\newcommand{\SUP}[1]{\underset{#1}\sup \,}
\newcommand{\INF}[1]{\underset{#1}\inf \,}
\newcommand{\MAX}[1]{\underset{#1}\max \,}
\newcommand{\MIN}[1]{\underset{#1}\min \,}

%%%%%%%
% infinite sums, limits
%%%%%%%

\newcommand{\SUMK}{\SUM{k=1}{\infty}}
\newcommand{\SUMN}{\SUM{n=1}{\infty}}
\newcommand{\SUMKZ}{\SUM{k=0}{\infty}}
\newcommand{\LIM}[1]{\underset{#1}\lim\,}
\newcommand{\IWOB}[1]{\LIM{#1 \to \infty}}
\newcommand{\LIMK}{\IWOB{k}}
\newcommand{\LIMN}{\IWOB{n}}
\newcommand{\LIMX}{\IWOB{x}}
\newcommand{\NIWOB}{\LIM{n \to \infty}}
\newcommand{\LIMSUPK}{\underset{k\to\infty}\limsup \,}
\newcommand{\LIMSUPN}{\underset{n\to\infty}\limsup \,}
\newcommand{\LIMINFK}{\underset{k\to\infty}\liminf \,}
\newcommand{\LIMINFN}{\underset{n\to\infty}\liminf \,}
\newcommand{\ROOTRULE}[1]{\LIMSUPK \BARS{#1}^{1/k}}

\newcommand{\CUPK}{\bigcup_{k=1}^{\infty}}
\newcommand{\CAPK}{\bigcap_{k=1}^{\infty}}
\newcommand{\CUPN}{\bigcup_{n=1}^{\infty}}
\newcommand{\CAPN}{\bigcap_{n=1}^{\infty}}

%%%%%%%
% number systems (real, rational, etc.)
%%%%%%%

\newcommand{\REALS}{\mathbb{R}}
\newcommand{\RATIONALS}{\mathbb{Q}}
\newcommand{\IRRATIONALS}{\REALS \backslash \RATIONALS}
\newcommand{\INTEGERS}{\mathbb{Z}}
\newcommand{\NUMBERS}{\mathbb{N}}
\newcommand{\COMPLEX}{\mathbb{C}}
\newcommand{\DISC}{\mathbb{D}}
\newcommand{\HPLANE}{\mathbb{H}}

\newcommand{\R}{\mathbb{R}}
\newcommand{\Q}{\mathbb{Q}}
\newcommand{\Z}{\mathbb{Z}}
\newcommand{\N}{\mathbb{N}}
\newcommand{\C}{\mathbb{C}}
\newcommand{\T}{\mathbb{T}}
\newcommand{\COUNTABLE}{\aleph_0}
\newcommand{\UNCOUNTABLE}{\aleph_1}


%%%%%%%
% Arithmetic/Algebraic operators
%%%%%%%


\DeclareMathOperator{\MOD}{mod}
%\newcommand{\MOD}[1]{\mod #1}
\newcommand{\BAR}[1]{\overline{#1}}
\newcommand{\LCM}{\text{ lcm}}
\newcommand{\ZMOD}[1]{\Z/#1\Z}
\DeclareMathOperator{\VAR}{Var}
%%%%%%%
% complex operators
%%%%%%%

\DeclareMathOperator{\RR}{Re}
%\newcommand{\RE}{\text{Re}}
\DeclareMathOperator{\IM}{Im}
%\newcommand{\IM}{\text{Im}}
\newcommand{\CONJ}[1]{\overline{#1}}
\DeclareMathOperator{\LOG}{Log}
%\newcommand{\LOG}{\text{ Log }}
\newcommand{\RES}[2]{\underset{#1}{\text{res}} #2}

%%%%%%%
% Group operators
%%%%%%%

\newcommand{\AUT}{\text{Aut}\,}
\newcommand{\KER}{\text{ker}\,}
\newcommand{\END}{\text{End}}
\newcommand{\HOM}{\text{Hom}}
\newcommand{\CYCLE}[1]{(\begin{array}{cccccccccc}
		#1
	\end{array})}
\newcommand{\SUBGROUP}{\underset{\text{group}}\subseteq}	
%\newcommand{\SUBGROUP}{\subseteq_g}
\newcommand{\SUBRING}{\underset{\text{ring}}\subseteq}
\newcommand{\SUBMOD}{\underset{\text{mod}}\subseteq}
\newcommand{\SUBFIELD}{\underset{\text{field}}\subseteq}
\newcommand{\ISO}{\underset{\text{iso}}\longrightarrow}
\newcommand{\HOMO}{\underset{\text{homo}}\longrightarrow}

%%%%%%%
% grouping (parenthesis, absolute value, square, multi-level brackets).
%%%%%%%

\newcommand{\PAREN}[1]{\left (\, #1 \,\right )}
\newcommand{\BRACKET}[1]{\left \{\, #1 \,\right \}}
\newcommand{\SQBRACKET}[1]{\left [\, #1 \,\right ]}
\newcommand{\ABRACKET}[1]{\left \langle\, #1 \,\right \rangle}
\newcommand{\BARS}[1]{\left |\, #1 \,\right |}
\newcommand{\DBARS}[1]{\left \| \, #1 \,\right \|}
\newcommand{\LBRACKET}[1]{\left \{ #1 \right .} 
\newcommand{\RBRACKET}[1]{\left . #1 \right \]}
\newcommand{\RBAR}[1]{\left . #1 \, \right |}
\newcommand{\LBAR}[1]{\left | \, #1 \right .}
\newcommand{\BLBRACKET}[2]{\BRACKET{\RBAR{#1}#2}}
\newcommand{\GEN}[1]{\ABRACKET{#1}}
\newcommand{\BINDEF}[2]{\LBRACKET{\begin{array}{ll}
     #1\\
     #2
\end{array}}}

%%%%%%%
% Fourier Analysis
%%%%%%%

\newcommand{\ONEOTWOPI}{\frac{1}{2\pi}}
\newcommand{\FHAT}{\hat{f}(n)}
\newcommand{\FINT}{\int_{-\pi}^\pi}
\newcommand{\FINTWO}{\int_{0}^{2\pi}}
\newcommand{\FSUMN}[1]{\SUM{n=-#1}{#1}}
%\newcommand{\FSUM}{\SUMN{\infty}}
\newcommand{\EIN}[1]{e^{in#1}}
\newcommand{\NEIN}[1]{e^{-in#1}}
\newcommand{\INTALL}{\int_{-\infty}^{\infty}}
\newcommand{\FTINT}[1]{\INTALL #1 e^{2\pi inx\xi} dx}
\newcommand{\GAUSS}{e^{-\pi x^2}}

%%%%%%%
% formatting 
%%%%%%%

\newcommand{\LEFTBOLD}[1]{\noindent\textbf{#1}}
\newcommand{\SEQ}[1]{\{#1\,\}}
\newcommand{\WIP}{\footnote{work in progress}}
\newcommand{\QED}{\hfill\square}
\newcommand{\ts}{\textsuperscript}
\newcommand{\HLINE}{\noindent\rule{7in}{1pt}\\}

%%%%%%%
% Mathematical note taking (definitions, theorems, etc.)
%%%%%%%

\newcommand{\REM}{\noindent\textbf{\\Remark: }}
\newcommand{\DEF}{\noindent\textbf{\\Definition: }}
\newcommand{\THE}{\noindent\textbf{\\Theorem: }}
\newcommand{\COR}{\noindent\textbf{\\Corollary: }}
\newcommand{\LEM}{\noindent\textbf{\\Lemma: }}
\newcommand{\PROP}{\noindent\textbf{\\Proposition: }}
\newcommand{\PROOF}{\noindent\textbf{\\Proof: }}
\newcommand{\EXP}{\noindent\textbf{\\Example: }}
\newcommand{\TRICKS}{\noindent\textbf{\\Tricks: }}


%%%%%%%
% text highlighting
%%%%%%%

\newcommand{\B}[1]{\textbf{#1}}
\newcommand{\CAL}[1]{\mathcal{#1}}
\newcommand{\UL}[1]{\underline{#1}}

%%%%%%
% Linear Algebra
%%%%%%

\newcommand{\COLVECTOR}[1]{\PAREN{\begin{array}{c}
#1
\end{array} }}
\newcommand{\TWOXTWO}[4]{\PAREN{ \begin{array}{c c} #1&#2 \\ #3 & #4 \end{array} }}
\newcommand{\DTWOXTWO}[4]{\BARS{ \begin{array}{c c} #1&#2 \\ #3 & #4 \end{array} }}
\newcommand{\THREEXTHREE}[9]{\PAREN{ \begin{array}{c c c} #1&#2&#3 \\ #4 & #5 & #6 \\ #7 & #8 & #9 \end{array} }}
\newcommand{\DTHREEXTHREE}[9]{\BARS{ \begin{array}{c c c} #1&#2&#3 \\ #4 & #5 & #6 \\ #7 & #8 & #9 \end{array} }}
\newcommand{\NXN}{\PAREN{ \begin{array}{c c c c} 
			a_{11} & a_{12} & \cdots & a_{1n} \\
			a_{21} & a_{22} & \cdots & a_{2n} \\
			\vdots & \vdots & \ddots & a_{1n} \\
			a_{n1} & a_{n2} & \cdots & a_{nn} \\
		\end{array} }}
\newcommand{\SLR}{SL_2(\R)}
\newcommand{\GLR}{GL_2(\R)}
\DeclareMathOperator{\TR}{tr}
\DeclareMathOperator{\BIL}{Bil}
\DeclareMathOperator{\SPAN}{span}

%%%%%%%
%  White space
%%%%%%%

\newcommand{\BOXIT}[1]{\noindent\fbox{\parbox{\textwidth}{#1}}}


\newtheorem{theorem}{Theorem}[section]
\newtheorem{corollary}{Corollary}[theorem]
\newtheorem{lemma}[theorem]{Lemma}

\theoremstyle{definition}
\newtheorem{definition}[theorem]{Definition}
\newtheorem{prop}[theorem]{Proposition}

\theoremstyle{remark}
\newtheorem{remark}[theorem]{Remark}
\newtheorem{example}[theorem]{Example}
%\newtheorem*{proof}[theorem]{Proof}



\newcommand{\RED}[1]{\textcolor{red}{#1}}
\newcommand{\BLUE}[1]{\textcolor{blue}{#1}}
\newcommand{\GREEN}[1]{\textcolor{black!30!green}{#1}}
\newcommand{\ORANGE}[1]{\textcolor{orange}{#1}}
\newcommand{\F}{\textbf{F}}

\title{Advanced Linear Algebra}
\author{The Unforgetable Someone}
\date{Summer 2023}

\newcommand{\NORM}[1]{\,\left \Vert #1 \right \Vert\,}
\begin{document}

\begin{center}
	\Large{\CLASSNAME -- \SEMESTER} \\
\end{center}
\begin{center}
	\STUDENTNAME \\
	\ASSIGNMENT -- \DUEDATE\\
\end{center} 
\vskip 0.5cm
p. 81 \#7.  If $\dim Y < \infty$ in Riesz's lemma 2.5-4, show that one can even choose $\theta = 1$.\\
\\
To restate the theorem: \textbf{F. Riesz's Lemma}. Let $Y$ and $Z$ be subspaces of a normed space $X$ (of any dimension), and suppose that $Y$ is closed and is a proper
subset of $Z$. Then for every real number $\theta$ in the interval (0,1) there is a
$z \in Z$ such that $\NORM{z}= 1,\NORM{z - y} < \theta$ for all $y \in Y$.\\
\\
Suppose that we had a sequence $\theta_m$ such that $\LIM{m->\infty} \theta_m =1$ and there is a corresponding sequence $z_m$ such that $\NORM{z_m} = 1$ and $\NORM{z_m-y}<\theta_m$ for all $y \in Y$ and $m \in \N$.  Since, $\dim Y < \infty$, $Y$ is closed and bounded, hence every sequence converges.  Thus, the sequence $\NORM{z_m -y }$ converges to 1 and can include 1.

\newpage
 p. 101 \#3, 5, 6, 7, 8, 9. 
\begin{enumerate}
	\setcounter{enumi}{2}
	\item If $T \ne 0$ is a bounded linear operator, show that for any $x \in \mathcal{D}(T)$ such that $\NORM{x} < 1$ we have the strict inequality $\NORM{Tx} < \NORM{T}$.\\
	\\
	Since $T$ is bounded, $\exists c \to \NORM{Tx}\le c||x||$ and $c = \NORM{T}$.  When, $\NORM{x} = 1$ we have $\NORM{Tx} \le \NORM{T}$ and otherwise
	\begin{align*}
		\NORM{T} &= \sup_{x\in\mathcal{D}(T),x\ne 0} \frac{\NORM{Tx}}{\NORM{x}} \\
		\NORM{Tx} &\le  \sup_{x\in\mathcal{D}(T),x\ne 0} \frac{\NORM{Tx}}{\NORM{x}} \\
		1 & \le \sup_{x\in\mathcal{D}(T),x\ne 0} \frac{1}{\NORM{x}}
	\end{align*}which is a strict inequality for $\NORM{x}<1$.
	
	\setcounter{enumi}{4}
	\item Show that the operator $T: \ell^\infty \to \ell^\infty$ defined by $y = (\eta_i)=Tx, \eta_j=\xi_j/j, x=(\xi_j),$ is linear and bounded.\\
	\begin{align*}
		T : \ell^\infty &\to \ell^\infty \\
		x=(\xi_j) &\mapsto y = (\xi_j/j)\\
		\\
		\text{let } u&= (\zeta_j)\\
		T(\alpha x + \beta u) &= \PAREN{ \PAREN{\alpha (\xi_j)+ \beta (\zeta_j)}/j } \\
		&= \PAREN{ \alpha (\xi_j)/j+ \beta (\zeta_j)/j } \\
		&= \PAREN{ \alpha (\xi_j/j)+ \beta (\zeta_j/j) } \\
		&= \PAREN{ \alpha (\xi_j/j) }+ \PAREN{\beta (\zeta_j/j) } \\
		&= \alpha \PAREN{ (\xi_j/j) }+ \beta\PAREN{ (\zeta_j/j) } \\
		&= \alpha Tx + \beta Tu \\
		\\
		\exists c>0 \to \NORM{Tx} &\le c \NORM{x}, \, \forall x\in \ell^\infty
	\end{align*}Let $j\in \N$ be such that $\NORM{x} = \xi_j$.  Even if $j=1$ we can see that $\NORM{Tx} \le \xi_j$ because $\xi_j \ge \xi_j/j$ for all $j$.  Thus, $\NORM{Tx} \le \NORM{x}$, hence $T$ is bounded.
	\item \textbf{(Range)} Show that the range $\mathcal{R}(T)$ of a bounded linear operator $T: X\to Y$ need not be closed in $Y$.  \textit{Hint.} Use $T$ in Prob 5.\\
	\\
	Let $x=(\xi_m) \in \ell^\infty$ and $\LIM{m\to \infty} \xi_m = 0$.  Then, from Prob 5, 
	\begin{align*}
		T : \ell^\infty &\to \ell^\infty \\
		x=(\xi_j) &\mapsto y = (\xi_j/j)\\
		Tx &= (\xi_j/j)
	\end{align*}Notice that if $1/j< \xi_j$ then $\xi_j/j > 1$, that is, if there exists $N$ such that $n>N$ implies that $1/n < \xi_n$ then $Tx$ does not converge to zero.  Hence, the range of $Tx$ is open.\\
	\\
	\item \textbf{(Inverse operator)} Let $T$ be a bounded linear operator from a normed space $X$ onto a normed space $Y$.  If there is a positive $b$ such that $$ \NORM{Tx}\ge b\NORM{x} \text{ for all } x \in X$$ show that then $T^{-1}:Y\to X$ exists and is bounded.\\
	
	Notice that $\NORM{Tx} \ge b\NORM{x}$ implies that $T0 = 0$ which makes it injective, hence an inverse exists.  Then,
	\begin{align*}
		\NORM{T^{-1}x} &\le b\NORM{T^{-1}Tx} \le b\NORM{x}
	\end{align*}means that $T$ is bounded.
	\newpage
	\item Show that the inverse $T^{-1}:\mathcal{R}(T)\to X$ of a bounded linear operator $T:X\to Y$ need not be bounded.  \textit{Hint.}  Use $T$ in Prob. 5.\\
	\\
	Using $T$ as defined in Prob. 5, $T^{-1}y = (\eta_j j)$.  Clearly there is no $c$ such that $\NORM{T^{-1}y} \le c\NORM{y}$ for all $y$, therefore $T^{-1}$ can be unbounded.
	\\
	\item Let $T: C[0,1] \to C[0,1]$ be defined by 
	\begin{align*}
		y(t) = \int_0^t x(\tau) d\tau.
	\end{align*}Find $\mathcal{R}(T)$ and $T^{-1}:\mathcal{R}(T)\to C[0,1]$.  Is $T^{-1}$ linear and bounded?\\
	\\
	After integration, each $y(t)$ will be the anti-derivative of $x(\tau)$, which is a differentiable function.  That is $\mathcal{R}(T)$ will be the set of differentiable functions on $[0,1]$.  $T^{-1}(z) = z'$.  $\NORM{T^{-1}z} =\SUP{t\in[0,1]} z'(t)$.  However, notice that given a polynomial $z^n$ then $T^{-1}(z^n)= nz^{n-1}$ and $\NORM{T^{-1}(z^n)} = n$.  $n$ is arbitrary, therefore, $T^{-1}$ is not bounded.
\end{enumerate}
\newpage 

p. 109 \#2, 3, 4.
\begin{enumerate}
	\setcounter{enumi}{1}
	\item Show that the functionals defined on $C[a,b]$ by 
	\begin{align*}
		f_1(x) &= \int_a^b x(t)y_0(t)dt & (y_o\in C[a,b]) \\
		f_2(x) &= \alpha x(a) + \beta x(b) & (\alpha, \beta \text{ fixed})
	\end{align*}are linear and bounded.\\
	Let $p,q \in C[a,b]$
	\begin{align*}
		f_1(\alpha p + q) &= \int_a^b (\alpha p+q)(t)y_0(t)dt = \alpha\int_a^b p(t)y_0(t)dt + \int_a^b q(t)y_0(t)dt = \alpha f_1(p)+f_1(q)\\ \\
		f_2(\gamma p+q) &= \alpha (\gamma p+q)(a) + \beta (\gamma p+q)(b) \\
		&= \alpha \gamma p(a)+\alpha q(a) + \beta \gamma p(b)+\beta q(b) \\
		&= \gamma f_2(p)+f_2(q)
	\end{align*}
	\begin{align*}
		\NORM{f_1(x)} &\le \max_{t\in [a,b]} (|x(t)y_0(t)|) \\
		&\le \max_{t\in [a,b]} (|x(t)| |y_0(t)|) \\
		&\le \NORM{y_0} \NORM{x}
	\end{align*}$\NORM{y_0}$ is a constant, hence $f_1$ is bounded.  Then, with the extreme value theorem, there exists $s \in [a,b]$ such that $x(s) \ge x(t), \forall t \in [a,b]$ 
	\begin{align*}
		\NORM{f_2(x)} &\le | \alpha x(s) + \beta x(s) | \\
		&\le | x(s) (\alpha + \beta)| \\
		&\le |\alpha + \beta| |x(s)| \\
		&\le |\alpha + \beta| \NORM{x(s)}
	\end{align*}hence $f_2$ is bounded.
	\\
	\item Find the norm of the linear functional $f$ defined on $C[-1,1]$ by 
	\begin{align*}
		f(x) = \int_{-1}^0 x(t)dt-\int_0^1 x(t) dt.
	\end{align*}Let $g, h$ linear functionals on $C[-1,1]$ and $g(x) = \int_{-1}^0 x(t)dt,\, h(x) = -\int_0^1 x(t) dt$ then
	\begin{align*}
		f(x) &= (g+h)(x) \\
		\NORM{f} &\le \NORM{g} + \NORM{h} \\
		\NORM{g(x)} &= \BARS{\int_{-1}^0 x(t)dt} \le \max_{t\in [-1,0]} \BARS{x(t)}\\
		\NORM{h(x)} &= \BARS{-\int_0^1 x(t)dt} \le \max_{t\in [0,1]} \BARS{x(t)}\\
		\NORM{f} &\le \max\PAREN{\NORM{g},\NORM{h}} = \max_{t\in [-1,1]} \BARS{x(t)}\\
	\end{align*} 
	\item Show that
	\begin{align*}
		f_1(x) &= \max_{t\in J} x(t) \\
		 & & J=[a,b]\\
		 f(2) &= \min_{t \in J} x(t)
	\end{align*}define functionals on $C[a,b]$.  Are they linear? Bounded?\\
	\\
	It seems pretty clear that 
	\begin{align*}
		\max_{t\in J} (x+y)(t) &\ne \max_{t\in J}x(t) + \max_{t\in J} y(t) \\
		\text{ and } \min_{t\in J} (x+y)(t) &\ne \min_{t\in J}x(t) + \min_{t\in J} y(t) 
	\end{align*}so, they cannot be linear. They are, however, limited by that fact that being continuous functions each $x \in C[a,b]$ must have finite values on the entire interval (no points go to infinity).  Hence, bounded.
\end{enumerate}
\end{document}