\documentclass[10pt,a4paper]{report}
\usepackage[utf8]{inputenc}
\usepackage{amsmath}
\usepackage{amsfonts}
\usepackage{amssymb}
\usepackage{amsthm}
\usepackage{hyperref}

\usepackage{multicol}
\usepackage{fancyhdr}
\usepackage{enumitem}
\usepackage{tikz}
\usepackage{tikz-cd}
\usetikzlibrary{calc}
\usetikzlibrary{shapes.geometric}
\usepackage[margin=0.5in]{geometry}
\usepackage{xcolor}
\DeclareMathOperator{\RANGE}{range}
\DeclareMathOperator{\NULL}{null}

\hypersetup{
    colorlinks=true,
    linkcolor=blue,
    filecolor=magenta,      
    urlcolor=cyan,
    pdftitle={Tensors},
    pdfpagemode=FullScreen,
    }

%\urlstyle{same}

\newcommand{\CLASSNAME}{Functional Analysis}
\newcommand{\STUDENTNAME}{Paul Carmody}
\newcommand{\ASSIGNMENT}{Assignment \#6}
\newcommand{\DUEDATE}{May 2, 2024}
\newcommand{\SEMESTER}{Spring 2024}
\newcommand{\SCHEDULE}{T/Th 9:30 -- 10:45}
\newcommand{\ROOM}{Remote}

\pagestyle{fancy}
\fancyhf{}
\chead{ \fancyplain{}{\CLASSNAME} }
%\chead{ \fancyplain{}{\STUDENTNAME} }
\rhead{\thepage}
\newcommand{\LET}{\text{Let }}
%\newcommand{\IF}{\text{if }}
\newcommand{\AND}{\text{ and }}
\newcommand{\OR}{\text{ or }}
\newcommand{\FORSOME}{\text{ for some }}
\newcommand{\FORALL}{\text{ for all }}
\newcommand{\WHERE}{\text{ where }}
\newcommand{\WTS}{\text{ WTS }}
\newcommand{\WLOG}{\text{ WLOG }}
\newcommand{\BS}{\backslash}
\newcommand{\DEFINE}[1]{\textbf{\emph{#1}}}
\newcommand{\IF}{$(\Rightarrow)$}
\newcommand{\ONLYIF}{$(\Leftarrow)$}
\newcommand{\ITH}{\textsuperscript{th} }
\newcommand{\FST}{\textsuperscript{st} }
\newcommand{\SND}{\textsuperscript{nd} }
\newcommand{\TRD}{\textsuperscript{rd} }
\newcommand{\INV}{\textsuperscript{-1} }


%%%%%%%
% derivatives
%%%%%%%

\newcommand{\PART}[2]{\frac{\partial #1}{\partial #2}}
\newcommand{\SPART}[2]{\frac{\partial^2 #1}{\partial #2^2}}
\newcommand{\DERIV}[2]{\frac{d #1}{d #2}}
\newcommand{\LAPLACIAN}[1]{\frac{\partial^2 #1}{\partial x^2} + \frac{\partial^2 #1}{\partial y^2}}

%%%%%%%
% sum, product, union, intersections
%%%%%%%

\newcommand{\SUM}[2]{\underset{#1}{\overset{#2}{\sum}}}
\newcommand{\PROD}[2]{\underset{#1}{\overset{#2}{\prod}}}
\newcommand{\UNION}[2]{\underset{#1}{\overset{#2}{\bigcup}}}
\newcommand{\INTERSECT}[2]{\underset{#1}{\overset{#2}{\bigcap}}}
\newcommand{\FSUM}{\SUM{n=-\infty}{\infty}}
       

%%%%%%%
% supremum and infimum
%%%%%%%

\newcommand{\SUP}[1]{\underset{#1}\sup \,}
\newcommand{\INF}[1]{\underset{#1}\inf \,}
\newcommand{\MAX}[1]{\underset{#1}\max \,}
\newcommand{\MIN}[1]{\underset{#1}\min \,}

%%%%%%%
% infinite sums, limits
%%%%%%%

\newcommand{\SUMK}{\SUM{k=1}{\infty}}
\newcommand{\SUMN}{\SUM{n=1}{\infty}}
\newcommand{\SUMKZ}{\SUM{k=0}{\infty}}
\newcommand{\LIM}[1]{\underset{#1}\lim\,}
\newcommand{\IWOB}[1]{\LIM{#1 \to \infty}}
\newcommand{\LIMK}{\IWOB{k}}
\newcommand{\LIMN}{\IWOB{n}}
\newcommand{\LIMX}{\IWOB{x}}
\newcommand{\NIWOB}{\LIM{n \to \infty}}
\newcommand{\LIMSUPK}{\underset{k\to\infty}\limsup \,}
\newcommand{\LIMSUPN}{\underset{n\to\infty}\limsup \,}
\newcommand{\LIMINFK}{\underset{k\to\infty}\liminf \,}
\newcommand{\LIMINFN}{\underset{n\to\infty}\liminf \,}
\newcommand{\ROOTRULE}[1]{\LIMSUPK \BARS{#1}^{1/k}}

\newcommand{\CUPK}{\bigcup_{k=1}^{\infty}}
\newcommand{\CAPK}{\bigcap_{k=1}^{\infty}}
\newcommand{\CUPN}{\bigcup_{n=1}^{\infty}}
\newcommand{\CAPN}{\bigcap_{n=1}^{\infty}}

%%%%%%%
% number systems (real, rational, etc.)
%%%%%%%

\newcommand{\REALS}{\mathbb{R}}
\newcommand{\RATIONALS}{\mathbb{Q}}
\newcommand{\IRRATIONALS}{\REALS \backslash \RATIONALS}
\newcommand{\INTEGERS}{\mathbb{Z}}
\newcommand{\NUMBERS}{\mathbb{N}}
\newcommand{\COMPLEX}{\mathbb{C}}
\newcommand{\DISC}{\mathbb{D}}
\newcommand{\HPLANE}{\mathbb{H}}

\newcommand{\R}{\mathbb{R}}
\newcommand{\Q}{\mathbb{Q}}
\newcommand{\Z}{\mathbb{Z}}
\newcommand{\N}{\mathbb{N}}
\newcommand{\C}{\mathbb{C}}
\newcommand{\T}{\mathbb{T}}
\newcommand{\COUNTABLE}{\aleph_0}
\newcommand{\UNCOUNTABLE}{\aleph_1}


%%%%%%%
% Arithmetic/Algebraic operators
%%%%%%%


\DeclareMathOperator{\MOD}{mod}
%\newcommand{\MOD}[1]{\mod #1}
\newcommand{\BAR}[1]{\overline{#1}}
\newcommand{\LCM}{\text{ lcm}}
\newcommand{\ZMOD}[1]{\Z/#1\Z}
\DeclareMathOperator{\VAR}{Var}
%%%%%%%
% complex operators
%%%%%%%

\DeclareMathOperator{\RR}{Re}
%\newcommand{\RE}{\text{Re}}
\DeclareMathOperator{\IM}{Im}
%\newcommand{\IM}{\text{Im}}
\newcommand{\CONJ}[1]{\overline{#1}}
\DeclareMathOperator{\LOG}{Log}
%\newcommand{\LOG}{\text{ Log }}
\newcommand{\RES}[2]{\underset{#1}{\text{res}} #2}

%%%%%%%
% Group operators
%%%%%%%

\newcommand{\AUT}{\text{Aut}\,}
\newcommand{\KER}{\text{ker}\,}
\newcommand{\END}{\text{End}}
\newcommand{\HOM}{\text{Hom}}
\newcommand{\CYCLE}[1]{(\begin{array}{cccccccccc}
		#1
	\end{array})}
\newcommand{\SUBGROUP}{\underset{\text{group}}\subseteq}	
%\newcommand{\SUBGROUP}{\subseteq_g}
\newcommand{\SUBRING}{\underset{\text{ring}}\subseteq}
\newcommand{\SUBMOD}{\underset{\text{mod}}\subseteq}
\newcommand{\SUBFIELD}{\underset{\text{field}}\subseteq}
\newcommand{\ISO}{\underset{\text{iso}}\longrightarrow}
\newcommand{\HOMO}{\underset{\text{homo}}\longrightarrow}

%%%%%%%
% grouping (parenthesis, absolute value, square, multi-level brackets).
%%%%%%%

\newcommand{\PAREN}[1]{\left (\, #1 \,\right )}
\newcommand{\BRACKET}[1]{\left \{\, #1 \,\right \}}
\newcommand{\SQBRACKET}[1]{\left [\, #1 \,\right ]}
\newcommand{\ABRACKET}[1]{\left \langle\, #1 \,\right \rangle}
\newcommand{\BARS}[1]{\left |\, #1 \,\right |}
\newcommand{\DBARS}[1]{\left \| \, #1 \,\right \|}
\newcommand{\LBRACKET}[1]{\left \{ #1 \right .} 
\newcommand{\RBRACKET}[1]{\left . #1 \right \]}
\newcommand{\RBAR}[1]{\left . #1 \, \right |}
\newcommand{\LBAR}[1]{\left | \, #1 \right .}
\newcommand{\BLBRACKET}[2]{\BRACKET{\RBAR{#1}#2}}
\newcommand{\GEN}[1]{\ABRACKET{#1}}
\newcommand{\BINDEF}[2]{\LBRACKET{\begin{array}{ll}
     #1\\
     #2
\end{array}}}

%%%%%%%
% Fourier Analysis
%%%%%%%

\newcommand{\ONEOTWOPI}{\frac{1}{2\pi}}
\newcommand{\FHAT}{\hat{f}(n)}
\newcommand{\FINT}{\int_{-\pi}^\pi}
\newcommand{\FINTWO}{\int_{0}^{2\pi}}
\newcommand{\FSUMN}[1]{\SUM{n=-#1}{#1}}
%\newcommand{\FSUM}{\SUMN{\infty}}
\newcommand{\EIN}[1]{e^{in#1}}
\newcommand{\NEIN}[1]{e^{-in#1}}
\newcommand{\INTALL}{\int_{-\infty}^{\infty}}
\newcommand{\FTINT}[1]{\INTALL #1 e^{2\pi inx\xi} dx}
\newcommand{\GAUSS}{e^{-\pi x^2}}

%%%%%%%
% formatting 
%%%%%%%

\newcommand{\LEFTBOLD}[1]{\noindent\textbf{#1}}
\newcommand{\SEQ}[1]{\{#1\,\}}
\newcommand{\WIP}{\footnote{work in progress}}
\newcommand{\QED}{\hfill\square}
\newcommand{\ts}{\textsuperscript}
\newcommand{\HLINE}{\noindent\rule{7in}{1pt}\\}

%%%%%%%
% Mathematical note taking (definitions, theorems, etc.)
%%%%%%%

\newcommand{\REM}{\noindent\textbf{\\Remark: }}
\newcommand{\DEF}{\noindent\textbf{\\Definition: }}
\newcommand{\THE}{\noindent\textbf{\\Theorem: }}
\newcommand{\COR}{\noindent\textbf{\\Corollary: }}
\newcommand{\LEM}{\noindent\textbf{\\Lemma: }}
\newcommand{\PROP}{\noindent\textbf{\\Proposition: }}
\newcommand{\PROOF}{\noindent\textbf{\\Proof: }}
\newcommand{\EXP}{\noindent\textbf{\\Example: }}
\newcommand{\TRICKS}{\noindent\textbf{\\Tricks: }}


%%%%%%%
% text highlighting
%%%%%%%

\newcommand{\B}[1]{\textbf{#1}}
\newcommand{\CAL}[1]{\mathcal{#1}}
\newcommand{\UL}[1]{\underline{#1}}

%%%%%%
% Linear Algebra
%%%%%%

\newcommand{\COLVECTOR}[1]{\PAREN{\begin{array}{c}
#1
\end{array} }}
\newcommand{\TWOXTWO}[4]{\PAREN{ \begin{array}{c c} #1&#2 \\ #3 & #4 \end{array} }}
\newcommand{\THREEXTHREE}[9]{\PAREN{ \begin{array}{c c c} #1&#2&#3 \\ #4 & #5 & #6 \\ #7 & #8 & #9 \end{array} }}
\newcommand{\NXN}{\PAREN{ \begin{array}{c c c c} 
			a_{11} & a_{12} & \cdots & a_{1n} \\
			a_{21} & a_{22} & \cdots & a_{2n} \\
			\vdots & \vdots & \ddots & a_{1n} \\
			a_{n1} & a_{n2} & \cdots & a_{nn} \\
		\end{array} }}
\newcommand{\SLR}{SL_2(\R)}
\newcommand{\GLR}{GL_2(\R)}
\DeclareMathOperator{\TR}{tr}
\DeclareMathOperator{\BIL}{Bil}
\DeclareMathOperator{\SPAN}{span}

%%%%%%%
%  White space
%%%%%%%

\newcommand{\BOXIT}[1]{\noindent\fbox{\parbox{\textwidth}{#1}}}


\newtheorem{theorem}{Theorem}[section]
\newtheorem{corollary}{Corollary}[theorem]
\newtheorem{lemma}[theorem]{Lemma}

\theoremstyle{definition}
\newtheorem{definition}[theorem]{Definition}
\newtheorem{prop}[theorem]{Proposition}

\theoremstyle{remark}
\newtheorem{remark}[theorem]{Remark}
\newtheorem{example}[theorem]{Example}
%\newtheorem*{proof}[theorem]{Proof}



\newcommand{\RED}[1]{\textcolor{red}{#1}}
\newcommand{\BLUE}[1]{\textcolor{blue}{#1}}
\newcommand{\GREEN}[1]{\textcolor{black!30!green}{#1}}
\newcommand{\ORANGE}[1]{\textcolor{orange}{#1}}
\newcommand{\F}{\textbf{F}}
\newcommand{\NLL}{\mathcal{N}}

\title{Advanced Linear Algebra}
\author{The Unforgetable Someone}
\date{Summer 2023}

\newcommand{\NORM}[1]{\,\left \Vert #1 \right \Vert}
%\newcommand{\HAT}[1]{}
\begin{document}

\begin{center}
	\Large{\CLASSNAME -- \SEMESTER} \\
\end{center}
\begin{center}
	\STUDENTNAME \\
	\ASSIGNMENT -- \DUEDATE\\
\end{center} 


p. 224 \#4, 7, 8, 9, 

\begin{enumerate}
	\setcounter{enumi}{3}
	\item Let $p$ be defined on a vector space $X$ and satisfy (1) and (2). Show that for any given $x_0 \in X$ there is a linear functional $\tilde{f}$ on $X$ such that $\tilde{f}(x_0)=p(x_0)$ and $|\tilde{f}(x)| \le p(x)$ for all $x \in X$.\\
	\\
	Let $f \in X'$ and $f(x_0)=p(x_0)$.  Clearly, $f$ is defined on the subspace spanned by $x_0$, that is, $f$ is linear and $f(\alpha x_0)=\alpha f(x_0)$.  The Hahn-Banach Theorem says that there exists an extension of $f$, namely, $\tilde{f} \in X'$ such that $|\tilde{f}(x)| \le p(x)$ for all $x \in X$.
	
	\setcounter{enumi}{6}
	\item Give another proof of Theorem 4.3-3 in the case of a Hilbert space.
	
	\textbf{Theorem 4.3-3a: (Bounded linear functionals, Hilbert).}  Let $X$ be a Hibert space and let $x_0\ne 0$ be any element in $X$.  Then there exists a bounded linear functional $\tilde{f}$ on $X$ such that 
	\begin{align*}
	\NORM{\tilde{f}} &= 1, &\tilde{f}(x_0)= \NORM{x_0}
	\end{align*}
	
	Proof: Let $x_0 \in X$, then $Z$ is the subspace spanned by $x_0$. Any Cauchy sequence in $Z$ will converge because that same sequence is in $X$ which is complete.  Hence. $Z$ is also complete.  We know that, for any $f_g \in Z'$ there exists a $g \in Z$ such that $f_g(x) = \ABRACKET{x,g}$ and $\NORM{f_g} = 1$.  By Hahn-Bannach, there exists an extension $\tilde{f} \in X'$ such that $\NORM{\tilde{f}} = 1$ and $|f(x_0)| = \NORM{\tilde{f}} \NORM{x_0} = \NORM{x}$.
	
	\item Let $X$ be a normed space and $X'$ its dual space. If $X \ne \{0\}$, show that $X'$ cannot be $\{0\}$.
	
	Let $f(x) = \NORM{x}$, this is linear by definition. Therefore, $f \in X'$.  We can see that when $x \ne 0$ that $f(x) \ne 0$. Therefore $f$ is not the zero function and $X' \ne \{ 0 \}$.
	
	\item Show that for a separable normed space $X$, theorem 4.3-2 can be proved directly, without the use of Zorn's Lemma (which was used indirectly, namely, in the proof of Theorem 4.2-1).
	
	We still need a function $p$ defined over all of $X$.  We can still use the $p$ defined in the proof, that is
	\begin{align*}
		p(x) = \NORM{f}_Z \NORM{x}
	\end{align*}and we know that it satisfies conditiosn (1) and (2) as well by 
	\begin{align*}
		p(x+y) &= \NORM{f}_Z \NORM{x+y} \le \NORM{f}_Z \PAREN{\NORM{x}+\NORM{y}} = p(x)+p(y)\\
		p(\alpha x) &= \NORM{f}_Z \NORM{\alpha x} = |\alpha|\NORM{f}_Z \NORM{x} = \alpha \NORM{x} \\
	\end{align*}What we need now is an $\tilde{f}$ which is a maximal function such that $\tilde{f} \le  p(x)$ for all $x \in X$.  Let $f_1$ be a linear extension of $f$ over $\mathcal{D}(f_1)$.  That is $f_1(x) = f(x)$ for all $x \in Z$.  We know that $f_1$ exists because at the very least $f_1 = f$ and $\mathcal{D}(f_1) = Z$.  When  $x_1 \in X \backslash Z$ we now have $Z \subset \mathcal{D}(f_1)$.  We can repeat this for $f_2$ giving us $x_2, \mathcal{D}(f_2)$, and so on.  For any number $n$ we have a set $(f_i)$ such that $f(x) \le f_i(x) \le p(x)$ for all $x \in \bigcup_{i=1}^n \mathcal{D}(f_i)$.  Since, $X$ is dense we know that there exists an $f_{n+1}$ and $x_{n+1}$, thus $\bigcup_{i=1}^\infty\mathcal{D}(f_i) = X$.  Let $f_i \to \tilde{f}$.  Thus $\tilde{f}(x) = f(x)$ when $x \in Z$ and $\tilde{f}(x) \le p(x)$ when $x \in X$.  Further, each $f_i$ is an extension of $f$ we know that $\NORM{f}_Z = \NORM{f_i}_{\mathcal{D}(f_i)}$ thus $\NORM{f}_Z = \NORM{\tilde{f}}_X.\QED$
	
\end{enumerate}
\newpage
p. 255 \#10, 11, 13, 14,

\begin{enumerate}
	\setcounter{enumi}{9}
	\item \textbf{(Space $c_0$)}  Let $y=(\eta_j), \eta_j \in \C$, be such that $\sum  \xi_j\eta_j$ converges for every $x=(\xi_j)\in c_0$, where $c_0\in l^\infty$ is the subspace of all complex sequences converge to zero. Show that $\sum |\eta_j|< \infty$.  (Use 4.7-3)
	
	Let $T: c_0 \to \C$ such that $T_n(x) = \SUM{i=1}{n} \xi_i \eta_i$.  $(T_n(x))$ is bounded, thus, $(T_n)$ is bounded.  By the Unified Boundedness Theorem there exists $c > 0$ such that $\NORM{T_n}<c$ for all $n \in \N$.  For each $n$ there is a sequence $x_n = (\mu_1, \mu_2, \dots, \mu_n, \dots)$ where $\mu_i$ are on the unit sphere, that is $\NORM{\mu_i}=1$ and $\mu_i\eta_i = \eta_i$.  Since $\NORM{T_n} \le c$ for all $n\in N$, we have $T_n(x_n) = \SUM{i=1}{n}\NORM{\mu_i\eta_i} = \SUM{i=1}{n}\NORM{\eta_i} < c$. Since this is true for all $n$,  $y$ is convergent.
	
	\item Let $X$ be a Banach space, $Y$ a normed space and $T_n\in B(X,Y)$ such that $(T_nx)$ is Cauchy in $Y$ for every $x \in X$.  Show that $(\NORM{T_n})$ is bounded.
	
	For any $j\in \N$ we know that $T_j$ is Cauchy in $Y$.  Therefore, given any convergent sequence $(x_k) \in X$ we know that $(T_j x_k)$ converges, there must be some number $y_j\in Y$ such $\NORM{T_j x_k} =  \NORM{T_j x_k} \NORM{x_k}\le\NORM{T_j} \NORM{x_k}$ for all $x_k$.  That is, there exists $c_j$ such that $\NORM{T_j} \le c_j$.  By Uniform Boundedness Thoeorem, there exists a $c$ such that $\NORM{T_n}\le c$ for all $n \in \N$, thus $(\NORM{T_n})$ is bounded.
	
	\setcounter{enumi}{12}
	\item  If $(x_n)$ in a Banach space $X$ is such that $(f(x_n))$ is bounded for all $ f\in X'$, show that $(\NORM{x_n})$ is bounded.

	Let $g: X \to X''$ be such that $g(x)f=f(x)$ and $f \in X'$. Then $|g(x)f| \le |f(x)| \le \NORM{f}\NORM{x}$ and hence bounded.  Further, $(g(x_n)f)$ is bounded because $(f(x_n))$ is bounded for all $f \in X'$.  $X$ is complete therefore, $(|g(x_n)f|)$ is bounded implies that $(\NORM{x_n})$ is bounded.
	
	\item if $X$ and $Y$ are Banach spaces and $T_n\in B(X,Y), n=1,2,\cdots,$ show that equivalent statements are:
	\begin{enumerate}
		\item $(\NORM{T_n})$ is bounded.
		\item $(\NORM{T_nx})$ is bounded for all $x \in X$.
		\item $(|g(T_nx)|)$ is bounded for all $x\in X$ and all $g \in Y'$.
	\end{enumerate}
	
	Let's show that $B(X,Y)$ is complete.  Let $T_n \in B(X,Y)$ be a Cauchy sequence.  For each $x \in X$, we have
	\begin{align*}
		\NORM{T_nx-T_mx} \le \NORM{T_n-T_m}\NORM{x},
	\end{align*}which shows that $(T_n x)$ is Cauchy in Y.  Since $Y$ is complete, there is a $y \in Y$ such that $T_nx \to y$.  We can see that $T: X\to Y$ where $T_nx =y$ forms a linear map.  For any $\epsilon >0$, let $N_\epsilon$ be such that $\NORM{T_n-T_m} < \epsilon/2$ for all $n,m \ge N_\epsilon$.  Whenever, $n \ge N_\epsilon$, for each $x \in X$, there is an $m_x \ge N_\epsilon$ such that $\NORM{T_{m_x}x-Tx} \le \epsilon/2$.  If $\NORM{x} = 1$ we have 
	\begin{align*}
		\NORM{T_nx-Tx} \le \NORM{T_nx-T_{m_x}x}+\NORM{T_{m_x}x-Tx} \le \epsilon
	\end{align*}If follows that as $n \ge N_\epsilon$, then 
	\begin{align*}
		\NORM{Tx}\le \NORM{T_nx} + \NORM{Tx-T_nx} \le \NORM{T_n} + \epsilon
	\end{align*}for all $x$ with $\NORM{x}=1$, so $T$ is bounded.  If follows that $\LIM{n\to \infty} \NORM{T_n-T} = 0$.  Therefore, $T_n \to T$ and $B(X,Y)$ is complete.\\
	\\
	(a) $Tn \to T$ thus $\NORM{T_n} \to \NORM{T}$ thus $(\NORM{T_n})$ is bounded.\\ \\
	(b) $\NORM{T_nx} \le \NORM{T_n}\NORM{x}$ for all $x \in X$ and $\NORM{T_n}$ is bounded thus $(\NORM{T_n x})$ is bounded.\\ \\
	(c) $Y$ is a Banach space $Y'$ is also a Banach space.  Thus, given any $g \in Y'$ and any convergent sequence $x_n \in X$ then $(|g(x_n)|)$ converges.  Let $x_n = T_nx_0$ for some fixed $x_0 \in X$ and we can see that $(|g(T_n x_0)|)$ converges.  Since $x_0$ and $g$ are arbitrary we see that $(|g(T_nx)|)$ converges for all $x \in X$ and $g \in Y'$.
\end{enumerate}
\end{document}