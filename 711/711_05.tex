	\documentclass[10pt,a4paper]{report}
\usepackage[utf8]{inputenc}
\usepackage{amsmath}
\usepackage{amsfonts}
\usepackage{amssymb}
\usepackage{amsthm}
\usepackage{hyperref}

\usepackage{multicol}
\usepackage{fancyhdr}
\usepackage{enumitem}
\usepackage{tikz}
\usepackage{tikz-cd}
\usetikzlibrary{calc}
\usetikzlibrary{shapes.geometric}
\usepackage[margin=0.5in]{geometry}
\usepackage{xcolor}
\DeclareMathOperator{\RANGE}{range}
\DeclareMathOperator{\NULL}{null}

\hypersetup{
    colorlinks=true,
    linkcolor=blue,
    filecolor=magenta,      
    urlcolor=cyan,
    pdftitle={Tensors},
    pdfpagemode=FullScreen,
    }

%\urlstyle{same}

\newcommand{\CLASSNAME}{Functional Analysis}
\newcommand{\STUDENTNAME}{Paul Carmody}
\newcommand{\ASSIGNMENT}{Assignment \#5}
\newcommand{\DUEDATE}{April 18, 2024}
\newcommand{\SEMESTER}{Spring 2024}
\newcommand{\SCHEDULE}{T/Th 9:30 -- 10:45}
\newcommand{\ROOM}{Remote}

\pagestyle{fancy}
\fancyhf{}
\chead{ \fancyplain{}{\CLASSNAME} }
%\chead{ \fancyplain{}{\STUDENTNAME} }
\rhead{\thepage}
\newcommand{\LET}{\text{Let }}
%\newcommand{\IF}{\text{if }}
\newcommand{\AND}{\text{ and }}
\newcommand{\OR}{\text{ or }}
\newcommand{\FORSOME}{\text{ for some }}
\newcommand{\FORALL}{\text{ for all }}
\newcommand{\WHERE}{\text{ where }}
\newcommand{\WTS}{\text{ WTS }}
\newcommand{\WLOG}{\text{ WLOG }}
\newcommand{\BS}{\backslash}
\newcommand{\DEFINE}[1]{\textbf{\emph{#1}}}
\newcommand{\IF}{$(\Rightarrow)$}
\newcommand{\ONLYIF}{$(\Leftarrow)$}
\newcommand{\ITH}{\textsuperscript{th} }
\newcommand{\FST}{\textsuperscript{st} }
\newcommand{\SND}{\textsuperscript{nd} }
\newcommand{\TRD}{\textsuperscript{rd} }
\newcommand{\INV}{\textsuperscript{-1} }

\newcommand{\XXX}{\mathfrak{X}}
\newcommand{\MMM}{\mathfrak{M}}
%\newcommand{\????}{\textfrak{A}}
%\newcommand{\????}{\textgoth{A}}
%\newcommand{\????}{\textswab{A}}

\DeclareMathOperator{\DER}{Der}
\DeclareMathOperator{\SGN}{sgn}

%%%%%%%
% derivatives
%%%%%%%

\newcommand{\PART}[2]{\frac{\partial #1}{\partial #2}}
\newcommand{\SPART}[2]{\frac{\partial^2 #1}{\partial #2^2}}
\newcommand{\DERIV}[2]{\frac{d #1}{d #2}}
\newcommand{\LAPLACIAN}[1]{\frac{\partial^2 #1}{\partial x^2} + \frac{\partial^2 #1}{\partial y^2}}

%%%%%%%
% sum, product, union, intersections
%%%%%%%

\newcommand{\SUM}[2]{\underset{#1}{\overset{#2}{\sum}}}
\newcommand{\PROD}[2]{\underset{#1}{\overset{#2}{\prod}}}
\newcommand{\UNION}[2]{\underset{#1}{\overset{#2}{\bigcup}}}
\newcommand{\INTERSECT}[2]{\underset{#1}{\overset{#2}{\bigcap}}}
\newcommand{\FSUM}{\SUM{n=-\infty}{\infty}}
       

%%%%%%%
% supremum and infimum
%%%%%%%

\newcommand{\SUP}[1]{\underset{#1}\sup \,}
\newcommand{\INF}[1]{\underset{#1}\inf \,}
\newcommand{\MAX}[1]{\underset{#1}\max \,}
\newcommand{\MIN}[1]{\underset{#1}\min \,}

%%%%%%%
% infinite sums, limits
%%%%%%%

\newcommand{\SUMK}{\SUM{k=1}{\infty}}
\newcommand{\SUMN}{\SUM{n=1}{\infty}}
\newcommand{\SUMKZ}{\SUM{k=0}{\infty}}
\newcommand{\LIM}[1]{\underset{#1}\lim\,}
\newcommand{\IWOB}[1]{\LIM{#1 \to \infty}}
\newcommand{\LIMK}{\IWOB{k}}
\newcommand{\LIMN}{\IWOB{n}}
\newcommand{\LIMX}{\IWOB{x}}
\newcommand{\NIWOB}{\LIM{n \to \infty}}
\newcommand{\LIMSUPK}{\underset{k\to\infty}\limsup \,}
\newcommand{\LIMSUPN}{\underset{n\to\infty}\limsup \,}
\newcommand{\LIMINFK}{\underset{k\to\infty}\liminf \,}
\newcommand{\LIMINFN}{\underset{n\to\infty}\liminf \,}
\newcommand{\ROOTRULE}[1]{\LIMSUPK \BARS{#1}^{1/k}}

\newcommand{\CUPK}{\bigcup_{k=1}^{\infty}}
\newcommand{\CAPK}{\bigcap_{k=1}^{\infty}}
\newcommand{\CUPN}{\bigcup_{n=1}^{\infty}}
\newcommand{\CAPN}{\bigcap_{n=1}^{\infty}}

%%%%%%%
% number systems (real, rational, etc.)
%%%%%%%

\newcommand{\REALS}{\mathbb{R}}
\newcommand{\RATIONALS}{\mathbb{Q}}
\newcommand{\IRRATIONALS}{\REALS \backslash \RATIONALS}
\newcommand{\INTEGERS}{\mathbb{Z}}
\newcommand{\NUMBERS}{\mathbb{N}}
\newcommand{\COMPLEX}{\mathbb{C}}
\newcommand{\DISC}{\mathbb{D}}
\newcommand{\HPLANE}{\mathbb{H}}

\newcommand{\R}{\mathbb{R}}
\newcommand{\Q}{\mathbb{Q}}
\newcommand{\Z}{\mathbb{Z}}
\newcommand{\N}{\mathbb{N}}
\newcommand{\C}{\mathbb{C}}
\newcommand{\T}{\mathbb{T}}
\newcommand{\COUNTABLE}{\aleph_0}
\newcommand{\UNCOUNTABLE}{\aleph_1}


%%%%%%%
% Arithmetic/Algebraic operators
%%%%%%%


\DeclareMathOperator{\MOD}{mod}
%\newcommand{\MOD}[1]{\mod #1}
\newcommand{\BAR}[1]{\overline{#1}}
\newcommand{\LCM}{\text{ lcm}}
\newcommand{\ZMOD}[1]{\Z/#1\Z}
\DeclareMathOperator{\VAR}{Var}
%%%%%%%
% complex operators
%%%%%%%

\DeclareMathOperator{\RR}{Re}
%\newcommand{\RE}{\text{Re}}
\DeclareMathOperator{\IM}{Im}
%\newcommand{\IM}{\text{Im}}
\newcommand{\CONJ}[1]{\overline{#1}}
\DeclareMathOperator{\LOG}{Log}
%\newcommand{\LOG}{\text{ Log }}
\newcommand{\RES}[2]{\underset{#1}{\text{res}} #2}

%%%%%%%
% Group operators
%%%%%%%

\newcommand{\AUT}{\text{Aut}\,}
\newcommand{\KER}{\text{ker}\,}
\newcommand{\END}{\text{End}}
\newcommand{\HOM}{\text{Hom}}
\newcommand{\CYCLE}[1]{(\begin{array}{cccccccccc}
		#1
	\end{array})}
\newcommand{\SUBGROUP}{\underset{\text{group}}\subseteq}	
%\newcommand{\SUBGROUP}{\subseteq_g}
\newcommand{\SUBRING}{\underset{\text{ring}}\subseteq}
\newcommand{\SUBMOD}{\underset{\text{mod}}\subseteq}
\newcommand{\SUBFIELD}{\underset{\text{field}}\subseteq}
\newcommand{\ISO}{\underset{\text{iso}}\longrightarrow}
\newcommand{\HOMO}{\underset{\text{homo}}\longrightarrow}

%%%%%%%
% grouping (parenthesis, absolute value, square, multi-level brackets).
%%%%%%%

\newcommand{\PAREN}[1]{\left (\, #1 \,\right )}
\newcommand{\BRACKET}[1]{\left \{\, #1 \,\right \}}
\newcommand{\SQBRACKET}[1]{\left [\, #1 \,\right ]}
\newcommand{\ABRACKET}[1]{\left \langle\, #1 \,\right \rangle}
\newcommand{\BARS}[1]{\left |\, #1 \,\right |}
\newcommand{\DBARS}[1]{\left \| \, #1 \,\right \|}
\newcommand{\LBRACKET}[1]{\left \{ #1 \right .} 
\newcommand{\RBRACKET}[1]{\left . #1 \right \]}
\newcommand{\RBAR}[1]{\left . #1 \, \right |}
\newcommand{\LBAR}[1]{\left | \, #1 \right .}
\newcommand{\BLBRACKET}[2]{\BRACKET{\RBAR{#1}#2}}
\newcommand{\GEN}[1]{\ABRACKET{#1}}
\newcommand{\BINDEF}[2]{\LBRACKET{\begin{array}{ll}
     #1\\
     #2
\end{array}}}

%%%%%%%
% Fourier Analysis
%%%%%%%

\newcommand{\ONEOTWOPI}{\frac{1}{2\pi}}
\newcommand{\FHAT}{\hat{f}(n)}
\newcommand{\FINT}{\int_{-\pi}^\pi}
\newcommand{\FINTWO}{\int_{0}^{2\pi}}
\newcommand{\FSUMN}[1]{\SUM{n=-#1}{#1}}
%\newcommand{\FSUM}{\SUMN{\infty}}
\newcommand{\EIN}[1]{e^{in#1}}
\newcommand{\NEIN}[1]{e^{-in#1}}
\newcommand{\INTALL}{\int_{-\infty}^{\infty}}
\newcommand{\FTINT}[1]{\INTALL #1 e^{2\pi inx\xi} dx}
\newcommand{\GAUSS}{e^{-\pi x^2}}

%%%%%%%
% formatting 
%%%%%%%

\newcommand{\LEFTBOLD}[1]{\noindent\textbf{#1}}
\newcommand{\SEQ}[1]{\{#1\,\}}
\newcommand{\WIP}{\footnote{work in progress}}
\newcommand{\QED}{\hfill\square}
\newcommand{\ts}{\textsuperscript}
\newcommand{\HLINE}{\noindent\rule{7in}{1pt}\\}

%%%%%%%
% Mathematical note taking (definitions, theorems, etc.)
%%%%%%%

\newcommand{\REM}{\noindent\textbf{\\Remark: }}
\newcommand{\DEF}{\noindent\textbf{\\Definition: }}
\newcommand{\THE}{\noindent\textbf{\\Theorem: }}
\newcommand{\COR}{\noindent\textbf{\\Corollary: }}
\newcommand{\LEM}{\noindent\textbf{\\Lemma: }}
\newcommand{\PROP}{\noindent\textbf{\\Proposition: }}
\newcommand{\PROOF}{\noindent\textbf{\\Proof: }}
\newcommand{\EXP}{\noindent\textbf{\\Example: }}
\newcommand{\TRICKS}{\noindent\textbf{\\Tricks: }}


%%%%%%%
% text highlighting
%%%%%%%

\newcommand{\B}[1]{\textbf{#1}}
\newcommand{\CAL}[1]{\mathcal{#1}}
\newcommand{\UL}[1]{\underline{#1}}

%%%%%%
% Linear Algebra
%%%%%%

\newcommand{\COLVECTOR}[1]{\PAREN{\begin{array}{c}
#1
\end{array} }}
\newcommand{\TWOXTWO}[4]{\PAREN{ \begin{array}{c c} #1&#2 \\ #3 & #4 \end{array} }}
\newcommand{\DTWOXTWO}[4]{\BARS{ \begin{array}{c c} #1&#2 \\ #3 & #4 \end{array} }}
\newcommand{\THREEXTHREE}[9]{\PAREN{ \begin{array}{c c c} #1&#2&#3 \\ #4 & #5 & #6 \\ #7 & #8 & #9 \end{array} }}
\newcommand{\DTHREEXTHREE}[9]{\BARS{ \begin{array}{c c c} #1&#2&#3 \\ #4 & #5 & #6 \\ #7 & #8 & #9 \end{array} }}
\newcommand{\NXN}{\PAREN{ \begin{array}{c c c c} 
			a_{11} & a_{12} & \cdots & a_{1n} \\
			a_{21} & a_{22} & \cdots & a_{2n} \\
			\vdots & \vdots & \ddots & a_{1n} \\
			a_{n1} & a_{n2} & \cdots & a_{nn} \\
		\end{array} }}
\newcommand{\SLR}{SL_2(\R)}
\newcommand{\GLR}{GL_2(\R)}
\DeclareMathOperator{\TR}{tr}
\DeclareMathOperator{\BIL}{Bil}
\DeclareMathOperator{\SPAN}{span}

%%%%%%%
%  White space
%%%%%%%

\newcommand{\BOXIT}[1]{\noindent\fbox{\parbox{\textwidth}{#1}}}


\newtheorem{theorem}{Theorem}[section]
\newtheorem{corollary}{Corollary}[theorem]
\newtheorem{lemma}[theorem]{Lemma}

\theoremstyle{definition}
\newtheorem{definition}[theorem]{Definition}
\newtheorem{prop}[theorem]{Proposition}

\theoremstyle{remark}
\newtheorem{remark}[theorem]{Remark}
\newtheorem{example}[theorem]{Example}
%\newtheorem*{proof}[theorem]{Proof}



\newcommand{\RED}[1]{\textcolor{red}{#1}}
\newcommand{\BLUE}[1]{\textcolor{blue}{#1}}
\newcommand{\GREEN}[1]{\textcolor{black!30!green}{#1}}
\newcommand{\ORANGE}[1]{\textcolor{orange}{#1}}
\newcommand{\F}{\textbf{F}}
\newcommand{\NLL}{\mathcal{N}}

\title{Advanced Linear Algebra}
\author{The Unforgetable Someone}
\date{Summer 2023}

\newcommand{\NORM}[1]{\,\left \Vert #1 \right \Vert}
\begin{document}

\begin{center}
	\Large{\CLASSNAME -- \SEMESTER} \\
\end{center}
\begin{center}
	\STUDENTNAME \\
	\ASSIGNMENT -- \DUEDATE\\
\end{center} 

p. 200 \#2,3,4,5,6,10. 
\begin{enumerate}
	\setcounter{enumi}{1}
	\item Let $H$ be a Hilbert space and $T: H \to H$ a bijective bounded linear operator whose inverse is bounded.  Show that $(T^*)^{-1}$ exists and 
	\begin{align*}
		(T^*)^{-1}=(T^{-1})^*
	\end{align*}
	
	\begin{align*}
		(TT^*)^{-1} &= (T^*)^{-1}T^{-1} \\
		(TT^*)^{-1}T &= (T^*)^{-1}\\
		\therefore (T^*)^{-1} &\text{ exists}\\
		\\
		\ABRACKET{Tx, y} &= \ABRACKET{x, T^*y}\\
		\ABRACKET{T^{-1}Tx, y} &= \ABRACKET{T^{-1}x, T^*y}\\
		\ABRACKET{x,y} &= \ABRACKET{T^{-1}x, T^*y}\\
		\ABRACKET{x, (T^*)^{-1}y} &= \ABRACKET{T^{-1}x, (T^*)^{-1}T^*y}\\
		&= \ABRACKET{T^{-1}x, y} \\
		&= \ABRACKET{x, (T^{-1})^*y} \\
		(T^*)^{-1}&=(T^{-1})^*
	\end{align*}
	
	\item If $(T_n)$ is a sequence of bounded linear operators on a Hilbert space and $T_n \to T$, show that $T_n^*\to T^*$.
	
	\begin{align*}
		\NORM{T_n -T}^2 &\ge \NORM{T_n}^2 - \NORM{T}^2 = \NORM{T_n^*}^2 - \NORM{T^*}^2 \ge \NORM{T_n^*-T^*}^2\\
		\text{similarly, }  \NORM{T_n^*-T^*}^2 &\ge  \NORM{T_n^*}^2 - \NORM{T^*}^2 = \NORM{T_n}^2 - \NORM{T}^2 \ge \NORM{T_n -T}^2 \\
		\text{hence } \NORM{T_n -T}^2 &= \NORM{T_n^*-T^*}^2
	\end{align*}	We know that given any $N>0$ then for all $n > N$ when $\NORM{T_n -T} < \epsilon$ implies that $\NORM{T_n^*-T^*}<\epsilon$.  Therefore, $T_n^* \to T^*$.
	
	\item Let $H_1$ and $H_2$ be Hilbert spaces and $T: H_1 \to H_2$ a bounded linear operator.  If $M_1\subset H_1$ and $M_2 \subset H_2$ are such that $T(M_1) \subset M_2$, show that $M_1^\perp \subset T^*(M_2^\perp)$.
	
	 Let $x \in M_1$ and $z \in M_2^\perp$ and $x \not \in \mathcal{N}(T)$.Then, $\ABRACKET{Tx, z}=0$ implies $\ABRACKET{x, T^*z}=0$ and either $T^*z \in \mathcal{N}(T^*)$ or $T^*z \perp x$.  $x$ is arbitrary, therefore $T^*z \perp \SPAN(M_1)$ or $T^*z \in M_1^\perp$.  Thus, $T^*z \in \mathcal{N}(T^*) \cup M_1^\perp$.  $z$ is arbitrary so $T^*(M_2^\perp) = \mathcal{N}(T^*) \cup M_1^\perp$, hence, $M_1^\perp \subset T^*(M_2^\perp)$.
	
	\item Let $M_1$ and $M_2$ in Prob. 4 be closed subspaces.   Show that $T(M_1)\subset M_2$ if and only if $M_1^\perp \supset T^*(M_2^\perp)$.
	
	$(\Rightarrow)$ Assuming that $T(M_1) \subset M_2$.  We can see that $H_1 = M_1 \oplus M_1^\perp$.  Then, let $x \in H_1, x= a+b$ for some $a \in M_1$ and $b \in M_1^\perp$ and $z \in M_2^\perp$ such that $z \ne 0$.  Then, 
	\begin{align*}
		\ABRACKET{Tx, z} &= \ABRACKET{Ta,z}+\ABRACKET{Tb,z} \\
		&= \ABRACKET{Tb,z} \\
		&= \ABRACKET{b, T^*z} \\
		z \ne 0 &\implies T^*z \in M_1^\perp
	\end{align*}$z$ is arbitrary, thus $T^*(M_2^\perp) \subset M_1^\perp$.
	
	$(\Leftarrow)$ Assuming that $M^\perp \supset T^*(M_2^\perp)$.  We can see that $H_2 = M_2 \oplus M_2^\perp$.  Then, let $z \in H_2, z= c+d$ for some $c\in M_2$ and $d\in M_2^\perp$ and $x \in M_1$ and $x\ne 0$.  Then,
	\begin{align*}
		\ABRACKET{x, T^*z} &= \ABRACKET{x, T^*c}+\ABRACKET{x, T^*d} \\
		&= \ABRACKET{x,T^*c} \\
		&= \ABRACKET{Tx, c} \\
		x \ne 0 & \implies Tx \in M_2
	\end{align*}$x$ is arbitrary, thus $T(M_1) \subset M_2$
	
	\item If $M_1=\mathcal{N}(T)=\{x\,|\,Tx=0\}$ in Prob. 4, show that 
	\begin{enumerate}
		\item $T^*(H_2) \subset M_1^\perp$
		
		$\mathcal{N}(T)$ is a closed vector space and by Prob 5 we can see that if $M_2=\{0\}$ then $M_2^\perp = H_2$.  Thus, $T^*(M_2^\perp)=T^*(H_2)\subset M_1^\perp$.
		
		\item $[T(H_1)]^\perp \subset \mathcal{N}(T^*)$
		
		Let $x \in H_1 \backslash \NLL(T)$ and $z \in \NLL(T^*)$ then $0=\ABRACKET{x, T^*z}=\ABRACKET{Tx, z}$ implies that $Tx \in \NLL(T^*)^\perp$.\\
		Since $H_1=\NLL(T)\oplus \NLL(T)^\perp$ given any $x \in \NLL(T)$ then $Tx = 0 \in \NLL(T^*)$ or $x \in \NLL(T)^\perp$ then $Tx\in \NLL(T^*)^\perp$, then $[T(H_1)]^\perp \subset \NLL(T^*)$.
		
		\item $M_1=[T^*(H_2)]^\perp$
		
		Let $z \in H_2\backslash \NLL(T^*)$ and $x \in \NLL(T)$ then $0 = \ABRACKET{Tx, z}= \ABRACKET{x, T^*z}$ implies that $T^*z \in \NLL(T)^\perp$.\\
		Since $H_2 = \NLL(T^*)\oplus \NLL(T^*)^\perp$ given any $z\in \NLL(T^*)$ then $T^*z = 0 \in \NLL(T)$ or $z \in \NLL(T^*)^\perp$ then $T^*z \in \NLL(T)^\perp$,\\
		then $[T^*(H_2)]^\perp = \NLL(T)$.
	\end{enumerate}
	\setcounter{enumi}{9}
	\item \textbf{(Right shift operator)} Let $(e_n)$ be a total orthonormal sequence in a separable Hilbert space $H$ and define the \textit{right shift operator} to be the linear operator $T: H \to H$ such that $Te_n=e_{n+1}$ for $n=1,2,\cdots$.  Explain the name.  Find the range, null space, norm and Hilbert-adjoint operator of $T$.\\
	
	The \textit{right shift operator} gets its name by shifting the element back one position.\\
	The range of $\mathcal{R}(T)$ is the set of total orthornormal sequences.\\
	The null space is the first element if $e_1 = 1, 0, 0, 0, \dots $ for clearly $Te_1 = (0)$.  \\
	The norm $\NORM{T} = \SUP{x \in H, \NORM{x}=1} Tx = 1$\\
	The adjoint, $T^*$, note that 
	\begin{align*}
		\ABRACKET{Te_i, e_j} &= \ABRACKET{e_{i+1}, e_j} = \BINDEF{ 1 & j = i+1 }{0 & \text{otherwise}} \\
		\text{ then }&\\
		\ABRACKET{e_i, T^*e_j} &= \ABRACKET{Te_i, e_j} \\
		& \implies T^*e_j = e_{j-1}
	\end{align*}

\end{enumerate}

\newpage
p. 207 \#4, 5

\begin{enumerate}
	\setcounter{enumi}{3}
	\item Show that for any bounded linear operator $T$ on $H$, the operators 
	\begin{align*}
		T_1=\frac{1}{2}(T+T^*) \text{  and  } T_2=\frac{1}{2i}(T-T^*)
	\end{align*}are self-adjoint.  Show that 
	\begin{align*}
		T=T_1+iT_2 \text{ and }  T^*=T_1-iT_2.
	\end{align*}Show uniqueness, that is, $T_1+iT_2=S_1+iS_2$ implies $S_1=T_1$ and $S_2=T_2$; here, $S_1$ and $S_2$ are self-adjoint by assumption.

\begin{multicols}{2}	
	\begin{align*}
		T_1&=\frac{1}{2}(T+T^*) \\
		\ABRACKET{T_1x, y} &= \ABRACKET{\PAREN{\frac{1}{2}(T+T^*)}x, y} \\
		&= \frac{1}{2}\ABRACKET{Tx+T^*x, y} \\
		&= \frac{1}{2}\PAREN{\ABRACKET{Tx, y} +\ABRACKET{T^*x, y} }\\
		&= \frac{1}{2}\PAREN{\ABRACKET{x, T^*y} + \ABRACKET{x, Ty}} \\
		&= \frac{1}{2}\ABRACKET{x, T^*y +  Ty} \\
		&= \ABRACKET{x, \frac{1}{2}\PAREN{T^*y +  Ty}} \\
		&= \ABRACKET{x, T_1y}
	\end{align*}
	
	\begin{align*}
		T_2&=\frac{1}{2i}(T-T^*) \\
		\ABRACKET{x, T_2y} &= \ABRACKET{x, \frac{1}{2i}\PAREN{T-T^*} y} \\
		&= \frac{-1}{2}\ABRACKET{x, Ty-T^*y} \\
		&= \frac{-1}{2}\PAREN{\ABRACKET{x, Ty} -\ABRACKET{x, T^*y} }\\
		&= \frac{-1}{2}\PAREN{\ABRACKET{T^*x, y} - \ABRACKET{Tx, y}} \\
		&= \frac{-1}{2}\ABRACKET{T^*x -  Tx, y} \\
		&= \ABRACKET{\frac{-1}{2}\PAREN{T^*x -  Tx}, y} \\
		&= \ABRACKET{T_2x, y}
	\end{align*}	
\end{multicols}
\begin{multicols}{2}
	\begin{align*}
		T_1+iT_2 &= \frac{1}{2}(T+T^*) + \frac{1}{2i}(T-T^*) \\
		&= \frac{1}{2}T+\frac{1}{2}T^* + \frac{1}{2}T-\frac{1}{2}T^* \\
		&= T
	\end{align*}
	\begin{align*}
\\		T_1-iT_2 &= \frac{1}{2}(T+T^*) - \frac{1}{2i}(T-T^*) \\
		&= \frac{1}{2}T+\frac{1}{2}T^* - \frac{1}{2}T+\frac{1}{2}T^* \\
		&= T^*
	\end{align*}
\end{multicols}

	\item On $\C^2$ (cf. 3.1-4) let the operator $T: \C^2\to \C^2$ be defined by $Tx=(\xi_1+i\xi_2, \xi_1-i\xi_2),$ where $x=(\xi_1,\xi_2)$.  Find $T^*$.  Show that we have $T^*T=TT^*=2I$.  Find $T_1$ and $T_2$ as defined in prob. 4.

	\begin{align*}
		T(a+bi) &= a+ib+i(a-ib) = a+b+i(a+b) = (a+b)(1+i) \\
		\ABRACKET{T(a+bi), c+di} &= \ABRACKET{(a+b)(1+i), c+di}= (a+b)\ABRACKET{1+i,c+di} = (a+b)(c+d)\\
		(a+b)(c+d) &= \ABRACKET{a+bi, 1+i}(c+d) = \ABRACKET{a+bi, (c+d)(1+i)} \\
		T^*(c+di) &= (c+d)(1+i) = (c+id)+i(c-id) \\
		T^* &= T \\
		\NORM{T1}^2 &= \ABRACKET{T1, T1} = \ABRACKET{1+i, 1+i} = 2 \\
		\NORM{T} &= \sqrt{2} = \NORM{T^*} \\
		\NORM{TT^*} &= 2 \\
		TT^* &= 2I \\
		T_1 &= \frac{1}{2}\PAREN{(1+i)+(1+i)} = 1+i = T \\
		T_2 &= \frac{1}{2i}\PAREN{(1+i)-(1+i)} = 0 \\
	\end{align*}
\end{enumerate}
\end{document}