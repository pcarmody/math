\documentclass[10pt,a4paper]{report}
\usepackage[utf8]{inputenc}
\usepackage{amsmath}
\usepackage{amsfonts}
\usepackage{amssymb}
\usepackage{amsthm}
\usepackage{hyperref}

\usepackage{multicol}
\usepackage{fancyhdr}
\usepackage[inline]{enumitem}
\usepackage{color,soul}
\usepackage{tikz}
\usepackage{tikz-cd}
\usetikzlibrary{calc}
\usetikzlibrary{shapes.geometric}
\usepackage[margin=0.5in]{geometry}
\usepackage{xcolor}

\hypersetup{
    colorlinks=true,
    linkcolor=blue,
    filecolor=magenta,      
    urlcolor=cyan,
    pdftitle={Tensors},
    pdfpagemode=FullScreen,
    }

%\urlstyle{same}

\newcommand{\CLASSNAME}{Math 5411 -- Mathematical Statistics I}
\newcommand{\PROFESSOR}{Nezamoddini-Kachouie}
\newcommand{\STUDENTNAME}{Paul Carmody}
\newcommand{\ASSIGNMENT}{Homework \#8 }
\newcommand{\DUEDATE}{October 28, 2024}
\newcommand{\SEMESTER}{Fall 2024}
\newcommand{\SCHEDULE}{MW 5:30 -- 6:45}
\newcommand{\ROOM}{Remote}

\pagestyle{fancy}
\fancyhf{}
\chead{ \fancyplain{}{\CLASSNAME} }
%\chead{ \fancyplain{}{\STUDENTNAME} }
\rhead{\thepage}
\newcommand{\LET}{\text{Let }}
%\newcommand{\IF}{\text{if }}
\newcommand{\AND}{\text{ and }}
\newcommand{\OR}{\text{ or }}
\newcommand{\FORSOME}{\text{ for some }}
\newcommand{\FORALL}{\text{ for all }}
\newcommand{\WHERE}{\text{ where }}
\newcommand{\WTS}{\text{ WTS }}
\newcommand{\WLOG}{\text{ WLOG }}
\newcommand{\BS}{\backslash}
\newcommand{\DEFINE}[1]{\textbf{\emph{#1}}}
\newcommand{\IF}{$(\Rightarrow)$}
\newcommand{\ONLYIF}{$(\Leftarrow)$}
\newcommand{\ITH}{\textsuperscript{th} }
\newcommand{\FST}{\textsuperscript{st} }
\newcommand{\SND}{\textsuperscript{nd} }
\newcommand{\TRD}{\textsuperscript{rd} }
\newcommand{\INV}{\textsuperscript{-1} }

\newcommand{\XXX}{\mathfrak{X}}
\newcommand{\MMM}{\mathfrak{M}}
%\newcommand{\????}{\textfrak{A}}
%\newcommand{\????}{\textgoth{A}}
%\newcommand{\????}{\textswab{A}}

\DeclareMathOperator{\DER}{Der}
\DeclareMathOperator{\SGN}{sgn}

%%%%%%%
% derivatives
%%%%%%%

\newcommand{\PART}[2]{\frac{\partial #1}{\partial #2}}
\newcommand{\SPART}[2]{\frac{\partial^2 #1}{\partial #2^2}}
\newcommand{\DERIV}[2]{\frac{d #1}{d #2}}
\newcommand{\LAPLACIAN}[1]{\frac{\partial^2 #1}{\partial x^2} + \frac{\partial^2 #1}{\partial y^2}}

%%%%%%%
% sum, product, union, intersections
%%%%%%%

\newcommand{\SUM}[2]{\underset{#1}{\overset{#2}{\sum}}}
\newcommand{\PROD}[2]{\underset{#1}{\overset{#2}{\prod}}}
\newcommand{\UNION}[2]{\underset{#1}{\overset{#2}{\bigcup}}}
\newcommand{\INTERSECT}[2]{\underset{#1}{\overset{#2}{\bigcap}}}
\newcommand{\FSUM}{\SUM{n=-\infty}{\infty}}
       

%%%%%%%
% supremum and infimum
%%%%%%%

\newcommand{\SUP}[1]{\underset{#1}\sup \,}
\newcommand{\INF}[1]{\underset{#1}\inf \,}
\newcommand{\MAX}[1]{\underset{#1}\max \,}
\newcommand{\MIN}[1]{\underset{#1}\min \,}

%%%%%%%
% infinite sums, limits
%%%%%%%

\newcommand{\SUMK}{\SUM{k=1}{\infty}}
\newcommand{\SUMN}{\SUM{n=1}{\infty}}
\newcommand{\SUMKZ}{\SUM{k=0}{\infty}}
\newcommand{\LIM}[1]{\underset{#1}\lim\,}
\newcommand{\IWOB}[1]{\LIM{#1 \to \infty}}
\newcommand{\LIMK}{\IWOB{k}}
\newcommand{\LIMN}{\IWOB{n}}
\newcommand{\LIMX}{\IWOB{x}}
\newcommand{\NIWOB}{\LIM{n \to \infty}}
\newcommand{\LIMSUPK}{\underset{k\to\infty}\limsup \,}
\newcommand{\LIMSUPN}{\underset{n\to\infty}\limsup \,}
\newcommand{\LIMINFK}{\underset{k\to\infty}\liminf \,}
\newcommand{\LIMINFN}{\underset{n\to\infty}\liminf \,}
\newcommand{\ROOTRULE}[1]{\LIMSUPK \BARS{#1}^{1/k}}

\newcommand{\CUPK}{\bigcup_{k=1}^{\infty}}
\newcommand{\CAPK}{\bigcap_{k=1}^{\infty}}
\newcommand{\CUPN}{\bigcup_{n=1}^{\infty}}
\newcommand{\CAPN}{\bigcap_{n=1}^{\infty}}

%%%%%%%
% number systems (real, rational, etc.)
%%%%%%%

\newcommand{\REALS}{\mathbb{R}}
\newcommand{\RATIONALS}{\mathbb{Q}}
\newcommand{\IRRATIONALS}{\REALS \backslash \RATIONALS}
\newcommand{\INTEGERS}{\mathbb{Z}}
\newcommand{\NUMBERS}{\mathbb{N}}
\newcommand{\COMPLEX}{\mathbb{C}}
\newcommand{\DISC}{\mathbb{D}}
\newcommand{\HPLANE}{\mathbb{H}}

\newcommand{\R}{\mathbb{R}}
\newcommand{\Q}{\mathbb{Q}}
\newcommand{\Z}{\mathbb{Z}}
\newcommand{\N}{\mathbb{N}}
\newcommand{\C}{\mathbb{C}}
\newcommand{\T}{\mathbb{T}}
\newcommand{\COUNTABLE}{\aleph_0}
\newcommand{\UNCOUNTABLE}{\aleph_1}


%%%%%%%
% Arithmetic/Algebraic operators
%%%%%%%


\DeclareMathOperator{\MOD}{mod}
%\newcommand{\MOD}[1]{\mod #1}
\newcommand{\BAR}[1]{\overline{#1}}
\newcommand{\LCM}{\text{ lcm}}
\newcommand{\ZMOD}[1]{\Z/#1\Z}
\DeclareMathOperator{\VAR}{Var}
%%%%%%%
% complex operators
%%%%%%%

\DeclareMathOperator{\RR}{Re}
%\newcommand{\RE}{\text{Re}}
\DeclareMathOperator{\IM}{Im}
%\newcommand{\IM}{\text{Im}}
\newcommand{\CONJ}[1]{\overline{#1}}
\DeclareMathOperator{\LOG}{Log}
%\newcommand{\LOG}{\text{ Log }}
\newcommand{\RES}[2]{\underset{#1}{\text{res}} #2}

%%%%%%%
% Group operators
%%%%%%%

\newcommand{\AUT}{\text{Aut}\,}
\newcommand{\KER}{\text{ker}\,}
\newcommand{\END}{\text{End}}
\newcommand{\HOM}{\text{Hom}}
\newcommand{\CYCLE}[1]{(\begin{array}{cccccccccc}
		#1
	\end{array})}
\newcommand{\SUBGROUP}{\underset{\text{group}}\subseteq}	
%\newcommand{\SUBGROUP}{\subseteq_g}
\newcommand{\SUBRING}{\underset{\text{ring}}\subseteq}
\newcommand{\SUBMOD}{\underset{\text{mod}}\subseteq}
\newcommand{\SUBFIELD}{\underset{\text{field}}\subseteq}
\newcommand{\ISO}{\underset{\text{iso}}\longrightarrow}
\newcommand{\HOMO}{\underset{\text{homo}}\longrightarrow}

%%%%%%%
% grouping (parenthesis, absolute value, square, multi-level brackets).
%%%%%%%

\newcommand{\PAREN}[1]{\left (\, #1 \,\right )}
\newcommand{\BRACKET}[1]{\left \{\, #1 \,\right \}}
\newcommand{\SQBRACKET}[1]{\left [\, #1 \,\right ]}
\newcommand{\ABRACKET}[1]{\left \langle\, #1 \,\right \rangle}
\newcommand{\BARS}[1]{\left |\, #1 \,\right |}
\newcommand{\DBARS}[1]{\left \| \, #1 \,\right \|}
\newcommand{\LBRACKET}[1]{\left \{ #1 \right .} 
\newcommand{\RBRACKET}[1]{\left . #1 \right \]}
\newcommand{\RBAR}[1]{\left . #1 \, \right |}
\newcommand{\LBAR}[1]{\left | \, #1 \right .}
\newcommand{\BLBRACKET}[2]{\BRACKET{\RBAR{#1}#2}}
\newcommand{\GEN}[1]{\ABRACKET{#1}}
\newcommand{\BINDEF}[2]{\LBRACKET{\begin{array}{ll}
     #1\\
     #2
\end{array}}}

%%%%%%%
% Fourier Analysis
%%%%%%%

\newcommand{\ONEOTWOPI}{\frac{1}{2\pi}}
\newcommand{\FHAT}{\hat{f}(n)}
\newcommand{\FINT}{\int_{-\pi}^\pi}
\newcommand{\FINTWO}{\int_{0}^{2\pi}}
\newcommand{\FSUMN}[1]{\SUM{n=-#1}{#1}}
%\newcommand{\FSUM}{\SUMN{\infty}}
\newcommand{\EIN}[1]{e^{in#1}}
\newcommand{\NEIN}[1]{e^{-in#1}}
\newcommand{\INTALL}{\int_{-\infty}^{\infty}}
\newcommand{\FTINT}[1]{\INTALL #1 e^{2\pi inx\xi} dx}
\newcommand{\GAUSS}{e^{-\pi x^2}}

%%%%%%%
% formatting 
%%%%%%%

\newcommand{\LEFTBOLD}[1]{\noindent\textbf{#1}}
\newcommand{\SEQ}[1]{\{#1\,\}}
\newcommand{\WIP}{\footnote{work in progress}}
\newcommand{\QED}{\hfill\square}
\newcommand{\ts}{\textsuperscript}
\newcommand{\HLINE}{\noindent\rule{7in}{1pt}\\}

%%%%%%%
% Mathematical note taking (definitions, theorems, etc.)
%%%%%%%

\newcommand{\REM}{\noindent\textbf{\\Remark: }}
\newcommand{\DEF}{\noindent\textbf{\\Definition: }}
\newcommand{\THE}{\noindent\textbf{\\Theorem: }}
\newcommand{\COR}{\noindent\textbf{\\Corollary: }}
\newcommand{\LEM}{\noindent\textbf{\\Lemma: }}
\newcommand{\PROP}{\noindent\textbf{\\Proposition: }}
\newcommand{\PROOF}{\noindent\textbf{\\Proof: }}
\newcommand{\EXP}{\noindent\textbf{\\Example: }}
\newcommand{\TRICKS}{\noindent\textbf{\\Tricks: }}


%%%%%%%
% text highlighting
%%%%%%%

\newcommand{\B}[1]{\textbf{#1}}
\newcommand{\CAL}[1]{\mathcal{#1}}
\newcommand{\UL}[1]{\underline{#1}}

%%%%%%
% Linear Algebra
%%%%%%

\newcommand{\COLVECTOR}[1]{\PAREN{\begin{array}{c}
#1
\end{array} }}
\newcommand{\TWOXTWO}[4]{\PAREN{ \begin{array}{c c} #1&#2 \\ #3 & #4 \end{array} }}
\newcommand{\DTWOXTWO}[4]{\BARS{ \begin{array}{c c} #1&#2 \\ #3 & #4 \end{array} }}
\newcommand{\THREEXTHREE}[9]{\PAREN{ \begin{array}{c c c} #1&#2&#3 \\ #4 & #5 & #6 \\ #7 & #8 & #9 \end{array} }}
\newcommand{\DTHREEXTHREE}[9]{\BARS{ \begin{array}{c c c} #1&#2&#3 \\ #4 & #5 & #6 \\ #7 & #8 & #9 \end{array} }}
\newcommand{\NXN}{\PAREN{ \begin{array}{c c c c} 
			a_{11} & a_{12} & \cdots & a_{1n} \\
			a_{21} & a_{22} & \cdots & a_{2n} \\
			\vdots & \vdots & \ddots & a_{1n} \\
			a_{n1} & a_{n2} & \cdots & a_{nn} \\
		\end{array} }}
\newcommand{\SLR}{SL_2(\R)}
\newcommand{\GLR}{GL_2(\R)}
\DeclareMathOperator{\TR}{tr}
\DeclareMathOperator{\BIL}{Bil}
\DeclareMathOperator{\SPAN}{span}

%%%%%%%
%  White space
%%%%%%%

\newcommand{\BOXIT}[1]{\noindent\fbox{\parbox{\textwidth}{#1}}}


\newtheorem{theorem}{Theorem}[section]
\newtheorem{corollary}{Corollary}[theorem]
\newtheorem{lemma}[theorem]{Lemma}

\theoremstyle{definition}
\newtheorem{definition}[theorem]{Definition}
\newtheorem{prop}[theorem]{Proposition}

\theoremstyle{remark}
\newtheorem{remark}[theorem]{Remark}
\newtheorem{example}[theorem]{Example}
%\newtheorem*{proof}[theorem]{Proof}



\newcommand{\RED}[1]{\textcolor{red}{#1}}
\newcommand{\BLUE}[1]{\textcolor{blue}{#1}}
\newcommand{\HIGHLIGHT}[1]{\fcolorbox{black}{yellow}{#1}}

\begin{document}

\begin{center}
	\Large{\CLASSNAME -- \SEMESTER} \\
	\large{w/\PROFESSOR}
\end{center}
\begin{center}
	\STUDENTNAME \\
	\ASSIGNMENT -- \DUEDATE\\
\end{center}

1, 2, 3, 4(a,b,d,e)
\begin{enumerate}
	\item Consider two random variables $X$ and $Y$ with joint PMF given in Table 5.5.3.
	\begin{align*}
		\begin{array}{|c|c|c|c|}
			\hline     & Y=2 & Y=4 & Y=5 \\
			\hline X=1 & \frac{1}{12} & \frac{1}{24} & \frac{1}{24} \\
			\hline X=2 & \frac{1}{6} & \frac{1}{12} & \frac{1}{8} \\
			\hline X=3 & \frac{1}{4} & \frac{1}{8} & \frac{1}{12} \\
			\hline
		\end{array}
	\end{align*}
	\begin{enumerate}
	\item Find $P(X \le 2, Y \le 4)$.
	
	\BLUE{\begin{align*}
		P(X \le 2, Y \le 4) &= P(X =1, Y \le 4) + P(X=2, Y \le 4)\\
		&= P(X =1, Y =2) + P(X=2, Y = 2) + P(X =1, Y = 4) + P(X=2, Y = 4) \\
		&= \frac{1}{12}+\frac{1}{6}+\frac{1}{24}+\frac{1}{12} \\
		&= \frac{2}{24}+\frac{4}{24}+\frac{1}{24}+\frac{1}{24} \\
		&= \frac{8}{24} = \frac{1}{3}
	\end{align*}
	} 
	
	\item Find the marginal PMFs of $X$ and $Y$.
	
	\BLUE{\begin{align*}
		P(X=1) &= \sum_{y= 2,4,5}P(X=1,Y=y) \\
			&= P(X=1,Y=2) + P(X=1,Y=4) + P(X=1,Y=5) \\
			&= \frac{1}{12} + \frac{1}{24} + \frac{1}{24} \\
			&= \frac{4}{24} = \frac{1}{6}\\
		P(X=2) &= \sum_{y= 2,4,5}P(X=2,Y=y) \\
			&= P(X=2,Y=2) + P(X=2,Y=4) + P(X=2,Y=5) \\
			&= \frac{1}{6} + \frac{1}{12} + \frac{1}{8} \\
			&= \frac{9}{24} = \frac{3}{8}\\
		P(X=3) &= \sum_{y= 2,4,5}P(X=3,Y=y)\\
			&= P(X=3,Y=2) + P(X=3,Y=4) + P(X=3,Y=5) \\
			&= \frac{1}{4} + \frac{1}{8} + \frac{1}{12} \\
			&= \frac{11}{24}
	\end{align*}
	\begin{align*}
		P(Y=2) &= \sum_{x=1,2,3}P(X=x, Y=2) \\
			&= P(X=1, Y=2) + P(X=2, Y=2) + P(X=3, Y=2) \\
			&= \frac{1}{12} + \frac{1}{6} + \frac{1}{4}\\
			&= \frac{6}{12} = \frac{1}{2}\\
		P(Y=4) &= \sum_{x=1,2,3}P(X=x, Y=4) \\
			&= P(X=1, Y=4) + P(X=2, Y=4) + P(X=3, Y=4) \\
			&= \frac{1}{24} + \frac{1}{12} + \frac{1}{8}\\
			&= \frac{6}{24} = \frac{1}{4}\\
		P(Y=4) &= \sum_{x=1,2,3}P(X=x, Y=5) \\
			&= P(X=1, Y=5) + P(X=2, Y=5) + P(X=3, Y=5) \\
			&= \frac{1}{24} + \frac{1}{8} + \frac{1}{12}\\
			&= \frac{6}{24} = \frac{1}{4}
	\end{align*}
	}
	\item Find $P(Y=2|X=1)$.
	
	\BLUE{\begin{align*}
		P(Y=2|X=1) &= P(X=1, Y=2) = \frac{1}{12}
	\end{align*}
	}
	
	\item Are $X$ and $Y$ independent?	\BLUE{ Yes }
	\end{enumerate}
	
	\item I have a bag containing 40 blue marbles and 60 red marbles.  I choose 10 marbles (without replacement) at random.  Let $X$ be the number of blue marbles and $Y$ be the number of read marbles.  Find the joint PMF of $X$ and $Y$.
	
	\BLUE{This is a hypergeometric distribution in one variable.  Let $b$ be the number of blue marbles. then
	\begin{align*}
		P(X=b, Y=10-b) &= \frac{\binom{40}{b}\binom{60}{10-b}}{\binom{100}{10}} \text{ for } b=1\dots 10
	\end{align*}
	}
	
	\item Let $X$ and $Y$ be two independent discrete random variables with the same CDFs $F_X$ and $F_Y$.  Define 
	\begin{align*}
		Z = \max(X,Y)\\
		W = \min(X,Y)
	\end{align*}Find CDFs of $Z$ and $W$.
	
	\BLUE{When assessing the maximum we can see that when $Y > X$ the probability of $X$ is included. Thus,
	\begin{align*}
		F(Z=n) &= P(\max(X,Y)\le n) \\
			&= P(X \le n, Y \le n) \\
			&= P(X \le n)P(Y \le n) \\
			&= F_X(n)F_Y(n)
	\end{align*}When assessuing the minimum we can see that when $Y < X$ there is still a greater probability that $X$ could occur, thus we must assess this based on $1-\min(X,Y)$ and switch the direction of integration which would include the greater value probability.
	\begin{align*}
		F(W = n) &= P(\min(X,Y)\le n) \\
			&= 1-P(\min(X,Y)\ge n) \\
			&= 1-P(X\ge n)P(Y\ge n) \\
			&= 1-(1-P(X\le n))(1-P(Y\le n) \\
			&= 1-(1-F_X(n))(1-F_Y(n)) \\
			&= 1-(1-F_X(n)-F_Y(n)+F_X(n)F_Y(n))\\
			&= F_X(n)+F_Y(n)-F_X(n)F_Y(n)
	\end{align*}
	}
	
	\item Let $X$ and $Y$ be two discrete random variables, with range
	\begin{align*}
		R_{XY} &= \BRACKET{(i,j) \in \Z^2\,|\,i,j \ge 0,\, |i-j|\le 1}
	\end{align*} and joint PMF
	\begin{align*}
		P_{XY}(i,j) &= \frac{1}{6\cdot 2^{\min(i,j)}},\, \text{ for }(i,j) \in \R_{XY}
	\end{align*}
	\begin{enumerate}
		\item Pictorially show $R_{XY}$ in the $x-y$ plane.
		
		\BLUE{We have $i-j\le 1$ and $j-i \le 1$.\\
		\begin{tikzpicture}
			\draw (0,0) -> (0,4);
			\draw (0,0) -> (4,0);
			\filldraw[black] (0,1) circle (2pt);
			\filldraw[black] (1,2) circle (2pt);
			\filldraw[black] (2,3) circle (2pt);
			\filldraw[black] (0,0) circle (2pt);
			\filldraw[black] (1,1) circle (2pt);
			\filldraw[black] (2,2) circle (2pt);
			\filldraw[black] (3,3) circle (2pt);
			\filldraw[black] (1,0) circle (2pt);
			\filldraw[black] (2,1) circle (2pt);
			\filldraw[black] (3,2) circle (2pt);
			\draw[dotted] (0,1) -- (3,4);
			\draw[dotted] (0,0) -- (4,4);
			\draw[dotted] (1,0) -- (4,3);
		\end{tikzpicture}
		}
		\item Find the marginal PMFs $P_X(i), P_Y(j)$.	
		
		\BLUE{\begin{align*}
				P_X(X=0) &= P_{XY}(X=0, Y=0) + P_{XY}(X=0, Y=1) = \frac{1}{6} + \frac{1}{12} = \frac{1}{4} \\
				P_X(X=1) &= P_{XY}(X=1, Y=0) + P_{XY}(X=1, Y=1) + P_{XY}(X=1, Y=2) = \frac{1}{6} + \frac{1}{12} + \frac{1}{12} = \frac{1}{3} \\
				P_X(X=2) &= P_{XY}(X=2, Y=1) + P_{XY}(X=2, Y=2) + P_{XY}(X=2, Y=3) = \frac{1}{12} + \frac{1}{24} + \frac{1}{24} = \frac{1}{6} \\
			\end{align*}	each $P(X)$ had three terms and in general is defined for $n\ge 0$ as
			\begin{align*}
				P_X(n) &= \BINDEF{\frac{1}{3} & n=0}{\frac{1}{3\cdot 2^{n-1}} & n\ge 1}
			\end{align*}$P(Y)$ is the same.
			}
			
		\item Find $P(X=Y\,|\,X<2)$.
		
		\BLUE{\begin{align*}
			P(X=Y\,|\,X<2) &= P(X=Y\,|\,Y=1)+P(X=Y\,|\,Y=0) \\
				&= P(X=1\,|\,Y=1)+P(X=0\,|\,Y=0) \\
				&= \frac{1}{12} + \frac{1}{6} = \frac{1}{4}
		\end{align*}
		}
		
		\item skip
		\item $P(X=Y)$.
		
		\BLUE{\begin{align*}
			P(X=Y) &= \sum_{x=0}^\infty\frac{1}{6\cdot 2^{\min(x,x)}} =  \sum_{x=0}^\infty\frac{1}{6\cdot 2^x} = \frac{1}{6}\sum_{x=0}^\infty\frac{1}{2^x}=\frac{2}{6}=\frac{1}{3}
		\end{align*}
		}
	\end{enumerate}
\end{enumerate}
\end{document}