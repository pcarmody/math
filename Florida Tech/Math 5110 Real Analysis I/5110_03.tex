\documentclass[10pt,a4paper]{report}
\usepackage[utf8]{inputenc}
\usepackage{amsmath}
\usepackage{amsfonts}
\usepackage{amssymb}
\usepackage{amsthm}
\usepackage{hyperref}

\usepackage{multicol}
\usepackage{fancyhdr}
\usepackage[inline]{enumitem}
\usepackage{tikz}
\usepackage{tikz-cd}
\usetikzlibrary{calc}
\usetikzlibrary{shapes.geometric}
\usepackage[margin=0.5in]{geometry}
\usepackage{xcolor}

\hypersetup{
    colorlinks=true,
    linkcolor=blue,
    filecolor=magenta,      
    urlcolor=cyan,
    pdftitle={Tensors},
    pdfpagemode=FullScreen,
    }

%\urlstyle{same}

\newcommand{\CLASSNAME}{Math 5110 -- Real Analysis I}
\newcommand{\STUDENTNAME}{Paul Carmody}
\newcommand{\ASSIGNMENT}{Homework \#3 }
\newcommand{\DUEDATE}{TBD: October 31, 2024}
\newcommand{\SEMESTER}{Fall 2024}
\newcommand{\SCHEDULE}{MW 11:00 -- 12:15}
\newcommand{\ROOM}{Remote}

\newcommand{\MMN}{M_{m\times n}}
\newcommand{\FF}{\mathcal{F}}

\pagestyle{fancy}
\fancyhf{}
\chead{ \fancyplain{}{\CLASSNAME} }
%\chead{ \fancyplain{}{\STUDENTNAME} }
\rhead{\thepage}
\newcommand{\LET}{\text{Let }}
%\newcommand{\IF}{\text{if }}
\newcommand{\AND}{\text{ and }}
\newcommand{\OR}{\text{ or }}
\newcommand{\FORSOME}{\text{ for some }}
\newcommand{\FORALL}{\text{ for all }}
\newcommand{\WHERE}{\text{ where }}
\newcommand{\WTS}{\text{ WTS }}
\newcommand{\WLOG}{\text{ WLOG }}
\newcommand{\BS}{\backslash}
\newcommand{\DEFINE}[1]{\textbf{\emph{#1}}}
\newcommand{\IF}{$(\Rightarrow)$}
\newcommand{\ONLYIF}{$(\Leftarrow)$}
\newcommand{\ITH}{\textsuperscript{th} }
\newcommand{\FST}{\textsuperscript{st} }
\newcommand{\SND}{\textsuperscript{nd} }
\newcommand{\TRD}{\textsuperscript{rd} }
\newcommand{\INV}{\textsuperscript{-1} }

\newcommand{\XXX}{\mathfrak{X}}
\newcommand{\MMM}{\mathfrak{M}}
%\newcommand{\????}{\textfrak{A}}
%\newcommand{\????}{\textgoth{A}}
%\newcommand{\????}{\textswab{A}}

\DeclareMathOperator{\DER}{Der}
\DeclareMathOperator{\SGN}{sgn}

%%%%%%%
% derivatives
%%%%%%%

\newcommand{\PART}[2]{\frac{\partial #1}{\partial #2}}
\newcommand{\SPART}[2]{\frac{\partial^2 #1}{\partial #2^2}}
\newcommand{\DERIV}[2]{\frac{d #1}{d #2}}
\newcommand{\LAPLACIAN}[1]{\frac{\partial^2 #1}{\partial x^2} + \frac{\partial^2 #1}{\partial y^2}}

%%%%%%%
% sum, product, union, intersections
%%%%%%%

\newcommand{\SUM}[2]{\underset{#1}{\overset{#2}{\sum}}}
\newcommand{\PROD}[2]{\underset{#1}{\overset{#2}{\prod}}}
\newcommand{\UNION}[2]{\underset{#1}{\overset{#2}{\bigcup}}}
\newcommand{\INTERSECT}[2]{\underset{#1}{\overset{#2}{\bigcap}}}
\newcommand{\FSUM}{\SUM{n=-\infty}{\infty}}
       

%%%%%%%
% supremum and infimum
%%%%%%%

\newcommand{\SUP}[1]{\underset{#1}\sup \,}
\newcommand{\INF}[1]{\underset{#1}\inf \,}
\newcommand{\MAX}[1]{\underset{#1}\max \,}
\newcommand{\MIN}[1]{\underset{#1}\min \,}

%%%%%%%
% infinite sums, limits
%%%%%%%

\newcommand{\SUMK}{\SUM{k=1}{\infty}}
\newcommand{\SUMN}{\SUM{n=1}{\infty}}
\newcommand{\SUMKZ}{\SUM{k=0}{\infty}}
\newcommand{\LIM}[1]{\underset{#1}\lim\,}
\newcommand{\IWOB}[1]{\LIM{#1 \to \infty}}
\newcommand{\LIMK}{\IWOB{k}}
\newcommand{\LIMN}{\IWOB{n}}
\newcommand{\LIMX}{\IWOB{x}}
\newcommand{\NIWOB}{\LIM{n \to \infty}}
\newcommand{\LIMSUPK}{\underset{k\to\infty}\limsup \,}
\newcommand{\LIMSUPN}{\underset{n\to\infty}\limsup \,}
\newcommand{\LIMINFK}{\underset{k\to\infty}\liminf \,}
\newcommand{\LIMINFN}{\underset{n\to\infty}\liminf \,}
\newcommand{\ROOTRULE}[1]{\LIMSUPK \BARS{#1}^{1/k}}

\newcommand{\CUPK}{\bigcup_{k=1}^{\infty}}
\newcommand{\CAPK}{\bigcap_{k=1}^{\infty}}
\newcommand{\CUPN}{\bigcup_{n=1}^{\infty}}
\newcommand{\CAPN}{\bigcap_{n=1}^{\infty}}

%%%%%%%
% number systems (real, rational, etc.)
%%%%%%%

\newcommand{\REALS}{\mathbb{R}}
\newcommand{\RATIONALS}{\mathbb{Q}}
\newcommand{\IRRATIONALS}{\REALS \backslash \RATIONALS}
\newcommand{\INTEGERS}{\mathbb{Z}}
\newcommand{\NUMBERS}{\mathbb{N}}
\newcommand{\COMPLEX}{\mathbb{C}}
\newcommand{\DISC}{\mathbb{D}}
\newcommand{\HPLANE}{\mathbb{H}}

\newcommand{\R}{\mathbb{R}}
\newcommand{\Q}{\mathbb{Q}}
\newcommand{\Z}{\mathbb{Z}}
\newcommand{\N}{\mathbb{N}}
\newcommand{\C}{\mathbb{C}}
\newcommand{\T}{\mathbb{T}}
\newcommand{\COUNTABLE}{\aleph_0}
\newcommand{\UNCOUNTABLE}{\aleph_1}


%%%%%%%
% Arithmetic/Algebraic operators
%%%%%%%


\DeclareMathOperator{\MOD}{mod}
%\newcommand{\MOD}[1]{\mod #1}
\newcommand{\BAR}[1]{\overline{#1}}
\newcommand{\LCM}{\text{ lcm}}
\newcommand{\ZMOD}[1]{\Z/#1\Z}
\DeclareMathOperator{\VAR}{Var}
%%%%%%%
% complex operators
%%%%%%%

\DeclareMathOperator{\RR}{Re}
%\newcommand{\RE}{\text{Re}}
\DeclareMathOperator{\IM}{Im}
%\newcommand{\IM}{\text{Im}}
\newcommand{\CONJ}[1]{\overline{#1}}
\DeclareMathOperator{\LOG}{Log}
%\newcommand{\LOG}{\text{ Log }}
\newcommand{\RES}[2]{\underset{#1}{\text{res}} #2}

%%%%%%%
% Group operators
%%%%%%%

\newcommand{\AUT}{\text{Aut}\,}
\newcommand{\KER}{\text{ker}\,}
\newcommand{\END}{\text{End}}
\newcommand{\HOM}{\text{Hom}}
\newcommand{\CYCLE}[1]{(\begin{array}{cccccccccc}
		#1
	\end{array})}
\newcommand{\SUBGROUP}{\underset{\text{group}}\subseteq}	
%\newcommand{\SUBGROUP}{\subseteq_g}
\newcommand{\SUBRING}{\underset{\text{ring}}\subseteq}
\newcommand{\SUBMOD}{\underset{\text{mod}}\subseteq}
\newcommand{\SUBFIELD}{\underset{\text{field}}\subseteq}
\newcommand{\ISO}{\underset{\text{iso}}\longrightarrow}
\newcommand{\HOMO}{\underset{\text{homo}}\longrightarrow}

%%%%%%%
% grouping (parenthesis, absolute value, square, multi-level brackets).
%%%%%%%

\newcommand{\PAREN}[1]{\left (\, #1 \,\right )}
\newcommand{\BRACKET}[1]{\left \{\, #1 \,\right \}}
\newcommand{\SQBRACKET}[1]{\left [\, #1 \,\right ]}
\newcommand{\ABRACKET}[1]{\left \langle\, #1 \,\right \rangle}
\newcommand{\BARS}[1]{\left |\, #1 \,\right |}
\newcommand{\DBARS}[1]{\left \| \, #1 \,\right \|}
\newcommand{\LBRACKET}[1]{\left \{ #1 \right .} 
\newcommand{\RBRACKET}[1]{\left . #1 \right \]}
\newcommand{\RBAR}[1]{\left . #1 \, \right |}
\newcommand{\LBAR}[1]{\left | \, #1 \right .}
\newcommand{\BLBRACKET}[2]{\BRACKET{\RBAR{#1}#2}}
\newcommand{\GEN}[1]{\ABRACKET{#1}}
\newcommand{\BINDEF}[2]{\LBRACKET{\begin{array}{ll}
     #1\\
     #2
\end{array}}}

%%%%%%%
% Fourier Analysis
%%%%%%%

\newcommand{\ONEOTWOPI}{\frac{1}{2\pi}}
\newcommand{\FHAT}{\hat{f}(n)}
\newcommand{\FINT}{\int_{-\pi}^\pi}
\newcommand{\FINTWO}{\int_{0}^{2\pi}}
\newcommand{\FSUMN}[1]{\SUM{n=-#1}{#1}}
%\newcommand{\FSUM}{\SUMN{\infty}}
\newcommand{\EIN}[1]{e^{in#1}}
\newcommand{\NEIN}[1]{e^{-in#1}}
\newcommand{\INTALL}{\int_{-\infty}^{\infty}}
\newcommand{\FTINT}[1]{\INTALL #1 e^{2\pi inx\xi} dx}
\newcommand{\GAUSS}{e^{-\pi x^2}}

%%%%%%%
% formatting 
%%%%%%%

\newcommand{\LEFTBOLD}[1]{\noindent\textbf{#1}}
\newcommand{\SEQ}[1]{\{#1\,\}}
\newcommand{\WIP}{\footnote{work in progress}}
\newcommand{\QED}{\hfill\square}
\newcommand{\ts}{\textsuperscript}
\newcommand{\HLINE}{\noindent\rule{7in}{1pt}\\}

%%%%%%%
% Mathematical note taking (definitions, theorems, etc.)
%%%%%%%

\newcommand{\REM}{\noindent\textbf{\\Remark: }}
\newcommand{\DEF}{\noindent\textbf{\\Definition: }}
\newcommand{\THE}{\noindent\textbf{\\Theorem: }}
\newcommand{\COR}{\noindent\textbf{\\Corollary: }}
\newcommand{\LEM}{\noindent\textbf{\\Lemma: }}
\newcommand{\PROP}{\noindent\textbf{\\Proposition: }}
\newcommand{\PROOF}{\noindent\textbf{\\Proof: }}
\newcommand{\EXP}{\noindent\textbf{\\Example: }}
\newcommand{\TRICKS}{\noindent\textbf{\\Tricks: }}


%%%%%%%
% text highlighting
%%%%%%%

\newcommand{\B}[1]{\textbf{#1}}
\newcommand{\CAL}[1]{\mathcal{#1}}
\newcommand{\UL}[1]{\underline{#1}}

%%%%%%
% Linear Algebra
%%%%%%

\newcommand{\COLVECTOR}[1]{\PAREN{\begin{array}{c}
#1
\end{array} }}
\newcommand{\TWOXTWO}[4]{\PAREN{ \begin{array}{c c} #1&#2 \\ #3 & #4 \end{array} }}
\newcommand{\DTWOXTWO}[4]{\BARS{ \begin{array}{c c} #1&#2 \\ #3 & #4 \end{array} }}
\newcommand{\THREEXTHREE}[9]{\PAREN{ \begin{array}{c c c} #1&#2&#3 \\ #4 & #5 & #6 \\ #7 & #8 & #9 \end{array} }}
\newcommand{\DTHREEXTHREE}[9]{\BARS{ \begin{array}{c c c} #1&#2&#3 \\ #4 & #5 & #6 \\ #7 & #8 & #9 \end{array} }}
\newcommand{\NXN}{\PAREN{ \begin{array}{c c c c} 
			a_{11} & a_{12} & \cdots & a_{1n} \\
			a_{21} & a_{22} & \cdots & a_{2n} \\
			\vdots & \vdots & \ddots & a_{1n} \\
			a_{n1} & a_{n2} & \cdots & a_{nn} \\
		\end{array} }}
\newcommand{\SLR}{SL_2(\R)}
\newcommand{\GLR}{GL_2(\R)}
\DeclareMathOperator{\TR}{tr}
\DeclareMathOperator{\BIL}{Bil}
\DeclareMathOperator{\SPAN}{span}

%%%%%%%
%  White space
%%%%%%%

\newcommand{\BOXIT}[1]{\noindent\fbox{\parbox{\textwidth}{#1}}}


\newtheorem{theorem}{Theorem}[section]
\newtheorem{corollary}{Corollary}[theorem]
\newtheorem{lemma}[theorem]{Lemma}

\theoremstyle{definition}
\newtheorem{definition}[theorem]{Definition}
\newtheorem{prop}[theorem]{Proposition}

\theoremstyle{remark}
\newtheorem{remark}[theorem]{Remark}
\newtheorem{example}[theorem]{Example}
%\newtheorem*{proof}[theorem]{Proof}



\newcommand{\RED}[1]{\textcolor{red}{#1}}
\newcommand{\BLUE}[1]{\textcolor{blue}{#1}}

\begin{document}

\begin{center}
	\Large{\CLASSNAME -- \SEMESTER} \\
	\large{ w/Professor Liu}
\end{center}
\begin{center}
	\STUDENTNAME \\
	\ASSIGNMENT -- \DUEDATE\\
\end{center} 

\begin{enumerate}[label=\Roman*.]
\item Let $\Omega \subset \R^m, a \in \Omega^o$.  If $f:\Omega \to \R$ is continuous at $a, g: \Omega \to \R$ is differentiable at $a$ and $g(a) = 0$, show that $fg$ is differentiable at $a$.  (Note $fg$ is the function whose value at $x \in \Omega$ is $f(x)g(x))$.

\BLUE{$g$ is differentiable at $a$ means that there exists a transformation $L$ such that 
\begin{align*}
	0 &= \lim_{x \to a} \frac{g(x) - (g(a)-L(x-a))}{|x-a|} \\
	&= \lim_{x \to a} \frac{g(x) + L(x-a)}{|x-a|} \\
	&= L 
\end{align*}Let's look at the following
\begin{align*}
	\lim_{x \to a} \frac{f(x)g(x) - (f(a)g(a)-L(x-a))}{|x-a|} &= \lim_{x \to a} \frac{f(x)g(x) + L(x-a)}{|x-a|} \\
	&= \lim_{x \to a} \frac{f(x)g(x)}{|x-a|} + \lim_{x\to a} \frac{L(x-a)}{|x-a|} \\
	&= 0 + L
\end{align*}thus a transformation exists for $(fg)(a)$ that satisfies the definition for differentiation.
}

\item \RED{skip II}

\newpage
\item Find the total derivative (i.e., derivative matrices) of the following functions at the given points.
\begin{enumerate}[label=(\alph*)]
	\item $f(x_1,x_2,x_3) = \PAREN{\begin{array}{c}
		x_2 \\ 
		x_1x_3^2 \\
		x_1+x_2+x_3
	\end{array} }$ at $(x_1,x_2,x_3) = (1,0,1)$.
	
	\BLUE{\begin{align*}
		&\begin{array}{c|lccc}
			& & \partial_1 f_i & \partial_2 f_i & \partial_3 f_i\\ \hline
			f_1 & x_2 & 0 & 1 & 0\\
			f_2 & x_1x_3 & x_3 & 0 & x_1 \\
			f_3 & x_1+x_2+x_3 & 1 & 1 & 1
		\end{array} \\ \\
		J_f(x_1, x_2, x_3) &= \THREEXTHREE{0}{1}{0}{x_3}{0}{x_1}{1}{1}{1} \\
		J_f(1,0,1) &= \THREEXTHREE{0}{1}{0}{1}{0}{1}{1}{1}{1}
	\end{align*}
	}
	
	\item $f(x) = \binom{x^2}{e^x}$ at $x=3$.
	
	\BLUE{\begin{align*}
		f'(x) &= \binom{2x}{e^x} \AND f'(3) = \binom{6}{e^3}
\end{align*}	
	}
	
	\item $f(x_1,x_2,x_3,x_4) = x_1^2+2x_2x_4+\sin(x_3x_4)$ at $(x_1,x_2,x_3,x_4) = (1,1,0,1)$
	
	\BLUE{\begin{align*}
		\partial_{x_1}f &= 2x_1\\
		\partial_{x_2}f &= 2x_4\\
		\partial_{x_3}f &= x_4\cos(x_3x_4)\\
		\partial_{x_4}f &= 2x_2+x_3\cos(x_3x_4) \\
		\\
		J_f(x_1,x_2,x_3,x_4) &= \PAREN{\begin{array}{c}
			2x_1\\
			2x_4\\
			x_4\cos(x_3x_4)\\
			2x_2+x_3\cos(x_3x_4)		
		\end{array} }\\
		J_f(1,1,0,1) &= \PAREN{\begin{array}{c}
			2 \\
			2 \\
			1 \\
			2
		\end{array}
		} 
	\end{align*}
	}
\end{enumerate}

\newpage
\item Section 6.2 Problem 2.

	\textit{Exercise 6.2.2}.  Prove Lemma 6.2.4. (Hint: prove by contradiction.  If $L_1 \ne L_2$, then there exists a vector $v$ such that $L_1v \ne L_2v$; this vector must be non-zero (why?).  Now apply the definition of derivative, and try to specialize to the case where $x=x_0+tv$ for some scalar $t$, to obtain a contradiction.)

\newcommand{\RRR}{\mathbb{R}}	
	\textbf{Lemma 6.2.4} (Uniqueness of derivatives).  \textit{Let $E$ be  subset of $\RRR^n, f: E\to \RRR^m$ be a function, $x_0\in E$ be an interior point of $E$, and let $L_1: \RRR^n\to \RRR^m$ and $L_2: \RRR^n\to \RRR^m$ be linear transformations.  Suppose that $f$ is differentiable at $x_0$ with derivatives $L_1$, and also diffentiable at $x_0$ with derivative $L_2$.  Then $L_1=L_2$}
\newcommand{\NORM}[1]{\left \lVert\, #1\, \right \lVert}	

	\BLUE{Let $L_1, L_2: \RRR^n \to \RRR^m$ be linear transformations and $L_1 \ne L_2$.  Also, let $E \subset \RRR^n$ and $f: E \to \RRR^m$ be a function that is differentiable at a point $x_0 \in E^o$ with derivatives $L_1$ and $L_2$ at $x_0$.  First, $\det f'(x_0) \ne 0$ because $f$ is differentialable at $x_0$ and since $L_2 \ne L_1$ there exists a non-zero vector $v$ such that $L_1 v \ne L_2 v$.
	\begin{align*}
		\text{for any } x &= x_0+tv \AND x_0 \ne 0\\
		L_1x &= L_1(x_0+tv) \AND L_2x = L_2(x_0+tv) \\
		L_1x_0 &= L_1x+L_1(tv) \AND L_2x_0 = L_2x+L_2(tv) \\
		L_1x+L_1(tv) &= L_2x+L_2(tv) \\
		L_1x-L_2x &= L_1(tv)+L_2(tv) \\
		(L_1-L_2)x &= (L_1-L_2)(tv) \\
		x &= tv \\
		\therefore x_0 &= 0 \, \Rightarrow\Leftarrow
	\end{align*}hence $L_1 = L_2$ making it unique.
	}
	
	\newpage
	\item Section 6.3, problem 3 and problem 4.
	
	\textit{Exercise 6.3.3.}  Let $f: \RRR^2 \to \RRR$ be a function defined by $f(x,y) := \frac{x^3}{x^2+y^2}$ when $(x,y) \ne (0,0)$, and $f(0,0) := 0$.  Show that $f$ is not differentiable at $(0,0)$, despite being differentiable in every direction $v \in \RRR^2$ at $(0,0)$. Explain why this does not contradict Thoerem 6.3.8.
	
	\BLUE{\begin{align*}
		f(x,y) &= \frac{x^3}{x^2+y^2} \\
		\partial_x f(x,y) &= \frac{3x^2}{x^2+y^2}  -\frac{2x^3}{x^3+y^2}\\
		\partial_y f(x,y) &= \frac{-2x^3}{x^2+y^2} 
	\end{align*}if we hold $x$ constant as $y \to 0$ we can see that $\partial_y f(x,y) \to \infty$ which means that $\partial_y f(x,y)$ is not continuous.  Theorem 6.3.8 states that the first partial derivatives must be continuous at $(0,0)$ for $f(x,y)$ to be differentiable at $(0,0)$.\\
	}
	
	\textit{Exercise 6.3.4.}  Let $f: \RRR^n \to \RRR^m$ be a differentiable function such that $f'(x) = 0$ for all $x \in \RRR^n$.  Show that $f$ is constant.  (Hint: you may use the mean-value theorem or fundamental theorem of calculus for one-dimensional functions, but bear in mind that there is a direct analogue to these theorems for several-variable functions.  I would not advise proeeding via first principles.)  For a tougher challenge, replace the domain $\RRR^n$ by an open connected subet $\Omega$ of $\RRR^n$.
	
	\BLUE{\begin{align*}
		\LET \Omega &\subset \RRR^n \AND [a,b] \in \Omega\\
		\exists \xi \in [a,b] & \to |f(b)-f(a)|\le f'(\xi)|b-a| & \text{mean value theorem}\\
		|f(b)-f(a)| &\le 0\\
		f(b)=f(a)
\end{align*}	$a$ and $b$ are arbitrary thus $f(x)=f(a)$ for all $x\in \RRR^n$, thus $f$ is a constant function.
	}

\newpage
\item Let $f: \R^m \to \R$ be differentiable, $\alpha \in \R$.  If $f(tx)=t^\alpha f(x)$	for $\forall x \in \R^m$ and $t> 0$, we say that $f$ is homogeneous of order $\alpha$.  Show that $f$ is homogeneous of order $\alpha$ iff $x \cdot \nabla f(x)=\alpha f(x)$, that is
\begin{align*}
	x^1\partial_1 f(x)+\cdots+x^m\partial_m f(x) = \alpha f(x). \
\end{align*}This equation is classically written as 
\begin{align*}
	x^1 \PART{f}{x^1}+\cdots+x^m\PART{f}{x^m} = \alpha f(x).
\end{align*}Hint: As in the development of the theory in the text, a basic idea to study multivariable functions is to convert them into single-variable functions by restricting the variable $x$ in a fixed direction.  For example, for this problem you may consider the function $\varphi(t) = f(t)$.

\BLUE{\begin{description}
\item $(\Rightarrow)$ $f$ is homogenous of order $\alpha$, that is, $f(tx)= t^\alpha f(x)$. Then,
\begin{align*}
	\LET \varphi(t) &= f(tx) = t^\alpha f(x)\\
	\varphi'(t) &= f'(tx)\cdot x = \alpha t^{\alpha-1}f(x) \\
	\LET t=1 \to f'(x)\cdot x &= \alpha f(x)
\end{align*}
\item $(\Leftarrow)$ assume that $x\cdot \nabla f(x) = \alpha f(x)$.  Let $x = ty$ then
\begin{align*}
	\LET \varphi(t) &= f(xt) \\
	\varphi'(t) &= x\cdot f'(tx) = \alpha f(tx) = \alpha \varphi(t)
\end{align*}this is an ordinary differential equation whose solution is $\varphi(t) = Ct^\alpha$.  Notice $\varphi(1) = C = f(x)$.  Thus, $\varphi(t) = f(tx) = t^\alpha f(x)$.
\end{description}
}
	
\newpage
\item \begin{enumerate}[label=(\alph*)]

	\item Let $f: \R^m \to \R^m$ be a $C^1$-map,$$ |f(x)-f(y)|\ge |x-y|, \forall x,y \in \R^m,$$ then $\forall a \in \R^m, \det f'(a) \ne 0$.
	
	\item Let $f: \R^2\to \R^2$ be differentiable, and assume $f(0,0)=\ABRACKET{1,2}$, and $$ Df(0,0)=\TWOXTWO{1}{3}{2}{0}.$$ Let $g(x,y)=\ABRACKET{xy^2, y+2, 2x-3y}.$.  Find $D(g\circ f)(0,0)$.

	\BLUE{\begin{align*}
		g(x,y)&=\ABRACKET{xy^2, y+2, 2x-3y} \\
		g'(x,y) &= \PAREN{\begin{array}{cc}
			\frac{\partial g_1(x,y)}{\partial x} & \frac{\partial  g_1(x,y)}{\partial y} \\
			\frac{\partial g_2(x,y)}{\partial x} & \frac{\partial  g_2(x,y)}{\partial y} \\
			\frac{\partial g_3(x,y)}{\partial x} & \frac{\partial g_3(x,y)}{\partial y} \\
		\end{array}		 }\\
		&= \PAREN{\begin{array}{cc}
			y^2 & x \\
			0 & 1 \\
			2-3y & 2x-3
		\end{array}
		}\\
		D(g \circ f)(0,0) &= Dg(f(0,0))Df(0,0) \\
		&= Dg(1,2)Df(0,0) \\
		&= \PAREN{\begin{array}{cc}
				4 & 1 \\
				0 & 1 \\
				-4 & -1
			\end{array}		 }\TWOXTWO{1}{3}{2}{0} \\
		&= \PAREN{\begin{array}{cc}
			6 & 12 \\
			2 & 0 \\
			-6 & -12	
		\end{array}		 }
	\end{align*}
	}
		
\end{enumerate}

\newpage
	\item Let $f:E\to \R$ be defined on some open set $E \subset \R^2$, and assume the partial derivatives$\PART{f}{x_1}, \PART{f}{x_2}$ are bounded in $E$.  Prove that $f$ is continuous in $E$.
	
	\textit{Hint: } Proceed as in the proof of Theorem 6.3.8 (continuity of partial derivatives implies $f$ is differentiable) which we discussed in class.

	\BLUE{Since the partial derivatives are bounded, let $M_i =\max \PART{F}{x_i}$.  Then let $M = (M_1\,\, M_2)$.  They are bounded and therefore continuous. Thus we can say that
	\begin{align*}
		L &= f'(x_0) \\
		\forall \epsilon > 0, \, \epsilon &> \frac{ \BARS{ f(x)-f(x_0)-L(x - x_0) }}{|x-x_0|}, \, \text{whenever } \delta > |x-x_0| \text{ for some } \delta >0 \\
		&\le \frac{ \BARS{ f(x)-f(x_0)-M(x - x_0) }}{|x-x_0|} \\
		\epsilon |x-x_0| &\le \BARS{ f(x)-f(x_0)-M(x - x_0) } \\
		\epsilon\delta &\ge \BARS{ f(x)-f(x_0)}
	\end{align*}
	}
	
	\newpage
	\item Let $F(x,y,z) = \PAREN{\begin{array}{c}
		x+y \\
		x^2y \\
		z+2x	
	\end{array} }.$
	
	\begin{enumerate}[label=(\alph*)]
		\item At what points $(x_0, y_0, z_0)$ does $F$ have a local inverse, i.e., a function $F^{-1}$ defined on an open set $V$ containing $F(x_0, y_o, z_o)$, such that $F(F^{-1}(x,y,z))=(x,y,z)$ for all $(x,y,z) \in V$?
		
		\BLUE{The inverse exists wherever the Jacobian is valid.
		\begin{align*}
			F'(x,y,z) &= \THREEXTHREE{1}{1}{0}
			{2xy}{x^2}{0}
			{2}{0}{1} \\
			\det F'(x,y,z) &= x^2 - 2xy \\
			\det F'(x,y,z) &= 0 \implies x=2y
		\end{align*}Thus, $F^{-1}$ exists everywhere except on the line $x=2y$.
		}
		
		\item What is $D(F^{-1})(2,1,3)$?  (Hint: $F(1,1,1)=(2,1,3)$.)
		
		\BLUE{By utilizing the hint, 
		\begin{align*}
			D(F^{-1})(2,1,3) &= D(F^{-1})(F(1,1,1))\\
			&= F'(1,1,1)^{-1} \\
			&= \THREEXTHREE{1}{1}{0}
			{2}{1}{0}
			{2}{0}{1}^{-1} \\
			&= \THREEXTHREE{-1}{1}{0}
			{2}{-1}{0}
			{2}{-2}{1}
		\end{align*}
		}
	\end{enumerate}
	
	\newpage
	\item When does the equation $x_1^2+2x_2^3x_3-x^4+\ln(1+x_4^2)=1$ define a function $x_4=g(x_1,x_2,x_3)$ implicitly? Find $\nabla g(1,0,-1)$.
	
	\BLUE{\begin{align*}
		f(x_1,x_2,x_3, x_4) &= x_1^2+2x_2^3x_3-x_4+\ln(1+x_4^2)-1 = 0 \\
		\partial_{x_4} f &= -1 + \frac{2x_4}{1+x_4^2} \\
		&= \frac{-1-x_4^2+2x^4}{1+x_4^2} \\
		&= \frac{(2x_4+1)(x_4-1)}{1+x_4^2} \\
		\partial_{x_4} f = 0 &\text{ when } x_4 \in \BRACKET{ \frac{-1}{2}, 1}.
	\end{align*} there is an implicit function for $x_4=g(x_1,x_2,x_3)$ when $x_4 \not \in \BRACKET{ \frac{-1}{2}, 1}$.  Then we have 
	\begin{align*}
		\partial_{x_1} g &= \frac{-\partial_{x_1}f}{\partial_{x_4}f} = \frac{2x_1}{\frac{(2x_4+1)(x_4-1)}{1+x_4^2}} = \frac{2x_1(1+x_4^2)}{(2x_4+1)(x_4-1)} \\
		\partial_{x_2} g &= \frac{-\partial_{x_2}f}{\partial_{x_4}f} = \frac{6x_2^2x_3}{\frac{(2x_4+1)(x_4-1)}{1+x_4^2}} = \frac{6x_2^2x_3(1+x_4^2)}{(2x_4+1)(x_4-1)} \\
		\partial_{x_3} g &= \frac{-\partial_{x_3}f}{\partial_{x_4}f} = \frac{2x_2^3}{\frac{(2x_4+1)(x_4-1)}{1+x_4^2}} = \frac{2x_2^3(1+x_4^2)}{(2x_4+1)(x_4-1)} \\ \\
		\nabla g(1,0,-1) &= \ABRACKET{ \frac{2(1+x_4^2)}{(2x_4+1)(x_4-1)}, 0, 0  }
	\end{align*}
	}
\end{enumerate}

\end{document}
