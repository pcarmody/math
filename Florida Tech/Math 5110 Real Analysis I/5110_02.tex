\documentclass[10pt,a4paper]{report}
\usepackage[utf8]{inputenc}
\usepackage{amsmath}
\usepackage{amsfonts}
\usepackage{amssymb}
\usepackage{amsthm}
\usepackage{hyperref}

\usepackage{multicol}
\usepackage{fancyhdr}
\usepackage[inline]{enumitem}
\usepackage{tikz}
\usepackage{tikz-cd}
\usetikzlibrary{calc}
\usetikzlibrary{shapes.geometric}
\usepackage[margin=0.5in]{geometry}
\usepackage{xcolor}

\hypersetup{
    colorlinks=true,
    linkcolor=blue,
    filecolor=magenta,      
    urlcolor=cyan,
    pdftitle={Tensors},
    pdfpagemode=FullScreen,
    }

%\urlstyle{same}

\newcommand{\CLASSNAME}{Math 5110 -- Real Analysis I}
\newcommand{\STUDENTNAME}{Paul Carmody}
\newcommand{\ASSIGNMENT}{Homework \#2 }
\newcommand{\DUEDATE}{September 18, 2024}
\newcommand{\SEMESTER}{Fall 2024}
\newcommand{\SCHEDULE}{MW 11:00 -- 12:15}
\newcommand{\ROOM}{Remote}

\newcommand{\MMN}{M_{m\times n}}
\newcommand{\FF}{\mathcal{F}}

\pagestyle{fancy}
\fancyhf{}
\chead{ \fancyplain{}{\CLASSNAME} }
%\chead{ \fancyplain{}{\STUDENTNAME} }
\rhead{\thepage}
\newcommand{\LET}{\text{Let }}
%\newcommand{\IF}{\text{if }}
\newcommand{\AND}{\text{ and }}
\newcommand{\OR}{\text{ or }}
\newcommand{\FORSOME}{\text{ for some }}
\newcommand{\FORALL}{\text{ for all }}
\newcommand{\WHERE}{\text{ where }}
\newcommand{\WTS}{\text{ WTS }}
\newcommand{\WLOG}{\text{ WLOG }}
\newcommand{\BS}{\backslash}
\newcommand{\DEFINE}[1]{\textbf{\emph{#1}}}
\newcommand{\IF}{$(\Rightarrow)$}
\newcommand{\ONLYIF}{$(\Leftarrow)$}
\newcommand{\ITH}{\textsuperscript{th} }
\newcommand{\FST}{\textsuperscript{st} }
\newcommand{\SND}{\textsuperscript{nd} }
\newcommand{\TRD}{\textsuperscript{rd} }
\newcommand{\INV}{\textsuperscript{-1} }

\newcommand{\XXX}{\mathfrak{X}}
\newcommand{\MMM}{\mathfrak{M}}
%\newcommand{\????}{\textfrak{A}}
%\newcommand{\????}{\textgoth{A}}
%\newcommand{\????}{\textswab{A}}

\DeclareMathOperator{\DER}{Der}
\DeclareMathOperator{\SGN}{sgn}

%%%%%%%
% derivatives
%%%%%%%

\newcommand{\PART}[2]{\frac{\partial #1}{\partial #2}}
\newcommand{\SPART}[2]{\frac{\partial^2 #1}{\partial #2^2}}
\newcommand{\DERIV}[2]{\frac{d #1}{d #2}}
\newcommand{\LAPLACIAN}[1]{\frac{\partial^2 #1}{\partial x^2} + \frac{\partial^2 #1}{\partial y^2}}

%%%%%%%
% sum, product, union, intersections
%%%%%%%

\newcommand{\SUM}[2]{\underset{#1}{\overset{#2}{\sum}}}
\newcommand{\PROD}[2]{\underset{#1}{\overset{#2}{\prod}}}
\newcommand{\UNION}[2]{\underset{#1}{\overset{#2}{\bigcup}}}
\newcommand{\INTERSECT}[2]{\underset{#1}{\overset{#2}{\bigcap}}}
\newcommand{\FSUM}{\SUM{n=-\infty}{\infty}}
       

%%%%%%%
% supremum and infimum
%%%%%%%

\newcommand{\SUP}[1]{\underset{#1}\sup \,}
\newcommand{\INF}[1]{\underset{#1}\inf \,}
\newcommand{\MAX}[1]{\underset{#1}\max \,}
\newcommand{\MIN}[1]{\underset{#1}\min \,}

%%%%%%%
% infinite sums, limits
%%%%%%%

\newcommand{\SUMK}{\SUM{k=1}{\infty}}
\newcommand{\SUMN}{\SUM{n=1}{\infty}}
\newcommand{\SUMKZ}{\SUM{k=0}{\infty}}
\newcommand{\LIM}[1]{\underset{#1}\lim\,}
\newcommand{\IWOB}[1]{\LIM{#1 \to \infty}}
\newcommand{\LIMK}{\IWOB{k}}
\newcommand{\LIMN}{\IWOB{n}}
\newcommand{\LIMX}{\IWOB{x}}
\newcommand{\NIWOB}{\LIM{n \to \infty}}
\newcommand{\LIMSUPK}{\underset{k\to\infty}\limsup \,}
\newcommand{\LIMSUPN}{\underset{n\to\infty}\limsup \,}
\newcommand{\LIMINFK}{\underset{k\to\infty}\liminf \,}
\newcommand{\LIMINFN}{\underset{n\to\infty}\liminf \,}
\newcommand{\ROOTRULE}[1]{\LIMSUPK \BARS{#1}^{1/k}}

\newcommand{\CUPK}{\bigcup_{k=1}^{\infty}}
\newcommand{\CAPK}{\bigcap_{k=1}^{\infty}}
\newcommand{\CUPN}{\bigcup_{n=1}^{\infty}}
\newcommand{\CAPN}{\bigcap_{n=1}^{\infty}}

%%%%%%%
% number systems (real, rational, etc.)
%%%%%%%

\newcommand{\REALS}{\mathbb{R}}
\newcommand{\RATIONALS}{\mathbb{Q}}
\newcommand{\IRRATIONALS}{\REALS \backslash \RATIONALS}
\newcommand{\INTEGERS}{\mathbb{Z}}
\newcommand{\NUMBERS}{\mathbb{N}}
\newcommand{\COMPLEX}{\mathbb{C}}
\newcommand{\DISC}{\mathbb{D}}
\newcommand{\HPLANE}{\mathbb{H}}

\newcommand{\R}{\mathbb{R}}
\newcommand{\Q}{\mathbb{Q}}
\newcommand{\Z}{\mathbb{Z}}
\newcommand{\N}{\mathbb{N}}
\newcommand{\C}{\mathbb{C}}
\newcommand{\T}{\mathbb{T}}
\newcommand{\COUNTABLE}{\aleph_0}
\newcommand{\UNCOUNTABLE}{\aleph_1}


%%%%%%%
% Arithmetic/Algebraic operators
%%%%%%%


\DeclareMathOperator{\MOD}{mod}
%\newcommand{\MOD}[1]{\mod #1}
\newcommand{\BAR}[1]{\overline{#1}}
\newcommand{\LCM}{\text{ lcm}}
\newcommand{\ZMOD}[1]{\Z/#1\Z}
\DeclareMathOperator{\VAR}{Var}
%%%%%%%
% complex operators
%%%%%%%

\DeclareMathOperator{\RR}{Re}
%\newcommand{\RE}{\text{Re}}
\DeclareMathOperator{\IM}{Im}
%\newcommand{\IM}{\text{Im}}
\newcommand{\CONJ}[1]{\overline{#1}}
\DeclareMathOperator{\LOG}{Log}
%\newcommand{\LOG}{\text{ Log }}
\newcommand{\RES}[2]{\underset{#1}{\text{res}} #2}

%%%%%%%
% Group operators
%%%%%%%

\newcommand{\AUT}{\text{Aut}\,}
\newcommand{\KER}{\text{ker}\,}
\newcommand{\END}{\text{End}}
\newcommand{\HOM}{\text{Hom}}
\newcommand{\CYCLE}[1]{(\begin{array}{cccccccccc}
		#1
	\end{array})}
\newcommand{\SUBGROUP}{\underset{\text{group}}\subseteq}	
%\newcommand{\SUBGROUP}{\subseteq_g}
\newcommand{\SUBRING}{\underset{\text{ring}}\subseteq}
\newcommand{\SUBMOD}{\underset{\text{mod}}\subseteq}
\newcommand{\SUBFIELD}{\underset{\text{field}}\subseteq}
\newcommand{\ISO}{\underset{\text{iso}}\longrightarrow}
\newcommand{\HOMO}{\underset{\text{homo}}\longrightarrow}

%%%%%%%
% grouping (parenthesis, absolute value, square, multi-level brackets).
%%%%%%%

\newcommand{\PAREN}[1]{\left (\, #1 \,\right )}
\newcommand{\BRACKET}[1]{\left \{\, #1 \,\right \}}
\newcommand{\SQBRACKET}[1]{\left [\, #1 \,\right ]}
\newcommand{\ABRACKET}[1]{\left \langle\, #1 \,\right \rangle}
\newcommand{\BARS}[1]{\left |\, #1 \,\right |}
\newcommand{\DBARS}[1]{\left \| \, #1 \,\right \|}
\newcommand{\LBRACKET}[1]{\left \{ #1 \right .} 
\newcommand{\RBRACKET}[1]{\left . #1 \right \]}
\newcommand{\RBAR}[1]{\left . #1 \, \right |}
\newcommand{\LBAR}[1]{\left | \, #1 \right .}
\newcommand{\BLBRACKET}[2]{\BRACKET{\RBAR{#1}#2}}
\newcommand{\GEN}[1]{\ABRACKET{#1}}
\newcommand{\BINDEF}[2]{\LBRACKET{\begin{array}{ll}
     #1\\
     #2
\end{array}}}

%%%%%%%
% Fourier Analysis
%%%%%%%

\newcommand{\ONEOTWOPI}{\frac{1}{2\pi}}
\newcommand{\FHAT}{\hat{f}(n)}
\newcommand{\FINT}{\int_{-\pi}^\pi}
\newcommand{\FINTWO}{\int_{0}^{2\pi}}
\newcommand{\FSUMN}[1]{\SUM{n=-#1}{#1}}
%\newcommand{\FSUM}{\SUMN{\infty}}
\newcommand{\EIN}[1]{e^{in#1}}
\newcommand{\NEIN}[1]{e^{-in#1}}
\newcommand{\INTALL}{\int_{-\infty}^{\infty}}
\newcommand{\FTINT}[1]{\INTALL #1 e^{2\pi inx\xi} dx}
\newcommand{\GAUSS}{e^{-\pi x^2}}

%%%%%%%
% formatting 
%%%%%%%

\newcommand{\LEFTBOLD}[1]{\noindent\textbf{#1}}
\newcommand{\SEQ}[1]{\{#1\,\}}
\newcommand{\WIP}{\footnote{work in progress}}
\newcommand{\QED}{\hfill\square}
\newcommand{\ts}{\textsuperscript}
\newcommand{\HLINE}{\noindent\rule{7in}{1pt}\\}

%%%%%%%
% Mathematical note taking (definitions, theorems, etc.)
%%%%%%%

\newcommand{\REM}{\noindent\textbf{\\Remark: }}
\newcommand{\DEF}{\noindent\textbf{\\Definition: }}
\newcommand{\THE}{\noindent\textbf{\\Theorem: }}
\newcommand{\COR}{\noindent\textbf{\\Corollary: }}
\newcommand{\LEM}{\noindent\textbf{\\Lemma: }}
\newcommand{\PROP}{\noindent\textbf{\\Proposition: }}
\newcommand{\PROOF}{\noindent\textbf{\\Proof: }}
\newcommand{\EXP}{\noindent\textbf{\\Example: }}
\newcommand{\TRICKS}{\noindent\textbf{\\Tricks: }}


%%%%%%%
% text highlighting
%%%%%%%

\newcommand{\B}[1]{\textbf{#1}}
\newcommand{\CAL}[1]{\mathcal{#1}}
\newcommand{\UL}[1]{\underline{#1}}

%%%%%%
% Linear Algebra
%%%%%%

\newcommand{\COLVECTOR}[1]{\PAREN{\begin{array}{c}
#1
\end{array} }}
\newcommand{\TWOXTWO}[4]{\PAREN{ \begin{array}{c c} #1&#2 \\ #3 & #4 \end{array} }}
\newcommand{\DTWOXTWO}[4]{\BARS{ \begin{array}{c c} #1&#2 \\ #3 & #4 \end{array} }}
\newcommand{\THREEXTHREE}[9]{\PAREN{ \begin{array}{c c c} #1&#2&#3 \\ #4 & #5 & #6 \\ #7 & #8 & #9 \end{array} }}
\newcommand{\DTHREEXTHREE}[9]{\BARS{ \begin{array}{c c c} #1&#2&#3 \\ #4 & #5 & #6 \\ #7 & #8 & #9 \end{array} }}
\newcommand{\NXN}{\PAREN{ \begin{array}{c c c c} 
			a_{11} & a_{12} & \cdots & a_{1n} \\
			a_{21} & a_{22} & \cdots & a_{2n} \\
			\vdots & \vdots & \ddots & a_{1n} \\
			a_{n1} & a_{n2} & \cdots & a_{nn} \\
		\end{array} }}
\newcommand{\SLR}{SL_2(\R)}
\newcommand{\GLR}{GL_2(\R)}
\DeclareMathOperator{\TR}{tr}
\DeclareMathOperator{\BIL}{Bil}
\DeclareMathOperator{\SPAN}{span}

%%%%%%%
%  White space
%%%%%%%

\newcommand{\BOXIT}[1]{\noindent\fbox{\parbox{\textwidth}{#1}}}


\newtheorem{theorem}{Theorem}[section]
\newtheorem{corollary}{Corollary}[theorem]
\newtheorem{lemma}[theorem]{Lemma}

\theoremstyle{definition}
\newtheorem{definition}[theorem]{Definition}
\newtheorem{prop}[theorem]{Proposition}

\theoremstyle{remark}
\newtheorem{remark}[theorem]{Remark}
\newtheorem{example}[theorem]{Example}
%\newtheorem*{proof}[theorem]{Proof}



\newcommand{\RED}[1]{\textcolor{red}{#1}}
\newcommand{\BLUE}[1]{\textcolor{blue}{#1}}

\begin{document}

\begin{center}
	\Large{\CLASSNAME -- \SEMESTER} \\
	\large{ w/Professor Liu}
\end{center}
\begin{center}
	\STUDENTNAME \\
	\ASSIGNMENT -- \DUEDATE\\
\end{center} 

\begin{enumerate}[label=\Roman*.]
\item Consider a sequence $x_n$ of real numbers.  The \textit{limit inferior} and \textit{limit superior} of $x_n$ are defined by 
\begin{align*}
	\LIMINFN x_n = \lim_{n->\infty}\PAREN{\inf_{k \ge n} x_k }, \,\, \LIMSUPN x_n = \lim_{n->\infty}\PAREN{\sup_{k \ge n} x_k }
\end{align*}
	\begin{enumerate}[label=(\alph*)]
		\item Show that
			\begin{align*}
				\LIMINFN x_n = \sup_{n\ge 0} \PAREN{\inf_{k\ge n} x_l}
			\end{align*}and
			\begin{align*}
				\LIMSUPN x_n = \inf_{k\ge n}\PAREN{\sup_{k\ge n} x_n}
			\end{align*}
		
		\BLUE{If $\{ x_n \}$ were not bounded then this questions has no value.  Assuming, then, that $x_n$ is bounded above by its least upper bound, $M$, and below by its most lower bound, $L$.  We need to be aware of a repeating sequence or subsequence.  For example, $x_n = c$ where $c$ or $x_n = \{1/2, c, 1/4, c, 1/6, c, 1/8, \dots \}$.  These sequences are also bounded and form trivial solutions to this question.\\
		We will focus, then, on the convergent subsequence, $\{ x_{\alpha_k}\}$ where $\alpha_k \in A$ and $A$ is an infinite list of indices.  We have two types of these convergent sequences those that coverge increasingly and those that converge decreasingly.  
		\begin{description}
			\item \textbf{Converge Increasingly}.  These subsequences may have many values but as $n \to \infty$ we see that $|x_{\alpha_n} - M| \to 0$ and without loss of generality, allow it to be Cauchy. \\ Claim: given any $n\ge 0$, the $\INF{k\ge n} x_{\alpha_k} = x_{\alpha_n}$.  We can see that given any $\epsilon >0$ there is $N \in \N$ we have $|x_{\alpha_n} - x_{\alpha_k} | < \epsilon$ whenever $N < n < k$.  That is, the subsequence is increasing, thus, $x_{\alpha_n} < x_{\alpha_k}$ for all $k \ge n$.  Thus $\INF{k \ge n} x_{\alpha_k} = x_{\alpha_n}$.  Further, a $n \to \infty$ the infimum $\INF{k \ge n} x_{\alpha_k}$ increases thus the supremum of these values is
			\begin{align*}
				\LIMINFN x_{\alpha_n} = \sup_{n\ge 0} \PAREN{\inf_{k\ge n} x_{\alpha_k} }
			\end{align*}
\end{description}		
		}
		
		\item Show that $\LIMINFN x_n$ and $\LIMSUPN x_n$ are well-defined for any sequence $x_n$.  (Unlike $\LIM{n\to\infty} x_n$.)  We allow values of $\infty$ and $-\infty$
		
		\item Let $x_n$ be a bounded sequence, and let $L$ be the set of limit points of $x_n$, i.e., the set of all limits of subsequences of $x_n$.  Show $\LIMINFN x_n = \inf L$ and $\LIMSUPN = \sup L$.
		\item Let $x_n$ be a bounded sequence.  Conclude using (c) that $\LIMINFN x_n \le \LIMSUPN x_n$, with equality if and only if $x_n$ is convergent.
	\end{enumerate}
	
	\item Prove that for any (possibly uncountable) collection $(F_\alpha)_{\alpha\in A}$ of closed sets, the intersection $F = \bigcup_{\alpha \in A} F_\alpha$ is closed, in two ways.
	\begin{enumerate}[label=(\alph*)]
		\item Using the fact that any union of open sets is open, and DeMorgan's Laws from set theory, which state
		\begin{align*}
			X\backslash \PAREN{\bigcup_{\alpha\in A} E_\alpha } = \bigcap_{\alpha \in A} \PAREN{X \backslash E_\alpha} \AND X\backslash\PAREN{\bigcap_{\alpha \in A} E_\alpha} = \bigcup_{\alpha\in A} \PAREN{X \backslash E_\alpha}
		\end{align*}for all collection of sets $(E_\alpha)_{\alpha\in A}$
		
		\BLUE{Given that every open set, $E\in X$ is the union of other open sets $\bigcup_{\alpha \in A} E_\alpha$ for some index set $A$ (whether countable or uncountable).  We know that the complement is closed and the complement can be expressed as
		\begin{align*}
			E^c &= X\backslash E \\
			&= X\backslash \PAREN{\bigcup_{\alpha\in A} E_\alpha }\\
			&= \bigcap_{\alpha \in A} \PAREN{X \backslash E_\alpha}
		\end{align*}each $E_\alpha$ is the complement of an open set, hence they are closed.  Thus, $E^c$ which is closed is made up of the intersection of closed sets.
		}
		
		\item More directly, using the fact that a set $G$ is closed if and only if for any convergent sequence $(x_n)$ with all $x_n \in G$, the limit $x$ is also in $G$.
		
		\BLUE{Let $F, G \in X$ be closed sets and let $(x_n) \subset G$ and $(y_n) \subset F$ both be convergent sequences. Further, we let $(x_n), (y_n) \subset G\cap F$. Not that $F$ closed means that $(x_n) \in F$ implies that $\LIMN x_n \in F$, thus $\LIMN x_n \in G \cap F$ and a similar argument can be made for $y_n$ and $G$.  Thus sequences contained in $G\cap F$ must also contain their limits and $G\cap F$ is closed.  This can extend to any number of intersections.
		}
	\end{enumerate}
	
	\item\begin{enumerate}[label=(\alph*)]
		\item Let $(x_n)$ be a Cauchy sequence in a metric space $X$.  Show that if a subsequence $(x_{n_j})$ of $x_n$ converges to $x$, then the entire sequence also converges to $x$.
		\item Show that the metric space
		\begin{align*}
			C^1((-1,1))=\{f:(-1,1)\to\R, f \text{ is differentiable and }f'\text{ is continuous in }(1,-1)\}
		\end{align*}with the metric
		\begin{align*}
			d(f,g) = \sup_{x\in (-1,1)} |f(x)-g(x)|
		\end{align*}is not complete.  (Hint: similar to the proof that the rational numbers are not complete, find a sequence $C'((-1,1))$ that converges in $d$ metric to a function that is not in $C^1((-=1,1))$, and show that this sequence is Cauchy.)
,r to the proof that the rational numbers are not complete, find a sequence $C'((-1,1))$ that converges in $d$ metric to a function that is not in $C^1((-=1,1))$, and show that this sequence is Cauchy.)
	\end{enumerate}
	
	\item Let $A$ and $B$ be subsets of the metric space $X$.  which one of the following is true?
	\begin{align}
		(A\cup B)^o &= A^o \cup B^o, &\label{eq:eq1}\tag{2.1}\\
		(A \cup B)^o &\subset A^o\cup B^o, & \text{"=" fails for some $A$ and $B$} \label{eq:eq1}\tag{2.2} \\
		(A\cup B)^o &\supset A^o\cup B^o, &\text{"=" fails for some $A$ and $B$}\label{eq:eq1}\tag{2.3}
	\end{align}
	
	\item Let $C^0([a,b])$ be the space of continuous functions on $[a,b]$, with the metric $d(f,g) = \SUP{x\in [a,b]}|f(x)-g(x)|$.
	
	Show that the map $I: C^0([a,b]) \to \R$ defined by $I(f)=\int_a^b f(x) dx$ is continuous mapping from $C^0([a,b])$ to $\R$.
	
	\BLUE{$I$ is continuous if for ever $\epsilon > 0$ there exists $\delta > 0$ such that $d(I(f),I(g))<\epsilon$ whenever $d(f,g) < \delta$. Or 
	\begin{align*}
		d(I(f),I(g)) &= \SUP{x\in [a,b]}|I(f(x))-I(g(x))|\\
			&= \SUP{x\in [a,b]}\BARS{\int_a^b f(x) dx-\int_a^b g(x) dx}\\
			&= \SUP{x\in [a,b]}\BARS{\int_a^b f(x) - g(x) dx}\\
			&= \SUP{x\in [a,b]}\int_a^b \BARS{f(x) - g(x)} dx\\
			&\le \int_a^b \SUP{x\in [a,b]}\BARS{f(x) - g(x)} dx\\
			&\le \int_a^b d(f,g) dx\\
			&\le d(f,g)[b-a]
	\end{align*}Thus when $\epsilon > 0$ choose $\delta < [b-a]d(f,g)$.  Hence, $I$ is continuous.
	}
	
	\item Prove Propostion 2.3.2 in the text, in two different ways.:
	\begin{enumerate}[label=\alph*)]
		\item As a consequence of Theorem 2.3.1 in text.
		\item Directly, using the sequential definition of compactness.
		\textbf{Proposition 2.3.2} (Maximum principle). \textit{ Let $(X,d)$ be a compact metric space, and let $f:X \to \R$ be a contnuous function.  Then $f$ is bounded.  Furthermore, $f$ attains its maximum at some point $x_{\max} \in X$, and also attains its minimum at some point $x_{\min} \in X$. }
	\end{enumerate}
	
	\item Let $f:\R^n \to \R$ be a continuous function such that 
	\begin{align*}
		\lim_{|x| \to \infty} f(x) &= + \infty
	\end{align*}Prove that $f$ attains its minimum.
	
	Recall that by definition, the limit in (??) means that Given $A > 0$, there is $R>0$ such that 
	\begin{align*}
		f(x) >A \text{    for all } x \not\in B_R
	\end{align*}in other words, $f(x) >A$ whenever $|x| \ge R$. Here, $|x| = d_2(x,0)$ and $d_2$ is the standard Euclidean distance defined in Example 1.4.
\end{enumerate}
\end{document}
