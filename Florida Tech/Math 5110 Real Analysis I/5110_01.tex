\documentclass[10pt,a4paper]{report}
\usepackage[utf8]{inputenc}
\usepackage{amsmath}
\usepackage{amsfonts}
\usepackage{amssymb}
\usepackage{amsthm}
\usepackage{hyperref}

\usepackage{multicol}
\usepackage{fancyhdr}
\usepackage[inline]{enumitem}
\usepackage{tikz}
\usepackage{tikz-cd}
\usetikzlibrary{calc}
\usetikzlibrary{shapes.geometric}
\usepackage[margin=0.5in]{geometry}
\usepackage{xcolor}

\hypersetup{
    colorlinks=true,
    linkcolor=blue,
    filecolor=magenta,      
    urlcolor=cyan,
    pdftitle={Tensors},
    pdfpagemode=FullScreen,
    }

%\urlstyle{same}

\newcommand{\CLASSNAME}{Math 5110 -- Real Analysis I}
\newcommand{\STUDENTNAME}{Paul Carmody}
\newcommand{\ASSIGNMENT}{Homework \#1 }
\newcommand{\DUEDATE}{September 2, 2024}
\newcommand{\SEMESTER}{Fall 2024}
\newcommand{\SCHEDULE}{MW 11:00 -- 12:15}
\newcommand{\ROOM}{Remote}

\newcommand{\MMN}{M_{m\times n}}
\newcommand{\FF}{\mathcal{F}}

\pagestyle{fancy}
\fancyhf{}
\chead{ \fancyplain{}{\CLASSNAME} }
%\chead{ \fancyplain{}{\STUDENTNAME} }
\rhead{\thepage}
\newcommand{\LET}{\text{Let }}
%\newcommand{\IF}{\text{if }}
\newcommand{\AND}{\text{ and }}
\newcommand{\OR}{\text{ or }}
\newcommand{\FORSOME}{\text{ for some }}
\newcommand{\FORALL}{\text{ for all }}
\newcommand{\WHERE}{\text{ where }}
\newcommand{\WTS}{\text{ WTS }}
\newcommand{\WLOG}{\text{ WLOG }}
\newcommand{\BS}{\backslash}
\newcommand{\DEFINE}[1]{\textbf{\emph{#1}}}
\newcommand{\IF}{$(\Rightarrow)$}
\newcommand{\ONLYIF}{$(\Leftarrow)$}
\newcommand{\ITH}{\textsuperscript{th} }
\newcommand{\FST}{\textsuperscript{st} }
\newcommand{\SND}{\textsuperscript{nd} }
\newcommand{\TRD}{\textsuperscript{rd} }
\newcommand{\INV}{\textsuperscript{-1} }

\newcommand{\XXX}{\mathfrak{X}}
\newcommand{\MMM}{\mathfrak{M}}
%\newcommand{\????}{\textfrak{A}}
%\newcommand{\????}{\textgoth{A}}
%\newcommand{\????}{\textswab{A}}

\DeclareMathOperator{\DER}{Der}
\DeclareMathOperator{\SGN}{sgn}

%%%%%%%
% derivatives
%%%%%%%

\newcommand{\PART}[2]{\frac{\partial #1}{\partial #2}}
\newcommand{\SPART}[2]{\frac{\partial^2 #1}{\partial #2^2}}
\newcommand{\DERIV}[2]{\frac{d #1}{d #2}}
\newcommand{\LAPLACIAN}[1]{\frac{\partial^2 #1}{\partial x^2} + \frac{\partial^2 #1}{\partial y^2}}

%%%%%%%
% sum, product, union, intersections
%%%%%%%

\newcommand{\SUM}[2]{\underset{#1}{\overset{#2}{\sum}}}
\newcommand{\PROD}[2]{\underset{#1}{\overset{#2}{\prod}}}
\newcommand{\UNION}[2]{\underset{#1}{\overset{#2}{\bigcup}}}
\newcommand{\INTERSECT}[2]{\underset{#1}{\overset{#2}{\bigcap}}}
\newcommand{\FSUM}{\SUM{n=-\infty}{\infty}}
       

%%%%%%%
% supremum and infimum
%%%%%%%

\newcommand{\SUP}[1]{\underset{#1}\sup \,}
\newcommand{\INF}[1]{\underset{#1}\inf \,}
\newcommand{\MAX}[1]{\underset{#1}\max \,}
\newcommand{\MIN}[1]{\underset{#1}\min \,}

%%%%%%%
% infinite sums, limits
%%%%%%%

\newcommand{\SUMK}{\SUM{k=1}{\infty}}
\newcommand{\SUMN}{\SUM{n=1}{\infty}}
\newcommand{\SUMKZ}{\SUM{k=0}{\infty}}
\newcommand{\LIM}[1]{\underset{#1}\lim\,}
\newcommand{\IWOB}[1]{\LIM{#1 \to \infty}}
\newcommand{\LIMK}{\IWOB{k}}
\newcommand{\LIMN}{\IWOB{n}}
\newcommand{\LIMX}{\IWOB{x}}
\newcommand{\NIWOB}{\LIM{n \to \infty}}
\newcommand{\LIMSUPK}{\underset{k\to\infty}\limsup \,}
\newcommand{\LIMSUPN}{\underset{n\to\infty}\limsup \,}
\newcommand{\LIMINFK}{\underset{k\to\infty}\liminf \,}
\newcommand{\LIMINFN}{\underset{n\to\infty}\liminf \,}
\newcommand{\ROOTRULE}[1]{\LIMSUPK \BARS{#1}^{1/k}}

\newcommand{\CUPK}{\bigcup_{k=1}^{\infty}}
\newcommand{\CAPK}{\bigcap_{k=1}^{\infty}}
\newcommand{\CUPN}{\bigcup_{n=1}^{\infty}}
\newcommand{\CAPN}{\bigcap_{n=1}^{\infty}}

%%%%%%%
% number systems (real, rational, etc.)
%%%%%%%

\newcommand{\REALS}{\mathbb{R}}
\newcommand{\RATIONALS}{\mathbb{Q}}
\newcommand{\IRRATIONALS}{\REALS \backslash \RATIONALS}
\newcommand{\INTEGERS}{\mathbb{Z}}
\newcommand{\NUMBERS}{\mathbb{N}}
\newcommand{\COMPLEX}{\mathbb{C}}
\newcommand{\DISC}{\mathbb{D}}
\newcommand{\HPLANE}{\mathbb{H}}

\newcommand{\R}{\mathbb{R}}
\newcommand{\Q}{\mathbb{Q}}
\newcommand{\Z}{\mathbb{Z}}
\newcommand{\N}{\mathbb{N}}
\newcommand{\C}{\mathbb{C}}
\newcommand{\T}{\mathbb{T}}
\newcommand{\COUNTABLE}{\aleph_0}
\newcommand{\UNCOUNTABLE}{\aleph_1}


%%%%%%%
% Arithmetic/Algebraic operators
%%%%%%%


\DeclareMathOperator{\MOD}{mod}
%\newcommand{\MOD}[1]{\mod #1}
\newcommand{\BAR}[1]{\overline{#1}}
\newcommand{\LCM}{\text{ lcm}}
\newcommand{\ZMOD}[1]{\Z/#1\Z}
\DeclareMathOperator{\VAR}{Var}
%%%%%%%
% complex operators
%%%%%%%

\DeclareMathOperator{\RR}{Re}
%\newcommand{\RE}{\text{Re}}
\DeclareMathOperator{\IM}{Im}
%\newcommand{\IM}{\text{Im}}
\newcommand{\CONJ}[1]{\overline{#1}}
\DeclareMathOperator{\LOG}{Log}
%\newcommand{\LOG}{\text{ Log }}
\newcommand{\RES}[2]{\underset{#1}{\text{res}} #2}

%%%%%%%
% Group operators
%%%%%%%

\newcommand{\AUT}{\text{Aut}\,}
\newcommand{\KER}{\text{ker}\,}
\newcommand{\END}{\text{End}}
\newcommand{\HOM}{\text{Hom}}
\newcommand{\CYCLE}[1]{(\begin{array}{cccccccccc}
		#1
	\end{array})}
\newcommand{\SUBGROUP}{\underset{\text{group}}\subseteq}	
%\newcommand{\SUBGROUP}{\subseteq_g}
\newcommand{\SUBRING}{\underset{\text{ring}}\subseteq}
\newcommand{\SUBMOD}{\underset{\text{mod}}\subseteq}
\newcommand{\SUBFIELD}{\underset{\text{field}}\subseteq}
\newcommand{\ISO}{\underset{\text{iso}}\longrightarrow}
\newcommand{\HOMO}{\underset{\text{homo}}\longrightarrow}

%%%%%%%
% grouping (parenthesis, absolute value, square, multi-level brackets).
%%%%%%%

\newcommand{\PAREN}[1]{\left (\, #1 \,\right )}
\newcommand{\BRACKET}[1]{\left \{\, #1 \,\right \}}
\newcommand{\SQBRACKET}[1]{\left [\, #1 \,\right ]}
\newcommand{\ABRACKET}[1]{\left \langle\, #1 \,\right \rangle}
\newcommand{\BARS}[1]{\left |\, #1 \,\right |}
\newcommand{\DBARS}[1]{\left \| \, #1 \,\right \|}
\newcommand{\LBRACKET}[1]{\left \{ #1 \right .} 
\newcommand{\RBRACKET}[1]{\left . #1 \right \]}
\newcommand{\RBAR}[1]{\left . #1 \, \right |}
\newcommand{\LBAR}[1]{\left | \, #1 \right .}
\newcommand{\BLBRACKET}[2]{\BRACKET{\RBAR{#1}#2}}
\newcommand{\GEN}[1]{\ABRACKET{#1}}
\newcommand{\BINDEF}[2]{\LBRACKET{\begin{array}{ll}
     #1\\
     #2
\end{array}}}

%%%%%%%
% Fourier Analysis
%%%%%%%

\newcommand{\ONEOTWOPI}{\frac{1}{2\pi}}
\newcommand{\FHAT}{\hat{f}(n)}
\newcommand{\FINT}{\int_{-\pi}^\pi}
\newcommand{\FINTWO}{\int_{0}^{2\pi}}
\newcommand{\FSUMN}[1]{\SUM{n=-#1}{#1}}
%\newcommand{\FSUM}{\SUMN{\infty}}
\newcommand{\EIN}[1]{e^{in#1}}
\newcommand{\NEIN}[1]{e^{-in#1}}
\newcommand{\INTALL}{\int_{-\infty}^{\infty}}
\newcommand{\FTINT}[1]{\INTALL #1 e^{2\pi inx\xi} dx}
\newcommand{\GAUSS}{e^{-\pi x^2}}

%%%%%%%
% formatting 
%%%%%%%

\newcommand{\LEFTBOLD}[1]{\noindent\textbf{#1}}
\newcommand{\SEQ}[1]{\{#1\,\}}
\newcommand{\WIP}{\footnote{work in progress}}
\newcommand{\QED}{\hfill\square}
\newcommand{\ts}{\textsuperscript}
\newcommand{\HLINE}{\noindent\rule{7in}{1pt}\\}

%%%%%%%
% Mathematical note taking (definitions, theorems, etc.)
%%%%%%%

\newcommand{\REM}{\noindent\textbf{\\Remark: }}
\newcommand{\DEF}{\noindent\textbf{\\Definition: }}
\newcommand{\THE}{\noindent\textbf{\\Theorem: }}
\newcommand{\COR}{\noindent\textbf{\\Corollary: }}
\newcommand{\LEM}{\noindent\textbf{\\Lemma: }}
\newcommand{\PROP}{\noindent\textbf{\\Proposition: }}
\newcommand{\PROOF}{\noindent\textbf{\\Proof: }}
\newcommand{\EXP}{\noindent\textbf{\\Example: }}
\newcommand{\TRICKS}{\noindent\textbf{\\Tricks: }}


%%%%%%%
% text highlighting
%%%%%%%

\newcommand{\B}[1]{\textbf{#1}}
\newcommand{\CAL}[1]{\mathcal{#1}}
\newcommand{\UL}[1]{\underline{#1}}

%%%%%%
% Linear Algebra
%%%%%%

\newcommand{\COLVECTOR}[1]{\PAREN{\begin{array}{c}
#1
\end{array} }}
\newcommand{\TWOXTWO}[4]{\PAREN{ \begin{array}{c c} #1&#2 \\ #3 & #4 \end{array} }}
\newcommand{\DTWOXTWO}[4]{\BARS{ \begin{array}{c c} #1&#2 \\ #3 & #4 \end{array} }}
\newcommand{\THREEXTHREE}[9]{\PAREN{ \begin{array}{c c c} #1&#2&#3 \\ #4 & #5 & #6 \\ #7 & #8 & #9 \end{array} }}
\newcommand{\DTHREEXTHREE}[9]{\BARS{ \begin{array}{c c c} #1&#2&#3 \\ #4 & #5 & #6 \\ #7 & #8 & #9 \end{array} }}
\newcommand{\NXN}{\PAREN{ \begin{array}{c c c c} 
			a_{11} & a_{12} & \cdots & a_{1n} \\
			a_{21} & a_{22} & \cdots & a_{2n} \\
			\vdots & \vdots & \ddots & a_{1n} \\
			a_{n1} & a_{n2} & \cdots & a_{nn} \\
		\end{array} }}
\newcommand{\SLR}{SL_2(\R)}
\newcommand{\GLR}{GL_2(\R)}
\DeclareMathOperator{\TR}{tr}
\DeclareMathOperator{\BIL}{Bil}
\DeclareMathOperator{\SPAN}{span}

%%%%%%%
%  White space
%%%%%%%

\newcommand{\BOXIT}[1]{\noindent\fbox{\parbox{\textwidth}{#1}}}


\newtheorem{theorem}{Theorem}[section]
\newtheorem{corollary}{Corollary}[theorem]
\newtheorem{lemma}[theorem]{Lemma}

\theoremstyle{definition}
\newtheorem{definition}[theorem]{Definition}
\newtheorem{prop}[theorem]{Proposition}

\theoremstyle{remark}
\newtheorem{remark}[theorem]{Remark}
\newtheorem{example}[theorem]{Example}
%\newtheorem*{proof}[theorem]{Proof}



\newcommand{\RED}[1]{\textcolor{red}{#1}}
\newcommand{\BLUE}[1]{\textcolor{blue}{#1}}

\begin{document}

\begin{center}
	\Large{\CLASSNAME -- \SEMESTER} \\
	\large{ w/Professor Liu}
\end{center}
\begin{center}
	\STUDENTNAME \\
	\ASSIGNMENT -- \DUEDATE\\
\end{center} 

\begin{enumerate}[label=\Roman*.]
\item This problem review continuity for functions on real line.

We say a function $f: \R \to \R$ is \textit{continuous} at a point $a \in \R$ if for any $\epsilon > 0$, there is a $\delta > 0$ such that $|x-a| < \delta$ implies $|f(x)-f(a)|< \epsilon$.
\begin{enumerate}[label=(\alph*)]
\item Show that $f(x)=x^2$ is continuous at $x=2$.

\BLUE{Given an $\epsilon > 0$, when $|f(x)-4|< \epsilon$, $ |x^2 -4| < \epsilon$.  Let $\delta < \sqrt{\epsilon + 4}$ \\
If \begin{align*}
	(2+\delta)^2-4 &< \epsilon \\
	(2+\delta)^2&< \epsilon +4\\
	2+\delta < \sqrt{\epsilon +4} \\
	\delta < \sqrt{\epsilon+4}-2
\end{align*} }

\item Suppose that $f$ is continuous at $a$ and $f(a) \ne 0$. Show that $f$ is nonzero in some open interval containing $a$.

\BLUE{Since $f$ is continuous at $a$ and $f(a) \ne 0$ then for every $\epsilon > 0$ such that when $|f(x)-f(a)| < \epsilon$.  Without loss of generality, assume $f(a) > 0$.  Choose $\epsilon < f(a)$ then $0 < f(a)-\epsilon < f(x) < f(a) + \epsilon$.  Therefore, $f(x) \ne 0$ }

\end{enumerate}

\item This problem review derivatives.
\begin{enumerate}[label=(\alph*)]
\item Let $f(x)=x^n$ for some positive integer $n$.  Using the definition of the derivative, and the binomial theorem, show that $f'^{n-1}$.
\item Is the function 
\begin{align*}
	f(x) &= \BINDEF{x^2, & x \le 0,}{-x^2, & x \le 0}
\end{align*}differentiable at $x=0$.

\BLUE{
\begin{align*}
	\lim_{x \to 0^-} f(x) &= \lim_{x \to 0^-} x^2 = 0 \\
	\lim_{x \to 0^+} f(x) &= \lim_{x \to 0^+} x^2 = 0 \\
	\lim_{x \to 0^-} f(x) &= \lim_{x \to 0^+} f(x)
\end{align*}$f(x)$ is both continuous and differentiaable at $x=0$. }
\end{enumerate}

\newpage
\item This problem reviews $\sup$ and $\inf$.

For any subset $A \subset \R$, we say that $M$ is an \textit{upper bound} for $A$ if $x \le M$ for all $x \in A$.  If a set $A$ has a finite upper bound, we say it is \textit{boundared above}.  It is a theorem about the set $\R$ that \textit{for any set $A \subset \R$ that is bounded above, there exists a least (smallest) upper bound for $A$.}  This least upper bound is called supermum of $A$, and denoted $\sup A$.  By definition, the number $\sup A$ has two properties.
\begin{enumerate}[label = (\roman*)]
\item $x \le \sup A$ for all $x \in A$ (i.e., $\sup A$ is an upper bound for $M$).
\item for any $M$ that is an upper bound for $A$, we have $\sup A \le M$.
\end{enumerate}For sets that are not bounded above, we say that $\sup A = + \infty$.  we often write things like 
\begin{align*}
	\sup_{x \in A} f(x),
\end{align*}to denote the supremum of the set $\{ f(x): x \in A\}$, where $f$ is a some function.

Similarly, any set that is bounded below has a \textit{greatest lower bound} called the \textit{infimum}, denoted $\inf A$.  It satisfies the same properties as $\sup A$ with the inequalities reversed.
\begin{enumerate}[label=(\alph*)]

\item Find $\sup A$ and $\inf A$ for $A = (1,2]$, $A=\{1,\frac{1}{2},\frac{1}{3}, \cdots \}$, and $A = \{0,1,2,3,\dots\}$.

\begin{itemize}
\item $A = (1,2]$, \BLUE{ $\sup A = 2$ and $\inf A = 1$ }
\item $A=\{1,\frac{1}{2},\frac{1}{3}, \cdots \}$, \BLUE{ $\sup A = 1$, and $\inf A = \LIM{n\to \infty} \frac{1}{n} = 0$.}
\item $A = \{0,1,2,3,\dots\}$.  \BLUE{ $\sup A = \LIM{n\to \infty} = \infty$, and $\inf A = 0$}
\end{itemize}

\item Find $\SUP{x \in (0, 1)}(1+x^2)^{-1}$

\BLUE{ Let $f(x) = (1+x^2)^{-1}$.  On the interval $(0,1)$ we can see that it is strictly decreasing, that is $a<b \implies f(a)> f(b)$.  Thus, $\SUP{x \in (0, 1)}f(x) = f(0) = (1+0^2)^{-1} = 1$.}

\item Assume that $\sup A < \infty$, and show that for every $\epsilon > 0$, there exists $x \in A$ such that $x > \sup A - \epsilon$.

\BLUE{Given any $\epsilon > 0$ let $x > \sup A - \epsilon$.  If $x \not \in A$ then $x$ is an upper bound of $A$, i.e., $x \in M$ and $x< \sup A$, but that violates proper (ii).  Hence, $x \in A$.} 

\item For any two functions $f,g : \R \to \R$, and any set $A \subset \R$, show that $\SUP{x \in A}(f(x)+g(x)) \le \SUP{x\in A} f(x) + \SUP{x\in A} g(x)$.

\BLUE{
\begin{align*}
	f(x) &\le \SUP{x\in A} f(x) \AND g(x) \le \SUP{x\in A} g(x), \forall x \in {x \in A}\\
	\therefore f(x) + g(x) &\le \SUP{x\in A} f(x) + \SUP{x\in A} g(x), \forall x \in A \\
	\AND \SUP{x \in A}(f(x)+g(x)) &\le \SUP{x\in A} f(x) + \SUP{x\in A} g(x)
\end{align*} }

\end{enumerate}

\newpage
\item Section 1.1, Exercise 5, 6, 13.

\begin{description}
\item \textit{Exercise 1.1.5}. Let $n \ge 1$, and let $a_1, a_2, \dots, a_n$ and $b_1, b_2, \dots, b_n$ be real numbers verify the identity
\begin{align*}
	&\PAREN{\sum_{i=1}^na_ib_i}^2+\frac{1}{2}\sum_{i=1}^n\sum_{j=1}^n(a_ib_j-a_jb_i)^2= \PAREN{\sum_{i=1}^na_i^2}\PAREN{\sum_{j=1}^n b_j^2}, &(1.3)
\end{align*}and conclude \textit{Cauchy-Schwarz inequality}
\begin{align*}
	\BARS{\sum_{i=1}^n a_1b_i} \le \PAREN{\sum_{i=1}^na_i^2}^{1/2}\PAREN{\sum_{j=1}^nb_j^2}^{1/2}
\end{align*}Then use the Cauchy-Schwarz inequality to prove the \textit{triangle inequality}
\begin{align*}
	\PAREN{\sum_{i=1}^n(a_i+b_i)^2}^{1/2} \le \PAREN{\sum_{i=1}^n a_i^2}^{1/2}+\PAREN{\sum_{i=j}^n b_j^2}^{1/2}
\end{align*}

\BLUE{Let's start by expanding the center term\begin{align*}
	\sum_{i=1}^n\sum_{j=1}^n(a_ib_j-a_jb_i)^2 &= \sum_{i=1}^n\sum_{j=1}^n ( (a_ib_j)^2+(a_jb_i)^2 -(a_ib_ja_jb_i)^2)\\
	&= \sum_{i=1}^n a_i^2\sum_{j=1}^n b_j^2 + \sum_{i=1}^nb_i^2\sum_{j=1}^n a_i^2 - 2\sum_{i=1}^na_ib_i\sum_{j=1}^na_jb_j \\
	&= 2\PAREN{\sum_{i=1}^na_i^2}\PAREN{\sum_{j=1}^nb_j^2}-2\PAREN{\sum_{i=1}^na_ib_i}^2
\end{align*}Equation 1.3 then becomes
\begin{align*}
	\PAREN{\sum_{i=1}^na_ib_i}^2+\frac{1}{2}\sum_{i=1}^n\sum_{j=1}^n(a_ib_j-a_jb_i)^2= \PAREN{\sum_{i=1}^na_i^2}\PAREN{\sum_{j=1}^n b_j^2} \\
	\PAREN{\sum_{i=1}^na_ib_i}^2+\PAREN{\sum_{i=1}^na_i^2}\PAREN{\sum_{j=1}^nb_j^2}-\PAREN{\sum_{i=1}^na_ib_i}^2= \PAREN{\sum_{i=1}^na_i^2}\PAREN{\sum_{j=1}^n b_j^2}
\end{align*}which is true.  Since
\begin{align*}
	\PAREN{\sum_{i=1}^na_ib_i}^2+\frac{1}{2}\sum_{i=1}^n\sum_{j=1}^n(a_ib_j-a_jb_i)^2&= \PAREN{\sum_{i=1}^na_i^2}\PAREN{\sum_{j=1}^n b_j^2} \\
	\PAREN{\sum_{i=1}^na_ib_i}^2&=\PAREN{\sum_{i=1}^na_i^2}\PAREN{\sum_{j=1}^n b_j^2} - \frac{1}{2}\sum_{i=1}^n\sum_{j=1}^n(a_ib_j-a_jb_i)^2 \\
	\therefore \BARS{\sum_{i=1}^na_ib_i} &\le \PAREN{\sum_{i=1}^na_i^2}^{1/2}\PAREN{\sum_{j=1}^n b_j^2}^{1/2}
\end{align*}Let's start by taking the square of the distance from $a+b$ to zero using the $\ell^2$.
\begin{align*}
	d_{\ell^2}(a+b, 0)^2 &= \sum_{i=1}^n (a_i+b_i)^2 \\
	&= \sum_{i=1}^n (a_i^2 + b_i^2 + 2a_ib_i) \\
	&= \sum_{i=1}^n a_i^2 + \sum_{i=1}^nb_i^2 + 2\sum_{i=1}^na_ib_i 
\end{align*}apply Cauchy-Schwarz and factor.
\begin{align*}
	d_{\ell^2}(a+b, 0)^2 &\le d_{\ell^2}(a,0) + d_{\ell^2}(b, 0) + 2 \PAREN{\sum_{i=1}^na_i^2}^{1/2}\PAREN{\sum_{j=1}^n b_j^2}^{1/2} \\
	&\le d_{\ell^2}(a,0) + d_{\ell^2}(b, 0) + 2\PAREN{ d_{\ell^2}(a,0)\cdot d_{\ell^2}(b,0) }^{1/2} \\
	&\le \PAREN{ d_{\ell^2}(a,0)^{1/2} + d_{\ell^2}(b,0)^{1/2}}^2
\end{align*}Expand the $\ell^2$ metrics and take the square root of both sides and 
\begin{align*}
	\PAREN{\sum_{i=1}^n(a_i+b_i)^2}^{1/2} &\le \PAREN{\sum_{i=1}^n a_i^2}^{1/2}+\PAREN{\sum_{i=j}^n b_j^2}^{1/2}
\end{align*} }

\newpage
\item \textit{Exercise 1.1.6}  Show that $(\R^n, d_{l^2})$ in Example 1.1.6 is indeed a metric space. (Hint: use Exercise 1.1.5)\\
\textbf{Example 1.1.6} (Euclidean spaces). Let $n \ge 1$ be a natural number, and let $\R^n$ be the space of $n$-tupes of real numbers:
\begin{align*}
	\R^n = \{(x_1,x_2, \dots, x_n): x_1, \dots, x_n \in \R\}
\end{align*}We define the \textit{Euclidean metric} (also called the\textit{ $l^2$ metric}) $d_{l^2}: \R^n \times \R^n \to \R$ by
\begin{align*}
	d_{l^2} ((x_1, \dots, x_n), (y_1, \dots, y_n)) &= \sqrt{(x_1-y_1)^2+\cdots+ (x_n-y_n)^2}\\
	&= \PAREN{\sum_{i=1}^n (x_i-y_i)^2 }^{1/2}
\end{align*}

\BLUE{We must prove that $d_{\ell^2}$ is symmetric, positive definite and that the triangle inequality holds.
\begin{description}
	\item Symmetric: show that $d_{\ell^2}(x,y) = d_{\ell^2}(y,x)$.
		\begin{align*}
			d_{l^2} ((x_1, \dots, x_n), (y_1, \dots, y_n)) &= \PAREN{\sum_{i=1}^n (x_i-y_i)^2 }^{1/2} \\
			&= \PAREN{\sum_{i=1}^n (y_i-x_i)^2 }^{1/2} \\
			&= d_{l^2} ((y_1, \dots, y_n), (x_1, \dots, x_n))
		\end{align*}
	\item Positive Definite: show that $d_{\ell^2}(x,y)\ge 0$ and $d_{\ell^2}(x,y) = 0 \to x=y$.\\
	The square root is taken as a positive value.  $d_{l^2} ((x_1, \dots, x_n), (y_1, \dots, y_n))=0$ implies that each $x_i - y_i=0$ therefore $x=y$.
	\item Triangle Inequality: show that $d_{\ell^2}(x,z) \le d_{\ell^2}(x,y)+d_{\ell^2}(y,z)$\\
	Excercise 1.1.5 proves the triangulate inequality replacing $a_i = x_i$ and $b_i = y_i$.
\end{description}
}

\newpage
\item \textit{Exercise 1.1.13} Prove Proposition 1.1.19.\\
\textbf{Proposition 1.1.19} (Convergence in a the discrete metric).  \textit{Let $X$ be any set, and let $d_{\text{disc}}$ be the discrete metric on $X$.  Let $(x^{(n)})_{n=m}^\infty$ be a sequence of points in $X$, and let $x$ be a point in $X$. Then $(x^{(n)})_{n=m}^\infty$ convergent to $x$ with respect to the discrete metric $d_{\text{disc}}$ if and only if there exists $N \ge m$ such that $x^{(n)}=x$ for all $n \ge N$. }\\

\BLUE{
Remember that:
\begin{align*}
	d_{\text{disc}}(x,y)= \BINDEF{1 & x=y}{0 & x\ne y}
\end{align*}\begin{description}
	\item $(\Longrightarrow)$ assume that $x^{(n)} \to x$ under $d_{\text{disc}}$.
	Then, for any $\epsilon>0$ there exists $n\in \N$ such that $x^{(n)}-x < \epsilon$.  Clearly, $x^{(n)} - x$ can be equal to either 1 or 0.  Thus, $x^{(n)}-x=0$ or $x^{(n)} = x$ and hence true for all $n>N$.
	\item $(\Longleftarrow)$ assume that $\exists N>m$ such that when $n> N, x^{(n)}=x$.
	Given any $\epsilon > 0$ and $n>N$ we can see that $x^{(n)}-x=0<\epsilon$.  Therefore $x^{(n)}\to x$.
\end{description}
}

\end{description}

\newpage
\item For this problem only, you do not need to give proofs.  Just write the answers.

For each set, identify the boundary, interior, and closure of $A$, and say whether $A$ is open , closed, both or neither.  We are working in $\R^2$ with standard distance.  Unless othewise noted, the ambient space is $\R^2$.
\begin{enumerate}[label=(\alph*)]
\item $A = \{(x_1, x_2) \in \R^2 : x_1 < 1 \}$.

\BLUE{
Boundary: $\partial A = \{(x_1, x_2) \in \R^2 : x_1 = 1\}$\\
Interior: $A^o = \{(x_1, x_2) \in \R^2 : x_1 < 1\}$\\
Closure: $\CONJ{A} = \{(x_1, x_2) \in \R^2 : x_1 \le 1\}$
}

\item $A=\{(1/n, 2/n): n=1,2,3, \dots\}$ (Note: $(1/n, 2/n)$ is a vector in $\R^2$, not an open interval in $\R$.)

\BLUE{
Boundary: $\partial A=A$\\
Interior: $A^o = A$\\
Closure: $\CONJ{A} = A$
}

\item $A =\{x=(x_1,x_2) \in \R^2: x_1 >0, d(x,0) \le 1\}$, in the relative topology with respect to $Y = \{(x_1,x_2) \in \R^2:x_1 > 0 \}$.

\BLUE{
Boundary: $\partial_Y A = \{x=(x_1,x_2) \in \R^2:  x_1=0, d(x,0) = 1\}$ the right semi-circle combined with the y-axis from 1 to -1.\\
Interior: $A^o = \{x=(x_1,x_2) \in \R^2: x_1 >0, d(x,0) < 1\}$\\
Closure: $\CONJ{A} = \{x=(x_1,x_2) \in \R^2: x_1 \ge 0, d(x,0) \le 1\}$
}

\end{enumerate}

\item Let $(X,d)$ be a metric space.
\begin{enumerate}[label=(\alph*)]
\item For a given point $x_0 \in X$, show the singleton set $\{x_0\}$ is closed.
\item Let $x_0 \in X$ and $r > 0$.  Show that the ball 
\begin{align*}
	B(x_0,r) = \{x \in X: d(x,x_0) < r\}
\end{align*}is open.
\end{enumerate}

\end{enumerate}

\end{document}