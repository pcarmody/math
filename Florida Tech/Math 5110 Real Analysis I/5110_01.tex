\documentclass[10pt,a4paper]{report}
\usepackage[utf8]{inputenc}
\usepackage{amsmath}
\usepackage{amsfonts}
\usepackage{amssymb}
\usepackage{amsthm}
\usepackage{hyperref}

\usepackage{multicol}
\usepackage{fancyhdr}
\usepackage[inline]{enumitem}
\usepackage{tikz}
\usepackage{tikz-cd}
\usetikzlibrary{calc}
\usetikzlibrary{shapes.geometric}
\usepackage[margin=0.5in]{geometry}
\usepackage{xcolor}

\hypersetup{
    colorlinks=true,
    linkcolor=blue,
    filecolor=magenta,      
    urlcolor=cyan,
    pdftitle={Tensors},
    pdfpagemode=FullScreen,
    }

%\urlstyle{same}

\newcommand{\CLASSNAME}{Math 5102 -- Real Analysis I}
\newcommand{\STUDENTNAME}{Paul Carmody}
\newcommand{\ASSIGNMENT}{Homework \#1 }
\newcommand{\DUEDATE}{September 2, 2024}
\newcommand{\SEMESTER}{Fall 2024}
\newcommand{\SCHEDULE}{MW 11:00 -- 12:15}
\newcommand{\ROOM}{Remote}

\newcommand{\MMN}{M_{m\times n}}
\newcommand{\FF}{\mathcal{F}}

\pagestyle{fancy}
\fancyhf{}
\chead{ \fancyplain{}{\CLASSNAME} }
%\chead{ \fancyplain{}{\STUDENTNAME} }
\rhead{\thepage}
\newcommand{\LET}{\text{Let }}
%\newcommand{\IF}{\text{if }}
\newcommand{\AND}{\text{ and }}
\newcommand{\OR}{\text{ or }}
\newcommand{\FORSOME}{\text{ for some }}
\newcommand{\FORALL}{\text{ for all }}
\newcommand{\WHERE}{\text{ where }}
\newcommand{\WTS}{\text{ WTS }}
\newcommand{\WLOG}{\text{ WLOG }}
\newcommand{\BS}{\backslash}
\newcommand{\DEFINE}[1]{\textbf{\emph{#1}}}
\newcommand{\IF}{$(\Rightarrow)$}
\newcommand{\ONLYIF}{$(\Leftarrow)$}
\newcommand{\ITH}{\textsuperscript{th} }
\newcommand{\FST}{\textsuperscript{st} }
\newcommand{\SND}{\textsuperscript{nd} }
\newcommand{\TRD}{\textsuperscript{rd} }
\newcommand{\INV}{\textsuperscript{-1} }

\newcommand{\XXX}{\mathfrak{X}}
\newcommand{\MMM}{\mathfrak{M}}
%\newcommand{\????}{\textfrak{A}}
%\newcommand{\????}{\textgoth{A}}
%\newcommand{\????}{\textswab{A}}

\DeclareMathOperator{\DER}{Der}
\DeclareMathOperator{\SGN}{sgn}

%%%%%%%
% derivatives
%%%%%%%

\newcommand{\PART}[2]{\frac{\partial #1}{\partial #2}}
\newcommand{\SPART}[2]{\frac{\partial^2 #1}{\partial #2^2}}
\newcommand{\DERIV}[2]{\frac{d #1}{d #2}}
\newcommand{\LAPLACIAN}[1]{\frac{\partial^2 #1}{\partial x^2} + \frac{\partial^2 #1}{\partial y^2}}

%%%%%%%
% sum, product, union, intersections
%%%%%%%

\newcommand{\SUM}[2]{\underset{#1}{\overset{#2}{\sum}}}
\newcommand{\PROD}[2]{\underset{#1}{\overset{#2}{\prod}}}
\newcommand{\UNION}[2]{\underset{#1}{\overset{#2}{\bigcup}}}
\newcommand{\INTERSECT}[2]{\underset{#1}{\overset{#2}{\bigcap}}}
\newcommand{\FSUM}{\SUM{n=-\infty}{\infty}}
       

%%%%%%%
% supremum and infimum
%%%%%%%

\newcommand{\SUP}[1]{\underset{#1}\sup \,}
\newcommand{\INF}[1]{\underset{#1}\inf \,}
\newcommand{\MAX}[1]{\underset{#1}\max \,}
\newcommand{\MIN}[1]{\underset{#1}\min \,}

%%%%%%%
% infinite sums, limits
%%%%%%%

\newcommand{\SUMK}{\SUM{k=1}{\infty}}
\newcommand{\SUMN}{\SUM{n=1}{\infty}}
\newcommand{\SUMKZ}{\SUM{k=0}{\infty}}
\newcommand{\LIM}[1]{\underset{#1}\lim\,}
\newcommand{\IWOB}[1]{\LIM{#1 \to \infty}}
\newcommand{\LIMK}{\IWOB{k}}
\newcommand{\LIMN}{\IWOB{n}}
\newcommand{\LIMX}{\IWOB{x}}
\newcommand{\NIWOB}{\LIM{n \to \infty}}
\newcommand{\LIMSUPK}{\underset{k\to\infty}\limsup \,}
\newcommand{\LIMSUPN}{\underset{n\to\infty}\limsup \,}
\newcommand{\LIMINFK}{\underset{k\to\infty}\liminf \,}
\newcommand{\LIMINFN}{\underset{n\to\infty}\liminf \,}
\newcommand{\ROOTRULE}[1]{\LIMSUPK \BARS{#1}^{1/k}}

\newcommand{\CUPK}{\bigcup_{k=1}^{\infty}}
\newcommand{\CAPK}{\bigcap_{k=1}^{\infty}}
\newcommand{\CUPN}{\bigcup_{n=1}^{\infty}}
\newcommand{\CAPN}{\bigcap_{n=1}^{\infty}}

%%%%%%%
% number systems (real, rational, etc.)
%%%%%%%

\newcommand{\REALS}{\mathbb{R}}
\newcommand{\RATIONALS}{\mathbb{Q}}
\newcommand{\IRRATIONALS}{\REALS \backslash \RATIONALS}
\newcommand{\INTEGERS}{\mathbb{Z}}
\newcommand{\NUMBERS}{\mathbb{N}}
\newcommand{\COMPLEX}{\mathbb{C}}
\newcommand{\DISC}{\mathbb{D}}
\newcommand{\HPLANE}{\mathbb{H}}

\newcommand{\R}{\mathbb{R}}
\newcommand{\Q}{\mathbb{Q}}
\newcommand{\Z}{\mathbb{Z}}
\newcommand{\N}{\mathbb{N}}
\newcommand{\C}{\mathbb{C}}
\newcommand{\T}{\mathbb{T}}
\newcommand{\COUNTABLE}{\aleph_0}
\newcommand{\UNCOUNTABLE}{\aleph_1}


%%%%%%%
% Arithmetic/Algebraic operators
%%%%%%%


\DeclareMathOperator{\MOD}{mod}
%\newcommand{\MOD}[1]{\mod #1}
\newcommand{\BAR}[1]{\overline{#1}}
\newcommand{\LCM}{\text{ lcm}}
\newcommand{\ZMOD}[1]{\Z/#1\Z}
\DeclareMathOperator{\VAR}{Var}
%%%%%%%
% complex operators
%%%%%%%

\DeclareMathOperator{\RR}{Re}
%\newcommand{\RE}{\text{Re}}
\DeclareMathOperator{\IM}{Im}
%\newcommand{\IM}{\text{Im}}
\newcommand{\CONJ}[1]{\overline{#1}}
\DeclareMathOperator{\LOG}{Log}
%\newcommand{\LOG}{\text{ Log }}
\newcommand{\RES}[2]{\underset{#1}{\text{res}} #2}

%%%%%%%
% Group operators
%%%%%%%

\newcommand{\AUT}{\text{Aut}\,}
\newcommand{\KER}{\text{ker}\,}
\newcommand{\END}{\text{End}}
\newcommand{\HOM}{\text{Hom}}
\newcommand{\CYCLE}[1]{(\begin{array}{cccccccccc}
		#1
	\end{array})}
\newcommand{\SUBGROUP}{\underset{\text{group}}\subseteq}	
%\newcommand{\SUBGROUP}{\subseteq_g}
\newcommand{\SUBRING}{\underset{\text{ring}}\subseteq}
\newcommand{\SUBMOD}{\underset{\text{mod}}\subseteq}
\newcommand{\SUBFIELD}{\underset{\text{field}}\subseteq}
\newcommand{\ISO}{\underset{\text{iso}}\longrightarrow}
\newcommand{\HOMO}{\underset{\text{homo}}\longrightarrow}

%%%%%%%
% grouping (parenthesis, absolute value, square, multi-level brackets).
%%%%%%%

\newcommand{\PAREN}[1]{\left (\, #1 \,\right )}
\newcommand{\BRACKET}[1]{\left \{\, #1 \,\right \}}
\newcommand{\SQBRACKET}[1]{\left [\, #1 \,\right ]}
\newcommand{\ABRACKET}[1]{\left \langle\, #1 \,\right \rangle}
\newcommand{\BARS}[1]{\left |\, #1 \,\right |}
\newcommand{\DBARS}[1]{\left \| \, #1 \,\right \|}
\newcommand{\LBRACKET}[1]{\left \{ #1 \right .} 
\newcommand{\RBRACKET}[1]{\left . #1 \right \]}
\newcommand{\RBAR}[1]{\left . #1 \, \right |}
\newcommand{\LBAR}[1]{\left | \, #1 \right .}
\newcommand{\BLBRACKET}[2]{\BRACKET{\RBAR{#1}#2}}
\newcommand{\GEN}[1]{\ABRACKET{#1}}
\newcommand{\BINDEF}[2]{\LBRACKET{\begin{array}{ll}
     #1\\
     #2
\end{array}}}

%%%%%%%
% Fourier Analysis
%%%%%%%

\newcommand{\ONEOTWOPI}{\frac{1}{2\pi}}
\newcommand{\FHAT}{\hat{f}(n)}
\newcommand{\FINT}{\int_{-\pi}^\pi}
\newcommand{\FINTWO}{\int_{0}^{2\pi}}
\newcommand{\FSUMN}[1]{\SUM{n=-#1}{#1}}
%\newcommand{\FSUM}{\SUMN{\infty}}
\newcommand{\EIN}[1]{e^{in#1}}
\newcommand{\NEIN}[1]{e^{-in#1}}
\newcommand{\INTALL}{\int_{-\infty}^{\infty}}
\newcommand{\FTINT}[1]{\INTALL #1 e^{2\pi inx\xi} dx}
\newcommand{\GAUSS}{e^{-\pi x^2}}

%%%%%%%
% formatting 
%%%%%%%

\newcommand{\LEFTBOLD}[1]{\noindent\textbf{#1}}
\newcommand{\SEQ}[1]{\{#1\,\}}
\newcommand{\WIP}{\footnote{work in progress}}
\newcommand{\QED}{\hfill\square}
\newcommand{\ts}{\textsuperscript}
\newcommand{\HLINE}{\noindent\rule{7in}{1pt}\\}

%%%%%%%
% Mathematical note taking (definitions, theorems, etc.)
%%%%%%%

\newcommand{\REM}{\noindent\textbf{\\Remark: }}
\newcommand{\DEF}{\noindent\textbf{\\Definition: }}
\newcommand{\THE}{\noindent\textbf{\\Theorem: }}
\newcommand{\COR}{\noindent\textbf{\\Corollary: }}
\newcommand{\LEM}{\noindent\textbf{\\Lemma: }}
\newcommand{\PROP}{\noindent\textbf{\\Proposition: }}
\newcommand{\PROOF}{\noindent\textbf{\\Proof: }}
\newcommand{\EXP}{\noindent\textbf{\\Example: }}
\newcommand{\TRICKS}{\noindent\textbf{\\Tricks: }}


%%%%%%%
% text highlighting
%%%%%%%

\newcommand{\B}[1]{\textbf{#1}}
\newcommand{\CAL}[1]{\mathcal{#1}}
\newcommand{\UL}[1]{\underline{#1}}

%%%%%%
% Linear Algebra
%%%%%%

\newcommand{\COLVECTOR}[1]{\PAREN{\begin{array}{c}
#1
\end{array} }}
\newcommand{\TWOXTWO}[4]{\PAREN{ \begin{array}{c c} #1&#2 \\ #3 & #4 \end{array} }}
\newcommand{\DTWOXTWO}[4]{\BARS{ \begin{array}{c c} #1&#2 \\ #3 & #4 \end{array} }}
\newcommand{\THREEXTHREE}[9]{\PAREN{ \begin{array}{c c c} #1&#2&#3 \\ #4 & #5 & #6 \\ #7 & #8 & #9 \end{array} }}
\newcommand{\DTHREEXTHREE}[9]{\BARS{ \begin{array}{c c c} #1&#2&#3 \\ #4 & #5 & #6 \\ #7 & #8 & #9 \end{array} }}
\newcommand{\NXN}{\PAREN{ \begin{array}{c c c c} 
			a_{11} & a_{12} & \cdots & a_{1n} \\
			a_{21} & a_{22} & \cdots & a_{2n} \\
			\vdots & \vdots & \ddots & a_{1n} \\
			a_{n1} & a_{n2} & \cdots & a_{nn} \\
		\end{array} }}
\newcommand{\SLR}{SL_2(\R)}
\newcommand{\GLR}{GL_2(\R)}
\DeclareMathOperator{\TR}{tr}
\DeclareMathOperator{\BIL}{Bil}
\DeclareMathOperator{\SPAN}{span}

%%%%%%%
%  White space
%%%%%%%

\newcommand{\BOXIT}[1]{\noindent\fbox{\parbox{\textwidth}{#1}}}


\newtheorem{theorem}{Theorem}[section]
\newtheorem{corollary}{Corollary}[theorem]
\newtheorem{lemma}[theorem]{Lemma}

\theoremstyle{definition}
\newtheorem{definition}[theorem]{Definition}
\newtheorem{prop}[theorem]{Proposition}

\theoremstyle{remark}
\newtheorem{remark}[theorem]{Remark}
\newtheorem{example}[theorem]{Example}
%\newtheorem*{proof}[theorem]{Proof}



\newcommand{\RED}[1]{\textcolor{red}{#1}}
\newcommand{\BLUE}[1]{\textcolor{blue}{#1}}

\begin{document}

\begin{center}
	\Large{\CLASSNAME -- \SEMESTER} \\
	\large{ w/Professor Liu}
\end{center}
\begin{center}
	\STUDENTNAME \\
	\ASSIGNMENT -- \DUEDATE\\
\end{center} 
Page \#14: 7, 10, 11, 20, 21\\
\begin{enumerate}
	\setcounter{enumi}{6}
	\item Let $S = \{0,1\}$ and $F=R$.  If $\FF(S,R)$, show that $f=g$ and $f+g=h$, where $f(t) = 2t+1, g(t)=1+4t-2t^2$, and $h(t) = 5^t+1$.
	\begin{itemize}
		\item $f=g$
		
		$f(0)= 2(0)+1=1$ and $g(0)1+4(0)-2(0)^2=1 \to f(0) = g(0)$\\
		$f(1)= 2(1)+1=3$ and $g(1)1+4(1)-2(1)^2=1+4-2=3 \to f(1) = g(1)$\\
		$f=g$ since $f(s)=g(s)$ for all $s \in S$
		
		\item $f+g = h$
		
		\begin{align*}
			(f+g)(t) &= f(t)+g(t)\\
			&=2t+1+1+4t-2t^2 \\
			&= 2+6t-2t^2 \\
			(f+g)(0) &= 2+6(0)-2(0)^2 = 2 \\
			(f+g)(1) &= 2+6(1)-2(1)^2 = 6 \\ \\
			h(0) &= 5^0+1 = 2 \\
			h(1) &= 5^1+1 = 6 \\ \\
			(f+g)(0) &= h(0) \AND (f+g)(1) = h(1) \\ \\
			\therefore f+g &= h \text{ since } (f+g)(t) = h(t), \forall t \in S
		\end{align*}
	\end{itemize}
	
	\setcounter{enumi}{9}
	\item Let $V$ denote the set of all differentiable real-valued functions defined on the real line.  Prove that $V$ is a vector space with the operations of additons and scalar multiplication defined in Example 3.
	
	From Example 3 we have
	\begin{align*}
		(f+g)(s) = f(s)+g(s)\AND (cf)(s)= c\SQBRACKET{f(s)}, \forall s \in \R
	\end{align*}  From the additive rule of differentiation $(f+g)'(s) = f'(s)+g'(s)$ which implies that $(f+g) \in V$ and from the multiplicative rule of differentiation we know that $(cf)'(s)=c[f'(s)]$ which implies that $cf \in V$.  Hence $V$ is a vector space.
	
	\item Let $V=\{0\}$ consist of a single vector $0$ and define $0+0=0$ and $c0=0$ for each scalar $c$ in $F$.  Prove that $V$ is a vector space over $F$.  ($V$ is called the \textbf{zero vector space}).
	
	Given any $v,w \in V$ we have $v+w=0+0=0 \in V$ and for all $c\in F, cv=c(0)=0 \in V$.  Therefore $V$ is a vector space.
	\setcounter{enumi}{19}
	\item Let $V$ be the set of sequences $\{a_n\}$ of real numbers.  (See Example 5 for the definition of a sequence.)  For $\{a_n\},\{b_n\} \in V$ and any real number $t$, define
	\begin{align*}
		\{a_n\}+\{b_n\} = \{a_n+b _n\} \text{ and } t\{a_n\} = \{ta_n\}.
	\end{align*}Prove that, with these operations, $V$ is a vector space. over $\R$.
	
	Given any two real sequences $\{a_n\},\{b_n\} \in V$, we can see that the sequence $\{a_n + b_n\}$ is made up of real numbers, namely $a_i+b_i$ for all $i\in \N$, making it a real sequence.  We can also see that $\{c a_n\}$ is made up of real numbers, namely $ca_i$ for all $i \in \N$, making it a real sequence.  Hence, $V$ is a vector space.
	
	\item Let $V$ and $W$ be vector spaces over a field $F$.  Let 
	\begin{align*}
		Z = \BRACKET{\PAREN{v,w}: v \in V \AND w \in W}
	\end{align*}Prove that $Z$ is a vector space over $F$ with the operations
	\begin{align*}
	\PAREN{v_1, w_1}+\PAREN{v_2, w_2} = \PAREN{v_1+v_2, w_1+w_2} \AND c\PAREN{v_1, w_1}=\PAREN{cv_1.cw_1}
	\end{align*}
	
	Given $v_1, v_2 \in V$ and $w_1, w_2 \in W$ we can see that by definition $z_1=(v_1, w_1), z_2=(v_2, w_2) \in Z$.  We want to show that $z_1 + z_2 \in Z$ and $cz_1 \in Z$ for all $c \in F$.  \\
	Well, $z_1+z_2 = \PAREN{v_1, w_1}+\PAREN{v_2, w_2} = \PAREN{v_1+v_2, w_1+w_2}$ and since $V,W$ are vector spaces we know that $v_1+v_2\in V$ and $w_1+w_2 \in W$. Therefore, $z_1+z_2 \in Z$. \\
	Additionally, $cz_1 = c(v_1, w_1) = (cv_1, cw_1)$ and we know that $cv_1 \in V$ and $cw_1 \in W$ therefore $cz_1 \in Z$.\\
	Hence, $Z$ is a vector space.
\end{enumerate}
\newpage

Page \#20: 3, 4, 5, 10, 15, 19
\begin{enumerate}
	\setcounter{enumi}{2}
	\item Prove that $(aA+bB)^t = aA^t+bB^t$ for any $A,B\in \MMN(F)$ and any $a,b \in F$.
	
	\begin{align*}
		A^t &\implies [A_{ij}]^t = [A_{ji}] \\
		(aA+bB)_t &= \SQBRACKET{(aA + bB)_{ji}} \\
		 &= \SQBRACKET{ aA_{ji}+bB_{ji} } \\
		 &= a\SQBRACKET{A_{ji}}+b\SQBRACKET{B_{ji}} \\
		 &= aA^t + bB^t
	\end{align*}
	
	\item Prove that $(A^t)^t=A$ for each $A \in \MMN(F)$.
	
	Let $B = A^t = [A_{ji}]$ and $B^t = [A_{ji}]^t = [A_{ij}] = A$.
	
	\item Prove that $A+A^t$ is symmetric for any square matrix $A$.
	
	$A = [A_{ij}]$ and $A^t = [A_{ji}]$.  Therefore $A+A^t = [A_{ij}]+[A_{ji}]=[A_{ij}+A_{ji}$.  Given any $i, j in [1, \dots, n]$ we can see that $(A+A^t)_{ij}=A_{ij}+A_{ji}=A_{ji}+A_{ij}=(A+A^t)_{ji}$, hence symmetric.
	
	\setcounter{enumi}{9}
	\item Prove that $W_1= \BRACKET{\PAREN{a_1,a_2,\dots, a_n} \in F^n: a_1+a_2+\cdots+a_n=0}$ is a subspace of $F^n$, but $W_2= \BRACKET{\PAREN{a_1,a_2,\dots, a_n} \in F^n: a_1+a_2+\cdots+a_n=1}$ is not.
	
	Given $a=\PAREN{a_1,a_2,\dots, a_n},b=\PAREN{b_1,b_2,\dots, b_n} \in W_1$ we can see that $a+b = \PAREN{a_1+b_1, a_2+b_2, \dots, a_n+b_n}$ and $a_1+b_1+a_2+b_2+\cdots_n+b_n = a_1+a_2+\cdots+a_n+b_1+b_2+\cdots+b_n=0+0=0$ therefore $a+b \in W_1$.  Similarly, $ca = c\PAREN{a_1,a_2,\dots, a_n}= \PAREN{ca_1,ca_2,\dots, ca_n}$ and $ca_1+ca_2+\cdots+ca_n= c(a_1+a_2+\cdots+a_n)=c(0)=0.$ Therefore $W_1$ is a vector space. \\
	However, let $a=\PAREN{a_1,a_2,\dots, a_n},b=\PAREN{b_1,b_2,\dots, b_n} \in W_2$ we can see that $a+b = \PAREN{a_1+b_1, a_2+b_2, \dots, a_n+b_n}$ and $a_1+b_1+a_2+b_2+\cdots_n+b_n = a_1+a_2+\cdots+a_n+b_1+b_2+\cdots+b_n=1+1=2$ which means that $a+b \not \in W_2$.
	\setcounter{enumi}{14}
	\item  Is the set of all differentiable real-valued functions defined on $\R$ a subspace of $C(\R)$?  Justify your answer.
	\setcounter{enumi}{18}
	\item Let $W_1$ and $W_2$ be subspaces of a vector space $V$.  Prove that $W_1 \bigcup W_2$ is a subspace of $V$ if and only if $W_1 \subseteq W_2$ or $W_2 \subseteq W_1$.
	
\end{enumerate}
\end{document}