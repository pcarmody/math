\documentclass[10pt,a4paper]{report}
\usepackage[utf8]{inputenc}
\usepackage{amsmath}
\usepackage{amsfonts}
\usepackage{amssymb}
\usepackage{amsthm}
\usepackage{hyperref}

\usepackage{multicol}
\usepackage{fancyhdr}
\usepackage[inline]{enumitem}
\usepackage{tikz}
\usepackage{tikz-cd}
\usetikzlibrary{calc}
\usetikzlibrary{shapes.geometric}
\usepackage[margin=0.5in]{geometry}
\usepackage{xcolor}

\hypersetup{
    colorlinks=true,
    linkcolor=blue,
    filecolor=magenta,      
    urlcolor=cyan,
    pdftitle={Tensors},
    pdfpagemode=FullScreen,
    }

%\urlstyle{same}

\newcommand{\CLASSNAME}{Math 5102 -- Linear Algebra}
\newcommand{\STUDENTNAME}{Paul Carmody}
\newcommand{\ASSIGNMENT}{Homework \#7 }
\newcommand{\DUEDATE}{NONE}
\newcommand{\SEMESTER}{Fall 2024}
\newcommand{\SCHEDULE}{MW 11:00 -- 12:15}
\newcommand{\ROOM}{Remote}

\newcommand{\MMN}{M_{m\times n}}
\newcommand{\FF}{\mathcal{F}}

\pagestyle{fancy}
\fancyhf{}
\chead{ \fancyplain{}{\CLASSNAME} }
%\chead{ \fancyplain{}{\STUDENTNAME} }
\rhead{\thepage}
\newcommand{\LET}{\text{Let }}
%\newcommand{\IF}{\text{if }}
\newcommand{\AND}{\text{ and }}
\newcommand{\OR}{\text{ or }}
\newcommand{\FORSOME}{\text{ for some }}
\newcommand{\FORALL}{\text{ for all }}
\newcommand{\WHERE}{\text{ where }}
\newcommand{\WTS}{\text{ WTS }}
\newcommand{\WLOG}{\text{ WLOG }}
\newcommand{\BS}{\backslash}
\newcommand{\DEFINE}[1]{\textbf{\emph{#1}}}
\newcommand{\IF}{$(\Rightarrow)$}
\newcommand{\ONLYIF}{$(\Leftarrow)$}
\newcommand{\ITH}{\textsuperscript{th} }
\newcommand{\FST}{\textsuperscript{st} }
\newcommand{\SND}{\textsuperscript{nd} }
\newcommand{\TRD}{\textsuperscript{rd} }
\newcommand{\INV}{\textsuperscript{-1} }


%%%%%%%
% derivatives
%%%%%%%

\newcommand{\PART}[2]{\frac{\partial #1}{\partial #2}}
\newcommand{\SPART}[2]{\frac{\partial^2 #1}{\partial #2^2}}
\newcommand{\DERIV}[2]{\frac{d #1}{d #2}}
\newcommand{\LAPLACIAN}[1]{\frac{\partial^2 #1}{\partial x^2} + \frac{\partial^2 #1}{\partial y^2}}

%%%%%%%
% sum, product, union, intersections
%%%%%%%

\newcommand{\SUM}[2]{\underset{#1}{\overset{#2}{\sum}}}
\newcommand{\PROD}[2]{\underset{#1}{\overset{#2}{\prod}}}
\newcommand{\UNION}[2]{\underset{#1}{\overset{#2}{\bigcup}}}
\newcommand{\INTERSECT}[2]{\underset{#1}{\overset{#2}{\bigcap}}}
\newcommand{\FSUM}{\SUM{n=-\infty}{\infty}}
       

%%%%%%%
% supremum and infimum
%%%%%%%

\newcommand{\SUP}[1]{\underset{#1}\sup \,}
\newcommand{\INF}[1]{\underset{#1}\inf \,}
\newcommand{\MAX}[1]{\underset{#1}\max \,}
\newcommand{\MIN}[1]{\underset{#1}\min \,}

%%%%%%%
% infinite sums, limits
%%%%%%%

\newcommand{\SUMK}{\SUM{k=1}{\infty}}
\newcommand{\SUMN}{\SUM{n=1}{\infty}}
\newcommand{\SUMKZ}{\SUM{k=0}{\infty}}
\newcommand{\LIM}[1]{\underset{#1}\lim\,}
\newcommand{\IWOB}[1]{\LIM{#1 \to \infty}}
\newcommand{\LIMK}{\IWOB{k}}
\newcommand{\LIMN}{\IWOB{n}}
\newcommand{\LIMX}{\IWOB{x}}
\newcommand{\NIWOB}{\LIM{n \to \infty}}
\newcommand{\LIMSUPK}{\underset{k\to\infty}\limsup \,}
\newcommand{\LIMSUPN}{\underset{n\to\infty}\limsup \,}
\newcommand{\LIMINFK}{\underset{k\to\infty}\liminf \,}
\newcommand{\LIMINFN}{\underset{n\to\infty}\liminf \,}
\newcommand{\ROOTRULE}[1]{\LIMSUPK \BARS{#1}^{1/k}}

\newcommand{\CUPK}{\bigcup_{k=1}^{\infty}}
\newcommand{\CAPK}{\bigcap_{k=1}^{\infty}}
\newcommand{\CUPN}{\bigcup_{n=1}^{\infty}}
\newcommand{\CAPN}{\bigcap_{n=1}^{\infty}}

%%%%%%%
% number systems (real, rational, etc.)
%%%%%%%

\newcommand{\REALS}{\mathbb{R}}
\newcommand{\RATIONALS}{\mathbb{Q}}
\newcommand{\IRRATIONALS}{\REALS \backslash \RATIONALS}
\newcommand{\INTEGERS}{\mathbb{Z}}
\newcommand{\NUMBERS}{\mathbb{N}}
\newcommand{\COMPLEX}{\mathbb{C}}
\newcommand{\DISC}{\mathbb{D}}
\newcommand{\HPLANE}{\mathbb{H}}

\newcommand{\R}{\mathbb{R}}
\newcommand{\Q}{\mathbb{Q}}
\newcommand{\Z}{\mathbb{Z}}
\newcommand{\N}{\mathbb{N}}
\newcommand{\C}{\mathbb{C}}
\newcommand{\T}{\mathbb{T}}
\newcommand{\COUNTABLE}{\aleph_0}
\newcommand{\UNCOUNTABLE}{\aleph_1}


%%%%%%%
% Arithmetic/Algebraic operators
%%%%%%%


\DeclareMathOperator{\MOD}{mod}
%\newcommand{\MOD}[1]{\mod #1}
\newcommand{\BAR}[1]{\overline{#1}}
\newcommand{\LCM}{\text{ lcm}}
\newcommand{\ZMOD}[1]{\Z/#1\Z}
\DeclareMathOperator{\VAR}{Var}
%%%%%%%
% complex operators
%%%%%%%

\DeclareMathOperator{\RR}{Re}
%\newcommand{\RE}{\text{Re}}
\DeclareMathOperator{\IM}{Im}
%\newcommand{\IM}{\text{Im}}
\newcommand{\CONJ}[1]{\overline{#1}}
\DeclareMathOperator{\LOG}{Log}
%\newcommand{\LOG}{\text{ Log }}
\newcommand{\RES}[2]{\underset{#1}{\text{res}} #2}

%%%%%%%
% Group operators
%%%%%%%

\newcommand{\AUT}{\text{Aut}\,}
\newcommand{\KER}{\text{ker}\,}
\newcommand{\END}{\text{End}}
\newcommand{\HOM}{\text{Hom}}
\newcommand{\CYCLE}[1]{(\begin{array}{cccccccccc}
		#1
	\end{array})}
\newcommand{\SUBGROUP}{\underset{\text{group}}\subseteq}	
%\newcommand{\SUBGROUP}{\subseteq_g}
\newcommand{\SUBRING}{\underset{\text{ring}}\subseteq}
\newcommand{\SUBMOD}{\underset{\text{mod}}\subseteq}
\newcommand{\SUBFIELD}{\underset{\text{field}}\subseteq}
\newcommand{\ISO}{\underset{\text{iso}}\longrightarrow}
\newcommand{\HOMO}{\underset{\text{homo}}\longrightarrow}

%%%%%%%
% grouping (parenthesis, absolute value, square, multi-level brackets).
%%%%%%%

\newcommand{\PAREN}[1]{\left (\, #1 \,\right )}
\newcommand{\BRACKET}[1]{\left \{\, #1 \,\right \}}
\newcommand{\SQBRACKET}[1]{\left [\, #1 \,\right ]}
\newcommand{\ABRACKET}[1]{\left \langle\, #1 \,\right \rangle}
\newcommand{\BARS}[1]{\left |\, #1 \,\right |}
\newcommand{\DBARS}[1]{\left \| \, #1 \,\right \|}
\newcommand{\LBRACKET}[1]{\left \{ #1 \right .} 
\newcommand{\RBRACKET}[1]{\left . #1 \right \]}
\newcommand{\RBAR}[1]{\left . #1 \, \right |}
\newcommand{\LBAR}[1]{\left | \, #1 \right .}
\newcommand{\BLBRACKET}[2]{\BRACKET{\RBAR{#1}#2}}
\newcommand{\GEN}[1]{\ABRACKET{#1}}
\newcommand{\BINDEF}[2]{\LBRACKET{\begin{array}{ll}
     #1\\
     #2
\end{array}}}

%%%%%%%
% Fourier Analysis
%%%%%%%

\newcommand{\ONEOTWOPI}{\frac{1}{2\pi}}
\newcommand{\FHAT}{\hat{f}(n)}
\newcommand{\FINT}{\int_{-\pi}^\pi}
\newcommand{\FINTWO}{\int_{0}^{2\pi}}
\newcommand{\FSUMN}[1]{\SUM{n=-#1}{#1}}
%\newcommand{\FSUM}{\SUMN{\infty}}
\newcommand{\EIN}[1]{e^{in#1}}
\newcommand{\NEIN}[1]{e^{-in#1}}
\newcommand{\INTALL}{\int_{-\infty}^{\infty}}
\newcommand{\FTINT}[1]{\INTALL #1 e^{2\pi inx\xi} dx}
\newcommand{\GAUSS}{e^{-\pi x^2}}

%%%%%%%
% formatting 
%%%%%%%

\newcommand{\LEFTBOLD}[1]{\noindent\textbf{#1}}
\newcommand{\SEQ}[1]{\{#1\,\}}
\newcommand{\WIP}{\footnote{work in progress}}
\newcommand{\QED}{\hfill\square}
\newcommand{\ts}{\textsuperscript}
\newcommand{\HLINE}{\noindent\rule{7in}{1pt}\\}

%%%%%%%
% Mathematical note taking (definitions, theorems, etc.)
%%%%%%%

\newcommand{\REM}{\noindent\textbf{\\Remark: }}
\newcommand{\DEF}{\noindent\textbf{\\Definition: }}
\newcommand{\THE}{\noindent\textbf{\\Theorem: }}
\newcommand{\COR}{\noindent\textbf{\\Corollary: }}
\newcommand{\LEM}{\noindent\textbf{\\Lemma: }}
\newcommand{\PROP}{\noindent\textbf{\\Proposition: }}
\newcommand{\PROOF}{\noindent\textbf{\\Proof: }}
\newcommand{\EXP}{\noindent\textbf{\\Example: }}
\newcommand{\TRICKS}{\noindent\textbf{\\Tricks: }}


%%%%%%%
% text highlighting
%%%%%%%

\newcommand{\B}[1]{\textbf{#1}}
\newcommand{\CAL}[1]{\mathcal{#1}}
\newcommand{\UL}[1]{\underline{#1}}

%%%%%%
% Linear Algebra
%%%%%%

\newcommand{\COLVECTOR}[1]{\PAREN{\begin{array}{c}
#1
\end{array} }}
\newcommand{\TWOXTWO}[4]{\PAREN{ \begin{array}{c c} #1&#2 \\ #3 & #4 \end{array} }}
\newcommand{\THREEXTHREE}[9]{\PAREN{ \begin{array}{c c c} #1&#2&#3 \\ #4 & #5 & #6 \\ #7 & #8 & #9 \end{array} }}
\newcommand{\NXN}{\PAREN{ \begin{array}{c c c c} 
			a_{11} & a_{12} & \cdots & a_{1n} \\
			a_{21} & a_{22} & \cdots & a_{2n} \\
			\vdots & \vdots & \ddots & a_{1n} \\
			a_{n1} & a_{n2} & \cdots & a_{nn} \\
		\end{array} }}
\newcommand{\SLR}{SL_2(\R)}
\newcommand{\GLR}{GL_2(\R)}
\DeclareMathOperator{\TR}{tr}
\DeclareMathOperator{\BIL}{Bil}
\DeclareMathOperator{\SPAN}{span}

%%%%%%%
%  White space
%%%%%%%

\newcommand{\BOXIT}[1]{\noindent\fbox{\parbox{\textwidth}{#1}}}


\newtheorem{theorem}{Theorem}[section]
\newtheorem{corollary}{Corollary}[theorem]
\newtheorem{lemma}[theorem]{Lemma}

\theoremstyle{definition}
\newtheorem{definition}[theorem]{Definition}
\newtheorem{prop}[theorem]{Proposition}

\theoremstyle{remark}
\newtheorem{remark}[theorem]{Remark}
\newtheorem{example}[theorem]{Example}
%\newtheorem*{proof}[theorem]{Proof}



\newcommand{\RED}[1]{\textcolor{red}{#1}}
\newcommand{\BLUE}[1]{\textcolor{blue}{#1}}

\begin{document}

\begin{center}
	\Large{\CLASSNAME -- \SEMESTER} \\
	\large{ w/Professor Penera}
\end{center}
\begin{center}
	\STUDENTNAME \\
	\ASSIGNMENT -- \DUEDATE\\
\end{center} 

\noindent Page 116: 4, 11 \\

\noindent Page 116: 4. Let $T$ be the linear operator on $\R^2$ defined by 
\begin{align*}
	T\binom{a}{b}=\binom{2a+b}{a-3b}
\end{align*}let $\beta$ be the standard ordered basis for $\R^2$, and let
\begin{align*}
	\beta' &= \BRACKET{\binom{1}{1}, \binom{1}{2}}.
\end{align*}Use Theorem 2.23 and the fact that 
\begin{align*}
	\TWOXTWO{1}{1}{1}{2}^{-1}=\TWOXTWO{2}{-1}{-1}{1}
\end{align*}to find $[T]_{\beta'}$\\


\noindent Page 116: 11 Let $V$ be a finite-dimensional vector space with ordered bases $\alpha, \beta$ and $\gamma$.
	\begin{enumerate}[label=(\alph*)]
		\item Prove that if $Q$ and $R$ are the changed of coordinate matrices that change $\alpha$-coordinates in $\beta$-coordinates and $\beta$-coordiantes into $\gamma$-coordinates, respectiely, then $RQ$ is the change of coordinate matrix that changes $\alpha$-coordinates to $\gamma$-coordinates.
		\item Prove that if $Q$ changes $\alpha$-coordinates into $\beta$-coordinates, then $Q^{-1}$ changes $\beta$-coordinates into $\alpha$-coordinates.
	\end{enumerate} 

\noindent Page 124: 3, 6, 7, 11 \\

\noindent Page 124: 3. For each of the following vector spaces $V$ and bases $\beta$, find explicit formulas for vectors of the dual basis $\beta^*$ for $V^*$, as in Example 4.
\begin{enumerate}[label=(\alph*)]
	\item $V = \R^3; \beta=\{(1,0,1),(1,2,1),(0,0,1)\}$
	
	\BLUE{\begin{align*}
		\delta_{ij} &= f_i(\beta_j) \\
		&\begin{array}{cccc}
			i\backslash j & 1            & 2 & 3 \\
			1 & 1 = f_1(1,0,1) & 0=f_1(1,2,1) & 0=f_1 (0,0,1)\\
			  & 1 = a+c        & 0= a+2b+1 & 0=c \\
			  & f_1(x,y,z) = x-\frac{1}{2}y\\ \\
			2 & 0 = f_2(1,0,1) & 1=f_2(1,2,1) & 0=f_2 (0,0,1)\\
			  & d=-f & 1=d+2e+f & f=0 \\
			  & f_2(x,y,z) = \frac{1}{2}y \\ \\
			3 & 0 = f_3(1,0,1) & 0=f_3(1,2,1) & 1=f_3 (0,0,1)\\
			  & k = -m & 0=k+2l+m & 1=m \\
			  & f_3(x,y,z) = -x+z
		\end{array}
	\end{align*}
	}	
	
	\item $V=P_2(\R); \beta=\{1,x,x^2\}$
	
	\BLUE{\begin{align*}
		f_i(\beta_j) &= \delta_{ij}\\
		&\begin{array}{cccc}
		i\backslash j & 1 & x & x^2 \\
		1  & 1= f_1(1,0,0) & 0=f_1(0,1,0) & 0=f_1(0,0,1)\\
		   & f_1(x,y,z) = x \\
		2  & 0= f_2(1,0,0) & 1=f_2(0,1,0) & 0=f_2(0,0,1)\\
		   & f_2(x,y,z) = y \\
		3  & 0= f_3(1,0,0) & 0=f_3(0,1,0) & 1=f_3(0,0,1) \\
		   & f_3(x,y,z) = z
		\end{array}
	\end{align*}
	}

\end{enumerate}

\noindent Page 124: 6 Define $f \in (\R^2)^*$ by $f(x,y)=2x+y$ and $T: \R^2\to \R^2$ by $T(x,y) = (3x+2y,x)$.
\begin{enumerate}[label=(\alph*)]
\item Compute $T^t(f)$.
\BLUE{\begin{align*}
	T^t(f) &= (f\circ T)(x,y) =f(T(x,y))=f(3x+2y,x) = 2(3x+2y)+x = 7x+4y
\end{align*}
}

\item Compute $[T^t]_{\beta^*}$, where $\beta$ is the standard ordered basis for $\R^2$ and $\beta^*=\{f_1,f_2\}$ is the dual basis, by finding scalars $a,b,c,$ and $d$ such that $T^t(f_1)=af_1+cf_2$ and $T^t(f_2)=bf_1+df_2$.

\RED{\begin{align*}
	\beta^* &= \{ x, y\}\\
	T^t(x) &= ax+cy, \, T^t(y) = bx+dy \\
	T^t(f(x,y)) &= T^(2x+y) \\
		&= T^(2x)+T^(y) \\
		&= af_1(T(2,0))
\end{align*}
}

\item Compute $[T]_\beta$ and $\PAREN{[T]_\beta}^t$, and compare your results with (b).

\BLUE{\begin{align*}
	T(1,0) &= \binom{3}{1} \AND T(0,1) = \binom{2}{0}\\
	[T]_\beta &= \TWOXTWO{3}{2}{1}{0} \\
	\PAREN{[T]_\beta}^t &= \TWOXTWO{3}{1}{2}{0}
\end{align*}
}
\end{enumerate} 

\noindent Page 124: 7 Let $V=P_1(\R)$ and $W=\R^2$ with respective standard ordered bases $\beta$ and $\gamma$.  Define $T:V\to W$ by
\begin{align*}
	T(p(x))&= \PAREN{p(0)-2p(1),\,p(0)+p'(0)},
\end{align*}where $p'(x)$ is the derivative of $p(x)$.
\begin{enumerate}
	\item For $f \in W^*$ defined by $f(a,b)=a-2b$, compute $T^t(f)$.
	\item Compute $[T^t]_{\gamma^*}^{\beta^*}$ without appealing to Theorem 2.25.
	\item Compute $[T]_\beta^\gamma$ and its transpose, and compare your results with (b).
	
\end{enumerate}

\noindent Page 124: 11 let $V$ and $W$ be infinite-dimensional vector spaces over $F$, and let $\psi_1$ and $\psi_2$ be the isomorphisms between $V$ and $V^{**}$ and $W$ and $W^{**}$, respectively, as defined in Theorem 2.26.  Let $T: V \to W$ be linear, and defien $T^{tt}=(T^t)^t$.  Prove that the diagram depicted in Figure 2.6 commutes (i.e, prove that $\psi_2 T = T^{tt}\psi_1$).
\begin{align*}
	\begin{array}{ccc}
		V & \overset{T} \longrightarrow & W \\
		\psi_1 \bigcap\downarrow & & \big\downarrow \psi_2 \\
		V^{**} & \overset{T^{tt}} \longrightarrow  & W^{**}
	\end{array}
\end{align*}

\end{document}